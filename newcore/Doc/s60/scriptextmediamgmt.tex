% Copyright (c) 2009 Nokia Corporation
%
% Licensed under the Apache License, Version 2.0 (the "License");
% you may not use this file except in compliance with the License.
% You may obtain a copy of the License at
%
%     http://www.apache.org/licenses/LICENSE-2.0
%
% Unless required by applicable law or agreed to in writing, software
% distributed under the License is distributed on an "AS IS" BASIS,
% WITHOUT WARRANTIES OR CONDITIONS OF ANY KIND, either express or implied.
% See the License for the specific language governing permissions and
% limitations under the License.

\section{Media Management}
\label{sec:scriptextmediamgmt}

The Media Management service allows Python applications to retrieve information from the media files stored in the media gallery of an S60 device. \break

It is used to access information about different types of media including music, sounds, images, video, and streaming. \break

You can create applications like custom photo viewer or audio player that displays or, otherwise incorporate media, using the Media Management service.

The following sample code is used to load the provider:

\begin{verbatim}
import scriptext
msg_handle = scriptext.load('Service.MediaManagement', 'IDataSource') 
\end{verbatim}

The following table summarizes the Media Management Interface:
\begin{table}[htbp]
\begin{center}
\begin{tabular}{l|l}
\hline
{\bf Service provider} & \code{Service.MediaManagement}  \\
\hline
{\bf Supported interfaces} & \code{IDataSource}  \\
\end{tabular}
\end{center}
\end{table}

The following table lists the services available in Media Management:
\begin{table}[htbp]
\begin{center}
\begin{tabular}{l|l}
\hline
{\bf Services} & {\bf Description} \\
\hline
\code{Getlist} \ref{subsec:mediamgmtgetlist} & Retrieves information from a given service or data source on S60 device.  \\ 
\end{tabular}
\end{center}
\end{table}

\subsection{GetList}
\label{subsec:mediamgmtgetlist}

\code{GetList} takes a set of input parameters that define filter and sort criteria, and retrieves the metadata of media files based on media and metadata type. \break

\code{GetList} implements the main functionality of Media Management service. It is available only in asynchronous mode.

The following is an example for using \code{GetList}:

\begin{verbatim}
media_handle.call('GetList'[{'Type': string, 'Filter': map, 'Sort': map},
                            callback=callback_function])
\end{verbatim}

where, \code{callback_function} is user defined function.

The following table summarizes the specification of \code{GetList}:
\begin{table}[htbp]
\begin{center}
\begin{tabular}{l|l}
\hline
{\bf Interface} & \code{IDataSource}  \\
\hline
{\bf Description} & Retrieves the metadata of media files based on media and metadata type.  \\
\hline
{\bf Response Model} & Asynchronous  \\
\hline
{\bf Pre-condition} & Valid service object representing the provider and interface.  \\
\hline
{\bf Post-condition} & Nil  \\
\end{tabular}
\end{center}
\end{table}

{\bf Input Parameters} \break

Input parameter specifies the Type, the metadata of media file to fetch and the criteria for sorting. Input parameter has three properties: Type, Filter, and Sort.
\begin{table}[htbp]
\begin{center}
\begin{tabular}{l|l|p{3cm}|p{9cm}}
\hline
{\bf Name} & {\bf Type} & {\bf Range} & {\bf Description} \\
\hline
Type & unicode string & \code{Fileinfo} & Operation performed on the specified type. \break
This field is mandatory.  \\
\hline
Filter & map & For more information, refer table \ref{tab:getlistmediafiletype} & It specifies the type of media file to fetch, and key for filtering the media files with their range. \code{FileType} field is mandatory. \break

If key is specified, then it is mandatory to specify the range. \break
You must only mention the StartRange for keys, where EndRange is not applicable. \break

For example, if key is \code{FileName} then, mention the desired file name in the StartRange and leaving the EndRange empty.  \\
\hline
[Sort] & map & For more information, refer table \ref{tab:getlistkeyname} & It specifies the key name on which the resulting output will be sorted and can be one of the values mentioned in the {\bf Value} column. \break
By default, sorting is done in ascending order based on the \code{FileName}.  \\
\end{tabular}
\caption{Input parameters for GetList}
\end{center}
\end{table}

\begin{table}[htbp]
\begin{center}
\begin{tabular}{l|p{6cm}|p{2cm}}
\hline
{\bf Key} & {\bf Value} & {\bf Type}  \\
\hline
\code{FileType} & \code{Music} \break
\code{Sound} \break
\code{Image} \break
\code{Video} \break
\code{StreamingURL} & unicode string  \\
\hline
\code{[Key]} & \code{FileName} \break
\code{FileExtension} \break
\code{Drive} \break
\code{FileSize} \break
\code{FileDate} \break
\code{MimeType} \break
\code{FileNameAndPath} \break
\code{SongName} \break
\code{Artist} \break
\code{Album} \break
\code{Genre} \break
\code{TrackNumber} \break
\code{Composer} \break
\code{LinkFirstURL} & unicode string  \\
\hline
\code{[StartRange]} & Valid for all keys & unicode string  \\
\hline
\code{[EndRange]} & Valid for the following keys: \break

\code{FileSize}(bytes) \break
\code{FileDate}(YYYYMMDD:HHMMSS) \break

For example, 20070303:010101 & unicode string  \\
\end{tabular}
\caption{Media file type}
\label{tab:getlistmediafiletype}
\end{center}
\end{table}

\begin{table}[htbp]
\begin{center}
\begin{tabular}{l|p{4cm}|l}
\hline
{\bf Key} & {\bf Value} & {\bf Type}  \\
\hline
\code{[Key]} & \code{FileName} \break
\code{FileExtension} \break
\code{Drive} \break
\code{FileSize} \break
\code{FileDate} \break
\code{MimeType} \break
\code{FileNameAndPath} \break
\code{SongName} \break
\code{Artist} \break
\code{Album} \break
\code{Genre} \break
\code{TrackNumber} \break
\code{Composer} \break
\code{LinkFirstURL} & unicode string  \\
\hline
\code{[Order]} & Ascending or descending & unicode string  \\
\end{tabular}
\caption{Key name}
\label{tab:getlistkeyname}
\end{center}
\end{table}

{\bf Output Parameters} \break

Output contains \code{ReturnValue}. It also contains \code{ErrorCode} and an \code{ErrorMessage} if the operation fails.
\begin{table}[htbp]
\begin{center}
\begin{tabular}{l|l|p{4cm}|p{7cm}}
\hline
{\bf Name} & {\bf Type} & {\bf Range} & {\bf Description}  \\
\hline
\code{ErrorCode} & int & NA & Contains the SAPI specific error code when the operation fails.  \\
\hline
\code{ErrorMessage} & string & NA & Error Description in Engineering English.  \\
\hline
\code{ReturnValue} & Iterable maps & Type: string \break
\code{FileName}: string \break
\code{FileExtension}: string \break
\code{Drive}: string \break
\code{FileSize}: int \break
\code{FileDate}: datetime \break
\code{MediaType}: int \break
\code{FileNameAndPath}: string \break
\code{SongName}: string \break
\code{Artist}: string \break
\code{Album}: string \break
\code{Genre}: string \break
\code{TrackNumber}: string \break
\code{Composer}: string \break
\code{MimeType}: string \break
\code{LinkFirstURL}: string & The output is an iterable which on each invocation returns a map, which will be filled by the service provider. \break

Map stores the key names and its values. \break
The key-value pair that is, Property name and Value in the output map depends upon the file type in the input -Filter map.  \\
\end{tabular}
\caption{Output parameters for GetList}
\end{center}
\end{table}

{\bf Errors} \break

The following table lists the error codes and their values:
\begin{table}[htbp]
\begin{center}
\begin{tabular}{l|l}
\hline
{\bf Error code value} & {\bf Description}  \\
\hline
\code{1002} & Bad argument type  \\
\hline
\code{1003} & Missing argument  \\
\end{tabular}
\caption{Error codes}
\end{center}
\end{table}

{\bf Error Messages} \break

The following table lists the error messages and their description:
\begin{table}[htbp]
\begin{center}
\begin{tabular}{p{6cm}|p{7cm}}
\hline
{\bf Error messages} & {\bf Description}  \\
\hline
\code{MediaMgmt:GetList:Server busy} & Indicates provider is busy in processing another request.  \\
\hline
\code{MediaMgmt:GetList:Type Missing} & Indicates Type parameter is missing.  \\
\hline
\code{MediaMgmt:GetList:Type not supported(should be FileInfo)} & Indicates that the content type is incorrect.  \\
\hline
\code{MediaMgmt:GetList:Filter parameter missing} & Indicates that the Filter parameter which is mandatory is missing.  \\
\hline
\code{MediaMgmt:GetList:Filter parameter type mismatch} & Indicates that the type of Filter parameter is incorrect.  \\
\hline
\code{MediaMgmt:GetList:Sort parameter type mismatch} & Indicates that the type of Sort  parameter is incorrect.  \\
\hline
\code{MediaMgmt:GetList:FileType missing in Filter map} & Indicates that \code{FileType} parameter is not present in  Filter map or, \code{FileType} parameter type is incorrect.  \\
\end{tabular}
\caption{Error messages}
\end{center}
\end{table}

{\bf Example}

The following sample code illustrates how to get a list of all MP3s:

\begin{verbatim}
import scriptext
import e32

def media_callback(trans_id, event_id, output_params):
    if trans_id == media_trans_id:
        if event_id == scriptext.EventCompleted:
            song_list = []
            for item in output_params['ReturnValue']:
                song_list.append(item['FileName'])
            print "List of files retrieved:", song_list
        else:
            print "Event ID was not EventCompleted"
    else:
        print "Invalid Transaction ID"
    lock.signal()

lock = e32.Ao_lock()
media_handle = scriptext.load('Service.MediaManagement', 'IDataSource')

# Request for the list of mp3s in ascending order
media_trans_id = media_handle.call('GetList', {'Type': u'FileInfo',
                                               'Filter': {'FileType': u'Music',
                                                          'Key': u'FileExtension',
                                                          'StartRange': u'.mp3'},
                                               'Sort': {'Key': u'FileName',
                                                        'Order': u'Ascending'}},
                                   callback=media_callback)

lock.wait()
\end{verbatim}

\subsection{Key Values}
\label{subsec:mediamgmtkeyval}

{\bf File Types} \break
\begin{table}[htbp]
\begin{center}
\begin{tabular}{l|l|p{8cm}}
\hline
{\bf Key} & {\bf Description} & {\bf Supported Metadata}  \\
\hline
\code{Music} & Retrieves media files of Music type. & Artist, SongName, TrackNumber, Album, Genre, Composer, FileNameAndPath, FileName, FileExtension, Drive, MimeType, FileSize, FileDate.  \\
\hline
\code{Sound} & Retrieves media files of Sound type. & FileNameAndPath, FileName, FileExtension, Drive, MimeType, FileSize, FileDate.  \\
\hline
\code{Image} & Retrieves media files of Image type. & FileNameAndPath, FileName, FileExtension, Drive, MimeType, FileSize, FileDate.  \\
\hline
\code{Video} & Retrieves media files of Video type. & FileNameAndPath, FileName, FileExtension, Drive, MimeType, FileSize, FileDate.  \\
\hline
\code{StreamingUrl} & Retrieves media files of Link type. & LinkFirstURL, FileNameAndPath, FileName, FileExtension, Drive, MimeType, FileSize ,FileDate.  \\
\end{tabular}
\end{center}
\end{table}

{\bf Keys} \break
\begin{table}[htbp]
\begin{center}
\begin{tabular}{l|p{8cm}}
\hline
{\bf Key} & {\bf Description}  \\
\hline
Type & Always a media file.  \\
\hline
\code{FileName} & Filter/sort the result as per file name.  \\
\hline
\code{FileExtension} & Filter/sort the result based on file extension.  \\
\hline
\code{Drive} & Filter/sort the result as per file drive.  \\
\hline
\code{FileSize} & Filter/sort the result as per file size.  \\
\hline
\code{FileData} & Filter/sort the result as per file date.  \\
\hline
\code{MediaType} & Filter/sort the result as per file type. \break

{\bf Value}: {\bf Description} \break
0: Unknown media type \break
1: Music media type \break
2: Sound media type \break
3: Image media type \break
4: Video media type \break
5: Streaming URLs  \\
\hline
\code{FileNameAndPath} & Filter/sort the result as per full path of file.  \\
\hline
\code{SongName} & Filter/sort the result as per song name.  \\
\hline
\code{Artist} & Filter/sort the result as per artist name.  \\
\hline
\code{Album} & Filter/sort the result as per album name.  \\
\hline
\code{MimeType} & Filter/Sort the result based on mime type.  \\
\end{tabular}
\end{center}
\end{table}

{\bf Output maps for various values of FileType} \break
\begin{table}[htbp]
\begin{center}
\begin{tabular}{l|p{8cm}}
\hline
{\bf FileType} & {\bf Output map}  \\
\hline
\code{Image} & {\bf Key}: {\bf Value} \break
Type: string \break
\code{FileName}: string \break
\code{FileExtension}: string \break
\code{Drive}: string \break
\code{FileSize}: int \break
\code{FileDate}: datetime \break
\code{FileNameAndPath}: string \break
\code{MimeType}: string  \\
\hline
\code{Sound} & {\bf Key}: {\bf Value} \break
Type: string \break
\code{FileName}: string \break
\code{FileExtension}: string \break
\code{Drive}: string \break
\code{FileSize}: int \break
\code{FileDate}: datetime \break
\code{FileNameAndPath}: string \break
\code{MimeType}: string \\
\hline
\code{Video} & {\bf Key}: {\bf Value} \break
Type: string \break
\code{FileName}: string \break
\code{FileExtension}: string \break
\code{Drive}: string \break
\code{FileSize}: int \break
\code{FileDate}: datetime \break
\code{FileNameAndPath}: string \break
\code{MimeType}: string  \\
\hline
\code{Music} & {\bf Key}: {\bf Value} \break
Type: string \break
\code{FileName}: string \break
\code{FileExtension}: string \break
\code{Drive}: string \break
\code{FileSize}: int \break
\code{FileDate}: datetime \break
\code{MimeType}: string \break
\code{FileNameAndPath}: string \break
\code{SongName}: string \break
\code{Artist}: string \break
\code{Album}: string \break
\code{TrackNumber}: string \break
\code{Genre}: string \break
\code{Composer}: string  \\  \\
\hline
\code{StreamingUrl} & {\bf Key}: {\bf Value} \break
Type: string \break
\code{FileName}: string \break
\code{FileExtension}: string \break
\code{Drive}: string \break
\code{FileSize}: int \break
\code{FileDate}: datetime \break
\code{FileNameAndPath}: string \break
\code{LinkFirstURL}: string \break
\code{MimeType}: string  \\
\end{tabular}
\end{center}
\end{table}
\pagebreak






