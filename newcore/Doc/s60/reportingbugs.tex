% -*- tex -*-
% Copyright (c) 2008 Nokia Corporation
%
% Licensed under the Apache License, Version 2.0 (the "License");
% you may not use this file except in compliance with the License.
% You may obtain a copy of the License at
%
%     http://www.apache.org/licenses/LICENSE-2.0
%
% Unless required by applicable law or agreed to in writing, software
% distributed under the License is distributed on an "AS IS" BASIS,
% WITHOUT WARRANTIES OR CONDITIONS OF ANY KIND, either express or implied.
% See the License for the specific language governing permissions and
% limitations under the License.

\label{reporting-bugs}

In order to improve the quality of Python for S60 the developers would like to 
know of any deficiencies you find in Python for S60 or its documentation.

Before submitting a report, you will be required to log into garage.maemo.org;
this will make it possible for the developers to contact you
for additional information if needed.  It is not possible to submit a
bug report anonymously.

All bug reports should be submitted via the project Python for S60 Bug Tracker on 
garage.maemo.org (\url{https://garage.maemo.org/tracker/?group_id=854}). The bug 
tracker offers a Web form which allows pertinent information to be entered and 
submitted to the developers.

The first step in filing a report is to determine whether the problem
has already been reported.  The advantage in doing so, aside from
saving the developers time, is that you learn what has been done to
fix it; it may be that the problem has already been fixed for the next
release, or additional information is needed (in which case you are
welcome to provide it if you can!).  To do this, search the bug
database using the "Bugs: Browse" link present at the top of the page.

If the problem you're reporting is not already in the bug tracker, then click on the
"Submit New" link at the top of the page to open the bug reporting form.

The submission form has a number of fields.  The only fields that are required 
are the "Summary" and "Detailed descriptioin" fields.  For the summary, enter a 
\emph{very} short description of the problem; less than ten words is good.  In 
the Details field, describe the problem in detail, including what you
expected to happen and what did happen.  Be sure to include the
version of Python for S60 you used using the "Versioin" field, whether any extension modules were
involved and what hardware (the S60 device model or emulator) you were
using, including version information of the S60 SDK and your device
firmware version as appropriate. You can see the device firmware
version by entering \verb|*#0000#| on the device keypad - please
include all information that is shown by this code.

The only other field that you may want to set is the "Category"
field, which allows you to place the bug report into a broad category
(such as "Documentation" or "core").

Each bug report will be assigned to a developer who will determine
what needs to be done to correct the problem.  You will
receive an update each time action is taken on the bug.


\begin{seealso}
  \seetitle[http://www-mice.cs.ucl.ac.uk/multimedia/software/documentation/ReportingBugs.html]{How
        to Report Bugs Effectively}{Article which goes into some
        detail about how to create a useful bug report.  This
        describes what kind of information is useful and why it is
        useful.}

  \seetitle[http://www.mozilla.org/quality/bug-writing-guidelines.html]{Bug
        Writing Guidelines}{Information about writing a good bug
        report.  Some of this is specific to the Mozilla project, but
        describes general good practices.}
\end{seealso}
