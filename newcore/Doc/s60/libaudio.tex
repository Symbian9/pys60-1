% Copyright (c) 2005-2009 Nokia Corporation
%
% Licensed under the Apache License, Version 2.0 (the "License");
% you may not use this file except in compliance with the License.
% You may obtain a copy of the License at
%
%     http://www.apache.org/licenses/LICENSE-2.0
%
% Unless required by applicable law or agreed to in writing, software
% distributed under the License is distributed on an "AS IS" BASIS,
% WITHOUT WARRANTIES OR CONDITIONS OF ANY KIND, either express or implied.
% See the License for the specific language governing permissions and
% limitations under the License.

\section{\module{audio} ---
  An audio related services package}
\label{sec:audio}

\declaremodule{extension}{audio}
\platform{S60}
\modulesynopsis{An audio related services package.}

The \module{audio} module enables recording and playing audio files and access 
to device text-to-speech engine. The \module{audio} module supports all the 
formats supported by the device, typically: WAV, AMR, MIDI, MP3, AAC, and Real 
Audio\footnote{The dynamically loaded audio codec for the sound file is based on 
the MIME-type information inside the audio file and file extension.}. For more 
information on the audio types supported by different devices, see the 
\textit{Forum Nokia} Web site \cite{S60AudioVideo} and \textit{S60 Platform} Web 
site \cite{S60Developers}. 

The following \class{Sound} class static methods are defined in the 
\module{audio} module:

\begin{funcdesc}{Sound.open}{filename}

Returns a new initialized \class{Sound} object with the named file opened. 
Note that \var{filename} should be a full Unicode path name and 
must also include the file extension, for example \code{u'c:\e\e foo.wav'}.
\end{funcdesc}

The following data items for state information are available in \module{audio}:

\begin{datadesc}{ENotReady}
The \class{Sound} object has been constructed but no audio file is open.
\end{datadesc}

\begin{datadesc}{EOpen} 
An audio file is open but no playing or recording operation is in progress.
\end{datadesc}

\begin{datadesc}{EPlaying} 
An audio file is playing.
\end{datadesc}

\begin{datadesc}{ERecording}
An audio file is being recorded.
\end{datadesc}

The following data item is provided for continuous playback of an audio file:

\begin{datadesc}{KMdaRepeatForever}
Possible value for \var{times} parameter in \function{open}.
\end{datadesc}

The following method is available in the \module{audio} module:

\begin{funcdesc}{say}{text, prefix=audio.TTS_PREFIX}
Passes the \code{text} to the device text-to-speech engine. The default 
\code{prefix} is the text-to-speech prefix \code{"(tts)"}. 

\code{text} should be either Unicode or a UTF-8 encoded plain
string. The speech synthesizer pronounces the text according to the
current device language.
\end{funcdesc}

\subsection{Sound Objects}
\label{subsec:sound}

\begin{classdesc*}{Sound}

\class{Sound} objects have the following functions:

\begin{methoddesc}[Sound]{play}{\optional{times=1, interval=0, callback=None}}

Starts playback of an audio file from the beginning. Without the parameters 
\var{times} and \var{interval} it plays the audio file one time. 
\var{times} defines the number of times the audio file is played, the 
default being \var{1}. If the audio file is played several times, 
\var{interval} gives the time interval between the subsequent plays in 
microseconds. 

The optional callback is called when the playing starts and when the end of the 
sound file is reached. The callback should take three parameters: the previous 
state, the current state and the possible error code. The possible states given 
as parameters to the callback are data items in the module \code{audio}.

Other issues: 

\begin{itemize}
\item Calling \code{play(audio.KMdaRepeatForever)} will repeat the file forever. 
\item If an audio file is played but not stopped before exiting, the Python script will leave audio playing on; therefore \code{stop} needs to be called explicitly prior to exit.
\item Currently the module does not support playing simultaneous audio files, calling \code{play} to a second \class{Sound} instance while another audio file is playing, stops the earlier audio file and starts to play the second \class{Sound} instance.
\item Calling \code{play} while a telephone call is ongoing plays the sound file to uplink. In some devices the sound file is also played to the device speaker.
\item Calling \code{play} when already playing or recording results in \code{RuntimeError}. Calling \code{stop} prior to \code{play} will prevent this from happening.
\end{itemize}
\end{methoddesc}

\begin{methoddesc}[Sound]{stop}{}
Stops playback or recording of an audio file.
\end{methoddesc}

\begin{methoddesc}[Sound]{record}{}
Starts recording audio data to a file. If the file already exists, the 
operation appends to the file. For Nokia devices, WAV is typically supported 
for recording. For more information on the audio types supported by 
different devices, see the \textit{Forum Nokia} Web site 
\cite{S60AudioVideo} and \textit{S60 Platform} Web site 
\cite{S60Developers}. Other issues:

\begin{itemize}
\item Calling \code{record} while a telephone call is ongoing starts the recording of the telephone call. 
\item Calling \code{record} when already playing or recording results in \code{RuntimeError}. Calling \code{stop} prior to \code{record} will prevent this from happening.
\end{itemize}
\end{methoddesc}

\begin{methoddesc}[Sound]{close}{}
Closes an opened audio file.
\end{methoddesc}

\begin{methoddesc}[Sound]{state}{}
Returns the current state of the \class{Sound} type instance. The different 
states (constants) are defined in the \module{audio} module. The possible 
states\footnote{Descriptions for these options are based on information 
found in S60 SDK documentation \cite{S60Doc}.} are:

\begin{itemize}
\item \code{ENotReady} \newline
The \textsf{Sound} object has been constructed but no audio file is open.
\item \code{EOpen} \newline
An audio file is open but no playing or recording operation is in progress.
\item \code{EPlaying} \newline
An audio file is playing.
\item \code{ERecording} \newline
An audio file is being recorded.
\end{itemize}
\end{methoddesc}

\begin{methoddesc}[Sound]{max_volume}{}
Returns the maximum volume of the device.
\end{methoddesc}

\begin{methoddesc}[Sound]{set_volume}{volume}

Sets the volume. If the given volume is negative, then the volume is set to 
zero which mutes the device. If the volume is greater than \code{max_volume}, 
then \code{max_volume} is used.
\end{methoddesc}

\begin{methoddesc}[Sound]{current_volume}{}
Returns the current volume set.
\end{methoddesc}

\begin{methoddesc}[Sound]{duration}{}
Returns the duration of the file in microseconds.
\end{methoddesc}

\begin{methoddesc}[Sound]{set_position}{microseconds}
Set the position for the playhead.
\end{methoddesc}

\begin{methoddesc}[Sound]{current_position}{}
Returns the current playhead position in microseconds.
\end{methoddesc}

\end{classdesc*}
