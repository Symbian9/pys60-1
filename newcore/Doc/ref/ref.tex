\documentclass{manual}

\title{Python Reference Manual}

\author{Guido van Rossum\\
	Fred L. Drake, Jr., editor}
\authoraddress{
	\strong{Python Software Foundation}\\
	Email: \email{docs@python.org}
}

\date{23rd December, 2008}			% XXX update before final release!
\input{patchlevel}		% include Python version information


\makeindex

\begin{document}

\maketitle

\ifhtml
\chapter*{Front Matter\label{front}}
\fi

Copyright \copyright{} 2001-2008 Python Software Foundation.
All rights reserved.

Copyright \copyright{} 2000 BeOpen.com.
All rights reserved.

Copyright \copyright{} 1995-2000 Corporation for National Research Initiatives.
All rights reserved.

Copyright \copyright{} 1991-1995 Stichting Mathematisch Centrum.
All rights reserved.

See the end of this document for complete license and permissions
information.


\begin{abstract}

\noindent
Python is an interpreted, object-oriented, high-level programming
language with dynamic semantics.  Its high-level built in data
structures, combined with dynamic typing and dynamic binding, make it
very attractive for rapid application development, as well as for use
as a scripting or glue language to connect existing components
together.  Python's simple, easy to learn syntax emphasizes
readability and therefore reduces the cost of program
maintenance.  Python supports modules and packages, which encourages
program modularity and code reuse.  The Python interpreter and the
extensive standard library are available in source or binary form
without charge for all major platforms, and can be freely distributed.

This reference manual describes the syntax and ``core semantics'' of
the language.  It is terse, but attempts to be exact and complete.
The semantics of non-essential built-in object types and of the
built-in functions and modules are described in the
\citetitle[../lib/lib.html]{Python Library Reference}.  For an
informal introduction to the language, see the
\citetitle[../tut/tut.html]{Python Tutorial}.  For C or
\Cpp{} programmers, two additional manuals exist:
\citetitle[../ext/ext.html]{Extending and Embedding the Python
Interpreter} describes the high-level picture of how to write a Python
extension module, and the \citetitle[../api/api.html]{Python/C API
Reference Manual} describes the interfaces available to
C/\Cpp{} programmers in detail.

\end{abstract}

\tableofcontents

\chapter{Introduction\label{introduction}}

This reference manual describes the Python programming language.
It is not intended as a tutorial.

While I am trying to be as precise as possible, I chose to use English
rather than formal specifications for everything except syntax and
lexical analysis.  This should make the document more understandable
to the average reader, but will leave room for ambiguities.
Consequently, if you were coming from Mars and tried to re-implement
Python from this document alone, you might have to guess things and in
fact you would probably end up implementing quite a different language.
On the other hand, if you are using
Python and wonder what the precise rules about a particular area of
the language are, you should definitely be able to find them here.
If you would like to see a more formal definition of the language,
maybe you could volunteer your time --- or invent a cloning machine
:-).

It is dangerous to add too many implementation details to a language
reference document --- the implementation may change, and other
implementations of the same language may work differently.  On the
other hand, there is currently only one Python implementation in
widespread use (although alternate implementations exist), and
its particular quirks are sometimes worth being mentioned, especially
where the implementation imposes additional limitations.  Therefore,
you'll find short ``implementation notes'' sprinkled throughout the
text.

Every Python implementation comes with a number of built-in and
standard modules.  These are not documented here, but in the separate
\citetitle[../lib/lib.html]{Python Library Reference} document.  A few
built-in modules are mentioned when they interact in a significant way
with the language definition.


\section{Alternate Implementations\label{implementations}}

Though there is one Python implementation which is by far the most
popular, there are some alternate implementations which are of
particular interest to different audiences.

Known implementations include:

\begin{itemize}
\item[CPython]
This is the original and most-maintained implementation of Python,
written in C.  New language features generally appear here first.

\item[Jython]
Python implemented in Java.  This implementation can be used as a
scripting language for Java applications, or can be used to create
applications using the Java class libraries.  It is also often used to
create tests for Java libraries.  More information can be found at
\ulink{the Jython website}{http://www.jython.org/}.

\item[Python for .NET]
This implementation actually uses the CPython implementation, but is a
managed .NET application and makes .NET libraries available.  This was
created by Brian Lloyd.  For more information, see the \ulink{Python
for .NET home page}{http://www.zope.org/Members/Brian/PythonNet}.

\item[IronPython]
An alternate Python for\ .NET.  Unlike Python.NET, this is a complete
Python implementation that generates IL, and compiles Python code
directly to\ .NET assemblies.  It was created by Jim Hugunin, the
original creator of Jython.  For more information, see \ulink{the
IronPython website}{http://workspaces.gotdotnet.com/ironpython}.

\item[PyPy]
An implementation of Python written in Python; even the bytecode
interpreter is written in Python.  This is executed using CPython as
the underlying interpreter.  One of the goals of the project is to
encourage experimentation with the language itself by making it easier
to modify the interpreter (since it is written in Python).  Additional
information is available on \ulink{the PyPy project's home
page}{http://codespeak.net/pypy/}.
\end{itemize}

Each of these implementations varies in some way from the language as
documented in this manual, or introduces specific information beyond
what's covered in the standard Python documentation.  Please refer to
the implementation-specific documentation to determine what else you
need to know about the specific implementation you're using.


\section{Notation\label{notation}}

The descriptions of lexical analysis and syntax use a modified BNF
grammar notation.  This uses the following style of definition:
\index{BNF}
\index{grammar}
\index{syntax}
\index{notation}

\begin{productionlist}[*]
  \production{name}{\token{lc_letter} (\token{lc_letter} | "_")*}
  \production{lc_letter}{"a"..."z"}
\end{productionlist}

The first line says that a \code{name} is an \code{lc_letter} followed by
a sequence of zero or more \code{lc_letter}s and underscores.  An
\code{lc_letter} in turn is any of the single characters \character{a}
through \character{z}.  (This rule is actually adhered to for the
names defined in lexical and grammar rules in this document.)

Each rule begins with a name (which is the name defined by the rule)
and \code{::=}.  A vertical bar (\code{|}) is used to separate
alternatives; it is the least binding operator in this notation.  A
star (\code{*}) means zero or more repetitions of the preceding item;
likewise, a plus (\code{+}) means one or more repetitions, and a
phrase enclosed in square brackets (\code{[ ]}) means zero or one
occurrences (in other words, the enclosed phrase is optional).  The
\code{*} and \code{+} operators bind as tightly as possible;
parentheses are used for grouping.  Literal strings are enclosed in
quotes.  White space is only meaningful to separate tokens.
Rules are normally contained on a single line; rules with many
alternatives may be formatted alternatively with each line after the
first beginning with a vertical bar.

In lexical definitions (as the example above), two more conventions
are used: Two literal characters separated by three dots mean a choice
of any single character in the given (inclusive) range of \ASCII{}
characters.  A phrase between angular brackets (\code{<...>}) gives an
informal description of the symbol defined; e.g., this could be used
to describe the notion of `control character' if needed.
\index{lexical definitions}
\index{ASCII@\ASCII}

Even though the notation used is almost the same, there is a big
difference between the meaning of lexical and syntactic definitions:
a lexical definition operates on the individual characters of the
input source, while a syntax definition operates on the stream of
tokens generated by the lexical analysis.  All uses of BNF in the next
chapter (``Lexical Analysis'') are lexical definitions; uses in
subsequent chapters are syntactic definitions.
		% Introduction
\chapter{Lexical analysis\label{lexical}}

A Python program is read by a \emph{parser}.  Input to the parser is a
stream of \emph{tokens}, generated by the \emph{lexical analyzer}.  This
chapter describes how the lexical analyzer breaks a file into tokens.
\index{lexical analysis}
\index{parser}
\index{token}

Python uses the 7-bit \ASCII{} character set for program text.
\versionadded[An encoding declaration can be used to indicate that 
string literals and comments use an encoding different from ASCII]{2.3}
For compatibility with older versions, Python only warns if it finds
8-bit characters; those warnings should be corrected by either declaring
an explicit encoding, or using escape sequences if those bytes are binary
data, instead of characters.


The run-time character set depends on the I/O devices connected to the
program but is generally a superset of \ASCII.

\strong{Future compatibility note:} It may be tempting to assume that the
character set for 8-bit characters is ISO Latin-1 (an \ASCII{}
superset that covers most western languages that use the Latin
alphabet), but it is possible that in the future Unicode text editors
will become common.  These generally use the UTF-8 encoding, which is
also an \ASCII{} superset, but with very different use for the
characters with ordinals 128-255.  While there is no consensus on this
subject yet, it is unwise to assume either Latin-1 or UTF-8, even
though the current implementation appears to favor Latin-1.  This
applies both to the source character set and the run-time character
set.


\section{Line structure\label{line-structure}}

A Python program is divided into a number of \emph{logical lines}.
\index{line structure}


\subsection{Logical lines\label{logical}}

The end of
a logical line is represented by the token NEWLINE.  Statements cannot
cross logical line boundaries except where NEWLINE is allowed by the
syntax (e.g., between statements in compound statements).
A logical line is constructed from one or more \emph{physical lines}
by following the explicit or implicit \emph{line joining} rules.
\index{logical line}
\index{physical line}
\index{line joining}
\index{NEWLINE token}


\subsection{Physical lines\label{physical}}

A physical line is a sequence of characters terminated by an end-of-line
sequence.  In source files, any of the standard platform line
termination sequences can be used - the \UNIX{} form using \ASCII{} LF
(linefeed), the Windows form using the \ASCII{} sequence CR LF (return
followed by linefeed), or the Macintosh form using the \ASCII{} CR
(return) character.  All of these forms can be used equally, regardless
of platform.

When embedding Python, source code strings should be passed to Python
APIs using the standard C conventions for newline characters (the
\code{\e n} character, representing \ASCII{} LF, is the line
terminator).


\subsection{Comments\label{comments}}

A comment starts with a hash character (\code{\#}) that is not part of
a string literal, and ends at the end of the physical line.  A comment
signifies the end of the logical line unless the implicit line joining
rules are invoked.
Comments are ignored by the syntax; they are not tokens.
\index{comment}
\index{hash character}


\subsection{Encoding declarations\label{encodings}}
\index{source character set}
\index{encodings}

If a comment in the first or second line of the Python script matches
the regular expression \regexp{coding[=:]\e s*([-\e w.]+)}, this comment is
processed as an encoding declaration; the first group of this
expression names the encoding of the source code file. The recommended
forms of this expression are

\begin{verbatim}
# -*- coding: <encoding-name> -*-
\end{verbatim}

which is recognized also by GNU Emacs, and

\begin{verbatim}
# vim:fileencoding=<encoding-name>
\end{verbatim}

which is recognized by Bram Moolenaar's VIM. In addition, if the first
bytes of the file are the UTF-8 byte-order mark
(\code{'\e xef\e xbb\e xbf'}), the declared file encoding is UTF-8
(this is supported, among others, by Microsoft's \program{notepad}).

If an encoding is declared, the encoding name must be recognized by
Python. % XXX there should be a list of supported encodings.
The encoding is used for all lexical analysis, in particular to find
the end of a string, and to interpret the contents of Unicode literals.
String literals are converted to Unicode for syntactical analysis,
then converted back to their original encoding before interpretation
starts. The encoding declaration must appear on a line of its own.

\subsection{Explicit line joining\label{explicit-joining}}

Two or more physical lines may be joined into logical lines using
backslash characters (\code{\e}), as follows: when a physical line ends
in a backslash that is not part of a string literal or comment, it is
joined with the following forming a single logical line, deleting the
backslash and the following end-of-line character.  For example:
\index{physical line}
\index{line joining}
\index{line continuation}
\index{backslash character}
%
\begin{verbatim}
if 1900 < year < 2100 and 1 <= month <= 12 \
   and 1 <= day <= 31 and 0 <= hour < 24 \
   and 0 <= minute < 60 and 0 <= second < 60:   # Looks like a valid date
        return 1
\end{verbatim}

A line ending in a backslash cannot carry a comment.  A backslash does
not continue a comment.  A backslash does not continue a token except
for string literals (i.e., tokens other than string literals cannot be
split across physical lines using a backslash).  A backslash is
illegal elsewhere on a line outside a string literal.


\subsection{Implicit line joining\label{implicit-joining}}

Expressions in parentheses, square brackets or curly braces can be
split over more than one physical line without using backslashes.
For example:

\begin{verbatim}
month_names = ['Januari', 'Februari', 'Maart',      # These are the
               'April',   'Mei',      'Juni',       # Dutch names
               'Juli',    'Augustus', 'September',  # for the months
               'Oktober', 'November', 'December']   # of the year
\end{verbatim}

Implicitly continued lines can carry comments.  The indentation of the
continuation lines is not important.  Blank continuation lines are
allowed.  There is no NEWLINE token between implicit continuation
lines.  Implicitly continued lines can also occur within triple-quoted
strings (see below); in that case they cannot carry comments.


\subsection{Blank lines \label{blank-lines}}

\index{blank line}
A logical line that contains only spaces, tabs, formfeeds and possibly
a comment, is ignored (i.e., no NEWLINE token is generated).  During
interactive input of statements, handling of a blank line may differ
depending on the implementation of the read-eval-print loop.  In the
standard implementation, an entirely blank logical line (i.e.\ one
containing not even whitespace or a comment) terminates a multi-line
statement.


\subsection{Indentation\label{indentation}}

Leading whitespace (spaces and tabs) at the beginning of a logical
line is used to compute the indentation level of the line, which in
turn is used to determine the grouping of statements.
\index{indentation}
\index{whitespace}
\index{leading whitespace}
\index{space}
\index{tab}
\index{grouping}
\index{statement grouping}

First, tabs are replaced (from left to right) by one to eight spaces
such that the total number of characters up to and including the
replacement is a multiple of
eight (this is intended to be the same rule as used by \UNIX).  The
total number of spaces preceding the first non-blank character then
determines the line's indentation.  Indentation cannot be split over
multiple physical lines using backslashes; the whitespace up to the
first backslash determines the indentation.

\strong{Cross-platform compatibility note:} because of the nature of
text editors on non-UNIX platforms, it is unwise to use a mixture of
spaces and tabs for the indentation in a single source file.  It
should also be noted that different platforms may explicitly limit the
maximum indentation level.

A formfeed character may be present at the start of the line; it will
be ignored for the indentation calculations above.  Formfeed
characters occurring elsewhere in the leading whitespace have an
undefined effect (for instance, they may reset the space count to
zero).

The indentation levels of consecutive lines are used to generate
INDENT and DEDENT tokens, using a stack, as follows.
\index{INDENT token}
\index{DEDENT token}

Before the first line of the file is read, a single zero is pushed on
the stack; this will never be popped off again.  The numbers pushed on
the stack will always be strictly increasing from bottom to top.  At
the beginning of each logical line, the line's indentation level is
compared to the top of the stack.  If it is equal, nothing happens.
If it is larger, it is pushed on the stack, and one INDENT token is
generated.  If it is smaller, it \emph{must} be one of the numbers
occurring on the stack; all numbers on the stack that are larger are
popped off, and for each number popped off a DEDENT token is
generated.  At the end of the file, a DEDENT token is generated for
each number remaining on the stack that is larger than zero.

Here is an example of a correctly (though confusingly) indented piece
of Python code:

\begin{verbatim}
def perm(l):
        # Compute the list of all permutations of l
    if len(l) <= 1:
                  return [l]
    r = []
    for i in range(len(l)):
             s = l[:i] + l[i+1:]
             p = perm(s)
             for x in p:
              r.append(l[i:i+1] + x)
    return r
\end{verbatim}

The following example shows various indentation errors:

\begin{verbatim}
 def perm(l):                       # error: first line indented
for i in range(len(l)):             # error: not indented
    s = l[:i] + l[i+1:]
        p = perm(l[:i] + l[i+1:])   # error: unexpected indent
        for x in p:
                r.append(l[i:i+1] + x)
            return r                # error: inconsistent dedent
\end{verbatim}

(Actually, the first three errors are detected by the parser; only the
last error is found by the lexical analyzer --- the indentation of
\code{return r} does not match a level popped off the stack.)


\subsection{Whitespace between tokens\label{whitespace}}

Except at the beginning of a logical line or in string literals, the
whitespace characters space, tab and formfeed can be used
interchangeably to separate tokens.  Whitespace is needed between two
tokens only if their concatenation could otherwise be interpreted as a
different token (e.g., ab is one token, but a b is two tokens).


\section{Other tokens\label{other-tokens}}

Besides NEWLINE, INDENT and DEDENT, the following categories of tokens
exist: \emph{identifiers}, \emph{keywords}, \emph{literals},
\emph{operators}, and \emph{delimiters}.
Whitespace characters (other than line terminators, discussed earlier)
are not tokens, but serve to delimit tokens.
Where
ambiguity exists, a token comprises the longest possible string that
forms a legal token, when read from left to right.


\section{Identifiers and keywords\label{identifiers}}

Identifiers (also referred to as \emph{names}) are described by the following
lexical definitions:
\index{identifier}
\index{name}

\begin{productionlist}
  \production{identifier}
             {(\token{letter}|"_") (\token{letter} | \token{digit} | "_")*}
  \production{letter}
             {\token{lowercase} | \token{uppercase}}
  \production{lowercase}
             {"a"..."z"}
  \production{uppercase}
             {"A"..."Z"}
  \production{digit}
             {"0"..."9"}
\end{productionlist}

Identifiers are unlimited in length.  Case is significant.


\subsection{Keywords\label{keywords}}

The following identifiers are used as reserved words, or
\emph{keywords} of the language, and cannot be used as ordinary
identifiers.  They must be spelled exactly as written here:%
\index{keyword}%
\index{reserved word}

\begin{verbatim}
and       del       from      not       while    
as        elif      global    or        with     
assert    else      if        pass      yield    
break     except    import    print              
class     exec      in        raise              
continue  finally   is        return             
def       for       lambda    try 
\end{verbatim}

% When adding keywords, use reswords.py for reformatting

\versionchanged[\constant{None} became a constant and is now
recognized by the compiler as a name for the built-in object
\constant{None}.  Although it is not a keyword, you cannot assign
a different object to it]{2.4}

\versionchanged[Both \keyword{as} and \keyword{with} are only recognized
when the \code{with_statement} future feature has been enabled.
It will always be enabled in Python 2.6.  See section~\ref{with} for
details.  Note that using \keyword{as} and \keyword{with} as identifiers
will always issue a warning, even when the \code{with_statement} future
directive is not in effect]{2.5}


\subsection{Reserved classes of identifiers\label{id-classes}}

Certain classes of identifiers (besides keywords) have special
meanings.  These classes are identified by the patterns of leading and
trailing underscore characters:

\begin{description}

\item[\code{_*}]
  Not imported by \samp{from \var{module} import *}.  The special
  identifier \samp{_} is used in the interactive interpreter to store
  the result of the last evaluation; it is stored in the
  \module{__builtin__} module.  When not in interactive mode, \samp{_}
  has no special meaning and is not defined.
  See section~\ref{import}, ``The \keyword{import} statement.''

  \note{The name \samp{_} is often used in conjunction with
  internationalization; refer to the documentation for the
  \ulink{\module{gettext} module}{../lib/module-gettext.html} for more
  information on this convention.}

\item[\code{__*__}]
  System-defined names.  These names are defined by the interpreter
  and its implementation (including the standard library);
  applications should not expect to define additional names using this
  convention.  The set of names of this class defined by Python may be
  extended in future versions.
  See section~\ref{specialnames}, ``Special method names.''

\item[\code{__*}]
  Class-private names.  Names in this category, when used within the
  context of a class definition, are re-written to use a mangled form
  to help avoid name clashes between ``private'' attributes of base
  and derived classes.
  See section~\ref{atom-identifiers}, ``Identifiers (Names).''

\end{description}


\section{Literals\label{literals}}

Literals are notations for constant values of some built-in types.
\index{literal}
\index{constant}


\subsection{String literals\label{strings}}

String literals are described by the following lexical definitions:
\index{string literal}

\index{ASCII@\ASCII}
\begin{productionlist}
  \production{stringliteral}
             {[\token{stringprefix}](\token{shortstring} | \token{longstring})}
  \production{stringprefix}
             {"r" | "u" | "ur" | "R" | "U" | "UR" | "Ur" | "uR"}
  \production{shortstring}
             {"'" \token{shortstringitem}* "'"
              | '"' \token{shortstringitem}* '"'}
  \production{longstring}
             {"'''" \token{longstringitem}* "'''"}
  \productioncont{| '"""' \token{longstringitem}* '"""'}
  \production{shortstringitem}
             {\token{shortstringchar} | \token{escapeseq}}
  \production{longstringitem}
             {\token{longstringchar} | \token{escapeseq}}
  \production{shortstringchar}
             {<any source character except "\e" or newline or the quote>}
  \production{longstringchar}
             {<any source character except "\e">}
  \production{escapeseq}
             {"\e" <any ASCII character>}
\end{productionlist}

One syntactic restriction not indicated by these productions is that
whitespace is not allowed between the \grammartoken{stringprefix} and
the rest of the string literal. The source character set is defined
by the encoding declaration; it is \ASCII{} if no encoding declaration
is given in the source file; see section~\ref{encodings}.

\index{triple-quoted string}
\index{Unicode Consortium}
\index{string!Unicode}
In plain English: String literals can be enclosed in matching single
quotes (\code{'}) or double quotes (\code{"}).  They can also be
enclosed in matching groups of three single or double quotes (these
are generally referred to as \emph{triple-quoted strings}).  The
backslash (\code{\e}) character is used to escape characters that
otherwise have a special meaning, such as newline, backslash itself,
or the quote character.  String literals may optionally be prefixed
with a letter \character{r} or \character{R}; such strings are called
\dfn{raw strings}\index{raw string} and use different rules for interpreting
backslash escape sequences.  A prefix of \character{u} or \character{U}
makes the string a Unicode string.  Unicode strings use the Unicode character
set as defined by the Unicode Consortium and ISO~10646.  Some additional
escape sequences, described below, are available in Unicode strings.
The two prefix characters may be combined; in this case, \character{u} must
appear before \character{r}.

In triple-quoted strings,
unescaped newlines and quotes are allowed (and are retained), except
that three unescaped quotes in a row terminate the string.  (A
``quote'' is the character used to open the string, i.e. either
\code{'} or \code{"}.)

Unless an \character{r} or \character{R} prefix is present, escape
sequences in strings are interpreted according to rules similar
to those used by Standard C.  The recognized escape sequences are:
\index{physical line}
\index{escape sequence}
\index{Standard C}
\index{C}

\begin{tableiii}{l|l|c}{code}{Escape Sequence}{Meaning}{Notes}
\lineiii{\e\var{newline}} {Ignored}{}
\lineiii{\e\e}	{Backslash (\code{\e})}{}
\lineiii{\e'}	{Single quote (\code{'})}{}
\lineiii{\e"}	{Double quote (\code{"})}{}
\lineiii{\e a}	{\ASCII{} Bell (BEL)}{}
\lineiii{\e b}	{\ASCII{} Backspace (BS)}{}
\lineiii{\e f}	{\ASCII{} Formfeed (FF)}{}
\lineiii{\e n}	{\ASCII{} Linefeed (LF)}{}
\lineiii{\e N\{\var{name}\}}
        {Character named \var{name} in the Unicode database (Unicode only)}{}
\lineiii{\e r}	{\ASCII{} Carriage Return (CR)}{}
\lineiii{\e t}	{\ASCII{} Horizontal Tab (TAB)}{}
\lineiii{\e u\var{xxxx}}
        {Character with 16-bit hex value \var{xxxx} (Unicode only)}{(1)}
\lineiii{\e U\var{xxxxxxxx}}
        {Character with 32-bit hex value \var{xxxxxxxx} (Unicode only)}{(2)}
\lineiii{\e v}	{\ASCII{} Vertical Tab (VT)}{}
\lineiii{\e\var{ooo}} {Character with octal value \var{ooo}}{(3,5)}
\lineiii{\e x\var{hh}} {Character with hex value \var{hh}}{(4,5)}
\end{tableiii}
\index{ASCII@\ASCII}

\noindent
Notes:

\begin{itemize}
\item[(1)]
  Individual code units which form parts of a surrogate pair can be
  encoded using this escape sequence.
\item[(2)]
  Any Unicode character can be encoded this way, but characters
  outside the Basic Multilingual Plane (BMP) will be encoded using a
  surrogate pair if Python is compiled to use 16-bit code units (the
  default).  Individual code units which form parts of a surrogate
  pair can be encoded using this escape sequence.
\item[(3)]
  As in Standard C, up to three octal digits are accepted.
\item[(4)]
  Unlike in Standard C, exactly two hex digits are required.
\item[(5)]
  In a string literal, hexadecimal and octal escapes denote the
  byte with the given value; it is not necessary that the byte
  encodes a character in the source character set. In a Unicode
  literal, these escapes denote a Unicode character with the given
  value.
\end{itemize}


Unlike Standard \index{unrecognized escape sequence}C,
all unrecognized escape sequences are left in the string unchanged,
i.e., \emph{the backslash is left in the string}.  (This behavior is
useful when debugging: if an escape sequence is mistyped, the
resulting output is more easily recognized as broken.)  It is also
important to note that the escape sequences marked as ``(Unicode
only)'' in the table above fall into the category of unrecognized
escapes for non-Unicode string literals.

When an \character{r} or \character{R} prefix is present, a character
following a backslash is included in the string without change, and \emph{all
backslashes are left in the string}.  For example, the string literal
\code{r"\e n"} consists of two characters: a backslash and a lowercase
\character{n}.  String quotes can be escaped with a backslash, but the
backslash remains in the string; for example, \code{r"\e""} is a valid string
literal consisting of two characters: a backslash and a double quote;
\code{r"\e"} is not a valid string literal (even a raw string cannot
end in an odd number of backslashes).  Specifically, \emph{a raw
string cannot end in a single backslash} (since the backslash would
escape the following quote character).  Note also that a single
backslash followed by a newline is interpreted as those two characters
as part of the string, \emph{not} as a line continuation.

When an \character{r} or \character{R} prefix is used in conjunction
with a \character{u} or \character{U} prefix, then the \code{\e uXXXX}
and \code{\e UXXXXXXXX} escape sequences are processed while 
\emph{all other backslashes are left in the string}.
For example, the string literal
\code{ur"\e{}u0062\e n"} consists of three Unicode characters: `LATIN
SMALL LETTER B', `REVERSE SOLIDUS', and `LATIN SMALL LETTER N'.
Backslashes can be escaped with a preceding backslash; however, both
remain in the string.  As a result, \code{\e uXXXX} escape sequences
are only recognized when there are an odd number of backslashes.

\subsection{String literal concatenation\label{string-catenation}}

Multiple adjacent string literals (delimited by whitespace), possibly
using different quoting conventions, are allowed, and their meaning is
the same as their concatenation.  Thus, \code{"hello" 'world'} is
equivalent to \code{"helloworld"}.  This feature can be used to reduce
the number of backslashes needed, to split long strings conveniently
across long lines, or even to add comments to parts of strings, for
example:

\begin{verbatim}
re.compile("[A-Za-z_]"       # letter or underscore
           "[A-Za-z0-9_]*"   # letter, digit or underscore
          )
\end{verbatim}

Note that this feature is defined at the syntactical level, but
implemented at compile time.  The `+' operator must be used to
concatenate string expressions at run time.  Also note that literal
concatenation can use different quoting styles for each component
(even mixing raw strings and triple quoted strings).


\subsection{Numeric literals\label{numbers}}

There are four types of numeric literals: plain integers, long
integers, floating point numbers, and imaginary numbers.  There are no
complex literals (complex numbers can be formed by adding a real
number and an imaginary number).
\index{number}
\index{numeric literal}
\index{integer literal}
\index{plain integer literal}
\index{long integer literal}
\index{floating point literal}
\index{hexadecimal literal}
\index{octal literal}
\index{decimal literal}
\index{imaginary literal}
\index{complex!literal}

Note that numeric literals do not include a sign; a phrase like
\code{-1} is actually an expression composed of the unary operator
`\code{-}' and the literal \code{1}.


\subsection{Integer and long integer literals\label{integers}}

Integer and long integer literals are described by the following
lexical definitions:

\begin{productionlist}
  \production{longinteger}
             {\token{integer} ("l" | "L")}
  \production{integer}
             {\token{decimalinteger} | \token{octinteger} | \token{hexinteger}}
  \production{decimalinteger}
             {\token{nonzerodigit} \token{digit}* | "0"}
  \production{octinteger}
             {"0" \token{octdigit}+}
  \production{hexinteger}
             {"0" ("x" | "X") \token{hexdigit}+}
  \production{nonzerodigit}
             {"1"..."9"}
  \production{octdigit}
             {"0"..."7"}
  \production{hexdigit}
             {\token{digit} | "a"..."f" | "A"..."F"}
\end{productionlist}

Although both lower case \character{l} and upper case \character{L} are
allowed as suffix for long integers, it is strongly recommended to always
use \character{L}, since the letter \character{l} looks too much like the
digit \character{1}.

Plain integer literals that are above the largest representable plain
integer (e.g., 2147483647 when using 32-bit arithmetic) are accepted
as if they were long integers instead.\footnote{In versions of Python
prior to 2.4, octal and hexadecimal literals in the range just above
the largest representable plain integer but below the largest unsigned
32-bit number (on a machine using 32-bit arithmetic), 4294967296, were
taken as the negative plain integer obtained by subtracting 4294967296
from their unsigned value.}  There is no limit for long integer
literals apart from what can be stored in available memory.

Some examples of plain integer literals (first row) and long integer
literals (second and third rows):

\begin{verbatim}
7     2147483647                        0177
3L    79228162514264337593543950336L    0377L   0x100000000L
      79228162514264337593543950336             0xdeadbeef						    
\end{verbatim}


\subsection{Floating point literals\label{floating}}

Floating point literals are described by the following lexical
definitions:

\begin{productionlist}
  \production{floatnumber}
             {\token{pointfloat} | \token{exponentfloat}}
  \production{pointfloat}
             {[\token{intpart}] \token{fraction} | \token{intpart} "."}
  \production{exponentfloat}
             {(\token{intpart} | \token{pointfloat})
              \token{exponent}}
  \production{intpart}
             {\token{digit}+}
  \production{fraction}
             {"." \token{digit}+}
  \production{exponent}
             {("e" | "E") ["+" | "-"] \token{digit}+}
\end{productionlist}

Note that the integer and exponent parts of floating point numbers
can look like octal integers, but are interpreted using radix 10.  For
example, \samp{077e010} is legal, and denotes the same number
as \samp{77e10}.
The allowed range of floating point literals is
implementation-dependent.
Some examples of floating point literals:

\begin{verbatim}
3.14    10.    .001    1e100    3.14e-10    0e0
\end{verbatim}

Note that numeric literals do not include a sign; a phrase like
\code{-1} is actually an expression composed of the unary operator
\code{-} and the literal \code{1}.


\subsection{Imaginary literals\label{imaginary}}

Imaginary literals are described by the following lexical definitions:

\begin{productionlist}
  \production{imagnumber}{(\token{floatnumber} | \token{intpart}) ("j" | "J")}
\end{productionlist}

An imaginary literal yields a complex number with a real part of
0.0.  Complex numbers are represented as a pair of floating point
numbers and have the same restrictions on their range.  To create a
complex number with a nonzero real part, add a floating point number
to it, e.g., \code{(3+4j)}.  Some examples of imaginary literals:

\begin{verbatim}
3.14j   10.j    10j     .001j   1e100j  3.14e-10j 
\end{verbatim}


\section{Operators\label{operators}}

The following tokens are operators:
\index{operators}

\begin{verbatim}
+       -       *       **      /       //      %
<<      >>      &       |       ^       ~
<       >       <=      >=      ==      !=      <>
\end{verbatim}

The comparison operators \code{<>} and \code{!=} are alternate
spellings of the same operator.  \code{!=} is the preferred spelling;
\code{<>} is obsolescent.


\section{Delimiters\label{delimiters}}

The following tokens serve as delimiters in the grammar:
\index{delimiters}

\begin{verbatim}
(       )       [       ]       {       }      @
,       :       .       `       =       ;
+=      -=      *=      /=      //=     %=
&=      |=      ^=      >>=     <<=     **=
\end{verbatim}

The period can also occur in floating-point and imaginary literals.  A
sequence of three periods has a special meaning as an ellipsis in slices.
The second half of the list, the augmented assignment operators, serve
lexically as delimiters, but also perform an operation.

The following printing \ASCII{} characters have special meaning as part
of other tokens or are otherwise significant to the lexical analyzer:

\begin{verbatim}
'       "       #       \
\end{verbatim}

The following printing \ASCII{} characters are not used in Python.  Their
occurrence outside string literals and comments is an unconditional
error:
\index{ASCII@\ASCII}

\begin{verbatim}
$       ?
\end{verbatim}
		% Lexical analysis
\chapter{Data model\label{datamodel}}


\section{Objects, values and types\label{objects}}

\dfn{Objects} are Python's abstraction for data.  All data in a Python
program is represented by objects or by relations between objects.
(In a sense, and in conformance to Von Neumann's model of a
``stored program computer,'' code is also represented by objects.)
\index{object}
\index{data}

Every object has an identity, a type and a value.  An object's
\emph{identity} never changes once it has been created; you may think
of it as the object's address in memory.  The `\keyword{is}' operator
compares the identity of two objects; the
\function{id()}\bifuncindex{id} function returns an integer
representing its identity (currently implemented as its address).
An object's \dfn{type} is
also unchangeable.\footnote{Since Python 2.2, a gradual merging of
types and classes has been started that makes this and a few other
assertions made in this manual not 100\% accurate and complete:
for example, it \emph{is} now possible in some cases to change an
object's type, under certain controlled conditions.  Until this manual
undergoes extensive revision, it must now be taken as authoritative
only regarding ``classic classes'', that are still the default, for
compatibility purposes, in Python 2.2 and 2.3.  For more information,
see \url{http://www.python.org/doc/newstyle/}.}
An object's type determines the operations that the object
supports (e.g., ``does it have a length?'') and also defines the
possible values for objects of that type.  The
\function{type()}\bifuncindex{type} function returns an object's type
(which is an object itself).  The \emph{value} of some
objects can change.  Objects whose value can change are said to be
\emph{mutable}; objects whose value is unchangeable once they are
created are called \emph{immutable}.
(The value of an immutable container object that contains a reference
to a mutable object can change when the latter's value is changed;
however the container is still considered immutable, because the
collection of objects it contains cannot be changed.  So, immutability
is not strictly the same as having an unchangeable value, it is more
subtle.)
An object's mutability is determined by its type; for instance,
numbers, strings and tuples are immutable, while dictionaries and
lists are mutable.
\index{identity of an object}
\index{value of an object}
\index{type of an object}
\index{mutable object}
\index{immutable object}

Objects are never explicitly destroyed; however, when they become
unreachable they may be garbage-collected.  An implementation is
allowed to postpone garbage collection or omit it altogether --- it is
a matter of implementation quality how garbage collection is
implemented, as long as no objects are collected that are still
reachable.  (Implementation note: the current implementation uses a
reference-counting scheme with (optional) delayed detection of
cyclically linked garbage, which collects most objects as soon as they
become unreachable, but is not guaranteed to collect garbage
containing circular references.  See the
\citetitle[../lib/module-gc.html]{Python Library Reference} for
information on controlling the collection of cyclic garbage.)
\index{garbage collection}
\index{reference counting}
\index{unreachable object}

Note that the use of the implementation's tracing or debugging
facilities may keep objects alive that would normally be collectable.
Also note that catching an exception with a
`\keyword{try}...\keyword{except}' statement may keep objects alive.

Some objects contain references to ``external'' resources such as open
files or windows.  It is understood that these resources are freed
when the object is garbage-collected, but since garbage collection is
not guaranteed to happen, such objects also provide an explicit way to
release the external resource, usually a \method{close()} method.
Programs are strongly recommended to explicitly close such
objects.  The `\keyword{try}...\keyword{finally}' statement provides
a convenient way to do this.

Some objects contain references to other objects; these are called
\emph{containers}.  Examples of containers are tuples, lists and
dictionaries.  The references are part of a container's value.  In
most cases, when we talk about the value of a container, we imply the
values, not the identities of the contained objects; however, when we
talk about the mutability of a container, only the identities of
the immediately contained objects are implied.  So, if an immutable
container (like a tuple)
contains a reference to a mutable object, its value changes
if that mutable object is changed.
\index{container}

Types affect almost all aspects of object behavior.  Even the importance
of object identity is affected in some sense: for immutable types,
operations that compute new values may actually return a reference to
any existing object with the same type and value, while for mutable
objects this is not allowed.  E.g., after
\samp{a = 1; b = 1},
\code{a} and \code{b} may or may not refer to the same object with the
value one, depending on the implementation, but after
\samp{c = []; d = []}, \code{c} and \code{d}
are guaranteed to refer to two different, unique, newly created empty
lists.
(Note that \samp{c = d = []} assigns the same object to both
\code{c} and \code{d}.)


\section{The standard type hierarchy\label{types}}

Below is a list of the types that are built into Python.  Extension
modules (written in C, Java, or other languages, depending on
the implementation) can define additional types.  Future versions of
Python may add types to the type hierarchy (e.g., rational
numbers, efficiently stored arrays of integers, etc.).
\index{type}
\indexii{data}{type}
\indexii{type}{hierarchy}
\indexii{extension}{module}
\indexii{C}{language}

Some of the type descriptions below contain a paragraph listing
`special attributes.'  These are attributes that provide access to the
implementation and are not intended for general use.  Their definition
may change in the future.
\index{attribute}
\indexii{special}{attribute}
\indexiii{generic}{special}{attribute}

\begin{description}

\item[None]
This type has a single value.  There is a single object with this value.
This object is accessed through the built-in name \code{None}.
It is used to signify the absence of a value in many situations, e.g.,
it is returned from functions that don't explicitly return anything.
Its truth value is false.
\obindex{None}

\item[NotImplemented]
This type has a single value.  There is a single object with this value.
This object is accessed through the built-in name \code{NotImplemented}.
Numeric methods and rich comparison methods may return this value if
they do not implement the operation for the operands provided.  (The
interpreter will then try the reflected operation, or some other
fallback, depending on the operator.)  Its truth value is true.
\obindex{NotImplemented}

\item[Ellipsis]
This type has a single value.  There is a single object with this value.
This object is accessed through the built-in name \code{Ellipsis}.
It is used to indicate the presence of the \samp{...} syntax in a
slice.  Its truth value is true.
\obindex{Ellipsis}

\item[Numbers]
These are created by numeric literals and returned as results by
arithmetic operators and arithmetic built-in functions.  Numeric
objects are immutable; once created their value never changes.  Python
numbers are of course strongly related to mathematical numbers, but
subject to the limitations of numerical representation in computers.
\obindex{numeric}

Python distinguishes between integers, floating point numbers, and
complex numbers:

\begin{description}
\item[Integers]
These represent elements from the mathematical set of integers
(positive and negative).
\obindex{integer}

There are three types of integers:

\begin{description}

\item[Plain integers]
These represent numbers in the range -2147483648 through 2147483647.
(The range may be larger on machines with a larger natural word
size, but not smaller.)
When the result of an operation would fall outside this range, the
result is normally returned as a long integer (in some cases, the
exception \exception{OverflowError} is raised instead).
For the purpose of shift and mask operations, integers are assumed to
have a binary, 2's complement notation using 32 or more bits, and
hiding no bits from the user (i.e., all 4294967296 different bit
patterns correspond to different values).
\obindex{plain integer}
\withsubitem{(built-in exception)}{\ttindex{OverflowError}}

\item[Long integers]
These represent numbers in an unlimited range, subject to available
(virtual) memory only.  For the purpose of shift and mask operations,
a binary representation is assumed, and negative numbers are
represented in a variant of 2's complement which gives the illusion of
an infinite string of sign bits extending to the left.
\obindex{long integer}

\item[Booleans]
These represent the truth values False and True.  The two objects
representing the values False and True are the only Boolean objects.
The Boolean type is a subtype of plain integers, and Boolean values
behave like the values 0 and 1, respectively, in almost all contexts,
the exception being that when converted to a string, the strings
\code{"False"} or \code{"True"} are returned, respectively.
\obindex{Boolean}
\ttindex{False}
\ttindex{True}

\end{description} % Integers

The rules for integer representation are intended to give the most
meaningful interpretation of shift and mask operations involving
negative integers and the least surprises when switching between the
plain and long integer domains.  Any operation except left shift,
if it yields a result in the plain integer domain without causing
overflow, will yield the same result in the long integer domain or
when using mixed operands.
\indexii{integer}{representation}

\item[Floating point numbers]
These represent machine-level double precision floating point numbers.  
You are at the mercy of the underlying machine architecture (and
C or Java implementation) for the accepted range and handling of overflow.
Python does not support single-precision floating point numbers; the
savings in processor and memory usage that are usually the reason for using
these is dwarfed by the overhead of using objects in Python, so there
is no reason to complicate the language with two kinds of floating
point numbers.
\obindex{floating point}
\indexii{floating point}{number}
\indexii{C}{language}
\indexii{Java}{language}

\item[Complex numbers]
These represent complex numbers as a pair of machine-level double
precision floating point numbers.  The same caveats apply as for
floating point numbers.  The real and imaginary parts of a complex
number \code{z} can be retrieved through the read-only attributes
\code{z.real} and \code{z.imag}.
\obindex{complex}
\indexii{complex}{number}

\end{description} % Numbers


\item[Sequences]
These represent finite ordered sets indexed by non-negative numbers.
The built-in function \function{len()}\bifuncindex{len} returns the
number of items of a sequence.
When the length of a sequence is \var{n}, the
index set contains the numbers 0, 1, \ldots, \var{n}-1.  Item
\var{i} of sequence \var{a} is selected by \code{\var{a}[\var{i}]}.
\obindex{sequence}
\index{index operation}
\index{item selection}
\index{subscription}

Sequences also support slicing: \code{\var{a}[\var{i}:\var{j}]}
selects all items with index \var{k} such that \var{i} \code{<=}
\var{k} \code{<} \var{j}.  When used as an expression, a slice is a
sequence of the same type.  This implies that the index set is
renumbered so that it starts at 0.
\index{slicing}

Some sequences also support ``extended slicing'' with a third ``step''
parameter: \code{\var{a}[\var{i}:\var{j}:\var{k}]} selects all items
of \var{a} with index \var{x} where \code{\var{x} = \var{i} +
\var{n}*\var{k}}, \var{n} \code{>=} \code{0} and \var{i} \code{<=}
\var{x} \code{<} \var{j}.
\index{extended slicing}

Sequences are distinguished according to their mutability:

\begin{description}

\item[Immutable sequences]
An object of an immutable sequence type cannot change once it is
created.  (If the object contains references to other objects,
these other objects may be mutable and may be changed; however,
the collection of objects directly referenced by an immutable object
cannot change.)
\obindex{immutable sequence}
\obindex{immutable}

The following types are immutable sequences:

\begin{description}

\item[Strings]
The items of a string are characters.  There is no separate
character type; a character is represented by a string of one item.
Characters represent (at least) 8-bit bytes.  The built-in
functions \function{chr()}\bifuncindex{chr} and
\function{ord()}\bifuncindex{ord} convert between characters and
nonnegative integers representing the byte values.  Bytes with the
values 0-127 usually represent the corresponding \ASCII{} values, but
the interpretation of values is up to the program.  The string
data type is also used to represent arrays of bytes, e.g., to hold data
read from a file.
\obindex{string}
\index{character}
\index{byte}
\index{ASCII@\ASCII}

(On systems whose native character set is not \ASCII, strings may use
EBCDIC in their internal representation, provided the functions
\function{chr()} and \function{ord()} implement a mapping between \ASCII{} and
EBCDIC, and string comparison preserves the \ASCII{} order.
Or perhaps someone can propose a better rule?)
\index{ASCII@\ASCII}
\index{EBCDIC}
\index{character set}
\indexii{string}{comparison}
\bifuncindex{chr}
\bifuncindex{ord}

\item[Unicode]
The items of a Unicode object are Unicode code units.  A Unicode code
unit is represented by a Unicode object of one item and can hold
either a 16-bit or 32-bit value representing a Unicode ordinal (the
maximum value for the ordinal is given in \code{sys.maxunicode}, and
depends on how Python is configured at compile time).  Surrogate pairs
may be present in the Unicode object, and will be reported as two
separate items.  The built-in functions
\function{unichr()}\bifuncindex{unichr} and
\function{ord()}\bifuncindex{ord} convert between code units and
nonnegative integers representing the Unicode ordinals as defined in
the Unicode Standard 3.0. Conversion from and to other encodings are
possible through the Unicode method \method{encode()} and the built-in
function \function{unicode()}.\bifuncindex{unicode}
\obindex{unicode}
\index{character}
\index{integer}
\index{Unicode}

\item[Tuples]
The items of a tuple are arbitrary Python objects.
Tuples of two or more items are formed by comma-separated lists
of expressions.  A tuple of one item (a `singleton') can be formed
by affixing a comma to an expression (an expression by itself does
not create a tuple, since parentheses must be usable for grouping of
expressions).  An empty tuple can be formed by an empty pair of
parentheses.
\obindex{tuple}
\indexii{singleton}{tuple}
\indexii{empty}{tuple}

\end{description} % Immutable sequences

\item[Mutable sequences]
Mutable sequences can be changed after they are created.  The
subscription and slicing notations can be used as the target of
assignment and \keyword{del} (delete) statements.
\obindex{mutable sequence}
\obindex{mutable}
\indexii{assignment}{statement}
\index{delete}
\stindex{del}
\index{subscription}
\index{slicing}

There is currently a single intrinsic mutable sequence type:

\begin{description}

\item[Lists]
The items of a list are arbitrary Python objects.  Lists are formed
by placing a comma-separated list of expressions in square brackets.
(Note that there are no special cases needed to form lists of length 0
or 1.)
\obindex{list}

\end{description} % Mutable sequences

The extension module \module{array}\refstmodindex{array} provides an
additional example of a mutable sequence type.


\end{description} % Sequences


\item[Set types]
These represent unordered, finite sets of unique, immutable objects.
As such, they cannot be indexed by any subscript. However, they can be
iterated over, and the built-in function \function{len()} returns the
number of items in a set. Common uses for sets are
fast membership testing, removing duplicates from a sequence, and
computing mathematical operations such as intersection, union, difference,
and symmetric difference.
\bifuncindex{len}
\obindex{set type}

For set elements, the same immutability rules apply as for dictionary
keys. Note that numeric types obey the normal rules for numeric
comparison: if two numbers compare equal (e.g., \code{1} and
\code{1.0}), only one of them can be contained in a set.

There are currently two intrinsic set types:

\begin{description}

\item[Sets]
These\obindex{set} represent a mutable set. They are created by the
built-in \function{set()} constructor and can be modified afterwards
by several methods, such as \method{add()}.

\item[Frozen sets]
These\obindex{frozenset} represent an immutable set. They are created by
the built-in \function{frozenset()} constructor. As a frozenset is
immutable and hashable, it can be used again as an element of another set,
or as a dictionary key.

\end{description} % Set types


\item[Mappings]
These represent finite sets of objects indexed by arbitrary index sets.
The subscript notation \code{a[k]} selects the item indexed
by \code{k} from the mapping \code{a}; this can be used in
expressions and as the target of assignments or \keyword{del} statements.
The built-in function \function{len()} returns the number of items
in a mapping.
\bifuncindex{len}
\index{subscription}
\obindex{mapping}

There is currently a single intrinsic mapping type:

\begin{description}

\item[Dictionaries]
These\obindex{dictionary} represent finite sets of objects indexed by
nearly arbitrary values.  The only types of values not acceptable as
keys are values containing lists or dictionaries or other mutable
types that are compared by value rather than by object identity, the
reason being that the efficient implementation of dictionaries
requires a key's hash value to remain constant.
Numeric types used for keys obey the normal rules for numeric
comparison: if two numbers compare equal (e.g., \code{1} and
\code{1.0}) then they can be used interchangeably to index the same
dictionary entry.

Dictionaries are mutable; they can be created by the
\code{\{...\}} notation (see section~\ref{dict}, ``Dictionary
Displays'').

The extension modules \module{dbm}\refstmodindex{dbm},
\module{gdbm}\refstmodindex{gdbm}, and
\module{bsddb}\refstmodindex{bsddb} provide additional examples of
mapping types.

\end{description} % Mapping types

\item[Callable types]
These\obindex{callable} are the types to which the function call
operation (see section~\ref{calls}, ``Calls'') can be applied:
\indexii{function}{call}
\index{invocation}
\indexii{function}{argument}

\begin{description}

\item[User-defined functions]
A user-defined function object is created by a function definition
(see section~\ref{function}, ``Function definitions'').  It should be
called with an argument
list containing the same number of items as the function's formal
parameter list.
\indexii{user-defined}{function}
\obindex{function}
\obindex{user-defined function}

Special attributes: 

\begin{tableiii}{lll}{member}{Attribute}{Meaning}{}
  \lineiii{func_doc}{The function's documentation string, or
    \code{None} if unavailable}{Writable}

  \lineiii{__doc__}{Another way of spelling
    \member{func_doc}}{Writable}

  \lineiii{func_name}{The function's name}{Writable}

  \lineiii{__name__}{Another way of spelling
    \member{func_name}}{Writable}

  \lineiii{__module__}{The name of the module the function was defined
    in, or \code{None} if unavailable.}{Writable}

  \lineiii{func_defaults}{A tuple containing default argument values
    for those arguments that have defaults, or \code{None} if no
    arguments have a default value}{Writable}

  \lineiii{func_code}{The code object representing the compiled
    function body.}{Writable}

  \lineiii{func_globals}{A reference to the dictionary that holds the
    function's global variables --- the global namespace of the module
    in which the function was defined.}{Read-only}

  \lineiii{func_dict}{The namespace supporting arbitrary function
    attributes.}{Writable}

  \lineiii{func_closure}{\code{None} or a tuple of cells that contain
    bindings for the function's free variables.}{Read-only}
\end{tableiii}

Most of the attributes labelled ``Writable'' check the type of the
assigned value.

\versionchanged[\code{func_name} is now writable]{2.4}

Function objects also support getting and setting arbitrary
attributes, which can be used, for example, to attach metadata to
functions.  Regular attribute dot-notation is used to get and set such
attributes. \emph{Note that the current implementation only supports
function attributes on user-defined functions.  Function attributes on
built-in functions may be supported in the future.}

Additional information about a function's definition can be retrieved
from its code object; see the description of internal types below.

\withsubitem{(function attribute)}{
  \ttindex{func_doc}
  \ttindex{__doc__}
  \ttindex{__name__}
  \ttindex{__module__}
  \ttindex{__dict__}
  \ttindex{func_defaults}
  \ttindex{func_closure}
  \ttindex{func_code}
  \ttindex{func_globals}
  \ttindex{func_dict}}
\indexii{global}{namespace}

\item[User-defined methods]
A user-defined method object combines a class, a class instance (or
\code{None}) and any callable object (normally a user-defined
function).
\obindex{method}
\obindex{user-defined method}
\indexii{user-defined}{method}

Special read-only attributes: \member{im_self} is the class instance
object, \member{im_func} is the function object;
\member{im_class} is the class of \member{im_self} for bound methods
or the class that asked for the method for unbound methods;
\member{__doc__} is the method's documentation (same as
\code{im_func.__doc__}); \member{__name__} is the method name (same as
\code{im_func.__name__}); \member{__module__} is the name of the
module the method was defined in, or \code{None} if unavailable.
\versionchanged[\member{im_self} used to refer to the class that
                defined the method]{2.2}
\withsubitem{(method attribute)}{
  \ttindex{__doc__}
  \ttindex{__name__}
  \ttindex{__module__}
  \ttindex{im_func}
  \ttindex{im_self}}

Methods also support accessing (but not setting) the arbitrary
function attributes on the underlying function object.

User-defined method objects may be created when getting an attribute
of a class (perhaps via an instance of that class), if that attribute
is a user-defined function object, an unbound user-defined method object,
or a class method object.
When the attribute is a user-defined method object, a new
method object is only created if the class from which it is being
retrieved is the same as, or a derived class of, the class stored
in the original method object; otherwise, the original method object
is used as it is.

When a user-defined method object is created by retrieving
a user-defined function object from a class, its \member{im_self}
attribute is \code{None} and the method object is said to be unbound.
When one is created by retrieving a user-defined function object
from a class via one of its instances, its \member{im_self} attribute
is the instance, and the method object is said to be bound.
In either case, the new method's \member{im_class} attribute
is the class from which the retrieval takes place, and
its \member{im_func} attribute is the original function object.
\withsubitem{(method attribute)}{
  \ttindex{im_class}\ttindex{im_func}\ttindex{im_self}}

When a user-defined method object is created by retrieving another
method object from a class or instance, the behaviour is the same
as for a function object, except that the \member{im_func} attribute
of the new instance is not the original method object but its
\member{im_func} attribute.
\withsubitem{(method attribute)}{
  \ttindex{im_func}}

When a user-defined method object is created by retrieving a
class method object from a class or instance, its \member{im_self}
attribute is the class itself (the same as the \member{im_class}
attribute), and its \member{im_func} attribute is the function
object underlying the class method.
\withsubitem{(method attribute)}{
  \ttindex{im_class}\ttindex{im_func}\ttindex{im_self}}

When an unbound user-defined method object is called, the underlying
function (\member{im_func}) is called, with the restriction that the
first argument must be an instance of the proper class
(\member{im_class}) or of a derived class thereof.

When a bound user-defined method object is called, the underlying
function (\member{im_func}) is called, inserting the class instance
(\member{im_self}) in front of the argument list.  For instance, when
\class{C} is a class which contains a definition for a function
\method{f()}, and \code{x} is an instance of \class{C}, calling
\code{x.f(1)} is equivalent to calling \code{C.f(x, 1)}.

When a user-defined method object is derived from a class method object,
the ``class instance'' stored in \member{im_self} will actually be the
class itself, so that calling either \code{x.f(1)} or \code{C.f(1)} is
equivalent to calling \code{f(C,1)} where \code{f} is the underlying
function.

Note that the transformation from function object to (unbound or
bound) method object happens each time the attribute is retrieved from
the class or instance.  In some cases, a fruitful optimization is to
assign the attribute to a local variable and call that local variable.
Also notice that this transformation only happens for user-defined
functions; other callable objects (and all non-callable objects) are
retrieved without transformation.  It is also important to note that
user-defined functions which are attributes of a class instance are
not converted to bound methods; this \emph{only} happens when the
function is an attribute of the class.

\item[Generator functions\index{generator!function}\index{generator!iterator}]
A function or method which uses the \keyword{yield} statement (see
section~\ref{yield}, ``The \keyword{yield} statement'') is called a
\dfn{generator function}.  Such a function, when called, always
returns an iterator object which can be used to execute the body of
the function:  calling the iterator's \method{next()} method will
cause the function to execute until it provides a value using the
\keyword{yield} statement.  When the function executes a
\keyword{return} statement or falls off the end, a
\exception{StopIteration} exception is raised and the iterator will
have reached the end of the set of values to be returned.

\item[Built-in functions]
A built-in function object is a wrapper around a C function.  Examples
of built-in functions are \function{len()} and \function{math.sin()}
(\module{math} is a standard built-in module).
The number and type of the arguments are
determined by the C function.
Special read-only attributes: \member{__doc__} is the function's
documentation string, or \code{None} if unavailable; \member{__name__}
is the function's name; \member{__self__} is set to \code{None} (but see
the next item); \member{__module__} is the name of the module the
function was defined in or \code{None} if unavailable.
\obindex{built-in function}
\obindex{function}
\indexii{C}{language}

\item[Built-in methods]
This is really a different disguise of a built-in function, this time
containing an object passed to the C function as an implicit extra
argument.  An example of a built-in method is
\code{\var{alist}.append()}, assuming
\var{alist} is a list object.
In this case, the special read-only attribute \member{__self__} is set
to the object denoted by \var{list}.
\obindex{built-in method}
\obindex{method}
\indexii{built-in}{method}

\item[Class Types]
Class types, or ``new-style classes,'' are callable.  These objects
normally act as factories for new instances of themselves, but
variations are possible for class types that override
\method{__new__()}.  The arguments of the call are passed to
\method{__new__()} and, in the typical case, to \method{__init__()} to
initialize the new instance.

\item[Classic Classes]
Class objects are described below.  When a class object is called,
a new class instance (also described below) is created and
returned.  This implies a call to the class's \method{__init__()} method
if it has one.  Any arguments are passed on to the \method{__init__()}
method.  If there is no \method{__init__()} method, the class must be called
without arguments.
\withsubitem{(object method)}{\ttindex{__init__()}}
\obindex{class}
\obindex{class instance}
\obindex{instance}
\indexii{class object}{call}

\item[Class instances]
Class instances are described below.  Class instances are callable
only when the class has a \method{__call__()} method; \code{x(arguments)}
is a shorthand for \code{x.__call__(arguments)}.

\end{description}

\item[Modules]
Modules are imported by the \keyword{import} statement (see
section~\ref{import}, ``The \keyword{import} statement'').%
\stindex{import}\obindex{module}
A module object has a namespace implemented by a dictionary object
(this is the dictionary referenced by the func_globals attribute of
functions defined in the module).  Attribute references are translated
to lookups in this dictionary, e.g., \code{m.x} is equivalent to
\code{m.__dict__["x"]}.
A module object does not contain the code object used to
initialize the module (since it isn't needed once the initialization
is done).

Attribute assignment updates the module's namespace dictionary,
e.g., \samp{m.x = 1} is equivalent to \samp{m.__dict__["x"] = 1}.

Special read-only attribute: \member{__dict__} is the module's
namespace as a dictionary object.
\withsubitem{(module attribute)}{\ttindex{__dict__}}

Predefined (writable) attributes: \member{__name__}
is the module's name; \member{__doc__} is the
module's documentation string, or
\code{None} if unavailable; \member{__file__} is the pathname of the
file from which the module was loaded, if it was loaded from a file.
The \member{__file__} attribute is not present for C{} modules that are
statically linked into the interpreter; for extension modules loaded
dynamically from a shared library, it is the pathname of the shared
library file.
\withsubitem{(module attribute)}{
  \ttindex{__name__}
  \ttindex{__doc__}
  \ttindex{__file__}}
\indexii{module}{namespace}

\item[Classes]
Class objects are created by class definitions (see
section~\ref{class}, ``Class definitions'').
A class has a namespace implemented by a dictionary object.
Class attribute references are translated to
lookups in this dictionary,
e.g., \samp{C.x} is translated to \samp{C.__dict__["x"]}.
When the attribute name is not found
there, the attribute search continues in the base classes.  The search
is depth-first, left-to-right in the order of occurrence in the
base class list.

When a class attribute reference (for class \class{C}, say)
would yield a user-defined function object or
an unbound user-defined method object whose associated class is either
\class{C} or one of its base classes, it is transformed into an unbound
user-defined method object whose \member{im_class} attribute is~\class{C}.
When it would yield a class method object, it is transformed into
a bound user-defined method object whose \member{im_class} and
\member{im_self} attributes are both~\class{C}.  When it would yield
a static method object, it is transformed into the object wrapped
by the static method object. See section~\ref{descriptors} for another
way in which attributes retrieved from a class may differ from those
actually contained in its \member{__dict__}.
\obindex{class}
\obindex{class instance}
\obindex{instance}
\indexii{class object}{call}
\index{container}
\obindex{dictionary}
\indexii{class}{attribute}

Class attribute assignments update the class's dictionary, never the
dictionary of a base class.
\indexiii{class}{attribute}{assignment}

A class object can be called (see above) to yield a class instance (see
below).
\indexii{class object}{call}

Special attributes: \member{__name__} is the class name;
\member{__module__} is the module name in which the class was defined;
\member{__dict__} is the dictionary containing the class's namespace;
\member{__bases__} is a tuple (possibly empty or a singleton)
containing the base classes, in the order of their occurrence in the
base class list; \member{__doc__} is the class's documentation string,
or None if undefined.
\withsubitem{(class attribute)}{
  \ttindex{__name__}
  \ttindex{__module__}
  \ttindex{__dict__}
  \ttindex{__bases__}
  \ttindex{__doc__}}

\item[Class instances]
A class instance is created by calling a class object (see above).
A class instance has a namespace implemented as a dictionary which
is the first place in which
attribute references are searched.  When an attribute is not found
there, and the instance's class has an attribute by that name,
the search continues with the class attributes.  If a class attribute
is found that is a user-defined function object or an unbound
user-defined method object whose associated class is the class
(call it~\class{C}) of the instance for which the attribute reference
was initiated or one of its bases,
it is transformed into a bound user-defined method object whose
\member{im_class} attribute is~\class{C} and whose \member{im_self} attribute
is the instance. Static method and class method objects are also
transformed, as if they had been retrieved from class~\class{C};
see above under ``Classes''. See section~\ref{descriptors} for
another way in which attributes of a class retrieved via its
instances may differ from the objects actually stored in the
class's \member{__dict__}.
If no class attribute is found, and the object's class has a
\method{__getattr__()} method, that is called to satisfy the lookup.
\obindex{class instance}
\obindex{instance}
\indexii{class}{instance}
\indexii{class instance}{attribute}

Attribute assignments and deletions update the instance's dictionary,
never a class's dictionary.  If the class has a \method{__setattr__()} or
\method{__delattr__()} method, this is called instead of updating the
instance dictionary directly.
\indexiii{class instance}{attribute}{assignment}

Class instances can pretend to be numbers, sequences, or mappings if
they have methods with certain special names.  See
section~\ref{specialnames}, ``Special method names.''
\obindex{numeric}
\obindex{sequence}
\obindex{mapping}

Special attributes: \member{__dict__} is the attribute
dictionary; \member{__class__} is the instance's class.
\withsubitem{(instance attribute)}{
  \ttindex{__dict__}
  \ttindex{__class__}}

\item[Files]
A file\obindex{file} object represents an open file.  File objects are
created by the \function{open()}\bifuncindex{open} built-in function,
and also by
\withsubitem{(in module os)}{\ttindex{popen()}}\function{os.popen()},
\function{os.fdopen()}, and the
\method{makefile()}\withsubitem{(socket method)}{\ttindex{makefile()}}
method of socket objects (and perhaps by other functions or methods
provided by extension modules).  The objects
\ttindex{sys.stdin}\code{sys.stdin},
\ttindex{sys.stdout}\code{sys.stdout} and
\ttindex{sys.stderr}\code{sys.stderr} are initialized to file objects
corresponding to the interpreter's standard\index{stdio} input, output
and error streams.  See the \citetitle[../lib/lib.html]{Python Library
Reference} for complete documentation of file objects.
\withsubitem{(in module sys)}{
  \ttindex{stdin}
  \ttindex{stdout}
  \ttindex{stderr}}


\item[Internal types]
A few types used internally by the interpreter are exposed to the user.
Their definitions may change with future versions of the interpreter,
but they are mentioned here for completeness.
\index{internal type}
\index{types, internal}

\begin{description}

\item[Code objects]
Code objects represent \emph{byte-compiled} executable Python code, or 
\emph{bytecode}.
The difference between a code
object and a function object is that the function object contains an
explicit reference to the function's globals (the module in which it
was defined), while a code object contains no context; 
also the default argument values are stored in the function object,
not in the code object (because they represent values calculated at
run-time).  Unlike function objects, code objects are immutable and
contain no references (directly or indirectly) to mutable objects.
\index{bytecode}
\obindex{code}

Special read-only attributes: \member{co_name} gives the function
name; \member{co_argcount} is the number of positional arguments
(including arguments with default values); \member{co_nlocals} is the
number of local variables used by the function (including arguments);
\member{co_varnames} is a tuple containing the names of the local
variables (starting with the argument names); \member{co_cellvars} is
a tuple containing the names of local variables that are referenced by
nested functions; \member{co_freevars} is a tuple containing the names
of free variables; \member{co_code} is a string representing the
sequence of bytecode instructions;
\member{co_consts} is a tuple containing the literals used by the
bytecode; \member{co_names} is a tuple containing the names used by
the bytecode; \member{co_filename} is the filename from which the code
was compiled; \member{co_firstlineno} is the first line number of the
function; \member{co_lnotab} is a string encoding the mapping from
byte code offsets to line numbers (for details see the source code of
the interpreter); \member{co_stacksize} is the required stack size
(including local variables); \member{co_flags} is an integer encoding
a number of flags for the interpreter.

\withsubitem{(code object attribute)}{
  \ttindex{co_argcount}
  \ttindex{co_code}
  \ttindex{co_consts}
  \ttindex{co_filename}
  \ttindex{co_firstlineno}
  \ttindex{co_flags}
  \ttindex{co_lnotab}
  \ttindex{co_name}
  \ttindex{co_names}
  \ttindex{co_nlocals}
  \ttindex{co_stacksize}
  \ttindex{co_varnames}
  \ttindex{co_cellvars}
  \ttindex{co_freevars}}

The following flag bits are defined for \member{co_flags}: bit
\code{0x04} is set if the function uses the \samp{*arguments} syntax
to accept an arbitrary number of positional arguments; bit
\code{0x08} is set if the function uses the \samp{**keywords} syntax
to accept arbitrary keyword arguments; bit \code{0x20} is set if the
function is a generator.
\obindex{generator}

Future feature declarations (\samp{from __future__ import division})
also use bits in \member{co_flags} to indicate whether a code object
was compiled with a particular feature enabled: bit \code{0x2000} is
set if the function was compiled with future division enabled; bits
\code{0x10} and \code{0x1000} were used in earlier versions of Python.

Other bits in \member{co_flags} are reserved for internal use.

If\index{documentation string} a code object represents a function,
the first item in
\member{co_consts} is the documentation string of the function, or
\code{None} if undefined.

\item[Frame objects]
Frame objects represent execution frames.  They may occur in traceback
objects (see below).
\obindex{frame}

Special read-only attributes: \member{f_back} is to the previous
stack frame (towards the caller), or \code{None} if this is the bottom
stack frame; \member{f_code} is the code object being executed in this
frame; \member{f_locals} is the dictionary used to look up local
variables; \member{f_globals} is used for global variables;
\member{f_builtins} is used for built-in (intrinsic) names;
\member{f_restricted} is a flag indicating whether the function is
executing in restricted execution mode; \member{f_lasti} gives the
precise instruction (this is an index into the bytecode string of
the code object).
\withsubitem{(frame attribute)}{
  \ttindex{f_back}
  \ttindex{f_code}
  \ttindex{f_globals}
  \ttindex{f_locals}
  \ttindex{f_lasti}
  \ttindex{f_builtins}
  \ttindex{f_restricted}}

Special writable attributes: \member{f_trace}, if not \code{None}, is
a function called at the start of each source code line (this is used
by the debugger); \member{f_exc_type}, \member{f_exc_value},
\member{f_exc_traceback} represent the last exception raised in the
parent frame provided another exception was ever raised in the current
frame (in all other cases they are None); \member{f_lineno} is the
current line number of the frame --- writing to this from within a
trace function jumps to the given line (only for the bottom-most
frame).  A debugger can implement a Jump command (aka Set Next
Statement) by writing to f_lineno.
\withsubitem{(frame attribute)}{
  \ttindex{f_trace}
  \ttindex{f_exc_type}
  \ttindex{f_exc_value}
  \ttindex{f_exc_traceback}
  \ttindex{f_lineno}}

\item[Traceback objects] \label{traceback}
Traceback objects represent a stack trace of an exception.  A
traceback object is created when an exception occurs.  When the search
for an exception handler unwinds the execution stack, at each unwound
level a traceback object is inserted in front of the current
traceback.  When an exception handler is entered, the stack trace is
made available to the program.
(See section~\ref{try}, ``The \code{try} statement.'')
It is accessible as \code{sys.exc_traceback}, and also as the third
item of the tuple returned by \code{sys.exc_info()}.  The latter is
the preferred interface, since it works correctly when the program is
using multiple threads.
When the program contains no suitable handler, the stack trace is written
(nicely formatted) to the standard error stream; if the interpreter is
interactive, it is also made available to the user as
\code{sys.last_traceback}.
\obindex{traceback}
\indexii{stack}{trace}
\indexii{exception}{handler}
\indexii{execution}{stack}
\withsubitem{(in module sys)}{
  \ttindex{exc_info}
  \ttindex{exc_traceback}
  \ttindex{last_traceback}}
\ttindex{sys.exc_info}
\ttindex{sys.exc_traceback}
\ttindex{sys.last_traceback}

Special read-only attributes: \member{tb_next} is the next level in the
stack trace (towards the frame where the exception occurred), or
\code{None} if there is no next level; \member{tb_frame} points to the
execution frame of the current level; \member{tb_lineno} gives the line
number where the exception occurred; \member{tb_lasti} indicates the
precise instruction.  The line number and last instruction in the
traceback may differ from the line number of its frame object if the
exception occurred in a \keyword{try} statement with no matching
except clause or with a finally clause.
\withsubitem{(traceback attribute)}{
  \ttindex{tb_next}
  \ttindex{tb_frame}
  \ttindex{tb_lineno}
  \ttindex{tb_lasti}}
\stindex{try}

\item[Slice objects]
Slice objects are used to represent slices when \emph{extended slice
syntax} is used.  This is a slice using two colons, or multiple slices
or ellipses separated by commas, e.g., \code{a[i:j:step]}, \code{a[i:j,
k:l]}, or \code{a[..., i:j]}.  They are also created by the built-in
\function{slice()}\bifuncindex{slice} function.

Special read-only attributes: \member{start} is the lower bound;
\member{stop} is the upper bound; \member{step} is the step value; each is
\code{None} if omitted. These attributes can have any type.
\withsubitem{(slice object attribute)}{
  \ttindex{start}
  \ttindex{stop}
  \ttindex{step}}

Slice objects support one method:

\begin{methoddesc}[slice]{indices}{self, length}
This method takes a single integer argument \var{length} and computes
information about the extended slice that the slice object would
describe if applied to a sequence of \var{length} items.  It returns a
tuple of three integers; respectively these are the \var{start} and
\var{stop} indices and the \var{step} or stride length of the slice.
Missing or out-of-bounds indices are handled in a manner consistent
with regular slices.
\versionadded{2.3}
\end{methoddesc}

\item[Static method objects]
Static method objects provide a way of defeating the transformation
of function objects to method objects described above. A static method
object is a wrapper around any other object, usually a user-defined
method object. When a static method object is retrieved from a class
or a class instance, the object actually returned is the wrapped object,
which is not subject to any further transformation. Static method
objects are not themselves callable, although the objects they
wrap usually are. Static method objects are created by the built-in
\function{staticmethod()} constructor.

\item[Class method objects]
A class method object, like a static method object, is a wrapper
around another object that alters the way in which that object
is retrieved from classes and class instances. The behaviour of
class method objects upon such retrieval is described above,
under ``User-defined methods''. Class method objects are created
by the built-in \function{classmethod()} constructor.

\end{description} % Internal types

\end{description} % Types

%=========================================================================
\section{New-style and classic classes}

Classes and instances come in two flavors: old-style or classic, and new-style.  

Up to Python 2.1, old-style classes were the only flavour available to the
user.  The concept of (old-style) class is unrelated to the concept of type: if
\var{x} is an instance of an old-style class, then \code{x.__class__}
designates the class of \var{x}, but \code{type(x)} is always \code{<type
'instance'>}.  This reflects the fact that all old-style instances,
independently of their class, are implemented with a single built-in type,
called \code{instance}.

New-style classes were introduced in Python 2.2 to unify classes and types.  A
new-style class neither more nor less than a user-defined type.  If \var{x} is
an instance of a new-style class, then \code{type(x)} is the same as
\code{x.__class__}.

The major motivation for introducing new-style classes is to provide a unified
object model with a full meta-model.  It also has a number of immediate
benefits, like the ability to subclass most built-in types, or the introduction
of "descriptors", which enable computed properties.

For compatibility reasons, classes are still old-style by default.  New-style
classes are created by specifying another new-style class (i.e.\ a type) as a
parent class, or the "top-level type" \class{object} if no other parent is
needed.  The behaviour of new-style classes differs from that of old-style
classes in a number of important details in addition to what \function{type}
returns.  Some of these changes are fundamental to the new object model, like
the way special methods are invoked.  Others are "fixes" that could not be
implemented before for compatibility concerns, like the method resolution order
in case of multiple inheritance.

This manual is not up-to-date with respect to new-style classes.  For now,
please see \url{http://www.python.org/doc/newstyle/} for more information.

The plan is to eventually drop old-style classes, leaving only the semantics of
new-style classes.  This change will probably only be feasible in Python 3.0.
\index{class}{new-style}
\index{class}{classic}
\index{class}{old-style}

%=========================================================================
\section{Special method names\label{specialnames}}

A class can implement certain operations that are invoked by special
syntax (such as arithmetic operations or subscripting and slicing) by
defining methods with special names.\indexii{operator}{overloading}
This is Python's approach to \dfn{operator overloading}, allowing
classes to define their own behavior with respect to language
operators.  For instance, if a class defines
a method named \method{__getitem__()}, and \code{x} is an instance of
this class, then \code{x[i]} is equivalent\footnote{This, and other
statements, are only roughly true for instances of new-style
classes.} to
\code{x.__getitem__(i)}.  Except where mentioned, attempts to execute
an operation raise an exception when no appropriate method is defined.
\withsubitem{(mapping object method)}{\ttindex{__getitem__()}}

When implementing a class that emulates any built-in type, it is
important that the emulation only be implemented to the degree that it
makes sense for the object being modelled.  For example, some
sequences may work well with retrieval of individual elements, but
extracting a slice may not make sense.  (One example of this is the
\class{NodeList} interface in the W3C's Document Object Model.)


\subsection{Basic customization\label{customization}}

\begin{methoddesc}[object]{__new__}{cls\optional{, \moreargs}}
Called to create a new instance of class \var{cls}.  \method{__new__()}
is a static method (special-cased so you need not declare it as such)
that takes the class of which an instance was requested as its first
argument.  The remaining arguments are those passed to the object
constructor expression (the call to the class).  The return value of
\method{__new__()} should be the new object instance (usually an
instance of \var{cls}).

Typical implementations create a new instance of the class by invoking
the superclass's \method{__new__()} method using
\samp{super(\var{currentclass}, \var{cls}).__new__(\var{cls}[, ...])}
with appropriate arguments and then modifying the newly-created instance
as necessary before returning it.

If \method{__new__()} returns an instance of \var{cls}, then the new
instance's \method{__init__()} method will be invoked like
\samp{__init__(\var{self}[, ...])}, where \var{self} is the new instance
and the remaining arguments are the same as were passed to
\method{__new__()}.

If \method{__new__()} does not return an instance of \var{cls}, then the
new instance's \method{__init__()} method will not be invoked.

\method{__new__()} is intended mainly to allow subclasses of
immutable types (like int, str, or tuple) to customize instance
creation.
\end{methoddesc}

\begin{methoddesc}[object]{__init__}{self\optional{, \moreargs}}
Called\indexii{class}{constructor} when the instance is created.  The
arguments are those passed to the class constructor expression.  If a
base class has an \method{__init__()} method, the derived class's
\method{__init__()} method, if any, must explicitly call it to ensure proper
initialization of the base class part of the instance; for example:
\samp{BaseClass.__init__(\var{self}, [\var{args}...])}.  As a special
constraint on constructors, no value may be returned; doing so will
cause a \exception{TypeError} to be raised at runtime.
\end{methoddesc}


\begin{methoddesc}[object]{__del__}{self}
Called when the instance is about to be destroyed.  This is also
called a destructor\index{destructor}.  If a base class
has a \method{__del__()} method, the derived class's \method{__del__()}
method, if any,
must explicitly call it to ensure proper deletion of the base class
part of the instance.  Note that it is possible (though not recommended!)
for the \method{__del__()}
method to postpone destruction of the instance by creating a new
reference to it.  It may then be called at a later time when this new
reference is deleted.  It is not guaranteed that
\method{__del__()} methods are called for objects that still exist when
the interpreter exits.
\stindex{del}

\begin{notice}
\samp{del x} doesn't directly call
\code{x.__del__()} --- the former decrements the reference count for
\code{x} by one, and the latter is only called when \code{x}'s reference
count reaches zero.  Some common situations that may prevent the
reference count of an object from going to zero include: circular
references between objects (e.g., a doubly-linked list or a tree data
structure with parent and child pointers); a reference to the object
on the stack frame of a function that caught an exception (the
traceback stored in \code{sys.exc_traceback} keeps the stack frame
alive); or a reference to the object on the stack frame that raised an
unhandled exception in interactive mode (the traceback stored in
\code{sys.last_traceback} keeps the stack frame alive).  The first
situation can only be remedied by explicitly breaking the cycles; the
latter two situations can be resolved by storing \code{None} in
\code{sys.exc_traceback} or \code{sys.last_traceback}.  Circular
references which are garbage are detected when the option cycle
detector is enabled (it's on by default), but can only be cleaned up
if there are no Python-level \method{__del__()} methods involved.
Refer to the documentation for the \ulink{\module{gc}
module}{../lib/module-gc.html} for more information about how
\method{__del__()} methods are handled by the cycle detector,
particularly the description of the \code{garbage} value.
\end{notice}

\begin{notice}[warning]
Due to the precarious circumstances under which
\method{__del__()} methods are invoked, exceptions that occur during their
execution are ignored, and a warning is printed to \code{sys.stderr}
instead.  Also, when \method{__del__()} is invoked in response to a module
being deleted (e.g., when execution of the program is done), other
globals referenced by the \method{__del__()} method may already have been
deleted.  For this reason, \method{__del__()} methods should do the
absolute minimum needed to maintain external invariants.  Starting with
version 1.5, Python guarantees that globals whose name begins with a single
underscore are deleted from their module before other globals are deleted;
if no other references to such globals exist, this may help in assuring that
imported modules are still available at the time when the
\method{__del__()} method is called.
\end{notice}
\end{methoddesc}

\begin{methoddesc}[object]{__repr__}{self}
Called by the \function{repr()}\bifuncindex{repr} built-in function
and by string conversions (reverse quotes) to compute the ``official''
string representation of an object.  If at all possible, this should
look like a valid Python expression that could be used to recreate an
object with the same value (given an appropriate environment).  If
this is not possible, a string of the form \samp{<\var{...some useful
description...}>} should be returned.  The return value must be a
string object.
If a class defines \method{__repr__()} but not \method{__str__()},
then \method{__repr__()} is also used when an ``informal'' string
representation of instances of that class is required.		     

This is typically used for debugging, so it is important that the
representation is information-rich and unambiguous.
\indexii{string}{conversion}
\indexii{reverse}{quotes}
\indexii{backward}{quotes}
\index{back-quotes}
\end{methoddesc}

\begin{methoddesc}[object]{__str__}{self}
Called by the \function{str()}\bifuncindex{str} built-in function and
by the \keyword{print}\stindex{print} statement to compute the
``informal'' string representation of an object.  This differs from
\method{__repr__()} in that it does not have to be a valid Python
expression: a more convenient or concise representation may be used
instead.  The return value must be a string object.
\end{methoddesc}

\begin{methoddesc}[object]{__lt__}{self, other}
\methodline[object]{__le__}{self, other}
\methodline[object]{__eq__}{self, other}
\methodline[object]{__ne__}{self, other}
\methodline[object]{__gt__}{self, other}
\methodline[object]{__ge__}{self, other}
\versionadded{2.1}
These are the so-called ``rich comparison'' methods, and are called
for comparison operators in preference to \method{__cmp__()} below.
The correspondence between operator symbols and method names is as
follows:
\code{\var{x}<\var{y}} calls \code{\var{x}.__lt__(\var{y})},
\code{\var{x}<=\var{y}} calls \code{\var{x}.__le__(\var{y})},
\code{\var{x}==\var{y}} calls \code{\var{x}.__eq__(\var{y})},
\code{\var{x}!=\var{y}} and \code{\var{x}<>\var{y}} call
\code{\var{x}.__ne__(\var{y})},
\code{\var{x}>\var{y}} calls \code{\var{x}.__gt__(\var{y})}, and
\code{\var{x}>=\var{y}} calls \code{\var{x}.__ge__(\var{y})}.

A rich comparison method may return the singleton \code{NotImplemented} if it
does not implement the operation for a given pair of arguments.
By convention, \code{False} and \code{True} are returned for a successful
comparison. However, these methods can return any value, so if the
comparison operator is used in a Boolean context (e.g., in the condition
of an \code{if} statement), Python will call \function{bool()} on the
value to determine if the result is true or false.

There are no implied relationships among the comparison operators.
The truth of \code{\var{x}==\var{y}} does not imply that \code{\var{x}!=\var{y}}
is false.  Accordingly, when defining \method{__eq__()}, one should also
define \method{__ne__()} so that the operators will behave as expected.

There are no reflected (swapped-argument) versions of these methods
(to be used when the left argument does not support the operation but
the right argument does); rather, \method{__lt__()} and
\method{__gt__()} are each other's reflection, \method{__le__()} and
\method{__ge__()} are each other's reflection, and \method{__eq__()}
and \method{__ne__()} are their own reflection.

Arguments to rich comparison methods are never coerced.
\end{methoddesc}

\begin{methoddesc}[object]{__cmp__}{self, other}
Called by comparison operations if rich comparison (see above) is not
defined.  Should return a negative integer if \code{self < other},
zero if \code{self == other}, a positive integer if \code{self >
other}.  If no \method{__cmp__()}, \method{__eq__()} or
\method{__ne__()} operation is defined, class instances are compared
by object identity (``address'').  See also the description of
\method{__hash__()} for some important notes on creating objects which
support custom comparison operations and are usable as dictionary
keys.
(Note: the restriction that exceptions are not propagated by
\method{__cmp__()} has been removed since Python 1.5.)
\bifuncindex{cmp}
\index{comparisons}
\end{methoddesc}

\begin{methoddesc}[object]{__rcmp__}{self, other}
  \versionchanged[No longer supported]{2.1}
\end{methoddesc}

\begin{methoddesc}[object]{__hash__}{self}
Called for the key object for dictionary \obindex{dictionary}
operations, and by the built-in function
\function{hash()}\bifuncindex{hash}.  Should return a 32-bit integer
usable as a hash value
for dictionary operations.  The only required property is that objects
which compare equal have the same hash value; it is advised to somehow
mix together (e.g., using exclusive or) the hash values for the
components of the object that also play a part in comparison of
objects.  If a class does not define a \method{__cmp__()} method it should
not define a \method{__hash__()} operation either; if it defines
\method{__cmp__()} or \method{__eq__()} but not \method{__hash__()},
its instances will not be usable as dictionary keys.  If a class
defines mutable objects and implements a \method{__cmp__()} or
\method{__eq__()} method, it should not implement \method{__hash__()},
since the dictionary implementation requires that a key's hash value
is immutable (if the object's hash value changes, it will be in the
wrong hash bucket).

\versionchanged[\method{__hash__()} may now also return a long
integer object; the 32-bit integer is then derived from the hash
of that object]{2.5}

\withsubitem{(object method)}{\ttindex{__cmp__()}}
\end{methoddesc}

\begin{methoddesc}[object]{__nonzero__}{self}
Called to implement truth value testing, and the built-in operation
\code{bool()}; should return \code{False} or \code{True}, or their
integer equivalents \code{0} or \code{1}.
When this method is not defined, \method{__len__()} is
called, if it is defined (see below).  If a class defines neither
\method{__len__()} nor \method{__nonzero__()}, all its instances are
considered true.
\withsubitem{(mapping object method)}{\ttindex{__len__()}}
\end{methoddesc}

\begin{methoddesc}[object]{__unicode__}{self}
Called to implement \function{unicode()}\bifuncindex{unicode} builtin;
should return a Unicode object. When this method is not defined, string
conversion is attempted, and the result of string conversion is converted
to Unicode using the system default encoding.
\end{methoddesc}


\subsection{Customizing attribute access\label{attribute-access}}

The following methods can be defined to customize the meaning of
attribute access (use of, assignment to, or deletion of \code{x.name})
for class instances.

\begin{methoddesc}[object]{__getattr__}{self, name}
Called when an attribute lookup has not found the attribute in the
usual places (i.e. it is not an instance attribute nor is it found in
the class tree for \code{self}).  \code{name} is the attribute name.
This method should return the (computed) attribute value or raise an
\exception{AttributeError} exception.

Note that if the attribute is found through the normal mechanism,
\method{__getattr__()} is not called.  (This is an intentional
asymmetry between \method{__getattr__()} and \method{__setattr__()}.)
This is done both for efficiency reasons and because otherwise
\method{__setattr__()} would have no way to access other attributes of
the instance.  Note that at least for instance variables, you can fake
total control by not inserting any values in the instance attribute
dictionary (but instead inserting them in another object).  See the
\method{__getattribute__()} method below for a way to actually get
total control in new-style classes.
\withsubitem{(object method)}{\ttindex{__setattr__()}}
\end{methoddesc}

\begin{methoddesc}[object]{__setattr__}{self, name, value}
Called when an attribute assignment is attempted.  This is called
instead of the normal mechanism (i.e.\ store the value in the instance
dictionary).  \var{name} is the attribute name, \var{value} is the
value to be assigned to it.

If \method{__setattr__()} wants to assign to an instance attribute, it 
should not simply execute \samp{self.\var{name} = value} --- this
would cause a recursive call to itself.  Instead, it should insert the
value in the dictionary of instance attributes, e.g.,
\samp{self.__dict__[\var{name}] = value}.  For new-style classes,
rather than accessing the instance dictionary, it should call the base
class method with the same name, for example,
\samp{object.__setattr__(self, name, value)}.
\withsubitem{(instance attribute)}{\ttindex{__dict__}}
\end{methoddesc}

\begin{methoddesc}[object]{__delattr__}{self, name}
Like \method{__setattr__()} but for attribute deletion instead of
assignment.  This should only be implemented if \samp{del
obj.\var{name}} is meaningful for the object.
\end{methoddesc}

\subsubsection{More attribute access for new-style classes \label{new-style-attribute-access}}

The following methods only apply to new-style classes.

\begin{methoddesc}[object]{__getattribute__}{self, name}
Called unconditionally to implement attribute accesses for instances
of the class. If the class also defines \method{__getattr__()}, the latter 
will not be called unless \method{__getattribute__()} either calls it 
explicitly or raises an \exception{AttributeError}.
This method should return the (computed) attribute
value or raise an \exception{AttributeError} exception.
In order to avoid infinite recursion in this method, its
implementation should always call the base class method with the same
name to access any attributes it needs, for example,
\samp{object.__getattribute__(self, name)}.
\end{methoddesc}

\subsubsection{Implementing Descriptors \label{descriptors}}

The following methods only apply when an instance of the class
containing the method (a so-called \emph{descriptor} class) appears in
the class dictionary of another new-style class, known as the
\emph{owner} class. In the examples below, ``the attribute'' refers to
the attribute whose name is the key of the property in the owner
class' \code{__dict__}.  Descriptors can only be implemented as
new-style classes themselves.

\begin{methoddesc}[object]{__get__}{self, instance, owner}
Called to get the attribute of the owner class (class attribute access)
or of an instance of that class (instance attribute access).
\var{owner} is always the owner class, while \var{instance} is the
instance that the attribute was accessed through, or \code{None} when
the attribute is accessed through the \var{owner}.  This method should
return the (computed) attribute value or raise an
\exception{AttributeError} exception.
\end{methoddesc}

\begin{methoddesc}[object]{__set__}{self, instance, value}
Called to set the attribute on an instance \var{instance} of the owner
class to a new value, \var{value}.
\end{methoddesc}

\begin{methoddesc}[object]{__delete__}{self, instance}
Called to delete the attribute on an instance \var{instance} of the
owner class.
\end{methoddesc}


\subsubsection{Invoking Descriptors \label{descriptor-invocation}}

In general, a descriptor is an object attribute with ``binding behavior'',
one whose attribute access has been overridden by methods in the descriptor
protocol:  \method{__get__()}, \method{__set__()}, and \method{__delete__()}.
If any of those methods are defined for an object, it is said to be a
descriptor.

The default behavior for attribute access is to get, set, or delete the
attribute from an object's dictionary. For instance, \code{a.x} has a
lookup chain starting with \code{a.__dict__['x']}, then
\code{type(a).__dict__['x']}, and continuing 
through the base classes of \code{type(a)} excluding metaclasses.

However, if the looked-up value is an object defining one of the descriptor
methods, then Python may override the default behavior and invoke the
descriptor method instead.  Where this occurs in the precedence chain depends
on which descriptor methods were defined and how they were called.  Note that
descriptors are only invoked for new style objects or classes
(ones that subclass \class{object()} or \class{type()}).

The starting point for descriptor invocation is a binding, \code{a.x}.
How the arguments are assembled depends on \code{a}:

\begin{itemize}
                      
  \item[Direct Call] The simplest and least common call is when user code
    directly invokes a descriptor method:    \code{x.__get__(a)}.

  \item[Instance Binding]  If binding to a new-style object instance,
    \code{a.x} is transformed into the call:
    \code{type(a).__dict__['x'].__get__(a, type(a))}.
                     
  \item[Class Binding]  If binding to a new-style class, \code{A.x}
    is transformed into the call: \code{A.__dict__['x'].__get__(None, A)}.

  \item[Super Binding] If \code{a} is an instance of \class{super},
    then the binding \code{super(B, obj).m()} searches
    \code{obj.__class__.__mro__} for the base class \code{A} immediately
    preceding \code{B} and then invokes the descriptor with the call:
    \code{A.__dict__['m'].__get__(obj, A)}.
                     
\end{itemize}

For instance bindings, the precedence of descriptor invocation depends
on the which descriptor methods are defined.  Data descriptors define
both \method{__get__()} and \method{__set__()}.  Non-data descriptors have
just the \method{__get__()} method.  Data descriptors always override
a redefinition in an instance dictionary.  In contrast, non-data
descriptors can be overridden by instances.

Python methods (including \function{staticmethod()} and \function{classmethod()})
are implemented as non-data descriptors.  Accordingly, instances can
redefine and override methods.  This allows individual instances to acquire
behaviors that differ from other instances of the same class.                     

The \function{property()} function is implemented as a data descriptor.
Accordingly, instances cannot override the behavior of a property.


\subsubsection{__slots__\label{slots}}

By default, instances of both old and new-style classes have a dictionary
for attribute storage.  This wastes space for objects having very few instance
variables.  The space consumption can become acute when creating large numbers
of instances.

The default can be overridden by defining \var{__slots__} in a new-style class
definition.  The \var{__slots__} declaration takes a sequence of instance
variables and reserves just enough space in each instance to hold a value
for each variable.  Space is saved because \var{__dict__} is not created for
each instance.
    
\begin{datadesc}{__slots__}
This class variable can be assigned a string, iterable, or sequence of strings
with variable names used by instances.  If defined in a new-style class,
\var{__slots__} reserves space for the declared variables
and prevents the automatic creation of \var{__dict__} and \var{__weakref__}
for each instance.
\versionadded{2.2}                     
\end{datadesc}

\noindent
Notes on using \var{__slots__}

\begin{itemize}

\item Without a \var{__dict__} variable, instances cannot be assigned new
variables not listed in the \var{__slots__} definition.  Attempts to assign
to an unlisted variable name raises \exception{AttributeError}. If dynamic
assignment of new variables is desired, then add \code{'__dict__'} to the
sequence of strings in the \var{__slots__} declaration.                                     
\versionchanged[Previously, adding \code{'__dict__'} to the \var{__slots__}
declaration would not enable the assignment of new attributes not
specifically listed in the sequence of instance variable names]{2.3}                     

\item Without a \var{__weakref__} variable for each instance, classes
defining \var{__slots__} do not support weak references to its instances.
If weak reference support is needed, then add \code{'__weakref__'} to the
sequence of strings in the \var{__slots__} declaration.                    
\versionchanged[Previously, adding \code{'__weakref__'} to the \var{__slots__}
declaration would not enable support for weak references]{2.3}                                            

\item \var{__slots__} are implemented at the class level by creating
descriptors (\ref{descriptors}) for each variable name.  As a result,
class attributes cannot be used to set default values for instance
variables defined by \var{__slots__}; otherwise, the class attribute would
overwrite the descriptor assignment. 

\item If a class defines a slot also defined in a base class, the instance
variable defined by the base class slot is inaccessible (except by retrieving
its descriptor directly from the base class). This renders the meaning of the
program undefined.  In the future, a check may be added to prevent this.

\item The action of a \var{__slots__} declaration is limited to the class
where it is defined.  As a result, subclasses will have a \var{__dict__}
unless they also define  \var{__slots__}.                     

\item \var{__slots__} do not work for classes derived from ``variable-length''
built-in types such as \class{long}, \class{str} and \class{tuple}. 

\item Any non-string iterable may be assigned to \var{__slots__}.
Mappings may also be used; however, in the future, special meaning may
be assigned to the values corresponding to each key.                      

\end{itemize}


\subsection{Customizing class creation\label{metaclasses}}

By default, new-style classes are constructed using \function{type()}.
A class definition is read into a separate namespace and the value
of class name is bound to the result of \code{type(name, bases, dict)}.

When the class definition is read, if \var{__metaclass__} is defined
then the callable assigned to it will be called instead of \function{type()}.
The allows classes or functions to be written which monitor or alter the class
creation process:

\begin{itemize}
\item Modifying the class dictionary prior to the class being created.
\item Returning an instance of another class -- essentially performing
the role of a factory function.
\end{itemize}

\begin{datadesc}{__metaclass__}
This variable can be any callable accepting arguments for \code{name},
\code{bases}, and \code{dict}.  Upon class creation, the callable is
used instead of the built-in \function{type()}.
\versionadded{2.2}                     
\end{datadesc}

The appropriate metaclass is determined by the following precedence rules:

\begin{itemize}

\item If \code{dict['__metaclass__']} exists, it is used.

\item Otherwise, if there is at least one base class, its metaclass is used
(this looks for a \var{__class__} attribute first and if not found, uses its
type).

\item Otherwise, if a global variable named __metaclass__ exists, it is used.

\item Otherwise, the old-style, classic metaclass (types.ClassType) is used.

\end{itemize}      

The potential uses for metaclasses are boundless. Some ideas that have
been explored including logging, interface checking, automatic delegation,
automatic property creation, proxies, frameworks, and automatic resource
locking/synchronization.


\subsection{Emulating callable objects\label{callable-types}}

\begin{methoddesc}[object]{__call__}{self\optional{, args...}}
Called when the instance is ``called'' as a function; if this method
is defined, \code{\var{x}(arg1, arg2, ...)} is a shorthand for
\code{\var{x}.__call__(arg1, arg2, ...)}.
\indexii{call}{instance}
\end{methoddesc}


\subsection{Emulating container types\label{sequence-types}}

The following methods can be defined to implement container
objects.  Containers usually are sequences (such as lists or tuples)
or mappings (like dictionaries), but can represent other containers as
well.  The first set of methods is used either to emulate a
sequence or to emulate a mapping; the difference is that for a
sequence, the allowable keys should be the integers \var{k} for which
\code{0 <= \var{k} < \var{N}} where \var{N} is the length of the
sequence, or slice objects, which define a range of items. (For backwards
compatibility, the method \method{__getslice__()} (see below) can also be
defined to handle simple, but not extended slices.) It is also recommended
that mappings provide the methods \method{keys()}, \method{values()},
\method{items()}, \method{has_key()}, \method{get()}, \method{clear()},
\method{setdefault()}, \method{iterkeys()}, \method{itervalues()},
\method{iteritems()}, \method{pop()}, \method{popitem()},		     
\method{copy()}, and \method{update()} behaving similar to those for
Python's standard dictionary objects.  The \module{UserDict} module
provides a \class{DictMixin} class to help create those methods
from a base set of \method{__getitem__()}, \method{__setitem__()},
\method{__delitem__()}, and \method{keys()}.		     
Mutable sequences should provide
methods \method{append()}, \method{count()}, \method{index()},
\method{extend()},		     
\method{insert()}, \method{pop()}, \method{remove()}, \method{reverse()}
and \method{sort()}, like Python standard list objects.  Finally,
sequence types should implement addition (meaning concatenation) and
multiplication (meaning repetition) by defining the methods
\method{__add__()}, \method{__radd__()}, \method{__iadd__()},
\method{__mul__()}, \method{__rmul__()} and \method{__imul__()} described
below; they should not define \method{__coerce__()} or other numerical
operators.  It is recommended that both mappings and sequences
implement the \method{__contains__()} method to allow efficient use of
the \code{in} operator; for mappings, \code{in} should be equivalent
of \method{has_key()}; for sequences, it should search through the
values.  It is further recommended that both mappings and sequences
implement the \method{__iter__()} method to allow efficient iteration
through the container; for mappings, \method{__iter__()} should be
the same as \method{iterkeys()}; for sequences, it should iterate
through the values.
\withsubitem{(mapping object method)}{
  \ttindex{keys()}
  \ttindex{values()}
  \ttindex{items()}
  \ttindex{iterkeys()}
  \ttindex{itervalues()}
  \ttindex{iteritems()}    
  \ttindex{has_key()}
  \ttindex{get()}
  \ttindex{setdefault()}
  \ttindex{pop()}      
  \ttindex{popitem()}    
  \ttindex{clear()}
  \ttindex{copy()}
  \ttindex{update()}
  \ttindex{__contains__()}}
\withsubitem{(sequence object method)}{
  \ttindex{append()}
  \ttindex{count()}
  \ttindex{extend()}    
  \ttindex{index()}
  \ttindex{insert()}
  \ttindex{pop()}
  \ttindex{remove()}
  \ttindex{reverse()}
  \ttindex{sort()}
  \ttindex{__add__()}
  \ttindex{__radd__()}
  \ttindex{__iadd__()}
  \ttindex{__mul__()}
  \ttindex{__rmul__()}
  \ttindex{__imul__()}
  \ttindex{__contains__()}
  \ttindex{__iter__()}}		     
\withsubitem{(numeric object method)}{\ttindex{__coerce__()}}

\begin{methoddesc}[container object]{__len__}{self}
Called to implement the built-in function
\function{len()}\bifuncindex{len}.  Should return the length of the
object, an integer \code{>=} 0.  Also, an object that doesn't define a
\method{__nonzero__()} method and whose \method{__len__()} method
returns zero is considered to be false in a Boolean context.
\withsubitem{(object method)}{\ttindex{__nonzero__()}}
\end{methoddesc}

\begin{methoddesc}[container object]{__getitem__}{self, key}
Called to implement evaluation of \code{\var{self}[\var{key}]}.
For sequence types, the accepted keys should be integers and slice
objects.\obindex{slice}  Note that
the special interpretation of negative indexes (if the class wishes to
emulate a sequence type) is up to the \method{__getitem__()} method.
If \var{key} is of an inappropriate type, \exception{TypeError} may be
raised; if of a value outside the set of indexes for the sequence
(after any special interpretation of negative values),
\exception{IndexError} should be raised.
For mapping types, if \var{key} is missing (not in the container),
\exception{KeyError} should be raised.                     
\note{\keyword{for} loops expect that an
\exception{IndexError} will be raised for illegal indexes to allow
proper detection of the end of the sequence.}
\end{methoddesc}

\begin{methoddesc}[container object]{__setitem__}{self, key, value}
Called to implement assignment to \code{\var{self}[\var{key}]}.  Same
note as for \method{__getitem__()}.  This should only be implemented
for mappings if the objects support changes to the values for keys, or
if new keys can be added, or for sequences if elements can be
replaced.  The same exceptions should be raised for improper
\var{key} values as for the \method{__getitem__()} method.
\end{methoddesc}

\begin{methoddesc}[container object]{__delitem__}{self, key}
Called to implement deletion of \code{\var{self}[\var{key}]}.  Same
note as for \method{__getitem__()}.  This should only be implemented
for mappings if the objects support removal of keys, or for sequences
if elements can be removed from the sequence.  The same exceptions
should be raised for improper \var{key} values as for the
\method{__getitem__()} method.
\end{methoddesc}

\begin{methoddesc}[container object]{__iter__}{self}
This method is called when an iterator is required for a container.
This method should return a new iterator object that can iterate over
all the objects in the container.  For mappings, it should iterate
over the keys of the container, and should also be made available as
the method \method{iterkeys()}.

Iterator objects also need to implement this method; they are required
to return themselves.  For more information on iterator objects, see
``\ulink{Iterator Types}{../lib/typeiter.html}'' in the
\citetitle[../lib/lib.html]{Python Library Reference}.
\end{methoddesc}

The membership test operators (\keyword{in} and \keyword{not in}) are
normally implemented as an iteration through a sequence.  However,
container objects can supply the following special method with a more
efficient implementation, which also does not require the object be a
sequence.

\begin{methoddesc}[container object]{__contains__}{self, item}
Called to implement membership test operators.  Should return true if
\var{item} is in \var{self}, false otherwise.  For mapping objects,
this should consider the keys of the mapping rather than the values or
the key-item pairs.
\end{methoddesc}


\subsection{Additional methods for emulation of sequence types
  \label{sequence-methods}}

The following optional methods can be defined to further emulate sequence
objects.  Immutable sequences methods should at most only define
\method{__getslice__()}; mutable sequences might define all three
methods.

\begin{methoddesc}[sequence object]{__getslice__}{self, i, j}
\deprecated{2.0}{Support slice objects as parameters to the
\method{__getitem__()} method.}
Called to implement evaluation of \code{\var{self}[\var{i}:\var{j}]}.
The returned object should be of the same type as \var{self}.  Note
that missing \var{i} or \var{j} in the slice expression are replaced
by zero or \code{sys.maxint}, respectively.  If negative indexes are
used in the slice, the length of the sequence is added to that index.
If the instance does not implement the \method{__len__()} method, an
\exception{AttributeError} is raised.
No guarantee is made that indexes adjusted this way are not still
negative.  Indexes which are greater than the length of the sequence
are not modified.
If no \method{__getslice__()} is found, a slice
object is created instead, and passed to \method{__getitem__()} instead.
\end{methoddesc}

\begin{methoddesc}[sequence object]{__setslice__}{self, i, j, sequence}
Called to implement assignment to \code{\var{self}[\var{i}:\var{j}]}.
Same notes for \var{i} and \var{j} as for \method{__getslice__()}.

This method is deprecated. If no \method{__setslice__()} is found,
or for extended slicing of the form
\code{\var{self}[\var{i}:\var{j}:\var{k}]}, a
slice object is created, and passed to \method{__setitem__()},
instead of \method{__setslice__()} being called.
\end{methoddesc}

\begin{methoddesc}[sequence object]{__delslice__}{self, i, j}
Called to implement deletion of \code{\var{self}[\var{i}:\var{j}]}.
Same notes for \var{i} and \var{j} as for \method{__getslice__()}.
This method is deprecated. If no \method{__delslice__()} is found,
or for extended slicing of the form
\code{\var{self}[\var{i}:\var{j}:\var{k}]}, a
slice object is created, and passed to \method{__delitem__()},
instead of \method{__delslice__()} being called.
\end{methoddesc}

Notice that these methods are only invoked when a single slice with a
single colon is used, and the slice method is available.  For slice
operations involving extended slice notation, or in absence of the
slice methods, \method{__getitem__()}, \method{__setitem__()} or
\method{__delitem__()} is called with a slice object as argument.

The following example demonstrate how to make your program or module
compatible with earlier versions of Python (assuming that methods
\method{__getitem__()}, \method{__setitem__()} and \method{__delitem__()}
support slice objects as arguments):

\begin{verbatim}
class MyClass:
    ...
    def __getitem__(self, index):
        ...
    def __setitem__(self, index, value):
        ...
    def __delitem__(self, index):
        ...

    if sys.version_info < (2, 0):
        # They won't be defined if version is at least 2.0 final

        def __getslice__(self, i, j):
            return self[max(0, i):max(0, j):]
        def __setslice__(self, i, j, seq):
            self[max(0, i):max(0, j):] = seq
        def __delslice__(self, i, j):
            del self[max(0, i):max(0, j):]
    ...
\end{verbatim}

Note the calls to \function{max()}; these are necessary because of
the handling of negative indices before the
\method{__*slice__()} methods are called.  When negative indexes are
used, the \method{__*item__()} methods receive them as provided, but
the \method{__*slice__()} methods get a ``cooked'' form of the index
values.  For each negative index value, the length of the sequence is
added to the index before calling the method (which may still result
in a negative index); this is the customary handling of negative
indexes by the built-in sequence types, and the \method{__*item__()}
methods are expected to do this as well.  However, since they should
already be doing that, negative indexes cannot be passed in; they must
be constrained to the bounds of the sequence before being passed to
the \method{__*item__()} methods.
Calling \code{max(0, i)} conveniently returns the proper value.


\subsection{Emulating numeric types\label{numeric-types}}

The following methods can be defined to emulate numeric objects.
Methods corresponding to operations that are not supported by the
particular kind of number implemented (e.g., bitwise operations for
non-integral numbers) should be left undefined.

\begin{methoddesc}[numeric object]{__add__}{self, other}
\methodline[numeric object]{__sub__}{self, other}
\methodline[numeric object]{__mul__}{self, other}
\methodline[numeric object]{__floordiv__}{self, other}
\methodline[numeric object]{__mod__}{self, other}
\methodline[numeric object]{__divmod__}{self, other}
\methodline[numeric object]{__pow__}{self, other\optional{, modulo}}
\methodline[numeric object]{__lshift__}{self, other}
\methodline[numeric object]{__rshift__}{self, other}
\methodline[numeric object]{__and__}{self, other}
\methodline[numeric object]{__xor__}{self, other}
\methodline[numeric object]{__or__}{self, other}
These methods are
called to implement the binary arithmetic operations (\code{+},
\code{-}, \code{*}, \code{//}, \code{\%},
\function{divmod()}\bifuncindex{divmod},
\function{pow()}\bifuncindex{pow}, \code{**}, \code{<<},
\code{>>}, \code{\&}, \code{\^}, \code{|}).  For instance, to
evaluate the expression \var{x}\code{+}\var{y}, where \var{x} is an
instance of a class that has an \method{__add__()} method,
\code{\var{x}.__add__(\var{y})} is called.  The \method{__divmod__()}
method should be the equivalent to using \method{__floordiv__()} and
\method{__mod__()}; it should not be related to \method{__truediv__()}
(described below).  Note that
\method{__pow__()} should be defined to accept an optional third
argument if the ternary version of the built-in
\function{pow()}\bifuncindex{pow} function is to be supported.

If one of those methods does not support the operation with the
supplied arguments, it should return \code{NotImplemented}.
\end{methoddesc}

\begin{methoddesc}[numeric object]{__div__}{self, other}
\methodline[numeric object]{__truediv__}{self, other}
The division operator (\code{/}) is implemented by these methods.  The
\method{__truediv__()} method is used when \code{__future__.division}
is in effect, otherwise \method{__div__()} is used.  If only one of
these two methods is defined, the object will not support division in
the alternate context; \exception{TypeError} will be raised instead.
\end{methoddesc}

\begin{methoddesc}[numeric object]{__radd__}{self, other}
\methodline[numeric object]{__rsub__}{self, other}
\methodline[numeric object]{__rmul__}{self, other}
\methodline[numeric object]{__rdiv__}{self, other}
\methodline[numeric object]{__rtruediv__}{self, other}
\methodline[numeric object]{__rfloordiv__}{self, other}	     
\methodline[numeric object]{__rmod__}{self, other}
\methodline[numeric object]{__rdivmod__}{self, other}
\methodline[numeric object]{__rpow__}{self, other}
\methodline[numeric object]{__rlshift__}{self, other}
\methodline[numeric object]{__rrshift__}{self, other}
\methodline[numeric object]{__rand__}{self, other}
\methodline[numeric object]{__rxor__}{self, other}
\methodline[numeric object]{__ror__}{self, other}
These methods are
called to implement the binary arithmetic operations (\code{+},
\code{-}, \code{*}, \code{/}, \code{\%},
\function{divmod()}\bifuncindex{divmod},
\function{pow()}\bifuncindex{pow}, \code{**}, \code{<<},
\code{>>}, \code{\&}, \code{\^}, \code{|}) with reflected
(swapped) operands.  These functions are only called if the left
operand does not support the corresponding operation and the
operands are of different types.\footnote{
    For operands of the same type, it is assumed that if the
    non-reflected method (such as \method{__add__()}) fails the
    operation is not supported, which is why the reflected method
    is not called.} 
For instance, to evaluate the expression \var{x}\code{-}\var{y},
where \var{y} is an instance of a class that has an
\method{__rsub__()} method, \code{\var{y}.__rsub__(\var{x})}
is called if \code{\var{x}.__sub__(\var{y})} returns
\var{NotImplemented}.

Note that ternary
\function{pow()}\bifuncindex{pow} will not try calling
\method{__rpow__()} (the coercion rules would become too
complicated).

\note{If the right operand's type is a subclass of the left operand's
      type and that subclass provides the reflected method for the
      operation, this method will be called before the left operand's
      non-reflected method.  This behavior allows subclasses to
      override their ancestors' operations.}
\end{methoddesc}

\begin{methoddesc}[numeric object]{__iadd__}{self, other}
\methodline[numeric object]{__isub__}{self, other}
\methodline[numeric object]{__imul__}{self, other}
\methodline[numeric object]{__idiv__}{self, other}
\methodline[numeric object]{__itruediv__}{self, other}
\methodline[numeric object]{__ifloordiv__}{self, other}
\methodline[numeric object]{__imod__}{self, other}		     
\methodline[numeric object]{__ipow__}{self, other\optional{, modulo}}
\methodline[numeric object]{__ilshift__}{self, other}
\methodline[numeric object]{__irshift__}{self, other}
\methodline[numeric object]{__iand__}{self, other}
\methodline[numeric object]{__ixor__}{self, other}
\methodline[numeric object]{__ior__}{self, other}
These methods are called to implement the augmented arithmetic
operations (\code{+=}, \code{-=}, \code{*=}, \code{/=}, \code{//=},
\code{\%=}, \code{**=}, \code{<<=}, \code{>>=}, \code{\&=},
\code{\textasciicircum=}, \code{|=}).  These methods should attempt to do the
operation in-place (modifying \var{self}) and return the result (which
could be, but does not have to be, \var{self}).  If a specific method
is not defined, the augmented operation falls back to the normal
methods.  For instance, to evaluate the expression
\var{x}\code{+=}\var{y}, where \var{x} is an instance of a class that
has an \method{__iadd__()} method, \code{\var{x}.__iadd__(\var{y})} is
called.  If \var{x} is an instance of a class that does not define a
\method{__iadd__()} method, \code{\var{x}.__add__(\var{y})} and
\code{\var{y}.__radd__(\var{x})} are considered, as with the
evaluation of \var{x}\code{+}\var{y}.
\end{methoddesc}

\begin{methoddesc}[numeric object]{__neg__}{self}
\methodline[numeric object]{__pos__}{self}
\methodline[numeric object]{__abs__}{self}
\methodline[numeric object]{__invert__}{self}
Called to implement the unary arithmetic operations (\code{-},
\code{+}, \function{abs()}\bifuncindex{abs} and \code{\~{}}).
\end{methoddesc}

\begin{methoddesc}[numeric object]{__complex__}{self}
\methodline[numeric object]{__int__}{self}
\methodline[numeric object]{__long__}{self}
\methodline[numeric object]{__float__}{self}
Called to implement the built-in functions
\function{complex()}\bifuncindex{complex},
\function{int()}\bifuncindex{int}, \function{long()}\bifuncindex{long},
and \function{float()}\bifuncindex{float}.  Should return a value of
the appropriate type.
\end{methoddesc}

\begin{methoddesc}[numeric object]{__oct__}{self}
\methodline[numeric object]{__hex__}{self}
Called to implement the built-in functions
\function{oct()}\bifuncindex{oct} and
\function{hex()}\bifuncindex{hex}.  Should return a string value.
\end{methoddesc}

\begin{methoddesc}[numeric object]{__index__}{self}
Called to implement \function{operator.index()}.  Also called whenever
Python needs an integer object (such as in slicing).  Must return an
integer (int or long).
\versionadded{2.5}
\end{methoddesc}

\begin{methoddesc}[numeric object]{__coerce__}{self, other}
Called to implement ``mixed-mode'' numeric arithmetic.  Should either
return a 2-tuple containing \var{self} and \var{other} converted to
a common numeric type, or \code{None} if conversion is impossible.  When
the common type would be the type of \code{other}, it is sufficient to
return \code{None}, since the interpreter will also ask the other
object to attempt a coercion (but sometimes, if the implementation of
the other type cannot be changed, it is useful to do the conversion to
the other type here).  A return value of \code{NotImplemented} is
equivalent to returning \code{None}.
\end{methoddesc}

\subsection{Coercion rules\label{coercion-rules}}

This section used to document the rules for coercion.  As the language
has evolved, the coercion rules have become hard to document
precisely; documenting what one version of one particular
implementation does is undesirable.  Instead, here are some informal
guidelines regarding coercion.  In Python 3.0, coercion will not be
supported.

\begin{itemize}

\item

If the left operand of a \% operator is a string or Unicode object, no
coercion takes place and the string formatting operation is invoked
instead.

\item

It is no longer recommended to define a coercion operation.
Mixed-mode operations on types that don't define coercion pass the
original arguments to the operation.

\item

New-style classes (those derived from \class{object}) never invoke the
\method{__coerce__()} method in response to a binary operator; the only
time \method{__coerce__()} is invoked is when the built-in function
\function{coerce()} is called.

\item

For most intents and purposes, an operator that returns
\code{NotImplemented} is treated the same as one that is not
implemented at all.

\item

Below, \method{__op__()} and \method{__rop__()} are used to signify
the generic method names corresponding to an operator;
\method{__iop__()} is used for the corresponding in-place operator.  For
example, for the operator `\code{+}', \method{__add__()} and
\method{__radd__()} are used for the left and right variant of the
binary operator, and \method{__iadd__()} for the in-place variant.

\item

For objects \var{x} and \var{y}, first \code{\var{x}.__op__(\var{y})}
is tried.  If this is not implemented or returns \code{NotImplemented},
\code{\var{y}.__rop__(\var{x})} is tried.  If this is also not
implemented or returns \code{NotImplemented}, a \exception{TypeError}
exception is raised.  But see the following exception:

\item

Exception to the previous item: if the left operand is an instance of
a built-in type or a new-style class, and the right operand is an instance
of a proper subclass of that type or class and overrides the base's
\method{__rop__()} method, the right operand's \method{__rop__()} method
is tried \emph{before} the left operand's \method{__op__()} method.

This is done so that a subclass can completely override binary operators.
Otherwise, the left operand's \method{__op__()} method would always
accept the right operand: when an instance of a given class is expected,
an instance of a subclass of that class is always acceptable.

\item

When either operand type defines a coercion, this coercion is called
before that type's \method{__op__()} or \method{__rop__()} method is
called, but no sooner.  If the coercion returns an object of a
different type for the operand whose coercion is invoked, part of the
process is redone using the new object.

\item

When an in-place operator (like `\code{+=}') is used, if the left
operand implements \method{__iop__()}, it is invoked without any
coercion.  When the operation falls back to \method{__op__()} and/or
\method{__rop__()}, the normal coercion rules apply.

\item

In \var{x}\code{+}\var{y}, if \var{x} is a sequence that implements
sequence concatenation, sequence concatenation is invoked.

\item

In \var{x}\code{*}\var{y}, if one operator is a sequence that
implements sequence repetition, and the other is an integer
(\class{int} or \class{long}), sequence repetition is invoked.

\item

Rich comparisons (implemented by methods \method{__eq__()} and so on)
never use coercion.  Three-way comparison (implemented by
\method{__cmp__()}) does use coercion under the same conditions as
other binary operations use it.

\item

In the current implementation, the built-in numeric types \class{int},
\class{long} and \class{float} do not use coercion; the type
\class{complex} however does use it.  The difference can become
apparent when subclassing these types.  Over time, the type
\class{complex} may be fixed to avoid coercion.  All these types
implement a \method{__coerce__()} method, for use by the built-in
\function{coerce()} function.

\end{itemize}

\subsection{With Statement Context Managers\label{context-managers}}

\versionadded{2.5}

A \dfn{context manager} is an object that defines the runtime
context to be established when executing a \keyword{with}
statement. The context manager handles the entry into,
and the exit from, the desired runtime context for the execution
of the block of code.  Context managers are normally invoked using
the \keyword{with} statement (described in section~\ref{with}), but
can also be used by directly invoking their methods.

\stindex{with}
\index{context manager}

Typical uses of context managers include saving and
restoring various kinds of global state, locking and unlocking
resources, closing opened files, etc.

For more information on context managers, see
``\ulink{Context Types}{../lib/typecontextmanager.html}'' in the
\citetitle[../lib/lib.html]{Python Library Reference}.

\begin{methoddesc}[context manager]{__enter__}{self}
Enter the runtime context related to this object. The \keyword{with}
statement will bind this method's return value to the target(s)
specified in the \keyword{as} clause of the statement, if any.
\end{methoddesc}

\begin{methoddesc}[context manager]{__exit__}
{self, exc_type, exc_value, traceback}
Exit the runtime context related to this object. The parameters
describe the exception that caused the context to be exited. If
the context was exited without an exception, all three arguments
will be \constant{None}.

If an exception is supplied, and the method wishes to suppress the
exception (i.e., prevent it from being propagated), it should return a
true value. Otherwise, the exception will be processed normally upon
exit from this method.

Note that \method{__exit__} methods should not reraise the passed-in
exception; this is the caller's responsibility.
\end{methoddesc}

\begin{seealso}
  \seepep{0343}{The "with" statement}
         {The specification, background, and examples for the
          Python \keyword{with} statement.}
\end{seealso}

		% Data model
\chapter{Execution model \label{execmodel}}
\index{execution model}


\section{Naming and binding \label{naming}}
\indexii{code}{block}
\index{namespace}
\index{scope}

\dfn{Names}\index{name} refer to objects.  Names are introduced by
name binding operations.  Each occurrence of a name in the program
text refers to the \dfn{binding}\indexii{binding}{name} of that name
established in the innermost function block containing the use.

A \dfn{block}\index{block} is a piece of Python program text that is
executed as a unit.  The following are blocks: a module, a function
body, and a class definition.  Each command typed interactively is a
block.  A script file (a file given as standard input to the
interpreter or specified on the interpreter command line the first
argument) is a code block.  A script command (a command specified on
the interpreter command line with the `\strong{-c}' option) is a code
block.  The file read by the built-in function \function{execfile()}
is a code block.  The string argument passed to the built-in function
\function{eval()} and to the \keyword{exec} statement is a code block.
The expression read and evaluated by the built-in function
\function{input()} is a code block.

A code block is executed in an \dfn{execution
frame}\indexii{execution}{frame}.  A frame contains some
administrative information (used for debugging) and determines where
and how execution continues after the code block's execution has
completed.

A \dfn{scope}\index{scope} defines the visibility of a name within a
block.  If a local variable is defined in a block, its scope includes
that block.  If the definition occurs in a function block, the scope
extends to any blocks contained within the defining one, unless a
contained block introduces a different binding for the name.  The
scope of names defined in a class block is limited to the class block;
it does not extend to the code blocks of methods.

When a name is used in a code block, it is resolved using the nearest
enclosing scope.  The set of all such scopes visible to a code block
is called the block's \dfn{environment}\index{environment}.  

If a name is bound in a block, it is a local variable of that block.
If a name is bound at the module level, it is a global variable.  (The
variables of the module code block are local and global.)  If a
variable is used in a code block but not defined there, it is a
\dfn{free variable}\indexii{free}{variable}.

When a name is not found at all, a
\exception{NameError}\withsubitem{(built-in
exception)}{\ttindex{NameError}} exception is raised.  If the name
refers to a local variable that has not been bound, a
\exception{UnboundLocalError}\ttindex{UnboundLocalError} exception is
raised.  \exception{UnboundLocalError} is a subclass of
\exception{NameError}.

The following constructs bind names: formal parameters to functions,
\keyword{import} statements, class and function definitions (these
bind the class or function name in the defining block), and targets
that are identifiers if occurring in an assignment, \keyword{for} loop
header, or in the second position of an \keyword{except} clause
header.  The \keyword{import} statement of the form ``\samp{from
\ldots import *}''\stindex{from} binds all names defined in the
imported module, except those beginning with an underscore.  This form
may only be used at the module level.

A target occurring in a \keyword{del} statement is also considered bound
for this purpose (though the actual semantics are to unbind the
name).  It is illegal to unbind a name that is referenced by an
enclosing scope; the compiler will report a \exception{SyntaxError}.

Each assignment or import statement occurs within a block defined by a
class or function definition or at the module level (the top-level
code block).

If a name binding operation occurs anywhere within a code block, all
uses of the name within the block are treated as references to the
current block.  This can lead to errors when a name is used within a
block before it is bound.
This rule is subtle.  Python lacks declarations and allows
name binding operations to occur anywhere within a code block.  The
local variables of a code block can be determined by scanning the
entire text of the block for name binding operations.

If the global statement occurs within a block, all uses of the name
specified in the statement refer to the binding of that name in the
top-level namespace.  Names are resolved in the top-level namespace by
searching the global namespace, i.e. the namespace of the module
containing the code block, and the builtin namespace, the namespace of
the module \module{__builtin__}.  The global namespace is searched
first.  If the name is not found there, the builtin namespace is
searched.  The global statement must precede all uses of the name.

The built-in namespace associated with the execution of a code block
is actually found by looking up the name \code{__builtins__} in its
global namespace; this should be a dictionary or a module (in the
latter case the module's dictionary is used).  By default, when in the
\module{__main__} module, \code{__builtins__} is the built-in module
\module{__builtin__} (note: no `s'); when in any other module,
\code{__builtins__} is an alias for the dictionary of the
\module{__builtin__} module itself.  \code{__builtins__} can be set
to a user-created dictionary to create a weak form of restricted
execution\indexii{restricted}{execution}.

\begin{notice}
  Users should not touch \code{__builtins__}; it is strictly an
  implementation detail.  Users wanting to override values in the
  built-in namespace should \keyword{import} the \module{__builtin__}
  (no `s') module and modify its attributes appropriately.
\end{notice}

The namespace for a module is automatically created the first time a
module is imported.  The main module for a script is always called
\module{__main__}\refbimodindex{__main__}.

The global statement has the same scope as a name binding operation
in the same block.  If the nearest enclosing scope for a free variable
contains a global statement, the free variable is treated as a global.

A class definition is an executable statement that may use and define
names.  These references follow the normal rules for name resolution.
The namespace of the class definition becomes the attribute dictionary
of the class.  Names defined at the class scope are not visible in
methods. 

\subsection{Interaction with dynamic features \label{dynamic-features}}

There are several cases where Python statements are illegal when
used in conjunction with nested scopes that contain free
variables.

If a variable is referenced in an enclosing scope, it is illegal
to delete the name.  An error will be reported at compile time.

If the wild card form of import --- \samp{import *} --- is used in a
function and the function contains or is a nested block with free
variables, the compiler will raise a \exception{SyntaxError}.

If \keyword{exec} is used in a function and the function contains or
is a nested block with free variables, the compiler will raise a
\exception{SyntaxError} unless the exec explicitly specifies the local
namespace for the \keyword{exec}.  (In other words, \samp{exec obj}
would be illegal, but \samp{exec obj in ns} would be legal.)

The \function{eval()}, \function{execfile()}, and \function{input()}
functions and the \keyword{exec} statement do not have access to the
full environment for resolving names.  Names may be resolved in the
local and global namespaces of the caller.  Free variables are not
resolved in the nearest enclosing namespace, but in the global
namespace.\footnote{This limitation occurs because the code that is
    executed by these operations is not available at the time the
    module is compiled.}
The \keyword{exec} statement and the \function{eval()} and
\function{execfile()} functions have optional arguments to override
the global and local namespace.  If only one namespace is specified,
it is used for both.

\section{Exceptions \label{exceptions}}
\index{exception}

Exceptions are a means of breaking out of the normal flow of control
of a code block in order to handle errors or other exceptional
conditions.  An exception is
\emph{raised}\index{raise an exception} at the point where the error
is detected; it may be \emph{handled}\index{handle an exception} by
the surrounding code block or by any code block that directly or
indirectly invoked the code block where the error occurred.
\index{exception handler}
\index{errors}
\index{error handling}

The Python interpreter raises an exception when it detects a run-time
error (such as division by zero).  A Python program can also
explicitly raise an exception with the \keyword{raise} statement.
Exception handlers are specified with the \keyword{try} ... \keyword{except}
statement.  The \keyword{try} ... \keyword{finally} statement
specifies cleanup code which does not handle the exception, but is
executed whether an exception occurred or not in the preceding code.

Python uses the ``termination''\index{termination model} model of
error handling: an exception handler can find out what happened and
continue execution at an outer level, but it cannot repair the cause
of the error and retry the failing operation (except by re-entering
the offending piece of code from the top).

When an exception is not handled at all, the interpreter terminates
execution of the program, or returns to its interactive main loop.  In
either case, it prints a stack backtrace, except when the exception is 
\exception{SystemExit}\withsubitem{(built-in
exception)}{\ttindex{SystemExit}}.

Exceptions are identified by class instances.  The \keyword{except}
clause is selected depending on the class of the instance: it must
reference the class of the instance or a base class thereof.  The
instance can be received by the handler and can carry additional
information about the exceptional condition.

Exceptions can also be identified by strings, in which case the
\keyword{except} clause is selected by object identity.  An arbitrary
value can be raised along with the identifying string which can be
passed to the handler.

\deprecated{2.5}{String exceptions should not be used in new code.
They will not be supported in a future version of Python.  Old code
should be rewritten to use class exceptions instead.}

\begin{notice}[warning]
Messages to exceptions are not part of the Python API.  Their contents may
change from one version of Python to the next without warning and should not
be relied on by code which will run under multiple versions of the
interpreter.
\end{notice}

See also the description of the \keyword{try} statement in
section~\ref{try} and \keyword{raise} statement in
section~\ref{raise}.
		% Execution model
\chapter{Expressions\label{expressions}}
\index{expression}

This chapter explains the meaning of the elements of expressions in
Python.

\strong{Syntax Notes:} In this and the following chapters, extended
BNF\index{BNF} notation will be used to describe syntax, not lexical
analysis.  When (one alternative of) a syntax rule has the form

\begin{productionlist}[*]
  \production{name}{\token{othername}}
\end{productionlist}

and no semantics are given, the semantics of this form of \code{name}
are the same as for \code{othername}.
\index{syntax}


\section{Arithmetic conversions\label{conversions}}
\indexii{arithmetic}{conversion}

When a description of an arithmetic operator below uses the phrase
``the numeric arguments are converted to a common type,'' the
arguments are coerced using the coercion rules listed at
~\ref{coercion-rules}.  If both arguments are standard numeric types,
the following coercions are applied:

\begin{itemize}
\item	If either argument is a complex number, the other is converted
	to complex;
\item	otherwise, if either argument is a floating point number,
	the other is converted to floating point;
\item	otherwise, if either argument is a long integer,
	the other is converted to long integer;
\item	otherwise, both must be plain integers and no conversion
	is necessary.
\end{itemize}

Some additional rules apply for certain operators (e.g., a string left
argument to the `\%' operator). Extensions can define their own
coercions.


\section{Atoms\label{atoms}}
\index{atom}

Atoms are the most basic elements of expressions.  The simplest atoms
are identifiers or literals.  Forms enclosed in
reverse quotes or in parentheses, brackets or braces are also
categorized syntactically as atoms.  The syntax for atoms is:

\begin{productionlist}
  \production{atom}
             {\token{identifier} | \token{literal} | \token{enclosure}}
  \production{enclosure}
             {\token{parenth_form} | \token{list_display}}
  \productioncont{| \token{generator_expression} | \token{dict_display}}
  \productioncont{| \token{string_conversion} | \token{yield_atom}}
\end{productionlist}


\subsection{Identifiers (Names)\label{atom-identifiers}}
\index{name}
\index{identifier}

An identifier occurring as an atom is a name.  See
section \ref{identifiers} for lexical definition and
section~\ref{naming} for documentation of naming and binding.

When the name is bound to an object, evaluation of the atom yields
that object.  When a name is not bound, an attempt to evaluate it
raises a \exception{NameError} exception.
\exindex{NameError}

\strong{Private name mangling:}
\indexii{name}{mangling}%
\indexii{private}{names}%
When an identifier that textually occurs in a class definition begins
with two or more underscore characters and does not end in two or more
underscores, it is considered a \dfn{private name} of that class.
Private names are transformed to a longer form before code is
generated for them.  The transformation inserts the class name in
front of the name, with leading underscores removed, and a single
underscore inserted in front of the class name.  For example, the
identifier \code{__spam} occurring in a class named \code{Ham} will be
transformed to \code{_Ham__spam}.  This transformation is independent
of the syntactical context in which the identifier is used.  If the
transformed name is extremely long (longer than 255 characters),
implementation defined truncation may happen.  If the class name
consists only of underscores, no transformation is done.


\subsection{Literals\label{atom-literals}}
\index{literal}

Python supports string literals and various numeric literals:

\begin{productionlist}
  \production{literal}
             {\token{stringliteral} | \token{integer} | \token{longinteger}}
  \productioncont{| \token{floatnumber} | \token{imagnumber}}
\end{productionlist}

Evaluation of a literal yields an object of the given type (string,
integer, long integer, floating point number, complex number) with the
given value.  The value may be approximated in the case of floating
point and imaginary (complex) literals.  See section \ref{literals}
for details.

All literals correspond to immutable data types, and hence the
object's identity is less important than its value.  Multiple
evaluations of literals with the same value (either the same
occurrence in the program text or a different occurrence) may obtain
the same object or a different object with the same value.
\indexiii{immutable}{data}{type}
\indexii{immutable}{object}


\subsection{Parenthesized forms\label{parenthesized}}
\index{parenthesized form}

A parenthesized form is an optional expression list enclosed in
parentheses:

\begin{productionlist}
  \production{parenth_form}
             {"(" [\token{expression_list}] ")"}
\end{productionlist}

A parenthesized expression list yields whatever that expression list
yields: if the list contains at least one comma, it yields a tuple;
otherwise, it yields the single expression that makes up the
expression list.

An empty pair of parentheses yields an empty tuple object.  Since
tuples are immutable, the rules for literals apply (i.e., two
occurrences of the empty tuple may or may not yield the same object).
\indexii{empty}{tuple}

Note that tuples are not formed by the parentheses, but rather by use
of the comma operator.  The exception is the empty tuple, for which
parentheses \emph{are} required --- allowing unparenthesized ``nothing''
in expressions would cause ambiguities and allow common typos to
pass uncaught.
\index{comma}
\indexii{tuple}{display}


\subsection{List displays\label{lists}}
\indexii{list}{display}
\indexii{list}{comprehensions}

A list display is a possibly empty series of expressions enclosed in
square brackets:

\begin{productionlist}
  \production{list_display}
             {"[" [\token{expression_list} | \token{list_comprehension}] "]"}
  \production{list_comprehension}
             {\token{expression} \token{list_for}}
  \production{list_for}
             {"for" \token{target_list} "in" \token{old_expression_list}
              [\token{list_iter}]}
  \production{old_expression_list}
             {\token{old_expression}
              [("," \token{old_expression})+ [","]]}
  \production{list_iter}
             {\token{list_for} | \token{list_if}}
  \production{list_if}
             {"if" \token{old_expression} [\token{list_iter}]}
\end{productionlist}

A list display yields a new list object.  Its contents are specified
by providing either a list of expressions or a list comprehension.
\indexii{list}{comprehensions}
When a comma-separated list of expressions is supplied, its elements are
evaluated from left to right and placed into the list object in that
order.  When a list comprehension is supplied, it consists of a
single expression followed by at least one \keyword{for} clause and zero or
more \keyword{for} or \keyword{if} clauses.  In this
case, the elements of the new list are those that would be produced
by considering each of the \keyword{for} or \keyword{if} clauses a block,
nesting from
left to right, and evaluating the expression to produce a list element
each time the innermost block is reached\footnote{In Python 2.3, a
list comprehension "leaks" the control variables of each
\samp{for} it contains into the containing scope.  However, this
behavior is deprecated, and relying on it will not work once this
bug is fixed in a future release}.
\obindex{list}
\indexii{empty}{list}


\subsection{Generator expressions\label{genexpr}}
\indexii{generator}{expression}

A generator expression is a compact generator notation in parentheses:

\begin{productionlist}
  \production{generator_expression}
             {"(" \token{expression} \token{genexpr_for} ")"}
  \production{genexpr_for}
             {"for" \token{target_list} "in" \token{or_test}
              [\token{genexpr_iter}]}
  \production{genexpr_iter}
             {\token{genexpr_for} | \token{genexpr_if}}
  \production{genexpr_if}
             {"if" \token{old_expression} [\token{genexpr_iter}]}
\end{productionlist}

A generator expression yields a new generator object.
\obindex{generator}
It consists of a single expression followed by at least one
\keyword{for} clause and zero or more \keyword{for} or \keyword{if}
clauses.  The iterating values of the new generator are those that
would be produced by considering each of the \keyword{for} or
\keyword{if} clauses a block, nesting from left to right, and
evaluating the expression to yield a value that is reached the
innermost block for each iteration.

Variables used in the generator expression are evaluated lazily
when the \method{next()} method is called for generator object
(in the same fashion as normal generators). However, the leftmost
\keyword{for} clause is immediately evaluated so that error produced
by it can be seen before any other possible error in the code that
handles the generator expression.
Subsequent \keyword{for} clauses cannot be evaluated immediately since
they may depend on the previous \keyword{for} loop.
For example: \samp{(x*y for x in range(10) for y in bar(x))}.

The parentheses can be omitted on calls with only one argument.
See section \ref{calls} for the detail.


\subsection{Dictionary displays\label{dict}}
\indexii{dictionary}{display}

A dictionary display is a possibly empty series of key/datum pairs
enclosed in curly braces:
\index{key}
\index{datum}
\index{key/datum pair}

\begin{productionlist}
  \production{dict_display}
             {"\{" [\token{key_datum_list}] "\}"}
  \production{key_datum_list}
             {\token{key_datum} ("," \token{key_datum})* [","]}
  \production{key_datum}
             {\token{expression} ":" \token{expression}}
\end{productionlist}

A dictionary display yields a new dictionary object.
\obindex{dictionary}

The key/datum pairs are evaluated from left to right to define the
entries of the dictionary: each key object is used as a key into the
dictionary to store the corresponding datum.

Restrictions on the types of the key values are listed earlier in
section \ref{types}.  (To summarize, the key type should be hashable,
which excludes all mutable objects.)  Clashes between duplicate keys
are not detected; the last datum (textually rightmost in the display)
stored for a given key value prevails.
\indexii{immutable}{object}


\subsection{String conversions\label{string-conversions}}
\indexii{string}{conversion}
\indexii{reverse}{quotes}
\indexii{backward}{quotes}
\index{back-quotes}

A string conversion is an expression list enclosed in reverse (a.k.a.
backward) quotes:

\begin{productionlist}
  \production{string_conversion}
             {"`" \token{expression_list} "`"}
\end{productionlist}

A string conversion evaluates the contained expression list and
converts the resulting object into a string according to rules
specific to its type.

If the object is a string, a number, \code{None}, or a tuple, list or
dictionary containing only objects whose type is one of these, the
resulting string is a valid Python expression which can be passed to
the built-in function \function{eval()} to yield an expression with the
same value (or an approximation, if floating point numbers are
involved).

(In particular, converting a string adds quotes around it and converts
``funny'' characters to escape sequences that are safe to print.)

Recursive objects (for example, lists or dictionaries that contain a
reference to themselves, directly or indirectly) use \samp{...} to
indicate a recursive reference, and the result cannot be passed to
\function{eval()} to get an equal value (\exception{SyntaxError} will
be raised instead).
\obindex{recursive}

The built-in function \function{repr()} performs exactly the same
conversion in its argument as enclosing it in parentheses and reverse
quotes does.  The built-in function \function{str()} performs a
similar but more user-friendly conversion.
\bifuncindex{repr}
\bifuncindex{str}


\subsection{Yield expressions\label{yieldexpr}}
\kwindex{yield}
\indexii{yield}{expression}
\indexii{generator}{function}

\begin{productionlist}
  \production{yield_atom}
             {"(" \token{yield_expression} ")"}
  \production{yield_expression}
             {"yield" [\token{expression_list}]}
\end{productionlist}

\versionadded{2.5}

The \keyword{yield} expression is only used when defining a generator
function, and can only be used in the body of a function definition.
Using a \keyword{yield} expression in a function definition is
sufficient to cause that definition to create a generator function
instead of a normal function.

When a generator function is called, it returns an iterator known as a
generator.  That generator then controls the execution of a generator
function.  The execution starts when one of the generator's methods is
called.  At that time, the execution proceeds to the first
\keyword{yield} expression, where it is suspended again, returning the
value of \grammartoken{expression_list} to generator's caller.  By
suspended we mean that all local state is retained, including the
current bindings of local variables, the instruction pointer, and the
internal evaluation stack.  When the execution is resumed by calling
one of the generator's methods, the function can proceed exactly as
if the \keyword{yield} expression was just another external call.
The value of the \keyword{yield} expression after resuming depends on
the method which resumed the execution.

\index{coroutine}

All of this makes generator functions quite similar to coroutines; they
yield multiple times, they have more than one entry point and their
execution can be suspended.  The only difference is that a generator
function cannot control where should the execution continue after it
yields; the control is always transfered to the generator's caller.

\obindex{generator}

The following generator's methods can be used to control the execution
of a generator function:

\exindex{StopIteration}

\begin{methoddesc}[generator]{next}{}
  Starts the execution of a generator function or resumes it at the
  last executed \keyword{yield} expression.  When a generator function
  is resumed with a \method{next()} method, the current \keyword{yield}
  expression always evaluates to \constant{None}.  The execution then
  continues to the next \keyword{yield} expression, where the generator
  is suspended again, and the value of the
  \grammartoken{expression_list} is returned to \method{next()}'s
  caller. If the generator exits without yielding another value, a
  \exception{StopIteration} exception is raised.
\end{methoddesc}

\begin{methoddesc}[generator]{send}{value}
  Resumes the execution and ``sends'' a value into the generator
  function.  The \code{value} argument becomes the result of the
  current \keyword{yield} expression.  The \method{send()} method
  returns the next value yielded by the generator, or raises
  \exception{StopIteration} if the generator exits without yielding
  another value.
  When \method{send()} is called to start the generator, it must be
  called with \constant{None} as the argument, because there is no
  \keyword{yield} expression that could receieve the value.
\end{methoddesc}

\begin{methoddesc}[generator]{throw}
                  {type\optional{, value\optional{, traceback}}}
  Raises an exception of type \code{type} at the point where generator
  was paused, and returns the next value yielded by the generator
  function.  If the generator exits without yielding another value, a
  \exception{StopIteration} exception is raised.  If the generator
  function does not catch the passed-in exception, or raises a
  different exception, then that exception propagates to the caller.
\end{methoddesc}

\exindex{GeneratorExit}

\begin{methoddesc}[generator]{close}{}
  Raises a \exception{GeneratorExit} at the point where the generator
  function was paused.  If the generator function then raises
  \exception{StopIteration} (by exiting normally, or due to already
  being closed) or \exception{GeneratorExit} (by not catching the
  exception), close returns to its caller.  If the generator yields a
  value, a \exception{RuntimeError} is raised.  If the generator raises
  any other exception, it is propagated to the caller.  \method{close}
  does nothing if the generator has already exited due to an exception
  or normal exit.
\end{methoddesc}

Here is a simple example that demonstrates the behavior of generators
and generator functions:

\begin{verbatim}
>>> def echo(value=None):
...     print "Execution starts when 'next()' is called for the first time."
...     try:
...         while True:
...             try:
...                 value = (yield value)
...             except GeneratorExit:
...                 # never catch GeneratorExit
...                 raise
...             except Exception, e:
...                 value = e
...     finally:
...         print "Don't forget to clean up when 'close()' is called."
...
>>> generator = echo(1)
>>> print generator.next()
Execution starts when 'next()' is called for the first time.
1
>>> print generator.next()
None
>>> print generator.send(2)
2
>>> generator.throw(TypeError, "spam")
TypeError('spam',)
>>> generator.close()
Don't forget to clean up when 'close()' is called.
\end{verbatim}

\begin{seealso}
  \seepep{0342}{Coroutines via Enhanced Generators}
         {The proposal to enhance the API and syntax of generators,
          making them usable as simple coroutines.}
\end{seealso}


\section{Primaries\label{primaries}}
\index{primary}

Primaries represent the most tightly bound operations of the language.
Their syntax is:

\begin{productionlist}
  \production{primary}
             {\token{atom} | \token{attributeref}
              | \token{subscription} | \token{slicing} | \token{call}}
\end{productionlist}


\subsection{Attribute references\label{attribute-references}}
\indexii{attribute}{reference}

An attribute reference is a primary followed by a period and a name:

\begin{productionlist}
  \production{attributeref}
             {\token{primary} "." \token{identifier}}
\end{productionlist}

The primary must evaluate to an object of a type that supports
attribute references, e.g., a module, list, or an instance.  This
object is then asked to produce the attribute whose name is the
identifier.  If this attribute is not available, the exception
\exception{AttributeError}\exindex{AttributeError} is raised.
Otherwise, the type and value of the object produced is determined by
the object.  Multiple evaluations of the same attribute reference may
yield different objects.
\obindex{module}
\obindex{list}


\subsection{Subscriptions\label{subscriptions}}
\index{subscription}

A subscription selects an item of a sequence (string, tuple or list)
or mapping (dictionary) object:
\obindex{sequence}
\obindex{mapping}
\obindex{string}
\obindex{tuple}
\obindex{list}
\obindex{dictionary}
\indexii{sequence}{item}

\begin{productionlist}
  \production{subscription}
             {\token{primary} "[" \token{expression_list} "]"}
\end{productionlist}

The primary must evaluate to an object of a sequence or mapping type.

If the primary is a mapping, the expression list must evaluate to an
object whose value is one of the keys of the mapping, and the
subscription selects the value in the mapping that corresponds to that
key.  (The expression list is a tuple except if it has exactly one
item.)

If the primary is a sequence, the expression (list) must evaluate to a
plain integer.  If this value is negative, the length of the sequence
is added to it (so that, e.g., \code{x[-1]} selects the last item of
\code{x}.)  The resulting value must be a nonnegative integer less
than the number of items in the sequence, and the subscription selects
the item whose index is that value (counting from zero).

A string's items are characters.  A character is not a separate data
type but a string of exactly one character.
\index{character}
\indexii{string}{item}


\subsection{Slicings\label{slicings}}
\index{slicing}
\index{slice}

A slicing selects a range of items in a sequence object (e.g., a
string, tuple or list).  Slicings may be used as expressions or as
targets in assignment or \keyword{del} statements.  The syntax for a
slicing:
\obindex{sequence}
\obindex{string}
\obindex{tuple}
\obindex{list}

\begin{productionlist}
  \production{slicing}
             {\token{simple_slicing} | \token{extended_slicing}}
  \production{simple_slicing}
             {\token{primary} "[" \token{short_slice} "]"}
  \production{extended_slicing}
             {\token{primary} "[" \token{slice_list} "]" }
  \production{slice_list}
             {\token{slice_item} ("," \token{slice_item})* [","]}
  \production{slice_item}
             {\token{expression} | \token{proper_slice} | \token{ellipsis}}
  \production{proper_slice}
             {\token{short_slice} | \token{long_slice}}
  \production{short_slice}
             {[\token{lower_bound}] ":" [\token{upper_bound}]}
  \production{long_slice}
             {\token{short_slice} ":" [\token{stride}]}
  \production{lower_bound}
             {\token{expression}}
  \production{upper_bound}
             {\token{expression}}
  \production{stride}
             {\token{expression}}
  \production{ellipsis}
             {"..."}
\end{productionlist}

There is ambiguity in the formal syntax here: anything that looks like
an expression list also looks like a slice list, so any subscription
can be interpreted as a slicing.  Rather than further complicating the
syntax, this is disambiguated by defining that in this case the
interpretation as a subscription takes priority over the
interpretation as a slicing (this is the case if the slice list
contains no proper slice nor ellipses).  Similarly, when the slice
list has exactly one short slice and no trailing comma, the
interpretation as a simple slicing takes priority over that as an
extended slicing.\indexii{extended}{slicing}

The semantics for a simple slicing are as follows.  The primary must
evaluate to a sequence object.  The lower and upper bound expressions,
if present, must evaluate to plain integers; defaults are zero and the
\code{sys.maxint}, respectively.  If either bound is negative, the
sequence's length is added to it.  The slicing now selects all items
with index \var{k} such that
\code{\var{i} <= \var{k} < \var{j}} where \var{i}
and \var{j} are the specified lower and upper bounds.  This may be an
empty sequence.  It is not an error if \var{i} or \var{j} lie outside the
range of valid indexes (such items don't exist so they aren't
selected).

The semantics for an extended slicing are as follows.  The primary
must evaluate to a mapping object, and it is indexed with a key that
is constructed from the slice list, as follows.  If the slice list
contains at least one comma, the key is a tuple containing the
conversion of the slice items; otherwise, the conversion of the lone
slice item is the key.  The conversion of a slice item that is an
expression is that expression.  The conversion of an ellipsis slice
item is the built-in \code{Ellipsis} object.  The conversion of a
proper slice is a slice object (see section \ref{types}) whose
\member{start}, \member{stop} and \member{step} attributes are the
values of the expressions given as lower bound, upper bound and
stride, respectively, substituting \code{None} for missing
expressions.
\withsubitem{(slice object attribute)}{\ttindex{start}
  \ttindex{stop}\ttindex{step}}


\subsection{Calls\label{calls}}
\index{call}

A call calls a callable object (e.g., a function) with a possibly empty
series of arguments:
\obindex{callable}

\begin{productionlist}
  \production{call}
             {\token{primary} "(" [\token{argument_list} [","]}
  \productioncont{            | \token{expression} \token{genexpr_for}] ")"}
  \production{argument_list}
             {\token{positional_arguments} ["," \token{keyword_arguments}]}
  \productioncont{                     ["," "*" \token{expression}]}
  \productioncont{                     ["," "**" \token{expression}]}
  \productioncont{| \token{keyword_arguments} ["," "*" \token{expression}]}
  \productioncont{                    ["," "**" \token{expression}]}
  \productioncont{| "*" \token{expression} ["," "**" \token{expression}]}
  \productioncont{| "**" \token{expression}}
  \production{positional_arguments}
             {\token{expression} ("," \token{expression})*}
  \production{keyword_arguments}
             {\token{keyword_item} ("," \token{keyword_item})*}
  \production{keyword_item}
             {\token{identifier} "=" \token{expression}}
\end{productionlist}

A trailing comma may be present after the positional and keyword
arguments but does not affect the semantics.

The primary must evaluate to a callable object (user-defined
functions, built-in functions, methods of built-in objects, class
objects, methods of class instances, and certain class instances
themselves are callable; extensions may define additional callable
object types).  All argument expressions are evaluated before the call
is attempted.  Please refer to section \ref{function} for the syntax
of formal parameter lists.

If keyword arguments are present, they are first converted to
positional arguments, as follows.  First, a list of unfilled slots is
created for the formal parameters.  If there are N positional
arguments, they are placed in the first N slots.  Next, for each
keyword argument, the identifier is used to determine the
corresponding slot (if the identifier is the same as the first formal
parameter name, the first slot is used, and so on).  If the slot is
already filled, a \exception{TypeError} exception is raised.
Otherwise, the value of the argument is placed in the slot, filling it
(even if the expression is \code{None}, it fills the slot).  When all
arguments have been processed, the slots that are still unfilled are
filled with the corresponding default value from the function
definition.  (Default values are calculated, once, when the function
is defined; thus, a mutable object such as a list or dictionary used
as default value will be shared by all calls that don't specify an
argument value for the corresponding slot; this should usually be
avoided.)  If there are any unfilled slots for which no default value
is specified, a \exception{TypeError} exception is raised.  Otherwise,
the list of filled slots is used as the argument list for the call.

If there are more positional arguments than there are formal parameter
slots, a \exception{TypeError} exception is raised, unless a formal
parameter using the syntax \samp{*identifier} is present; in this
case, that formal parameter receives a tuple containing the excess
positional arguments (or an empty tuple if there were no excess
positional arguments).

If any keyword argument does not correspond to a formal parameter
name, a \exception{TypeError} exception is raised, unless a formal
parameter using the syntax \samp{**identifier} is present; in this
case, that formal parameter receives a dictionary containing the
excess keyword arguments (using the keywords as keys and the argument
values as corresponding values), or a (new) empty dictionary if there
were no excess keyword arguments.

If the syntax \samp{*expression} appears in the function call,
\samp{expression} must evaluate to a sequence.  Elements from this
sequence are treated as if they were additional positional arguments;
if there are postional arguments \var{x1},...,\var{xN} , and
\samp{expression} evaluates to a sequence \var{y1},...,\var{yM}, this
is equivalent to a call with M+N positional arguments
\var{x1},...,\var{xN},\var{y1},...,\var{yM}.

A consequence of this is that although the \samp{*expression} syntax
appears \emph{after} any keyword arguments, it is processed
\emph{before} the keyword arguments (and the
\samp{**expression} argument, if any -- see below).  So:

\begin{verbatim}
>>> def f(a, b):
...  print a, b
...
>>> f(b=1, *(2,))
2 1
>>> f(a=1, *(2,))
Traceback (most recent call last):
  File "<stdin>", line 1, in ?
TypeError: f() got multiple values for keyword argument 'a'
>>> f(1, *(2,))
1 2
\end{verbatim}

It is unusual for both keyword arguments and the
\samp{*expression} syntax to be used in the same call, so in practice
this confusion does not arise.

If the syntax \samp{**expression} appears in the function call,
\samp{expression} must evaluate to a (subclass of) dictionary, the
contents of which are treated as additional keyword arguments.  In the
case of a keyword appearing in both \samp{expression} and as an
explicit keyword argument, a \exception{TypeError} exception is
raised.

Formal parameters using the syntax \samp{*identifier} or
\samp{**identifier} cannot be used as positional argument slots or
as keyword argument names.  Formal parameters using the syntax
\samp{(sublist)} cannot be used as keyword argument names; the
outermost sublist corresponds to a single unnamed argument slot, and
the argument value is assigned to the sublist using the usual tuple
assignment rules after all other parameter processing is done.

A call always returns some value, possibly \code{None}, unless it
raises an exception.  How this value is computed depends on the type
of the callable object.

If it is---

\begin{description}

\item[a user-defined function:] The code block for the function is
executed, passing it the argument list.  The first thing the code
block will do is bind the formal parameters to the arguments; this is
described in section \ref{function}.  When the code block executes a
\keyword{return} statement, this specifies the return value of the
function call.
\indexii{function}{call}
\indexiii{user-defined}{function}{call}
\obindex{user-defined function}
\obindex{function}

\item[a built-in function or method:] The result is up to the
interpreter; see the \citetitle[../lib/built-in-funcs.html]{Python
Library Reference} for the descriptions of built-in functions and
methods.
\indexii{function}{call}
\indexii{built-in function}{call}
\indexii{method}{call}
\indexii{built-in method}{call}
\obindex{built-in method}
\obindex{built-in function}
\obindex{method}
\obindex{function}

\item[a class object:] A new instance of that class is returned.
\obindex{class}
\indexii{class object}{call}

\item[a class instance method:] The corresponding user-defined
function is called, with an argument list that is one longer than the
argument list of the call: the instance becomes the first argument.
\obindex{class instance}
\obindex{instance}
\indexii{class instance}{call}

\item[a class instance:] The class must define a \method{__call__()}
method; the effect is then the same as if that method was called.
\indexii{instance}{call}
\withsubitem{(object method)}{\ttindex{__call__()}}

\end{description}


\section{The power operator\label{power}}

The power operator binds more tightly than unary operators on its
left; it binds less tightly than unary operators on its right.  The
syntax is:

\begin{productionlist}
  \production{power}
             {\token{primary} ["**" \token{u_expr}]}
\end{productionlist}

Thus, in an unparenthesized sequence of power and unary operators, the
operators are evaluated from right to left (this does not constrain
the evaluation order for the operands).

The power operator has the same semantics as the built-in
\function{pow()} function, when called with two arguments: it yields
its left argument raised to the power of its right argument.  The
numeric arguments are first converted to a common type.  The result
type is that of the arguments after coercion.

With mixed operand types, the coercion rules for binary arithmetic
operators apply. For int and long int operands, the result has the
same type as the operands (after coercion) unless the second argument
is negative; in that case, all arguments are converted to float and a
float result is delivered. For example, \code{10**2} returns \code{100},
but \code{10**-2} returns \code{0.01}. (This last feature was added in
Python 2.2. In Python 2.1 and before, if both arguments were of integer
types and the second argument was negative, an exception was raised).

Raising \code{0.0} to a negative power results in a
\exception{ZeroDivisionError}.  Raising a negative number to a
fractional power results in a \exception{ValueError}.


\section{Unary arithmetic operations \label{unary}}
\indexiii{unary}{arithmetic}{operation}
\indexiii{unary}{bit-wise}{operation}

All unary arithmetic (and bit-wise) operations have the same priority:

\begin{productionlist}
  \production{u_expr}
             {\token{power} | "-" \token{u_expr}
              | "+" \token{u_expr} | "{\~}" \token{u_expr}}
\end{productionlist}

The unary \code{-} (minus) operator yields the negation of its
numeric argument.
\index{negation}
\index{minus}

The unary \code{+} (plus) operator yields its numeric argument
unchanged.
\index{plus}

The unary \code{\~} (invert) operator yields the bit-wise inversion
of its plain or long integer argument.  The bit-wise inversion of
\code{x} is defined as \code{-(x+1)}.  It only applies to integral
numbers.
\index{inversion}

In all three cases, if the argument does not have the proper type,
a \exception{TypeError} exception is raised.
\exindex{TypeError}


\section{Binary arithmetic operations\label{binary}}
\indexiii{binary}{arithmetic}{operation}

The binary arithmetic operations have the conventional priority
levels.  Note that some of these operations also apply to certain
non-numeric types.  Apart from the power operator, there are only two
levels, one for multiplicative operators and one for additive
operators:

\begin{productionlist}
  \production{m_expr}
             {\token{u_expr} | \token{m_expr} "*" \token{u_expr}
              | \token{m_expr} "//" \token{u_expr}
              | \token{m_expr} "/" \token{u_expr}}
  \productioncont{| \token{m_expr} "\%" \token{u_expr}}
  \production{a_expr}
             {\token{m_expr} | \token{a_expr} "+" \token{m_expr}
              | \token{a_expr} "-" \token{m_expr}}
\end{productionlist}

The \code{*} (multiplication) operator yields the product of its
arguments.  The arguments must either both be numbers, or one argument
must be an integer (plain or long) and the other must be a sequence.
In the former case, the numbers are converted to a common type and
then multiplied together.  In the latter case, sequence repetition is
performed; a negative repetition factor yields an empty sequence.
\index{multiplication}

The \code{/} (division) and \code{//} (floor division) operators yield
the quotient of their arguments.  The numeric arguments are first
converted to a common type.  Plain or long integer division yields an
integer of the same type; the result is that of mathematical division
with the `floor' function applied to the result.  Division by zero
raises the
\exception{ZeroDivisionError} exception.
\exindex{ZeroDivisionError}
\index{division}

The \code{\%} (modulo) operator yields the remainder from the
division of the first argument by the second.  The numeric arguments
are first converted to a common type.  A zero right argument raises
the \exception{ZeroDivisionError} exception.  The arguments may be floating
point numbers, e.g., \code{3.14\%0.7} equals \code{0.34} (since
\code{3.14} equals \code{4*0.7 + 0.34}.)  The modulo operator always
yields a result with the same sign as its second operand (or zero);
the absolute value of the result is strictly smaller than the absolute
value of the second operand\footnote{
    While \code{abs(x\%y) < abs(y)} is true mathematically, for
    floats it may not be true numerically due to roundoff.  For
    example, and assuming a platform on which a Python float is an
    IEEE 754 double-precision number, in order that \code{-1e-100 \% 1e100}
    have the same sign as \code{1e100}, the computed result is
    \code{-1e-100 + 1e100}, which is numerically exactly equal
    to \code{1e100}.  Function \function{fmod()} in the \module{math}
    module returns a result whose sign matches the sign of the
    first argument instead, and so returns \code{-1e-100} in this case.
    Which approach is more appropriate depends on the application.
}.
\index{modulo}

The integer division and modulo operators are connected by the
following identity: \code{x == (x/y)*y + (x\%y)}.  Integer division and
modulo are also connected with the built-in function \function{divmod()}:
\code{divmod(x, y) == (x/y, x\%y)}.  These identities don't hold for
floating point numbers; there similar identities hold
approximately where \code{x/y} is replaced by \code{floor(x/y)} or
\code{floor(x/y) - 1}\footnote{
    If x is very close to an exact integer multiple of y, it's
    possible for \code{floor(x/y)} to be one larger than
    \code{(x-x\%y)/y} due to rounding.  In such cases, Python returns
    the latter result, in order to preserve that \code{divmod(x,y)[0]
    * y + x \%{} y} be very close to \code{x}.
}.

In addition to performing the modulo operation on numbers, the \code{\%}
operator is also overloaded by string and unicode objects to perform
string formatting (also known as interpolation). The syntax for string
formatting is described in the
\citetitle[../lib/typesseq-strings.html]{Python Library Reference},
section ``Sequence Types''.

\deprecated{2.3}{The floor division operator, the modulo operator,
and the \function{divmod()} function are no longer defined for complex
numbers.  Instead, convert to a floating point number using the
\function{abs()} function if appropriate.}

The \code{+} (addition) operator yields the sum of its arguments.
The arguments must either both be numbers or both sequences of the
same type.  In the former case, the numbers are converted to a common
type and then added together.  In the latter case, the sequences are
concatenated.
\index{addition}

The \code{-} (subtraction) operator yields the difference of its
arguments.  The numeric arguments are first converted to a common
type.
\index{subtraction}


\section{Shifting operations\label{shifting}}
\indexii{shifting}{operation}

The shifting operations have lower priority than the arithmetic
operations:

\begin{productionlist}
  \production{shift_expr}
             {\token{a_expr}
              | \token{shift_expr} ( "<<" | ">>" ) \token{a_expr}}
\end{productionlist}

These operators accept plain or long integers as arguments.  The
arguments are converted to a common type.  They shift the first
argument to the left or right by the number of bits given by the
second argument.

A right shift by \var{n} bits is defined as division by
\code{pow(2,\var{n})}.  A left shift by \var{n} bits is defined as
multiplication with \code{pow(2,\var{n})}; for plain integers there is
no overflow check so in that case the operation drops bits and flips
the sign if the result is not less than \code{pow(2,31)} in absolute
value.  Negative shift counts raise a \exception{ValueError}
exception.
\exindex{ValueError}


\section{Binary bit-wise operations\label{bitwise}}
\indexiii{binary}{bit-wise}{operation}

Each of the three bitwise operations has a different priority level:

\begin{productionlist}
  \production{and_expr}
             {\token{shift_expr} | \token{and_expr} "\&" \token{shift_expr}}
  \production{xor_expr}
             {\token{and_expr} | \token{xor_expr} "\textasciicircum" \token{and_expr}}
  \production{or_expr}
             {\token{xor_expr} | \token{or_expr} "|" \token{xor_expr}}
\end{productionlist}

The \code{\&} operator yields the bitwise AND of its arguments, which
must be plain or long integers.  The arguments are converted to a
common type.
\indexii{bit-wise}{and}

The \code{\^} operator yields the bitwise XOR (exclusive OR) of its
arguments, which must be plain or long integers.  The arguments are
converted to a common type.
\indexii{bit-wise}{xor}
\indexii{exclusive}{or}

The \code{|} operator yields the bitwise (inclusive) OR of its
arguments, which must be plain or long integers.  The arguments are
converted to a common type.
\indexii{bit-wise}{or}
\indexii{inclusive}{or}


\section{Comparisons\label{comparisons}}
\index{comparison}

Unlike C, all comparison operations in Python have the same priority,
which is lower than that of any arithmetic, shifting or bitwise
operation.  Also unlike C, expressions like \code{a < b < c} have the
interpretation that is conventional in mathematics:
\indexii{C}{language}

\begin{productionlist}
  \production{comparison}
             {\token{or_expr} ( \token{comp_operator} \token{or_expr} )*}
  \production{comp_operator}
             {"<" | ">" | "==" | ">=" | "<=" | "<>" | "!="}
  \productioncont{| "is" ["not"] | ["not"] "in"}
\end{productionlist}

Comparisons yield boolean values: \code{True} or \code{False}.

Comparisons can be chained arbitrarily, e.g., \code{x < y <= z} is
equivalent to \code{x < y and y <= z}, except that \code{y} is
evaluated only once (but in both cases \code{z} is not evaluated at all
when \code{x < y} is found to be false).
\indexii{chaining}{comparisons}

Formally, if \var{a}, \var{b}, \var{c}, \ldots, \var{y}, \var{z} are
expressions and \var{opa}, \var{opb}, \ldots, \var{opy} are comparison
operators, then \var{a opa b opb c} \ldots \var{y opy z} is equivalent
to \var{a opa b} \keyword{and} \var{b opb c} \keyword{and} \ldots
\var{y opy z}, except that each expression is evaluated at most once.

Note that \var{a opa b opb c} doesn't imply any kind of comparison
between \var{a} and \var{c}, so that, e.g., \code{x < y > z} is
perfectly legal (though perhaps not pretty).

The forms \code{<>} and \code{!=} are equivalent; for consistency with
C, \code{!=} is preferred; where \code{!=} is mentioned below
\code{<>} is also accepted.  The \code{<>} spelling is considered
obsolescent.

The operators \code{<}, \code{>}, \code{==}, \code{>=}, \code{<=}, and
\code{!=} compare
the values of two objects.  The objects need not have the same type.
If both are numbers, they are converted to a common type.  Otherwise,
objects of different types \emph{always} compare unequal, and are
ordered consistently but arbitrarily.  You can control comparison
behavior of objects of non-builtin types by defining a \code{__cmp__}
method or rich comparison methods like \code{__gt__}, described in
section~\ref{specialnames}.

(This unusual definition of comparison was used to simplify the
definition of operations like sorting and the \keyword{in} and
\keyword{not in} operators.  In the future, the comparison rules for
objects of different types are likely to change.)

Comparison of objects of the same type depends on the type:

\begin{itemize}

\item
Numbers are compared arithmetically.

\item
Strings are compared lexicographically using the numeric equivalents
(the result of the built-in function \function{ord()}) of their
characters.  Unicode and 8-bit strings are fully interoperable in this
behavior.

\item
Tuples and lists are compared lexicographically using comparison of
corresponding elements.  This means that to compare equal, each
element must compare equal and the two sequences must be of the same
type and have the same length.

If not equal, the sequences are ordered the same as their first
differing elements.  For example, \code{cmp([1,2,x], [1,2,y])} returns
the same as \code{cmp(x,y)}.  If the corresponding element does not
exist, the shorter sequence is ordered first (for example,
\code{[1,2] < [1,2,3]}).

\item
Mappings (dictionaries) compare equal if and only if their sorted
(key, value) lists compare equal.\footnote{The implementation computes
   this efficiently, without constructing lists or sorting.}
Outcomes other than equality are resolved consistently, but are not
otherwise defined.\footnote{Earlier versions of Python used
  lexicographic comparison of the sorted (key, value) lists, but this
  was very expensive for the common case of comparing for equality.  An
  even earlier version of Python compared dictionaries by identity only,
  but this caused surprises because people expected to be able to test
  a dictionary for emptiness by comparing it to \code{\{\}}.}

\item
Most other objects of builtin types compare unequal unless they are
the same object;
the choice whether one object is considered smaller or larger than
another one is made arbitrarily but consistently within one
execution of a program.

\end{itemize}

The operators \keyword{in} and \keyword{not in} test for set
membership.  \code{\var{x} in \var{s}} evaluates to true if \var{x}
is a member of the set \var{s}, and false otherwise.  \code{\var{x}
not in \var{s}} returns the negation of \code{\var{x} in \var{s}}.
The set membership test has traditionally been bound to sequences; an
object is a member of a set if the set is a sequence and contains an
element equal to that object.  However, it is possible for an object
to support membership tests without being a sequence.  In particular,
dictionaries support membership testing as a nicer way of spelling
\code{\var{key} in \var{dict}}; other mapping types may follow suit.

For the list and tuple types, \code{\var{x} in \var{y}} is true if and
only if there exists an index \var{i} such that
\code{\var{x} == \var{y}[\var{i}]} is true.

For the Unicode and string types, \code{\var{x} in \var{y}} is true if
and only if \var{x} is a substring of \var{y}.  An equivalent test is
\code{y.find(x) != -1}.  Note, \var{x} and \var{y} need not be the
same type; consequently, \code{u'ab' in 'abc'} will return \code{True}.
Empty strings are always considered to be a substring of any other string,
so \code{"" in "abc"} will return \code{True}.
\versionchanged[Previously, \var{x} was required to be a string of
length \code{1}]{2.3}

For user-defined classes which define the \method{__contains__()} method,
\code{\var{x} in \var{y}} is true if and only if
\code{\var{y}.__contains__(\var{x})} is true.

For user-defined classes which do not define \method{__contains__()} and
do define \method{__getitem__()}, \code{\var{x} in \var{y}} is true if
and only if there is a non-negative integer index \var{i} such that
\code{\var{x} == \var{y}[\var{i}]}, and all lower integer indices
do not raise \exception{IndexError} exception. (If any other exception
is raised, it is as if \keyword{in} raised that exception).

The operator \keyword{not in} is defined to have the inverse true value
of \keyword{in}.
\opindex{in}
\opindex{not in}
\indexii{membership}{test}
\obindex{sequence}

The operators \keyword{is} and \keyword{is not} test for object identity:
\code{\var{x} is \var{y}} is true if and only if \var{x} and \var{y}
are the same object.  \code{\var{x} is not \var{y}} yields the inverse
truth value.
\opindex{is}
\opindex{is not}
\indexii{identity}{test}


\section{Boolean operations\label{Booleans}}
\indexii{Conditional}{expression}
\indexii{Boolean}{operation}

Boolean operations have the lowest priority of all Python operations:

\begin{productionlist}
  \production{expression}
             {\token{conditional_expression} | \token{lambda_form}}
  \production{old_expression}
             {\token{or_test} | \token{old_lambda_form}}
  \production{conditional_expression}
             {\token{or_test} ["if" \token{or_test} "else" \token{expression}]}
  \production{or_test}
             {\token{and_test} | \token{or_test} "or" \token{and_test}}
  \production{and_test}
             {\token{not_test} | \token{and_test} "and" \token{not_test}}
  \production{not_test}
             {\token{comparison} | "not" \token{not_test}}
\end{productionlist}

In the context of Boolean operations, and also when expressions are
used by control flow statements, the following values are interpreted
as false: \code{False}, \code{None}, numeric zero of all types, and empty
strings and containers (including strings, tuples, lists, dictionaries,
sets and frozensets).  All other values are interpreted as true.

The operator \keyword{not} yields \code{True} if its argument is false,
\code{False} otherwise.
\opindex{not}

The expression \code{\var{x} if \var{C} else \var{y}} first evaluates
\var{C} (\emph{not} \var{x}); if \var{C} is true, \var{x} is evaluated and
its value is returned; otherwise, \var{y} is evaluated and its value is
returned.  \versionadded{2.5}

The expression \code{\var{x} and \var{y}} first evaluates \var{x}; if
\var{x} is false, its value is returned; otherwise, \var{y} is
evaluated and the resulting value is returned.
\opindex{and}

The expression \code{\var{x} or \var{y}} first evaluates \var{x}; if
\var{x} is true, its value is returned; otherwise, \var{y} is
evaluated and the resulting value is returned.
\opindex{or}

(Note that neither \keyword{and} nor \keyword{or} restrict the value
and type they return to \code{False} and \code{True}, but rather return the
last evaluated argument.
This is sometimes useful, e.g., if \code{s} is a string that should be
replaced by a default value if it is empty, the expression
\code{s or 'foo'} yields the desired value.  Because \keyword{not} has to
invent a value anyway, it does not bother to return a value of the
same type as its argument, so e.g., \code{not 'foo'} yields \code{False},
not \code{''}.)

\section{Lambdas\label{lambdas}}
\indexii{lambda}{expression}
\indexii{lambda}{form}
\indexii{anonymous}{function}

\begin{productionlist}
  \production{lambda_form}
             {"lambda" [\token{parameter_list}]: \token{expression}}
  \production{old_lambda_form}
             {"lambda" [\token{parameter_list}]: \token{old_expression}}
\end{productionlist}

Lambda forms (lambda expressions) have the same syntactic position as
expressions.  They are a shorthand to create anonymous functions; the
expression \code{lambda \var{arguments}: \var{expression}}
yields a function object.  The unnamed object behaves like a function
object defined with

\begin{verbatim}
def name(arguments):
    return expression
\end{verbatim}

See section \ref{function} for the syntax of parameter lists.  Note
that functions created with lambda forms cannot contain statements.
\label{lambda}

\section{Expression lists\label{exprlists}}
\indexii{expression}{list}

\begin{productionlist}
  \production{expression_list}
             {\token{expression} ( "," \token{expression} )* [","]}
\end{productionlist}

An expression list containing at least one comma yields a
tuple.  The length of the tuple is the number of expressions in the
list.  The expressions are evaluated from left to right.
\obindex{tuple}

The trailing comma is required only to create a single tuple (a.k.a. a
\emph{singleton}); it is optional in all other cases.  A single
expression without a trailing comma doesn't create a
tuple, but rather yields the value of that expression.
(To create an empty tuple, use an empty pair of parentheses:
\code{()}.)
\indexii{trailing}{comma}

\section{Evaluation order\label{evalorder}}
\indexii{evaluation}{order}

Python evaluates expressions from left to right. Notice that while
evaluating an assignment, the right-hand side is evaluated before
the left-hand side.

In the following lines, expressions will be evaluated in the
arithmetic order of their suffixes:

\begin{verbatim}
expr1, expr2, expr3, expr4
(expr1, expr2, expr3, expr4)
{expr1: expr2, expr3: expr4}
expr1 + expr2 * (expr3 - expr4)
func(expr1, expr2, *expr3, **expr4)
expr3, expr4 = expr1, expr2
\end{verbatim}

\section{Summary\label{summary}}

The following table summarizes the operator
precedences\indexii{operator}{precedence} in Python, from lowest
precedence (least binding) to highest precedence (most binding).
Operators in the same box have the same precedence.  Unless the syntax
is explicitly given, operators are binary.  Operators in the same box
group left to right (except for comparisons, including tests, which all
have the same precedence and chain from left to right --- see section
\ref{comparisons} -- and exponentiation, which groups from right to left).

\begin{tableii}{c|l}{textrm}{Operator}{Description}
    \lineii{\keyword{lambda}}			{Lambda expression}
  \hline
    \lineii{\keyword{or}}			{Boolean OR}
  \hline
    \lineii{\keyword{and}}			{Boolean AND}
  \hline
    \lineii{\keyword{not} \var{x}}		{Boolean NOT}
  \hline
    \lineii{\keyword{in}, \keyword{not} \keyword{in}}{Membership tests}
    \lineii{\keyword{is}, \keyword{is not}}{Identity tests}
    \lineii{\code{<}, \code{<=}, \code{>}, \code{>=},
            \code{<>}, \code{!=}, \code{==}}
	   {Comparisons}
  \hline
    \lineii{\code{|}}				{Bitwise OR}
  \hline
    \lineii{\code{\^}}				{Bitwise XOR}
  \hline
    \lineii{\code{\&}}				{Bitwise AND}
  \hline
    \lineii{\code{<<}, \code{>>}}		{Shifts}
  \hline
    \lineii{\code{+}, \code{-}}{Addition and subtraction}
  \hline
    \lineii{\code{*}, \code{/}, \code{\%}}
           {Multiplication, division, remainder}
  \hline
    \lineii{\code{+\var{x}}, \code{-\var{x}}}	{Positive, negative}
    \lineii{\code{\~\var{x}}}			{Bitwise not}
  \hline
    \lineii{\code{**}}				{Exponentiation}
  \hline
    \lineii{\code{\var{x}.\var{attribute}}}	{Attribute reference}
    \lineii{\code{\var{x}[\var{index}]}}	{Subscription}
    \lineii{\code{\var{x}[\var{index}:\var{index}]}}	{Slicing}
    \lineii{\code{\var{f}(\var{arguments}...)}}	{Function call}
  \hline
    \lineii{\code{(\var{expressions}\ldots)}}	{Binding or tuple display}
    \lineii{\code{[\var{expressions}\ldots]}}	{List display}
    \lineii{\code{\{\var{key}:\var{datum}\ldots\}}}{Dictionary display}
    \lineii{\code{`\var{expressions}\ldots`}}	{String conversion}
\end{tableii}
		% Expressions and conditions
\chapter{Simple statements \label{simple}}
\indexii{simple}{statement}

Simple statements are comprised within a single logical line.
Several simple statements may occur on a single line separated
by semicolons.  The syntax for simple statements is:

\begin{productionlist}
  \production{simple_stmt}{\token{expression_stmt}}
  \productioncont{| \token{assert_stmt}}
  \productioncont{| \token{assignment_stmt}}
  \productioncont{| \token{augmented_assignment_stmt}}
  \productioncont{| \token{pass_stmt}}
  \productioncont{| \token{del_stmt}}
  \productioncont{| \token{print_stmt}}
  \productioncont{| \token{return_stmt}}
  \productioncont{| \token{yield_stmt}}
  \productioncont{| \token{raise_stmt}}
  \productioncont{| \token{break_stmt}}
  \productioncont{| \token{continue_stmt}}
  \productioncont{| \token{import_stmt}}
  \productioncont{| \token{global_stmt}}
  \productioncont{| \token{exec_stmt}}
\end{productionlist}


\section{Expression statements \label{exprstmts}}
\indexii{expression}{statement}

Expression statements are used (mostly interactively) to compute and
write a value, or (usually) to call a procedure (a function that
returns no meaningful result; in Python, procedures return the value
\code{None}).  Other uses of expression statements are allowed and
occasionally useful.  The syntax for an expression statement is:

\begin{productionlist}
  \production{expression_stmt}
             {\token{expression_list}}
\end{productionlist}

An expression statement evaluates the expression list (which may be a
single expression).
\indexii{expression}{list}

In interactive mode, if the value is not \code{None}, it is converted
to a string using the built-in \function{repr()}\bifuncindex{repr}
function and the resulting string is written to standard output (see
section~\ref{print}) on a line by itself.  (Expression statements
yielding \code{None} are not written, so that procedure calls do not
cause any output.)
\obindex{None}
\indexii{string}{conversion}
\index{output}
\indexii{standard}{output}
\indexii{writing}{values}
\indexii{procedure}{call}


\section{Assert statements \label{assert}}

Assert statements\stindex{assert} are a convenient way to insert
debugging assertions\indexii{debugging}{assertions} into a program:

\begin{productionlist}
  \production{assert_stmt}
             {"assert" \token{expression} ["," \token{expression}]}
\end{productionlist}

The simple form, \samp{assert expression}, is equivalent to

\begin{verbatim}
if __debug__:
   if not expression: raise AssertionError
\end{verbatim}

The extended form, \samp{assert expression1, expression2}, is
equivalent to

\begin{verbatim}
if __debug__:
   if not expression1: raise AssertionError, expression2
\end{verbatim}

These equivalences assume that \code{__debug__}\ttindex{__debug__} and
\exception{AssertionError}\exindex{AssertionError} refer to the built-in
variables with those names.  In the current implementation, the
built-in variable \code{__debug__} is \code{True} under normal
circumstances, \code{False} when optimization is requested (command line
option -O).  The current code generator emits no code for an assert
statement when optimization is requested at compile time.  Note that it
is unnecessary to include the source code for the expression that failed
in the error message;
it will be displayed as part of the stack trace.

Assignments to \code{__debug__} are illegal.  The value for the
built-in variable is determined when the interpreter starts.


\section{Assignment statements \label{assignment}}

Assignment statements\indexii{assignment}{statement} are used to
(re)bind names to values and to modify attributes or items of mutable
objects:
\indexii{binding}{name}
\indexii{rebinding}{name}
\obindex{mutable}
\indexii{attribute}{assignment}

\begin{productionlist}
  \production{assignment_stmt}
             {(\token{target_list} "=")+
              (\token{expression_list} | \token{yield_expression})}
  \production{target_list}
             {\token{target} ("," \token{target})* [","]}
  \production{target}
             {\token{identifier}}
  \productioncont{| "(" \token{target_list} ")"}
  \productioncont{| "[" \token{target_list} "]"}
  \productioncont{| \token{attributeref}}
  \productioncont{| \token{subscription}}
  \productioncont{| \token{slicing}}
\end{productionlist}

(See section~\ref{primaries} for the syntax definitions for the last
three symbols.)

An assignment statement evaluates the expression list (remember that
this can be a single expression or a comma-separated list, the latter
yielding a tuple) and assigns the single resulting object to each of
the target lists, from left to right.
\indexii{expression}{list}

Assignment is defined recursively depending on the form of the target
(list).  When a target is part of a mutable object (an attribute
reference, subscription or slicing), the mutable object must
ultimately perform the assignment and decide about its validity, and
may raise an exception if the assignment is unacceptable.  The rules
observed by various types and the exceptions raised are given with the
definition of the object types (see section~\ref{types}).
\index{target}
\indexii{target}{list}

Assignment of an object to a target list is recursively defined as
follows.
\indexiii{target}{list}{assignment}

\begin{itemize}
\item
If the target list is a single target: The object is assigned to that
target.

\item
If the target list is a comma-separated list of targets: The object
must be a sequence with the same number of items as there are
targets in the target list, and the items are assigned, from left to
right, to the corresponding targets.  (This rule is relaxed as of
Python 1.5; in earlier versions, the object had to be a tuple.  Since
strings are sequences, an assignment like \samp{a, b = "xy"} is
now legal as long as the string has the right length.)

\end{itemize}

Assignment of an object to a single target is recursively defined as
follows.

\begin{itemize} % nested

\item
If the target is an identifier (name):

\begin{itemize}

\item
If the name does not occur in a \keyword{global} statement in the current
code block: the name is bound to the object in the current local
namespace.
\stindex{global}

\item
Otherwise: the name is bound to the object in the current global
namespace.

\end{itemize} % nested

The name is rebound if it was already bound.  This may cause the
reference count for the object previously bound to the name to reach
zero, causing the object to be deallocated and its
destructor\index{destructor} (if it has one) to be called.

\item
If the target is a target list enclosed in parentheses or in square
brackets: The object must be a sequence with the same number of items
as there are targets in the target list, and its items are assigned,
from left to right, to the corresponding targets.

\item
If the target is an attribute reference: The primary expression in the
reference is evaluated.  It should yield an object with assignable
attributes; if this is not the case, \exception{TypeError} is raised.  That
object is then asked to assign the assigned object to the given
attribute; if it cannot perform the assignment, it raises an exception
(usually but not necessarily \exception{AttributeError}).
\indexii{attribute}{assignment}

\item
If the target is a subscription: The primary expression in the
reference is evaluated.  It should yield either a mutable sequence
object (such as a list) or a mapping object (such as a dictionary). Next,
the subscript expression is evaluated.
\indexii{subscription}{assignment}
\obindex{mutable}

If the primary is a mutable sequence object (such as a list), the subscript
must yield a plain integer.  If it is negative, the sequence's length
is added to it.  The resulting value must be a nonnegative integer
less than the sequence's length, and the sequence is asked to assign
the assigned object to its item with that index.  If the index is out
of range, \exception{IndexError} is raised (assignment to a subscripted
sequence cannot add new items to a list).
\obindex{sequence}
\obindex{list}

If the primary is a mapping object (such as a dictionary), the subscript must
have a type compatible with the mapping's key type, and the mapping is
then asked to create a key/datum pair which maps the subscript to
the assigned object.  This can either replace an existing key/value
pair with the same key value, or insert a new key/value pair (if no
key with the same value existed).
\obindex{mapping}
\obindex{dictionary}

\item
If the target is a slicing: The primary expression in the reference is
evaluated.  It should yield a mutable sequence object (such as a list).  The
assigned object should be a sequence object of the same type.  Next,
the lower and upper bound expressions are evaluated, insofar they are
present; defaults are zero and the sequence's length.  The bounds
should evaluate to (small) integers.  If either bound is negative, the
sequence's length is added to it.  The resulting bounds are clipped to
lie between zero and the sequence's length, inclusive.  Finally, the
sequence object is asked to replace the slice with the items of the
assigned sequence.  The length of the slice may be different from the
length of the assigned sequence, thus changing the length of the
target sequence, if the object allows it.
\indexii{slicing}{assignment}

\end{itemize}
        
(In the current implementation, the syntax for targets is taken
to be the same as for expressions, and invalid syntax is rejected
during the code generation phase, causing less detailed error
messages.)

WARNING: Although the definition of assignment implies that overlaps
between the left-hand side and the right-hand side are `safe' (for example
\samp{a, b = b, a} swaps two variables), overlaps \emph{within} the
collection of assigned-to variables are not safe!  For instance, the
following program prints \samp{[0, 2]}:

\begin{verbatim}
x = [0, 1]
i = 0
i, x[i] = 1, 2
print x
\end{verbatim}


\subsection{Augmented assignment statements \label{augassign}}

Augmented assignment is the combination, in a single statement, of a binary
operation and an assignment statement:
\indexii{augmented}{assignment}
\index{statement!assignment, augmented}

\begin{productionlist}
  \production{augmented_assignment_stmt}
             {\token{target} \token{augop}
              (\token{expression_list} | \token{yield_expression})}
  \production{augop}
             {"+=" | "-=" | "*=" | "/=" | "\%=" | "**="}
  \productioncont{| ">>=" | "<<=" | "\&=" | "\textasciicircum=" | "|="}
\end{productionlist}

(See section~\ref{primaries} for the syntax definitions for the last
three symbols.)

An augmented assignment evaluates the target (which, unlike normal
assignment statements, cannot be an unpacking) and the expression
list, performs the binary operation specific to the type of assignment
on the two operands, and assigns the result to the original
target.  The target is only evaluated once.

An augmented assignment expression like \code{x += 1} can be rewritten as
\code{x = x + 1} to achieve a similar, but not exactly equal effect. In the
augmented version, \code{x} is only evaluated once. Also, when possible, the
actual operation is performed \emph{in-place}, meaning that rather than
creating a new object and assigning that to the target, the old object is
modified instead.

With the exception of assigning to tuples and multiple targets in a single
statement, the assignment done by augmented assignment statements is handled
the same way as normal assignments. Similarly, with the exception of the
possible \emph{in-place} behavior, the binary operation performed by
augmented assignment is the same as the normal binary operations.

For targets which are attribute references, the initial value is
retrieved with a \method{getattr()} and the result is assigned with a
\method{setattr()}.  Notice that the two methods do not necessarily
refer to the same variable.  When \method{getattr()} refers to a class
variable, \method{setattr()} still writes to an instance variable.
For example:

\begin{verbatim}
class A:
    x = 3    # class variable
a = A()
a.x += 1     # writes a.x as 4 leaving A.x as 3
\end{verbatim}


\section{The \keyword{pass} statement \label{pass}}
\stindex{pass}

\begin{productionlist}
  \production{pass_stmt}
             {"pass"}
\end{productionlist}

\keyword{pass} is a null operation --- when it is executed, nothing
happens.  It is useful as a placeholder when a statement is
required syntactically, but no code needs to be executed, for example:
\indexii{null}{operation}

\begin{verbatim}
def f(arg): pass    # a function that does nothing (yet)

class C: pass       # a class with no methods (yet)
\end{verbatim}


\section{The \keyword{del} statement \label{del}}
\stindex{del}

\begin{productionlist}
  \production{del_stmt}
             {"del" \token{target_list}}
\end{productionlist}

Deletion is recursively defined very similar to the way assignment is
defined. Rather that spelling it out in full details, here are some
hints.
\indexii{deletion}{target}
\indexiii{deletion}{target}{list}

Deletion of a target list recursively deletes each target, from left
to right.

Deletion of a name removes the binding of that name 
from the local or global namespace, depending on whether the name
occurs in a \keyword{global} statement in the same code block.  If the
name is unbound, a \exception{NameError} exception will be raised.
\stindex{global}
\indexii{unbinding}{name}

It is illegal to delete a name from the local namespace if it occurs
as a free variable\indexii{free}{variable} in a nested block.

Deletion of attribute references, subscriptions and slicings
is passed to the primary object involved; deletion of a slicing
is in general equivalent to assignment of an empty slice of the
right type (but even this is determined by the sliced object).
\indexii{attribute}{deletion}


\section{The \keyword{print} statement \label{print}}
\stindex{print}

\begin{productionlist}
  \production{print_stmt}
             {"print" ([\token{expression} ("," \token{expression})* [","]}
  \productioncont{| ">>" \token{expression}
                  [("," \token{expression})+ [","])}
\end{productionlist}

\keyword{print} evaluates each expression in turn and writes the
resulting object to standard output (see below).  If an object is not
a string, it is first converted to a string using the rules for string
conversions.  The (resulting or original) string is then written.  A
space is written before each object is (converted and) written, unless
the output system believes it is positioned at the beginning of a
line.  This is the case (1) when no characters have yet been written
to standard output, (2) when the last character written to standard
output is \character{\e n}, or (3) when the last write operation on
standard output was not a \keyword{print} statement.  (In some cases
it may be functional to write an empty string to standard output for
this reason.)  \note{Objects which act like file objects but which are
not the built-in file objects often do not properly emulate this
aspect of the file object's behavior, so it is best not to rely on
this.}
\index{output}
\indexii{writing}{values}

A \character{\e n} character is written at the end, unless the
\keyword{print} statement ends with a comma.  This is the only action
if the statement contains just the keyword \keyword{print}.
\indexii{trailing}{comma}
\indexii{newline}{suppression}

Standard output is defined as the file object named \code{stdout}
in the built-in module \module{sys}.  If no such object exists, or if
it does not have a \method{write()} method, a \exception{RuntimeError}
exception is raised.
\indexii{standard}{output}
\refbimodindex{sys}
\withsubitem{(in module sys)}{\ttindex{stdout}}
\exindex{RuntimeError}

\keyword{print} also has an extended\index{extended print statement}
form, defined by the second portion of the syntax described above.
This form is sometimes referred to as ``\keyword{print} chevron.''
In this form, the first expression after the \code{>>} must
evaluate to a ``file-like'' object, specifically an object that has a
\method{write()} method as described above.  With this extended form,
the subsequent expressions are printed to this file object.  If the
first expression evaluates to \code{None}, then \code{sys.stdout} is
used as the file for output.


\section{The \keyword{return} statement \label{return}}
\stindex{return}

\begin{productionlist}
  \production{return_stmt}
             {"return" [\token{expression_list}]}
\end{productionlist}

\keyword{return} may only occur syntactically nested in a function
definition, not within a nested class definition.
\indexii{function}{definition}
\indexii{class}{definition}

If an expression list is present, it is evaluated, else \code{None}
is substituted.

\keyword{return} leaves the current function call with the expression
list (or \code{None}) as return value.

When \keyword{return} passes control out of a \keyword{try} statement
with a \keyword{finally} clause, that \keyword{finally} clause is executed
before really leaving the function.
\kwindex{finally}

In a generator function, the \keyword{return} statement is not allowed
to include an \grammartoken{expression_list}.  In that context, a bare
\keyword{return} indicates that the generator is done and will cause
\exception{StopIteration} to be raised.


\section{The \keyword{yield} statement \label{yield}}
\stindex{yield}

\begin{productionlist}
  \production{yield_stmt}
             {\token{yield_expression}}
\end{productionlist}

\index{generator!function}
\index{generator!iterator}
\index{function!generator}
\exindex{StopIteration}

The \keyword{yield} statement is only used when defining a generator
function, and is only used in the body of the generator function.
Using a \keyword{yield} statement in a function definition is
sufficient to cause that definition to create a generator function
instead of a normal function.

When a generator function is called, it returns an iterator known as a
generator iterator, or more commonly, a generator.  The body of the
generator function is executed by calling the generator's
\method{next()} method repeatedly until it raises an exception.

When a \keyword{yield} statement is executed, the state of the
generator is frozen and the value of \grammartoken{expression_list} is
returned to \method{next()}'s caller.  By ``frozen'' we mean that all
local state is retained, including the current bindings of local
variables, the instruction pointer, and the internal evaluation stack:
enough information is saved so that the next time \method{next()} is
invoked, the function can proceed exactly as if the \keyword{yield}
statement were just another external call.

As of Python version 2.5, the \keyword{yield} statement is now
allowed in the \keyword{try} clause of a \keyword{try} ...\ 
\keyword{finally} construct.  If the generator is not resumed before
it is finalized (by reaching a zero reference count or by being garbage
collected), the generator-iterator's \method{close()} method will be
called, allowing any pending \keyword{finally} clauses to execute.

\begin{notice}
In Python 2.2, the \keyword{yield} statement is only allowed
when the \code{generators} feature has been enabled.  It will always
be enabled in Python 2.3.  This \code{__future__} import statement can
be used to enable the feature:

\begin{verbatim}
from __future__ import generators
\end{verbatim}
\end{notice}


\begin{seealso}
  \seepep{0255}{Simple Generators}
         {The proposal for adding generators and the \keyword{yield}
          statement to Python.}

  \seepep{0342}{Coroutines via Enhanced Generators}
         {The proposal that, among other generator enhancements,
          proposed allowing \keyword{yield} to appear inside a
          \keyword{try} ... \keyword{finally} block.}
\end{seealso}


\section{The \keyword{raise} statement \label{raise}}
\stindex{raise}

\begin{productionlist}
  \production{raise_stmt}
             {"raise" [\token{expression} ["," \token{expression}
              ["," \token{expression}]]]}
\end{productionlist}

If no expressions are present, \keyword{raise} re-raises the last
exception that was active in the current scope.  If no exception is
active in the current scope, a \exception{TypeError} exception is
raised indicating that this is an error (if running under IDLE, a
\exception{Queue.Empty} exception is raised instead).
\index{exception}
\indexii{raising}{exception}

Otherwise, \keyword{raise} evaluates the expressions to get three
objects, using \code{None} as the value of omitted expressions.  The
first two objects are used to determine the \emph{type} and
\emph{value} of the exception.

If the first object is an instance, the type of the exception is the
class of the instance, the instance itself is the value, and the
second object must be \code{None}.

If the first object is a class, it becomes the type of the exception.
The second object is used to determine the exception value: If it is
an instance of the class, the instance becomes the exception value.
If the second object is a tuple, it is used as the argument list for
the class constructor; if it is \code{None}, an empty argument list is
used, and any other object is treated as a single argument to the
constructor.  The instance so created by calling the constructor is
used as the exception value.

If a third object is present and not \code{None}, it must be a
traceback\obindex{traceback} object (see section~\ref{traceback}), and
it is substituted instead of the current location as the place where
the exception occurred.  If the third object is present and not a
traceback object or \code{None}, a \exception{TypeError} exception is
raised.  The three-expression form of \keyword{raise} is useful to
re-raise an exception transparently in an except clause, but
\keyword{raise} with no expressions should be preferred if the
exception to be re-raised was the most recently active exception in
the current scope.

Additional information on exceptions can be found in
section~\ref{exceptions}, and information about handling exceptions is
in section~\ref{try}.


\section{The \keyword{break} statement \label{break}}
\stindex{break}

\begin{productionlist}
  \production{break_stmt}
             {"break"}
\end{productionlist}

\keyword{break} may only occur syntactically nested in a \keyword{for}
or \keyword{while} loop, but not nested in a function or class definition
within that loop.
\stindex{for}
\stindex{while}
\indexii{loop}{statement}

It terminates the nearest enclosing loop, skipping the optional
\keyword{else} clause if the loop has one.
\kwindex{else}

If a \keyword{for} loop is terminated by \keyword{break}, the loop control
target keeps its current value.
\indexii{loop control}{target}

When \keyword{break} passes control out of a \keyword{try} statement
with a \keyword{finally} clause, that \keyword{finally} clause is executed
before really leaving the loop.
\kwindex{finally}


\section{The \keyword{continue} statement \label{continue}}
\stindex{continue}

\begin{productionlist}
  \production{continue_stmt}
             {"continue"}
\end{productionlist}

\keyword{continue} may only occur syntactically nested in a \keyword{for} or
\keyword{while} loop, but not nested in a function or class definition or
\keyword{finally} statement within that loop.\footnote{It may
occur within an \keyword{except} or \keyword{else} clause.  The
restriction on occurring in the \keyword{try} clause is implementor's
laziness and will eventually be lifted.}
It continues with the next cycle of the nearest enclosing loop.
\stindex{for}
\stindex{while}
\indexii{loop}{statement}
\kwindex{finally}


\section{The \keyword{import} statement \label{import}}
\stindex{import}
\index{module!importing}
\indexii{name}{binding}
\kwindex{from}

\begin{productionlist}
  \production{import_stmt}
             {"import" \token{module} ["as" \token{name}]
                ( "," \token{module} ["as" \token{name}] )*}
  \productioncont{| "from" \token{relative_module} "import" \token{identifier}
                    ["as" \token{name}]}
  \productioncont{  ( "," \token{identifier} ["as" \token{name}] )*}
  \productioncont{| "from" \token{relative_module} "import" "("
                    \token{identifier} ["as" \token{name}]}
  \productioncont{  ( "," \token{identifier} ["as" \token{name}] )* [","] ")"}
  \productioncont{| "from" \token{module} "import" "*"}
  \production{module}
             {(\token{identifier} ".")* \token{identifier}}
  \production{relative_module}
             {"."* \token{module} | "."+}
  \production{name}
             {\token{identifier}}
\end{productionlist}

Import statements are executed in two steps: (1) find a module, and
initialize it if necessary; (2) define a name or names in the local
namespace (of the scope where the \keyword{import} statement occurs).
The first form (without \keyword{from}) repeats these steps for each
identifier in the list.  The form with \keyword{from} performs step
(1) once, and then performs step (2) repeatedly.

In this context, to ``initialize'' a built-in or extension module means to
call an initialization function that the module must provide for the purpose
(in the reference implementation, the function's name is obtained by
prepending string ``init'' to the module's name); to ``initialize'' a
Python-coded module means to execute the module's body.
  
The system maintains a table of modules that have been or are being
initialized,
indexed by module name.  This table is
accessible as \code{sys.modules}.  When a module name is found in
this table, step (1) is finished.  If not, a search for a module
definition is started.  When a module is found, it is loaded.  Details
of the module searching and loading process are implementation and
platform specific.  It generally involves searching for a ``built-in''
module with the given name and then searching a list of locations
given as \code{sys.path}.
\withsubitem{(in module sys)}{\ttindex{modules}}
\ttindex{sys.modules}
\indexii{module}{name}
\indexii{built-in}{module}
\indexii{user-defined}{module}
\refbimodindex{sys}
\indexii{filename}{extension}
\indexiii{module}{search}{path}

If a built-in module is found,\indexii{module}{initialization} its
built-in initialization code is executed and step (1) is finished.  If
no matching file is found,
\exception{ImportError}\exindex{ImportError} is raised.
\index{code block}If a file is found, it is parsed,
yielding an executable code block.  If a syntax error occurs,
\exception{SyntaxError}\exindex{SyntaxError} is raised.  Otherwise, an
empty module of the given name is created and inserted in the module
table, and then the code block is executed in the context of this
module.  Exceptions during this execution terminate step (1).

When step (1) finishes without raising an exception, step (2) can
begin.

The first form of \keyword{import} statement binds the module name in
the local namespace to the module object, and then goes on to import
the next identifier, if any.  If the module name is followed by
\keyword{as}, the name following \keyword{as} is used as the local
name for the module. 

The \keyword{from} form does not bind the module name: it goes through the
list of identifiers, looks each one of them up in the module found in step
(1), and binds the name in the local namespace to the object thus found. 
As with the first form of \keyword{import}, an alternate local name can be
supplied by specifying "\keyword{as} localname".  If a name is not found,
\exception{ImportError} is raised.  If the list of identifiers is replaced
by a star (\character{*}), all public names defined in the module are
bound in the local namespace of the \keyword{import} statement..
\indexii{name}{binding}
\exindex{ImportError}

The \emph{public names} defined by a module are determined by checking
the module's namespace for a variable named \code{__all__}; if
defined, it must be a sequence of strings which are names defined or
imported by that module.  The names given in \code{__all__} are all
considered public and are required to exist.  If \code{__all__} is not
defined, the set of public names includes all names found in the
module's namespace which do not begin with an underscore character
(\character{_}).  \code{__all__} should contain the entire public API.
It is intended to avoid accidentally exporting items that are not part
of the API (such as library modules which were imported and used within
the module).
\withsubitem{(optional module attribute)}{\ttindex{__all__}}

The \keyword{from} form with \samp{*} may only occur in a module
scope.  If the wild card form of import --- \samp{import *} --- is
used in a function and the function contains or is a nested block with
free variables, the compiler will raise a \exception{SyntaxError}.

\kwindex{from}
\stindex{from}

\strong{Hierarchical module names:}\indexiii{hierarchical}{module}{names}
when the module names contains one or more dots, the module search
path is carried out differently.  The sequence of identifiers up to
the last dot is used to find a ``package''\index{packages}; the final
identifier is then searched inside the package.  A package is
generally a subdirectory of a directory on \code{sys.path} that has a
file \file{__init__.py}.\ttindex{__init__.py}
%
[XXX Can't be bothered to spell this out right now; see the URL
\url{http://www.python.org/doc/essays/packages.html} for more details, also
about how the module search works from inside a package.]

The built-in function \function{__import__()} is provided to support
applications that determine which modules need to be loaded
dynamically; refer to \ulink{Built-in
Functions}{../lib/built-in-funcs.html} in the
\citetitle[../lib/lib.html]{Python Library Reference} for additional
information.
\bifuncindex{__import__}

\subsection{Future statements \label{future}}

A \dfn{future statement}\indexii{future}{statement} is a directive to
the compiler that a particular module should be compiled using syntax
or semantics that will be available in a specified future release of
Python.  The future statement is intended to ease migration to future
versions of Python that introduce incompatible changes to the
language.  It allows use of the new features on a per-module basis
before the release in which the feature becomes standard.

\begin{productionlist}[*]
  \production{future_statement}
             {"from" "__future__" "import" feature ["as" name]}
  \productioncont{  ("," feature ["as" name])*}
  \productioncont{| "from" "__future__" "import" "(" feature ["as" name]}
  \productioncont{  ("," feature ["as" name])* [","] ")"}
  \production{feature}{identifier}
  \production{name}{identifier}
\end{productionlist}

A future statement must appear near the top of the module.  The only
lines that can appear before a future statement are:

\begin{itemize}

\item the module docstring (if any),
\item comments,
\item blank lines, and
\item other future statements.

\end{itemize}

The features recognized by Python 2.5 are \samp{absolute_import},
\samp{division}, \samp{generators}, \samp{nested_scopes} and
\samp{with_statement}.  \samp{generators} and \samp{nested_scopes} 
are redundant in Python version 2.3 and above because they are always
enabled. 

A future statement is recognized and treated specially at compile
time: Changes to the semantics of core constructs are often
implemented by generating different code.  It may even be the case
that a new feature introduces new incompatible syntax (such as a new
reserved word), in which case the compiler may need to parse the
module differently.  Such decisions cannot be pushed off until
runtime.

For any given release, the compiler knows which feature names have been
defined, and raises a compile-time error if a future statement contains
a feature not known to it.

The direct runtime semantics are the same as for any import statement:
there is a standard module \module{__future__}, described later, and
it will be imported in the usual way at the time the future statement
is executed.

The interesting runtime semantics depend on the specific feature
enabled by the future statement.

Note that there is nothing special about the statement:

\begin{verbatim}
import __future__ [as name]
\end{verbatim}

That is not a future statement; it's an ordinary import statement with
no special semantics or syntax restrictions.

Code compiled by an \keyword{exec} statement or calls to the builtin functions
\function{compile()} and \function{execfile()} that occur in a module
\module{M} containing a future statement will, by default, use the new 
syntax or semantics associated with the future statement.  This can,
starting with Python 2.2 be controlled by optional arguments to
\function{compile()} --- see the documentation of that function in the
\citetitle[../lib/built-in-funcs.html]{Python Library Reference} for
details.

A future statement typed at an interactive interpreter prompt will
take effect for the rest of the interpreter session.  If an
interpreter is started with the \programopt{-i} option, is passed a
script name to execute, and the script includes a future statement, it
will be in effect in the interactive session started after the script
is executed.

\section{The \keyword{global} statement \label{global}}
\stindex{global}

\begin{productionlist}
  \production{global_stmt}
             {"global" \token{identifier} ("," \token{identifier})*}
\end{productionlist}

The \keyword{global} statement is a declaration which holds for the
entire current code block.  It means that the listed identifiers are to be
interpreted as globals.  It would be impossible to assign to a global
variable without \keyword{global}, although free variables may refer
to globals without being declared global.
\indexiii{global}{name}{binding}

Names listed in a \keyword{global} statement must not be used in the same
code block textually preceding that \keyword{global} statement.

Names listed in a \keyword{global} statement must not be defined as formal
parameters or in a \keyword{for} loop control target, \keyword{class}
definition, function definition, or \keyword{import} statement.

(The current implementation does not enforce the latter two
restrictions, but programs should not abuse this freedom, as future
implementations may enforce them or silently change the meaning of the
program.)

\strong{Programmer's note:}
the \keyword{global} is a directive to the parser.  It
applies only to code parsed at the same time as the \keyword{global}
statement.  In particular, a \keyword{global} statement contained in an
\keyword{exec} statement does not affect the code block \emph{containing}
the \keyword{exec} statement, and code contained in an \keyword{exec}
statement is unaffected by \keyword{global} statements in the code
containing the \keyword{exec} statement.  The same applies to the
\function{eval()}, \function{execfile()} and \function{compile()} functions.
\stindex{exec}
\bifuncindex{eval}
\bifuncindex{execfile}
\bifuncindex{compile}


\section{The \keyword{exec} statement \label{exec}}
\stindex{exec}

\begin{productionlist}
  \production{exec_stmt}
             {"exec" \token{or_expr}
              ["in" \token{expression} ["," \token{expression}]]}
\end{productionlist}

This statement supports dynamic execution of Python code.  The first
expression should evaluate to either a string, an open file object, or
a code object.  If it is a string, the string is parsed as a suite of
Python statements which is then executed (unless a syntax error
occurs).  If it is an open file, the file is parsed until \EOF{} and
executed.  If it is a code object, it is simply executed.  In all
cases, the code that's executed is expected to be valid as file
input (see section~\ref{file-input}, ``File input'').  Be aware that
the \keyword{return} and \keyword{yield} statements may not be used
outside of function definitions even within the context of code passed
to the \keyword{exec} statement.

In all cases, if the optional parts are omitted, the code is executed
in the current scope.  If only the first expression after \keyword{in}
is specified, it should be a dictionary, which will be used for both
the global and the local variables.  If two expressions are given,
they are used for the global and local variables, respectively.
If provided, \var{locals} can be any mapping object.
\versionchanged[formerly \var{locals} was required to be a dictionary]{2.4}

As a side effect, an implementation may insert additional keys into
the dictionaries given besides those corresponding to variable names
set by the executed code.  For example, the current implementation
may add a reference to the dictionary of the built-in module
\module{__builtin__} under the key \code{__builtins__} (!).
\ttindex{__builtins__}
\refbimodindex{__builtin__}

\strong{Programmer's hints:}
dynamic evaluation of expressions is supported by the built-in
function \function{eval()}.  The built-in functions
\function{globals()} and \function{locals()} return the current global
and local dictionary, respectively, which may be useful to pass around
for use by \keyword{exec}.
\bifuncindex{eval}
\bifuncindex{globals}
\bifuncindex{locals}

  



		% Simple statements
\chapter{Compound statements\label{compound}}
\indexii{compound}{statement}

Compound statements contain (groups of) other statements; they affect
or control the execution of those other statements in some way.  In
general, compound statements span multiple lines, although in simple
incarnations a whole compound statement may be contained in one line.

The \keyword{if}, \keyword{while} and \keyword{for} statements implement
traditional control flow constructs.  \keyword{try} specifies exception
handlers and/or cleanup code for a group of statements.  Function and
class definitions are also syntactically compound statements.

Compound statements consist of one or more `clauses.'  A clause
consists of a header and a `suite.'  The clause headers of a
particular compound statement are all at the same indentation level.
Each clause header begins with a uniquely identifying keyword and ends
with a colon.  A suite is a group of statements controlled by a
clause.  A suite can be one or more semicolon-separated simple
statements on the same line as the header, following the header's
colon, or it can be one or more indented statements on subsequent
lines.  Only the latter form of suite can contain nested compound
statements; the following is illegal, mostly because it wouldn't be
clear to which \keyword{if} clause a following \keyword{else} clause would
belong:
\index{clause}
\index{suite}

\begin{verbatim}
if test1: if test2: print x
\end{verbatim}

Also note that the semicolon binds tighter than the colon in this
context, so that in the following example, either all or none of the
\keyword{print} statements are executed:

\begin{verbatim}
if x < y < z: print x; print y; print z
\end{verbatim}

Summarizing:

\begin{productionlist}
  \production{compound_stmt}
             {\token{if_stmt}}
  \productioncont{| \token{while_stmt}}
  \productioncont{| \token{for_stmt}}
  \productioncont{| \token{try_stmt}}
  \productioncont{| \token{with_stmt}}
  \productioncont{| \token{funcdef}}
  \productioncont{| \token{classdef}}
  \production{suite}
             {\token{stmt_list} NEWLINE
              | NEWLINE INDENT \token{statement}+ DEDENT}
  \production{statement}
             {\token{stmt_list} NEWLINE | \token{compound_stmt}}
  \production{stmt_list}
             {\token{simple_stmt} (";" \token{simple_stmt})* [";"]}
\end{productionlist}

Note that statements always end in a
\code{NEWLINE}\index{NEWLINE token} possibly followed by a
\code{DEDENT}.\index{DEDENT token} Also note that optional
continuation clauses always begin with a keyword that cannot start a
statement, thus there are no ambiguities (the `dangling
\keyword{else}' problem is solved in Python by requiring nested
\keyword{if} statements to be indented).
\indexii{dangling}{else}

The formatting of the grammar rules in the following sections places
each clause on a separate line for clarity.


\section{The \keyword{if} statement\label{if}}
\stindex{if}

The \keyword{if} statement is used for conditional execution:

\begin{productionlist}
  \production{if_stmt}
             {"if" \token{expression} ":" \token{suite}}
  \productioncont{( "elif" \token{expression} ":" \token{suite} )*}
  \productioncont{["else" ":" \token{suite}]}
\end{productionlist}

It selects exactly one of the suites by evaluating the expressions one
by one until one is found to be true (see section~\ref{Booleans} for
the definition of true and false); then that suite is executed (and no
other part of the \keyword{if} statement is executed or evaluated).  If
all expressions are false, the suite of the \keyword{else} clause, if
present, is executed.
\kwindex{elif}
\kwindex{else}


\section{The \keyword{while} statement\label{while}}
\stindex{while}
\indexii{loop}{statement}

The \keyword{while} statement is used for repeated execution as long
as an expression is true:

\begin{productionlist}
  \production{while_stmt}
             {"while" \token{expression} ":" \token{suite}}
  \productioncont{["else" ":" \token{suite}]}
\end{productionlist}

This repeatedly tests the expression and, if it is true, executes the
first suite; if the expression is false (which may be the first time it
is tested) the suite of the \keyword{else} clause, if present, is
executed and the loop terminates.
\kwindex{else}

A \keyword{break} statement executed in the first suite terminates the
loop without executing the \keyword{else} clause's suite.  A
\keyword{continue} statement executed in the first suite skips the rest
of the suite and goes back to testing the expression.
\stindex{break}
\stindex{continue}


\section{The \keyword{for} statement\label{for}}
\stindex{for}
\indexii{loop}{statement}

The \keyword{for} statement is used to iterate over the elements of a
sequence (such as a string, tuple or list) or other iterable object:
\obindex{sequence}

\begin{productionlist}
  \production{for_stmt}
             {"for" \token{target_list} "in" \token{expression_list}
              ":" \token{suite}}
  \productioncont{["else" ":" \token{suite}]}
\end{productionlist}

The expression list is evaluated once; it should yield an iterable
object.  An iterator is created for the result of the
{}\code{expression_list}.  The suite is then executed once for each
item provided by the iterator, in the
order of ascending indices.  Each item in turn is assigned to the
target list using the standard rules for assignments, and then the
suite is executed.  When the items are exhausted (which is immediately
when the sequence is empty), the suite in the \keyword{else} clause, if
present, is executed, and the loop terminates.
\kwindex{in}
\kwindex{else}
\indexii{target}{list}

A \keyword{break} statement executed in the first suite terminates the
loop without executing the \keyword{else} clause's suite.  A
\keyword{continue} statement executed in the first suite skips the rest
of the suite and continues with the next item, or with the \keyword{else}
clause if there was no next item.
\stindex{break}
\stindex{continue}

The suite may assign to the variable(s) in the target list; this does
not affect the next item assigned to it.

The target list is not deleted when the loop is finished, but if the
sequence is empty, it will not have been assigned to at all by the
loop.  Hint: the built-in function \function{range()} returns a
sequence of integers suitable to emulate the effect of Pascal's
\code{for i := a to b do};
e.g., \code{range(3)} returns the list \code{[0, 1, 2]}.
\bifuncindex{range}
\indexii{Pascal}{language}

\warning{There is a subtlety when the sequence is being modified
by the loop (this can only occur for mutable sequences, i.e. lists).
An internal counter is used to keep track of which item is used next,
and this is incremented on each iteration.  When this counter has
reached the length of the sequence the loop terminates.  This means that
if the suite deletes the current (or a previous) item from the
sequence, the next item will be skipped (since it gets the index of
the current item which has already been treated).  Likewise, if the
suite inserts an item in the sequence before the current item, the
current item will be treated again the next time through the loop.
This can lead to nasty bugs that can be avoided by making a temporary
copy using a slice of the whole sequence, e.g.,
\index{loop!over mutable sequence}
\index{mutable sequence!loop over}}

\begin{verbatim}
for x in a[:]:
    if x < 0: a.remove(x)
\end{verbatim}


\section{The \keyword{try} statement\label{try}}
\stindex{try}

The \keyword{try} statement specifies exception handlers and/or cleanup
code for a group of statements:

\begin{productionlist}
  \production{try_stmt} {try1_stmt | try2_stmt}
  \production{try1_stmt}
             {"try" ":" \token{suite}}
  \productioncont{("except" [\token{expression}
                             ["," \token{target}]] ":" \token{suite})+}
  \productioncont{["else" ":" \token{suite}]}
  \productioncont{["finally" ":" \token{suite}]}
  \production{try2_stmt}
             {"try" ":" \token{suite}}
  \productioncont{"finally" ":" \token{suite}}
\end{productionlist}

\versionchanged[In previous versions of Python,
\keyword{try}...\keyword{except}...\keyword{finally} did not work.
\keyword{try}...\keyword{except} had to be nested in
\keyword{try}...\keyword{finally}]{2.5}

The \keyword{except} clause(s) specify one or more exception handlers.
When no exception occurs in the
\keyword{try} clause, no exception handler is executed.  When an
exception occurs in the \keyword{try} suite, a search for an exception
handler is started.  This search inspects the except clauses in turn until
one is found that matches the exception.  An expression-less except
clause, if present, must be last; it matches any exception.  For an
except clause with an expression, that expression is evaluated, and the
clause matches the exception if the resulting object is ``compatible''
with the exception.  An object is compatible with an exception if it
is the class or a base class of the exception object, a tuple
containing an item compatible with the exception, or, in the
(deprecated) case of string exceptions, is the raised string itself
(note that the object identities must match, i.e. it must be the same
string object, not just a string with the same value).
\kwindex{except}

If no except clause matches the exception, the search for an exception
handler continues in the surrounding code and on the invocation stack.
\footnote{The exception is propogated to the invocation stack only if
there is no \keyword{finally} clause that negates the exception.}

If the evaluation of an expression in the header of an except clause
raises an exception, the original search for a handler is canceled
and a search starts for the new exception in the surrounding code and
on the call stack (it is treated as if the entire \keyword{try} statement
raised the exception).

When a matching except clause is found, the exception is assigned to
the target specified in that except clause, if present, and the except
clause's suite is executed.  All except clauses must have an
executable block.  When the end of this block is reached, execution
continues normally after the entire try statement.  (This means that
if two nested handlers exist for the same exception, and the exception
occurs in the try clause of the inner handler, the outer handler will
not handle the exception.)

Before an except clause's suite is executed, details about the
exception are assigned to three variables in the
\module{sys}\refbimodindex{sys} module: \code{sys.exc_type} receives
the object identifying the exception; \code{sys.exc_value} receives
the exception's parameter; \code{sys.exc_traceback} receives a
traceback object\obindex{traceback} (see section~\ref{traceback})
identifying the point in the program where the exception occurred.
These details are also available through the \function{sys.exc_info()}
function, which returns a tuple \code{(\var{exc_type}, \var{exc_value},
\var{exc_traceback})}.  Use of the corresponding variables is
deprecated in favor of this function, since their use is unsafe in a
threaded program.  As of Python 1.5, the variables are restored to
their previous values (before the call) when returning from a function
that handled an exception.
\withsubitem{(in module sys)}{\ttindex{exc_type}
  \ttindex{exc_value}\ttindex{exc_traceback}}

The optional \keyword{else} clause is executed if and when control
flows off the end of the \keyword{try} clause.\footnote{
  Currently, control ``flows off the end'' except in the case of an
  exception or the execution of a \keyword{return},
  \keyword{continue}, or \keyword{break} statement.
} Exceptions in the \keyword{else} clause are not handled by the
preceding \keyword{except} clauses.
\kwindex{else}
\stindex{return}
\stindex{break}
\stindex{continue}

If \keyword{finally} is present, it specifies a `cleanup' handler.  The
\keyword{try} clause is executed, including any \keyword{except} and
\keyword{else} clauses.  If an exception occurs in any of the clauses
and is not handled, the exception is temporarily saved. The
\keyword{finally} clause is executed.  If there is a saved exception,
it is re-raised at the end of the \keyword{finally} clause.
If the \keyword{finally} clause raises another exception or
executes a \keyword{return} or \keyword{break} statement, the saved
exception is lost.  The exception information is not available to the
program during execution of the \keyword{finally} clause.
\kwindex{finally}

When a \keyword{return}, \keyword{break} or \keyword{continue} statement is
executed in the \keyword{try} suite of a \keyword{try}...\keyword{finally}
statement, the \keyword{finally} clause is also executed `on the way out.' A
\keyword{continue} statement is illegal in the \keyword{finally} clause.
(The reason is a problem with the current implementation --- this
restriction may be lifted in the future).
\stindex{return}
\stindex{break}
\stindex{continue}

Additional information on exceptions can be found in
section~\ref{exceptions}, and information on using the \keyword{raise}
statement to generate exceptions may be found in section~\ref{raise}.


\section{The \keyword{with} statement\label{with}}
\stindex{with}

\versionadded{2.5}

The \keyword{with} statement is used to wrap the execution of a block
with methods defined by a context manager (see
section~\ref{context-managers}). This allows common
\keyword{try}...\keyword{except}...\keyword{finally} usage patterns to
be encapsulated for convenient reuse.

\begin{productionlist}
  \production{with_stmt}
  {"with" \token{expression} ["as" \token{target}] ":" \token{suite}}
\end{productionlist}

The execution of the \keyword{with} statement proceeds as follows:

\begin{enumerate}

\item The context expression is evaluated to obtain a context manager.

\item The context manager's \method{__enter__()} method is invoked.

\item If a target was included in the \keyword{with}
statement, the return value from \method{__enter__()} is assigned to it.

\note{The \keyword{with} statement guarantees that if the
\method{__enter__()} method returns without an error, then
\method{__exit__()} will always be called. Thus, if an error occurs
during the assignment to the target list, it will be treated the same as
an error occurring within the suite would be. See step 5 below.}

\item The suite is executed.

\item The context manager's \method{__exit__()} method is invoked. If
an exception caused the suite to be exited, its type, value, and
traceback are passed as arguments to \method{__exit__()}. Otherwise,
three \constant{None} arguments are supplied.

If the suite was exited due to an exception, and the return
value from the \method{__exit__()} method was false, the exception is
reraised. If the return value was true, the exception is suppressed, and
execution continues with the statement following the \keyword{with}
statement.

If the suite was exited for any reason other than an exception, the
return value from \method{__exit__()} is ignored, and execution proceeds
at the normal location for the kind of exit that was taken.

\end{enumerate}

\begin{notice}
In Python 2.5, the \keyword{with} statement is only allowed
when the \code{with_statement} feature has been enabled.  It will always
be enabled in Python 2.6.  This \code{__future__} import statement can
be used to enable the feature:

\begin{verbatim}
from __future__ import with_statement
\end{verbatim}
\end{notice}

\begin{seealso}
  \seepep{0343}{The "with" statement}
         {The specification, background, and examples for the
          Python \keyword{with} statement.}
\end{seealso}

\section{Function definitions\label{function}}
\indexii{function}{definition}
\stindex{def}

A function definition defines a user-defined function object (see
section~\ref{types}):
\obindex{user-defined function}
\obindex{function}

\begin{productionlist}
  \production{funcdef}
             {[\token{decorators}] "def" \token{funcname} "(" [\token{parameter_list}] ")"
              ":" \token{suite}}
  \production{decorators}
             {\token{decorator}+}
  \production{decorator}
             {"@" \token{dotted_name} ["(" [\token{argument_list} [","]] ")"] NEWLINE}
  \production{dotted_name}
             {\token{identifier} ("." \token{identifier})*}
  \production{parameter_list}
                 {(\token{defparameter} ",")*}
  \productioncont{(~~"*" \token{identifier} [, "**" \token{identifier}]}
  \productioncont{ | "**" \token{identifier}}
  \productioncont{ | \token{defparameter} [","] )}
  \production{defparameter}
             {\token{parameter} ["=" \token{expression}]}
  \production{sublist}
             {\token{parameter} ("," \token{parameter})* [","]}
  \production{parameter}
             {\token{identifier} | "(" \token{sublist} ")"}
  \production{funcname}
             {\token{identifier}}
\end{productionlist}

A function definition is an executable statement.  Its execution binds
the function name in the current local namespace to a function object
(a wrapper around the executable code for the function).  This
function object contains a reference to the current global namespace
as the global namespace to be used when the function is called.
\indexii{function}{name}
\indexii{name}{binding}

The function definition does not execute the function body; this gets
executed only when the function is called.

A function definition may be wrapped by one or more decorator expressions.
Decorator expressions are evaluated when the function is defined, in the scope
that contains the function definition.  The result must be a callable,
which is invoked with the function object as the only argument.
The returned value is bound to the function name instead of the function
object.  Multiple decorators are applied in nested fashion.
For example, the following code:

\begin{verbatim}
@f1(arg)
@f2
def func(): pass
\end{verbatim}

is equivalent to:

\begin{verbatim}
def func(): pass
func = f1(arg)(f2(func))
\end{verbatim}

When one or more top-level parameters have the form \var{parameter}
\code{=} \var{expression}, the function is said to have ``default
parameter values.''  For a parameter with a
default value, the corresponding argument may be omitted from a call,
in which case the parameter's default value is substituted.  If a
parameter has a default value, all following parameters must also have
a default value --- this is a syntactic restriction that is not
expressed by the grammar.
\indexiii{default}{parameter}{value}

\strong{Default parameter values are evaluated when the function
definition is executed.}  This means that the expression is evaluated
once, when the function is defined, and that that same
``pre-computed'' value is used for each call.  This is especially
important to understand when a default parameter is a mutable object,
such as a list or a dictionary: if the function modifies the object
(e.g. by appending an item to a list), the default value is in effect
modified.  This is generally not what was intended.  A way around this 
is to use \code{None} as the default, and explicitly test for it in
the body of the function, e.g.:

\begin{verbatim}
def whats_on_the_telly(penguin=None):
    if penguin is None:
        penguin = []
    penguin.append("property of the zoo")
    return penguin
\end{verbatim}

Function call semantics are described in more detail in
section~\ref{calls}.
A function call always assigns values to all parameters mentioned in
the parameter list, either from position arguments, from keyword
arguments, or from default values.  If the form ``\code{*identifier}''
is present, it is initialized to a tuple receiving any excess
positional parameters, defaulting to the empty tuple.  If the form
``\code{**identifier}'' is present, it is initialized to a new
dictionary receiving any excess keyword arguments, defaulting to a
new empty dictionary.

It is also possible to create anonymous functions (functions not bound
to a name), for immediate use in expressions.  This uses lambda forms,
described in section~\ref{lambda}.  Note that the lambda form is
merely a shorthand for a simplified function definition; a function
defined in a ``\keyword{def}'' statement can be passed around or
assigned to another name just like a function defined by a lambda
form.  The ``\keyword{def}'' form is actually more powerful since it
allows the execution of multiple statements.
\indexii{lambda}{form}

\strong{Programmer's note:} Functions are first-class objects.  A
``\code{def}'' form executed inside a function definition defines a
local function that can be returned or passed around.  Free variables
used in the nested function can access the local variables of the
function containing the def.  See section~\ref{naming} for details.


\section{Class definitions\label{class}}
\indexii{class}{definition}
\stindex{class}

A class definition defines a class object (see section~\ref{types}):
\obindex{class}

\begin{productionlist}
  \production{classdef}
             {"class" \token{classname} [\token{inheritance}] ":"
              \token{suite}}
  \production{inheritance}
             {"(" [\token{expression_list}] ")"}
  \production{classname}
             {\token{identifier}}
\end{productionlist}

A class definition is an executable statement.  It first evaluates the
inheritance list, if present.  Each item in the inheritance list
should evaluate to a class object or class type which allows
subclassing.  The class's suite is then executed
in a new execution frame (see section~\ref{naming}), using a newly
created local namespace and the original global namespace.
(Usually, the suite contains only function definitions.)  When the
class's suite finishes execution, its execution frame is discarded but
its local namespace is saved.  A class object is then created using
the inheritance list for the base classes and the saved local
namespace for the attribute dictionary.  The class name is bound to this
class object in the original local namespace.
\index{inheritance}
\indexii{class}{name}
\indexii{name}{binding}
\indexii{execution}{frame}

\strong{Programmer's note:} Variables defined in the class definition
are class variables; they are shared by all instances.  To define
instance variables, they must be given a value in the
\method{__init__()} method or in another method.  Both class and
instance variables are accessible through the notation
``\code{self.name}'', and an instance variable hides a class variable
with the same name when accessed in this way.  Class variables with
immutable values can be used as defaults for instance variables.
For new-style classes, descriptors can be used to create instance
variables with different implementation details.
		% Compound statements
\chapter{Top-level components\label{top-level}}

The Python interpreter can get its input from a number of sources:
from a script passed to it as standard input or as program argument,
typed in interactively, from a module source file, etc.  This chapter
gives the syntax used in these cases.
\index{interpreter}


\section{Complete Python programs\label{programs}}
\index{program}

While a language specification need not prescribe how the language
interpreter is invoked, it is useful to have a notion of a complete
Python program.  A complete Python program is executed in a minimally
initialized environment: all built-in and standard modules are
available, but none have been initialized, except for \module{sys}
(various system services), \module{__builtin__} (built-in functions,
exceptions and \code{None}) and \module{__main__}.  The latter is used
to provide the local and global namespace for execution of the
complete program.
\refbimodindex{sys}
\refbimodindex{__main__}
\refbimodindex{__builtin__}

The syntax for a complete Python program is that for file input,
described in the next section.

The interpreter may also be invoked in interactive mode; in this case,
it does not read and execute a complete program but reads and executes
one statement (possibly compound) at a time.  The initial environment
is identical to that of a complete program; each statement is executed
in the namespace of \module{__main__}.
\index{interactive mode}
\refbimodindex{__main__}

Under \UNIX, a complete program can be passed to the interpreter in
three forms: with the \programopt{-c} \var{string} command line option, as a
file passed as the first command line argument, or as standard input.
If the file or standard input is a tty device, the interpreter enters
interactive mode; otherwise, it executes the file as a complete
program.
\index{UNIX}
\index{command line}
\index{standard input}


\section{File input\label{file-input}}

All input read from non-interactive files has the same form:

\begin{productionlist}
  \production{file_input}
             {(NEWLINE | \token{statement})*}
\end{productionlist}

This syntax is used in the following situations:

\begin{itemize}

\item when parsing a complete Python program (from a file or from a string);

\item when parsing a module;

\item when parsing a string passed to the \keyword{exec} statement;

\end{itemize}


\section{Interactive input\label{interactive}}

Input in interactive mode is parsed using the following grammar:

\begin{productionlist}
  \production{interactive_input}
             {[\token{stmt_list}] NEWLINE | \token{compound_stmt} NEWLINE}
\end{productionlist}

Note that a (top-level) compound statement must be followed by a blank
line in interactive mode; this is needed to help the parser detect the
end of the input.


\section{Expression input\label{expression-input}}
\index{input}

There are two forms of expression input.  Both ignore leading
whitespace.
The string argument to \function{eval()} must have the following form:
\bifuncindex{eval}

\begin{productionlist}
  \production{eval_input}
             {\token{expression_list} NEWLINE*}
\end{productionlist}

The input line read by \function{input()} must have the following form:
\bifuncindex{input}

\begin{productionlist}
  \production{input_input}
             {\token{expression_list} NEWLINE}
\end{productionlist}

Note: to read `raw' input line without interpretation, you can use the
built-in function \function{raw_input()} or the \method{readline()} method
of file objects.
\obindex{file}
\index{input!raw}
\index{raw input}
\bifuncindex{raw_input}
\withsubitem{(file method)}{\ttindex{readline()}}
		% Top-level components

\appendix

\chapter{History and License}
\section{History of the software}

Python was created in the early 1990s by Guido van Rossum at Stichting
Mathematisch Centrum (CWI, see \url{http://www.cwi.nl/}) in the Netherlands
as a successor of a language called ABC.  Guido remains Python's
principal author, although it includes many contributions from others.

In 1995, Guido continued his work on Python at the Corporation for
National Research Initiatives (CNRI, see \url{http://www.cnri.reston.va.us/})
in Reston, Virginia where he released several versions of the
software.

In May 2000, Guido and the Python core development team moved to
BeOpen.com to form the BeOpen PythonLabs team.  In October of the same
year, the PythonLabs team moved to Digital Creations (now Zope
Corporation; see \url{http://www.zope.com/}).  In 2001, the Python
Software Foundation (PSF, see \url{http://www.python.org/psf/}) was
formed, a non-profit organization created specifically to own
Python-related Intellectual Property.  Zope Corporation is a
sponsoring member of the PSF.

All Python releases are Open Source (see
\url{http://www.opensource.org/} for the Open Source Definition).
Historically, most, but not all, Python releases have also been
GPL-compatible; the table below summarizes the various releases.

\begin{tablev}{c|c|c|c|c}{textrm}%
  {Release}{Derived from}{Year}{Owner}{GPL compatible?}
  \linev{0.9.0 thru 1.2}{n/a}{1991-1995}{CWI}{yes}
  \linev{1.3 thru 1.5.2}{1.2}{1995-1999}{CNRI}{yes}
  \linev{1.6}{1.5.2}{2000}{CNRI}{no}
  \linev{2.0}{1.6}{2000}{BeOpen.com}{no}
  \linev{1.6.1}{1.6}{2001}{CNRI}{no}
  \linev{2.1}{2.0+1.6.1}{2001}{PSF}{no}
  \linev{2.0.1}{2.0+1.6.1}{2001}{PSF}{yes}
  \linev{2.1.1}{2.1+2.0.1}{2001}{PSF}{yes}
  \linev{2.2}{2.1.1}{2001}{PSF}{yes}
  \linev{2.1.2}{2.1.1}{2002}{PSF}{yes}
  \linev{2.1.3}{2.1.2}{2002}{PSF}{yes}
  \linev{2.2.1}{2.2}{2002}{PSF}{yes}
  \linev{2.2.2}{2.2.1}{2002}{PSF}{yes}
  \linev{2.2.3}{2.2.2}{2002-2003}{PSF}{yes}
  \linev{2.3}{2.2.2}{2002-2003}{PSF}{yes}
  \linev{2.3.1}{2.3}{2002-2003}{PSF}{yes}
  \linev{2.3.2}{2.3.1}{2003}{PSF}{yes}
  \linev{2.3.3}{2.3.2}{2003}{PSF}{yes}
  \linev{2.3.4}{2.3.3}{2004}{PSF}{yes}
  \linev{2.3.5}{2.3.4}{2005}{PSF}{yes}
  \linev{2.4}{2.3}{2004}{PSF}{yes}
  \linev{2.4.1}{2.4}{2005}{PSF}{yes}
  \linev{2.4.2}{2.4.1}{2005}{PSF}{yes}
  \linev{2.4.3}{2.4.2}{2006}{PSF}{yes}
  \linev{2.4.4}{2.4.3}{2006}{PSF}{yes}
  \linev{2.5}{2.4}{2006}{PSF}{yes}
  \linev{2.5.1}{2.5}{2007}{PSF}{yes}
  \linev{2.5.2}{2.5.1}{2008}{PSF}{yes}
\end{tablev}

\note{GPL-compatible doesn't mean that we're distributing
Python under the GPL.  All Python licenses, unlike the GPL, let you
distribute a modified version without making your changes open source.
The GPL-compatible licenses make it possible to combine Python with
other software that is released under the GPL; the others don't.}

Thanks to the many outside volunteers who have worked under Guido's
direction to make these releases possible.


\section{Terms and conditions for accessing or otherwise using Python}

\centerline{\strong{PSF LICENSE AGREEMENT FOR PYTHON \version}}

\begin{enumerate}
\item
This LICENSE AGREEMENT is between the Python Software Foundation
(``PSF''), and the Individual or Organization (``Licensee'') accessing
and otherwise using Python \version{} software in source or binary
form and its associated documentation.

\item
Subject to the terms and conditions of this License Agreement, PSF
hereby grants Licensee a nonexclusive, royalty-free, world-wide
license to reproduce, analyze, test, perform and/or display publicly,
prepare derivative works, distribute, and otherwise use Python
\version{} alone or in any derivative version, provided, however, that
PSF's License Agreement and PSF's notice of copyright, i.e.,
``Copyright \copyright{} 2001-2008 Python Software Foundation; All
Rights Reserved'' are retained in Python \version{} alone or in any
derivative version prepared by Licensee.

\item
In the event Licensee prepares a derivative work that is based on
or incorporates Python \version{} or any part thereof, and wants to
make the derivative work available to others as provided herein, then
Licensee hereby agrees to include in any such work a brief summary of
the changes made to Python \version.

\item
PSF is making Python \version{} available to Licensee on an ``AS IS''
basis.  PSF MAKES NO REPRESENTATIONS OR WARRANTIES, EXPRESS OR
IMPLIED.  BY WAY OF EXAMPLE, BUT NOT LIMITATION, PSF MAKES NO AND
DISCLAIMS ANY REPRESENTATION OR WARRANTY OF MERCHANTABILITY OR FITNESS
FOR ANY PARTICULAR PURPOSE OR THAT THE USE OF PYTHON \version{} WILL
NOT INFRINGE ANY THIRD PARTY RIGHTS.

\item
PSF SHALL NOT BE LIABLE TO LICENSEE OR ANY OTHER USERS OF PYTHON
\version{} FOR ANY INCIDENTAL, SPECIAL, OR CONSEQUENTIAL DAMAGES OR
LOSS AS A RESULT OF MODIFYING, DISTRIBUTING, OR OTHERWISE USING PYTHON
\version, OR ANY DERIVATIVE THEREOF, EVEN IF ADVISED OF THE
POSSIBILITY THEREOF.

\item
This License Agreement will automatically terminate upon a material
breach of its terms and conditions.

\item
Nothing in this License Agreement shall be deemed to create any
relationship of agency, partnership, or joint venture between PSF and
Licensee.  This License Agreement does not grant permission to use PSF
trademarks or trade name in a trademark sense to endorse or promote
products or services of Licensee, or any third party.

\item
By copying, installing or otherwise using Python \version, Licensee
agrees to be bound by the terms and conditions of this License
Agreement.
\end{enumerate}


\centerline{\strong{BEOPEN.COM LICENSE AGREEMENT FOR PYTHON 2.0}}

\centerline{\strong{BEOPEN PYTHON OPEN SOURCE LICENSE AGREEMENT VERSION 1}}

\begin{enumerate}
\item
This LICENSE AGREEMENT is between BeOpen.com (``BeOpen''), having an
office at 160 Saratoga Avenue, Santa Clara, CA 95051, and the
Individual or Organization (``Licensee'') accessing and otherwise
using this software in source or binary form and its associated
documentation (``the Software'').

\item
Subject to the terms and conditions of this BeOpen Python License
Agreement, BeOpen hereby grants Licensee a non-exclusive,
royalty-free, world-wide license to reproduce, analyze, test, perform
and/or display publicly, prepare derivative works, distribute, and
otherwise use the Software alone or in any derivative version,
provided, however, that the BeOpen Python License is retained in the
Software, alone or in any derivative version prepared by Licensee.

\item
BeOpen is making the Software available to Licensee on an ``AS IS''
basis.  BEOPEN MAKES NO REPRESENTATIONS OR WARRANTIES, EXPRESS OR
IMPLIED.  BY WAY OF EXAMPLE, BUT NOT LIMITATION, BEOPEN MAKES NO AND
DISCLAIMS ANY REPRESENTATION OR WARRANTY OF MERCHANTABILITY OR FITNESS
FOR ANY PARTICULAR PURPOSE OR THAT THE USE OF THE SOFTWARE WILL NOT
INFRINGE ANY THIRD PARTY RIGHTS.

\item
BEOPEN SHALL NOT BE LIABLE TO LICENSEE OR ANY OTHER USERS OF THE
SOFTWARE FOR ANY INCIDENTAL, SPECIAL, OR CONSEQUENTIAL DAMAGES OR LOSS
AS A RESULT OF USING, MODIFYING OR DISTRIBUTING THE SOFTWARE, OR ANY
DERIVATIVE THEREOF, EVEN IF ADVISED OF THE POSSIBILITY THEREOF.

\item
This License Agreement will automatically terminate upon a material
breach of its terms and conditions.

\item
This License Agreement shall be governed by and interpreted in all
respects by the law of the State of California, excluding conflict of
law provisions.  Nothing in this License Agreement shall be deemed to
create any relationship of agency, partnership, or joint venture
between BeOpen and Licensee.  This License Agreement does not grant
permission to use BeOpen trademarks or trade names in a trademark
sense to endorse or promote products or services of Licensee, or any
third party.  As an exception, the ``BeOpen Python'' logos available
at http://www.pythonlabs.com/logos.html may be used according to the
permissions granted on that web page.

\item
By copying, installing or otherwise using the software, Licensee
agrees to be bound by the terms and conditions of this License
Agreement.
\end{enumerate}


\centerline{\strong{CNRI LICENSE AGREEMENT FOR PYTHON 1.6.1}}

\begin{enumerate}
\item
This LICENSE AGREEMENT is between the Corporation for National
Research Initiatives, having an office at 1895 Preston White Drive,
Reston, VA 20191 (``CNRI''), and the Individual or Organization
(``Licensee'') accessing and otherwise using Python 1.6.1 software in
source or binary form and its associated documentation.

\item
Subject to the terms and conditions of this License Agreement, CNRI
hereby grants Licensee a nonexclusive, royalty-free, world-wide
license to reproduce, analyze, test, perform and/or display publicly,
prepare derivative works, distribute, and otherwise use Python 1.6.1
alone or in any derivative version, provided, however, that CNRI's
License Agreement and CNRI's notice of copyright, i.e., ``Copyright
\copyright{} 1995-2001 Corporation for National Research Initiatives;
All Rights Reserved'' are retained in Python 1.6.1 alone or in any
derivative version prepared by Licensee.  Alternately, in lieu of
CNRI's License Agreement, Licensee may substitute the following text
(omitting the quotes): ``Python 1.6.1 is made available subject to the
terms and conditions in CNRI's License Agreement.  This Agreement
together with Python 1.6.1 may be located on the Internet using the
following unique, persistent identifier (known as a handle):
1895.22/1013.  This Agreement may also be obtained from a proxy server
on the Internet using the following URL:
\url{http://hdl.handle.net/1895.22/1013}.''

\item
In the event Licensee prepares a derivative work that is based on
or incorporates Python 1.6.1 or any part thereof, and wants to make
the derivative work available to others as provided herein, then
Licensee hereby agrees to include in any such work a brief summary of
the changes made to Python 1.6.1.

\item
CNRI is making Python 1.6.1 available to Licensee on an ``AS IS''
basis.  CNRI MAKES NO REPRESENTATIONS OR WARRANTIES, EXPRESS OR
IMPLIED.  BY WAY OF EXAMPLE, BUT NOT LIMITATION, CNRI MAKES NO AND
DISCLAIMS ANY REPRESENTATION OR WARRANTY OF MERCHANTABILITY OR FITNESS
FOR ANY PARTICULAR PURPOSE OR THAT THE USE OF PYTHON 1.6.1 WILL NOT
INFRINGE ANY THIRD PARTY RIGHTS.

\item
CNRI SHALL NOT BE LIABLE TO LICENSEE OR ANY OTHER USERS OF PYTHON
1.6.1 FOR ANY INCIDENTAL, SPECIAL, OR CONSEQUENTIAL DAMAGES OR LOSS AS
A RESULT OF MODIFYING, DISTRIBUTING, OR OTHERWISE USING PYTHON 1.6.1,
OR ANY DERIVATIVE THEREOF, EVEN IF ADVISED OF THE POSSIBILITY THEREOF.

\item
This License Agreement will automatically terminate upon a material
breach of its terms and conditions.

\item
This License Agreement shall be governed by the federal
intellectual property law of the United States, including without
limitation the federal copyright law, and, to the extent such
U.S. federal law does not apply, by the law of the Commonwealth of
Virginia, excluding Virginia's conflict of law provisions.
Notwithstanding the foregoing, with regard to derivative works based
on Python 1.6.1 that incorporate non-separable material that was
previously distributed under the GNU General Public License (GPL), the
law of the Commonwealth of Virginia shall govern this License
Agreement only as to issues arising under or with respect to
Paragraphs 4, 5, and 7 of this License Agreement.  Nothing in this
License Agreement shall be deemed to create any relationship of
agency, partnership, or joint venture between CNRI and Licensee.  This
License Agreement does not grant permission to use CNRI trademarks or
trade name in a trademark sense to endorse or promote products or
services of Licensee, or any third party.

\item
By clicking on the ``ACCEPT'' button where indicated, or by copying,
installing or otherwise using Python 1.6.1, Licensee agrees to be
bound by the terms and conditions of this License Agreement.
\end{enumerate}

\centerline{ACCEPT}



\centerline{\strong{CWI LICENSE AGREEMENT FOR PYTHON 0.9.0 THROUGH 1.2}}

Copyright \copyright{} 1991 - 1995, Stichting Mathematisch Centrum
Amsterdam, The Netherlands.  All rights reserved.

Permission to use, copy, modify, and distribute this software and its
documentation for any purpose and without fee is hereby granted,
provided that the above copyright notice appear in all copies and that
both that copyright notice and this permission notice appear in
supporting documentation, and that the name of Stichting Mathematisch
Centrum or CWI not be used in advertising or publicity pertaining to
distribution of the software without specific, written prior
permission.

STICHTING MATHEMATISCH CENTRUM DISCLAIMS ALL WARRANTIES WITH REGARD TO
THIS SOFTWARE, INCLUDING ALL IMPLIED WARRANTIES OF MERCHANTABILITY AND
FITNESS, IN NO EVENT SHALL STICHTING MATHEMATISCH CENTRUM BE LIABLE
FOR ANY SPECIAL, INDIRECT OR CONSEQUENTIAL DAMAGES OR ANY DAMAGES
WHATSOEVER RESULTING FROM LOSS OF USE, DATA OR PROFITS, WHETHER IN AN
ACTION OF CONTRACT, NEGLIGENCE OR OTHER TORTIOUS ACTION, ARISING OUT
OF OR IN CONNECTION WITH THE USE OR PERFORMANCE OF THIS SOFTWARE.


\section{Licenses and Acknowledgements for Incorporated Software}

This section is an incomplete, but growing list of licenses and
acknowledgements for third-party software incorporated in the
Python distribution.


\subsection{Mersenne Twister}

The \module{_random} module includes code based on a download from
\url{http://www.math.keio.ac.jp/~matumoto/MT2002/emt19937ar.html}.
The following are the verbatim comments from the original code:

\begin{verbatim}
A C-program for MT19937, with initialization improved 2002/1/26.
Coded by Takuji Nishimura and Makoto Matsumoto.

Before using, initialize the state by using init_genrand(seed)
or init_by_array(init_key, key_length).

Copyright (C) 1997 - 2002, Makoto Matsumoto and Takuji Nishimura,
All rights reserved.

Redistribution and use in source and binary forms, with or without
modification, are permitted provided that the following conditions
are met:

 1. Redistributions of source code must retain the above copyright
    notice, this list of conditions and the following disclaimer.

 2. Redistributions in binary form must reproduce the above copyright
    notice, this list of conditions and the following disclaimer in the
    documentation and/or other materials provided with the distribution.

 3. The names of its contributors may not be used to endorse or promote
    products derived from this software without specific prior written
    permission.

THIS SOFTWARE IS PROVIDED BY THE COPYRIGHT HOLDERS AND CONTRIBUTORS
"AS IS" AND ANY EXPRESS OR IMPLIED WARRANTIES, INCLUDING, BUT NOT
LIMITED TO, THE IMPLIED WARRANTIES OF MERCHANTABILITY AND FITNESS FOR
A PARTICULAR PURPOSE ARE DISCLAIMED.  IN NO EVENT SHALL THE COPYRIGHT OWNER OR
CONTRIBUTORS BE LIABLE FOR ANY DIRECT, INDIRECT, INCIDENTAL, SPECIAL,
EXEMPLARY, OR CONSEQUENTIAL DAMAGES (INCLUDING, BUT NOT LIMITED TO,
PROCUREMENT OF SUBSTITUTE GOODS OR SERVICES; LOSS OF USE, DATA, OR
PROFITS; OR BUSINESS INTERRUPTION) HOWEVER CAUSED AND ON ANY THEORY OF
LIABILITY, WHETHER IN CONTRACT, STRICT LIABILITY, OR TORT (INCLUDING
NEGLIGENCE OR OTHERWISE) ARISING IN ANY WAY OUT OF THE USE OF THIS
SOFTWARE, EVEN IF ADVISED OF THE POSSIBILITY OF SUCH DAMAGE.


Any feedback is very welcome.
http://www.math.keio.ac.jp/matumoto/emt.html
email: matumoto@math.keio.ac.jp
\end{verbatim}



\subsection{Sockets}

The \module{socket} module uses the functions, \function{getaddrinfo},
and \function{getnameinfo}, which are coded in separate source files
from the WIDE Project, \url{http://www.wide.ad.jp/about/index.html}.

\begin{verbatim}      
Copyright (C) 1995, 1996, 1997, and 1998 WIDE Project.
All rights reserved.
 
Redistribution and use in source and binary forms, with or without
modification, are permitted provided that the following conditions
are met:
1. Redistributions of source code must retain the above copyright
   notice, this list of conditions and the following disclaimer.
2. Redistributions in binary form must reproduce the above copyright
   notice, this list of conditions and the following disclaimer in the
   documentation and/or other materials provided with the distribution.
3. Neither the name of the project nor the names of its contributors
   may be used to endorse or promote products derived from this software
   without specific prior written permission.

THIS SOFTWARE IS PROVIDED BY THE PROJECT AND CONTRIBUTORS ``AS IS'' AND
GAI_ANY EXPRESS OR IMPLIED WARRANTIES, INCLUDING, BUT NOT LIMITED TO, THE
IMPLIED WARRANTIES OF MERCHANTABILITY AND FITNESS FOR A PARTICULAR PURPOSE
ARE DISCLAIMED.  IN NO EVENT SHALL THE PROJECT OR CONTRIBUTORS BE LIABLE
FOR GAI_ANY DIRECT, INDIRECT, INCIDENTAL, SPECIAL, EXEMPLARY, OR CONSEQUENTIAL
DAMAGES (INCLUDING, BUT NOT LIMITED TO, PROCUREMENT OF SUBSTITUTE GOODS
OR SERVICES; LOSS OF USE, DATA, OR PROFITS; OR BUSINESS INTERRUPTION)
HOWEVER CAUSED AND ON GAI_ANY THEORY OF LIABILITY, WHETHER IN CONTRACT, STRICT
LIABILITY, OR TORT (INCLUDING NEGLIGENCE OR OTHERWISE) ARISING IN GAI_ANY WAY
OUT OF THE USE OF THIS SOFTWARE, EVEN IF ADVISED OF THE POSSIBILITY OF
SUCH DAMAGE.
\end{verbatim}



\subsection{Floating point exception control}

The source for the \module{fpectl} module includes the following notice:

\begin{verbatim}
     ---------------------------------------------------------------------  
    /                       Copyright (c) 1996.                           \ 
   |          The Regents of the University of California.                 |
   |                        All rights reserved.                           |
   |                                                                       |
   |   Permission to use, copy, modify, and distribute this software for   |
   |   any purpose without fee is hereby granted, provided that this en-   |
   |   tire notice is included in all copies of any software which is or   |
   |   includes  a  copy  or  modification  of  this software and in all   |
   |   copies of the supporting documentation for such software.           |
   |                                                                       |
   |   This  work was produced at the University of California, Lawrence   |
   |   Livermore National Laboratory under  contract  no.  W-7405-ENG-48   |
   |   between  the  U.S.  Department  of  Energy and The Regents of the   |
   |   University of California for the operation of UC LLNL.              |
   |                                                                       |
   |                              DISCLAIMER                               |
   |                                                                       |
   |   This  software was prepared as an account of work sponsored by an   |
   |   agency of the United States Government. Neither the United States   |
   |   Government  nor the University of California nor any of their em-   |
   |   ployees, makes any warranty, express or implied, or  assumes  any   |
   |   liability  or  responsibility  for the accuracy, completeness, or   |
   |   usefulness of any information,  apparatus,  product,  or  process   |
   |   disclosed,   or  represents  that  its  use  would  not  infringe   |
   |   privately-owned rights. Reference herein to any specific  commer-   |
   |   cial  products,  process,  or  service  by trade name, trademark,   |
   |   manufacturer, or otherwise, does not  necessarily  constitute  or   |
   |   imply  its endorsement, recommendation, or favoring by the United   |
   |   States Government or the University of California. The views  and   |
   |   opinions  of authors expressed herein do not necessarily state or   |
   |   reflect those of the United States Government or  the  University   |
   |   of  California,  and shall not be used for advertising or product   |
    \  endorsement purposes.                                              / 
     ---------------------------------------------------------------------
\end{verbatim}



\subsection{MD5 message digest algorithm}

The source code for the \module{md5} module contains the following notice:

\begin{verbatim}
  Copyright (C) 1999, 2002 Aladdin Enterprises.  All rights reserved.

  This software is provided 'as-is', without any express or implied
  warranty.  In no event will the authors be held liable for any damages
  arising from the use of this software.

  Permission is granted to anyone to use this software for any purpose,
  including commercial applications, and to alter it and redistribute it
  freely, subject to the following restrictions:

  1. The origin of this software must not be misrepresented; you must not
     claim that you wrote the original software. If you use this software
     in a product, an acknowledgment in the product documentation would be
     appreciated but is not required.
  2. Altered source versions must be plainly marked as such, and must not be
     misrepresented as being the original software.
  3. This notice may not be removed or altered from any source distribution.

  L. Peter Deutsch
  ghost@aladdin.com

  Independent implementation of MD5 (RFC 1321).

  This code implements the MD5 Algorithm defined in RFC 1321, whose
  text is available at
	http://www.ietf.org/rfc/rfc1321.txt
  The code is derived from the text of the RFC, including the test suite
  (section A.5) but excluding the rest of Appendix A.  It does not include
  any code or documentation that is identified in the RFC as being
  copyrighted.

  The original and principal author of md5.h is L. Peter Deutsch
  <ghost@aladdin.com>.  Other authors are noted in the change history
  that follows (in reverse chronological order):

  2002-04-13 lpd Removed support for non-ANSI compilers; removed
	references to Ghostscript; clarified derivation from RFC 1321;
	now handles byte order either statically or dynamically.
  1999-11-04 lpd Edited comments slightly for automatic TOC extraction.
  1999-10-18 lpd Fixed typo in header comment (ansi2knr rather than md5);
	added conditionalization for C++ compilation from Martin
	Purschke <purschke@bnl.gov>.
  1999-05-03 lpd Original version.
\end{verbatim}



\subsection{Asynchronous socket services}

The \module{asynchat} and \module{asyncore} modules contain the
following notice:

\begin{verbatim}      
 Copyright 1996 by Sam Rushing

                         All Rights Reserved

 Permission to use, copy, modify, and distribute this software and
 its documentation for any purpose and without fee is hereby
 granted, provided that the above copyright notice appear in all
 copies and that both that copyright notice and this permission
 notice appear in supporting documentation, and that the name of Sam
 Rushing not be used in advertising or publicity pertaining to
 distribution of the software without specific, written prior
 permission.

 SAM RUSHING DISCLAIMS ALL WARRANTIES WITH REGARD TO THIS SOFTWARE,
 INCLUDING ALL IMPLIED WARRANTIES OF MERCHANTABILITY AND FITNESS, IN
 NO EVENT SHALL SAM RUSHING BE LIABLE FOR ANY SPECIAL, INDIRECT OR
 CONSEQUENTIAL DAMAGES OR ANY DAMAGES WHATSOEVER RESULTING FROM LOSS
 OF USE, DATA OR PROFITS, WHETHER IN AN ACTION OF CONTRACT,
 NEGLIGENCE OR OTHER TORTIOUS ACTION, ARISING OUT OF OR IN
 CONNECTION WITH THE USE OR PERFORMANCE OF THIS SOFTWARE.
\end{verbatim}


\subsection{Cookie management}

The \module{Cookie} module contains the following notice:

\begin{verbatim}
 Copyright 2000 by Timothy O'Malley <timo@alum.mit.edu>

                All Rights Reserved

 Permission to use, copy, modify, and distribute this software
 and its documentation for any purpose and without fee is hereby
 granted, provided that the above copyright notice appear in all
 copies and that both that copyright notice and this permission
 notice appear in supporting documentation, and that the name of
 Timothy O'Malley  not be used in advertising or publicity
 pertaining to distribution of the software without specific, written
 prior permission.

 Timothy O'Malley DISCLAIMS ALL WARRANTIES WITH REGARD TO THIS
 SOFTWARE, INCLUDING ALL IMPLIED WARRANTIES OF MERCHANTABILITY
 AND FITNESS, IN NO EVENT SHALL Timothy O'Malley BE LIABLE FOR
 ANY SPECIAL, INDIRECT OR CONSEQUENTIAL DAMAGES OR ANY DAMAGES
 WHATSOEVER RESULTING FROM LOSS OF USE, DATA OR PROFITS,
 WHETHER IN AN ACTION OF CONTRACT, NEGLIGENCE OR OTHER TORTIOUS
 ACTION, ARISING OUT OF OR IN CONNECTION WITH THE USE OR
 PERFORMANCE OF THIS SOFTWARE.
\end{verbatim}      



\subsection{Profiling}

The \module{profile} and \module{pstats} modules contain
the following notice:

\begin{verbatim}
 Copyright 1994, by InfoSeek Corporation, all rights reserved.
 Written by James Roskind

 Permission to use, copy, modify, and distribute this Python software
 and its associated documentation for any purpose (subject to the
 restriction in the following sentence) without fee is hereby granted,
 provided that the above copyright notice appears in all copies, and
 that both that copyright notice and this permission notice appear in
 supporting documentation, and that the name of InfoSeek not be used in
 advertising or publicity pertaining to distribution of the software
 without specific, written prior permission.  This permission is
 explicitly restricted to the copying and modification of the software
 to remain in Python, compiled Python, or other languages (such as C)
 wherein the modified or derived code is exclusively imported into a
 Python module.

 INFOSEEK CORPORATION DISCLAIMS ALL WARRANTIES WITH REGARD TO THIS
 SOFTWARE, INCLUDING ALL IMPLIED WARRANTIES OF MERCHANTABILITY AND
 FITNESS. IN NO EVENT SHALL INFOSEEK CORPORATION BE LIABLE FOR ANY
 SPECIAL, INDIRECT OR CONSEQUENTIAL DAMAGES OR ANY DAMAGES WHATSOEVER
 RESULTING FROM LOSS OF USE, DATA OR PROFITS, WHETHER IN AN ACTION OF
 CONTRACT, NEGLIGENCE OR OTHER TORTIOUS ACTION, ARISING OUT OF OR IN
 CONNECTION WITH THE USE OR PERFORMANCE OF THIS SOFTWARE.
\end{verbatim}



\subsection{Execution tracing}

The \module{trace} module contains the following notice:

\begin{verbatim}
 portions copyright 2001, Autonomous Zones Industries, Inc., all rights...
 err...  reserved and offered to the public under the terms of the
 Python 2.2 license.
 Author: Zooko O'Whielacronx
 http://zooko.com/
 mailto:zooko@zooko.com

 Copyright 2000, Mojam Media, Inc., all rights reserved.
 Author: Skip Montanaro

 Copyright 1999, Bioreason, Inc., all rights reserved.
 Author: Andrew Dalke

 Copyright 1995-1997, Automatrix, Inc., all rights reserved.
 Author: Skip Montanaro

 Copyright 1991-1995, Stichting Mathematisch Centrum, all rights reserved.


 Permission to use, copy, modify, and distribute this Python software and
 its associated documentation for any purpose without fee is hereby
 granted, provided that the above copyright notice appears in all copies,
 and that both that copyright notice and this permission notice appear in
 supporting documentation, and that the name of neither Automatrix,
 Bioreason or Mojam Media be used in advertising or publicity pertaining to
 distribution of the software without specific, written prior permission.
\end{verbatim} 



\subsection{UUencode and UUdecode functions}

The \module{uu} module contains the following notice:

\begin{verbatim}
 Copyright 1994 by Lance Ellinghouse
 Cathedral City, California Republic, United States of America.
                        All Rights Reserved
 Permission to use, copy, modify, and distribute this software and its
 documentation for any purpose and without fee is hereby granted,
 provided that the above copyright notice appear in all copies and that
 both that copyright notice and this permission notice appear in
 supporting documentation, and that the name of Lance Ellinghouse
 not be used in advertising or publicity pertaining to distribution
 of the software without specific, written prior permission.
 LANCE ELLINGHOUSE DISCLAIMS ALL WARRANTIES WITH REGARD TO
 THIS SOFTWARE, INCLUDING ALL IMPLIED WARRANTIES OF MERCHANTABILITY AND
 FITNESS, IN NO EVENT SHALL LANCE ELLINGHOUSE CENTRUM BE LIABLE
 FOR ANY SPECIAL, INDIRECT OR CONSEQUENTIAL DAMAGES OR ANY DAMAGES
 WHATSOEVER RESULTING FROM LOSS OF USE, DATA OR PROFITS, WHETHER IN AN
 ACTION OF CONTRACT, NEGLIGENCE OR OTHER TORTIOUS ACTION, ARISING OUT
 OF OR IN CONNECTION WITH THE USE OR PERFORMANCE OF THIS SOFTWARE.

 Modified by Jack Jansen, CWI, July 1995:
 - Use binascii module to do the actual line-by-line conversion
   between ascii and binary. This results in a 1000-fold speedup. The C
   version is still 5 times faster, though.
 - Arguments more compliant with python standard
\end{verbatim}



\subsection{XML Remote Procedure Calls}

The \module{xmlrpclib} module contains the following notice:

\begin{verbatim}
     The XML-RPC client interface is

 Copyright (c) 1999-2002 by Secret Labs AB
 Copyright (c) 1999-2002 by Fredrik Lundh

 By obtaining, using, and/or copying this software and/or its
 associated documentation, you agree that you have read, understood,
 and will comply with the following terms and conditions:

 Permission to use, copy, modify, and distribute this software and
 its associated documentation for any purpose and without fee is
 hereby granted, provided that the above copyright notice appears in
 all copies, and that both that copyright notice and this permission
 notice appear in supporting documentation, and that the name of
 Secret Labs AB or the author not be used in advertising or publicity
 pertaining to distribution of the software without specific, written
 prior permission.

 SECRET LABS AB AND THE AUTHOR DISCLAIMS ALL WARRANTIES WITH REGARD
 TO THIS SOFTWARE, INCLUDING ALL IMPLIED WARRANTIES OF MERCHANT-
 ABILITY AND FITNESS.  IN NO EVENT SHALL SECRET LABS AB OR THE AUTHOR
 BE LIABLE FOR ANY SPECIAL, INDIRECT OR CONSEQUENTIAL DAMAGES OR ANY
 DAMAGES WHATSOEVER RESULTING FROM LOSS OF USE, DATA OR PROFITS,
 WHETHER IN AN ACTION OF CONTRACT, NEGLIGENCE OR OTHER TORTIOUS
 ACTION, ARISING OUT OF OR IN CONNECTION WITH THE USE OR PERFORMANCE
 OF THIS SOFTWARE.
\end{verbatim}


\documentclass{manual}

\title{Python Reference Manual}

\author{Guido van Rossum\\
	Fred L. Drake, Jr., editor}
\authoraddress{
	\strong{Python Software Foundation}\\
	Email: \email{docs@python.org}
}

\date{23rd December, 2008}			% XXX update before final release!
\input{patchlevel}		% include Python version information


\makeindex

\begin{document}

\maketitle

\ifhtml
\chapter*{Front Matter\label{front}}
\fi

Copyright \copyright{} 2001-2008 Python Software Foundation.
All rights reserved.

Copyright \copyright{} 2000 BeOpen.com.
All rights reserved.

Copyright \copyright{} 1995-2000 Corporation for National Research Initiatives.
All rights reserved.

Copyright \copyright{} 1991-1995 Stichting Mathematisch Centrum.
All rights reserved.

See the end of this document for complete license and permissions
information.


\begin{abstract}

\noindent
Python is an interpreted, object-oriented, high-level programming
language with dynamic semantics.  Its high-level built in data
structures, combined with dynamic typing and dynamic binding, make it
very attractive for rapid application development, as well as for use
as a scripting or glue language to connect existing components
together.  Python's simple, easy to learn syntax emphasizes
readability and therefore reduces the cost of program
maintenance.  Python supports modules and packages, which encourages
program modularity and code reuse.  The Python interpreter and the
extensive standard library are available in source or binary form
without charge for all major platforms, and can be freely distributed.

This reference manual describes the syntax and ``core semantics'' of
the language.  It is terse, but attempts to be exact and complete.
The semantics of non-essential built-in object types and of the
built-in functions and modules are described in the
\citetitle[../lib/lib.html]{Python Library Reference}.  For an
informal introduction to the language, see the
\citetitle[../tut/tut.html]{Python Tutorial}.  For C or
\Cpp{} programmers, two additional manuals exist:
\citetitle[../ext/ext.html]{Extending and Embedding the Python
Interpreter} describes the high-level picture of how to write a Python
extension module, and the \citetitle[../api/api.html]{Python/C API
Reference Manual} describes the interfaces available to
C/\Cpp{} programmers in detail.

\end{abstract}

\tableofcontents

\chapter{Introduction\label{introduction}}

This reference manual describes the Python programming language.
It is not intended as a tutorial.

While I am trying to be as precise as possible, I chose to use English
rather than formal specifications for everything except syntax and
lexical analysis.  This should make the document more understandable
to the average reader, but will leave room for ambiguities.
Consequently, if you were coming from Mars and tried to re-implement
Python from this document alone, you might have to guess things and in
fact you would probably end up implementing quite a different language.
On the other hand, if you are using
Python and wonder what the precise rules about a particular area of
the language are, you should definitely be able to find them here.
If you would like to see a more formal definition of the language,
maybe you could volunteer your time --- or invent a cloning machine
:-).

It is dangerous to add too many implementation details to a language
reference document --- the implementation may change, and other
implementations of the same language may work differently.  On the
other hand, there is currently only one Python implementation in
widespread use (although alternate implementations exist), and
its particular quirks are sometimes worth being mentioned, especially
where the implementation imposes additional limitations.  Therefore,
you'll find short ``implementation notes'' sprinkled throughout the
text.

Every Python implementation comes with a number of built-in and
standard modules.  These are not documented here, but in the separate
\citetitle[../lib/lib.html]{Python Library Reference} document.  A few
built-in modules are mentioned when they interact in a significant way
with the language definition.


\section{Alternate Implementations\label{implementations}}

Though there is one Python implementation which is by far the most
popular, there are some alternate implementations which are of
particular interest to different audiences.

Known implementations include:

\begin{itemize}
\item[CPython]
This is the original and most-maintained implementation of Python,
written in C.  New language features generally appear here first.

\item[Jython]
Python implemented in Java.  This implementation can be used as a
scripting language for Java applications, or can be used to create
applications using the Java class libraries.  It is also often used to
create tests for Java libraries.  More information can be found at
\ulink{the Jython website}{http://www.jython.org/}.

\item[Python for .NET]
This implementation actually uses the CPython implementation, but is a
managed .NET application and makes .NET libraries available.  This was
created by Brian Lloyd.  For more information, see the \ulink{Python
for .NET home page}{http://www.zope.org/Members/Brian/PythonNet}.

\item[IronPython]
An alternate Python for\ .NET.  Unlike Python.NET, this is a complete
Python implementation that generates IL, and compiles Python code
directly to\ .NET assemblies.  It was created by Jim Hugunin, the
original creator of Jython.  For more information, see \ulink{the
IronPython website}{http://workspaces.gotdotnet.com/ironpython}.

\item[PyPy]
An implementation of Python written in Python; even the bytecode
interpreter is written in Python.  This is executed using CPython as
the underlying interpreter.  One of the goals of the project is to
encourage experimentation with the language itself by making it easier
to modify the interpreter (since it is written in Python).  Additional
information is available on \ulink{the PyPy project's home
page}{http://codespeak.net/pypy/}.
\end{itemize}

Each of these implementations varies in some way from the language as
documented in this manual, or introduces specific information beyond
what's covered in the standard Python documentation.  Please refer to
the implementation-specific documentation to determine what else you
need to know about the specific implementation you're using.


\section{Notation\label{notation}}

The descriptions of lexical analysis and syntax use a modified BNF
grammar notation.  This uses the following style of definition:
\index{BNF}
\index{grammar}
\index{syntax}
\index{notation}

\begin{productionlist}[*]
  \production{name}{\token{lc_letter} (\token{lc_letter} | "_")*}
  \production{lc_letter}{"a"..."z"}
\end{productionlist}

The first line says that a \code{name} is an \code{lc_letter} followed by
a sequence of zero or more \code{lc_letter}s and underscores.  An
\code{lc_letter} in turn is any of the single characters \character{a}
through \character{z}.  (This rule is actually adhered to for the
names defined in lexical and grammar rules in this document.)

Each rule begins with a name (which is the name defined by the rule)
and \code{::=}.  A vertical bar (\code{|}) is used to separate
alternatives; it is the least binding operator in this notation.  A
star (\code{*}) means zero or more repetitions of the preceding item;
likewise, a plus (\code{+}) means one or more repetitions, and a
phrase enclosed in square brackets (\code{[ ]}) means zero or one
occurrences (in other words, the enclosed phrase is optional).  The
\code{*} and \code{+} operators bind as tightly as possible;
parentheses are used for grouping.  Literal strings are enclosed in
quotes.  White space is only meaningful to separate tokens.
Rules are normally contained on a single line; rules with many
alternatives may be formatted alternatively with each line after the
first beginning with a vertical bar.

In lexical definitions (as the example above), two more conventions
are used: Two literal characters separated by three dots mean a choice
of any single character in the given (inclusive) range of \ASCII{}
characters.  A phrase between angular brackets (\code{<...>}) gives an
informal description of the symbol defined; e.g., this could be used
to describe the notion of `control character' if needed.
\index{lexical definitions}
\index{ASCII@\ASCII}

Even though the notation used is almost the same, there is a big
difference between the meaning of lexical and syntactic definitions:
a lexical definition operates on the individual characters of the
input source, while a syntax definition operates on the stream of
tokens generated by the lexical analysis.  All uses of BNF in the next
chapter (``Lexical Analysis'') are lexical definitions; uses in
subsequent chapters are syntactic definitions.
		% Introduction
\chapter{Lexical analysis\label{lexical}}

A Python program is read by a \emph{parser}.  Input to the parser is a
stream of \emph{tokens}, generated by the \emph{lexical analyzer}.  This
chapter describes how the lexical analyzer breaks a file into tokens.
\index{lexical analysis}
\index{parser}
\index{token}

Python uses the 7-bit \ASCII{} character set for program text.
\versionadded[An encoding declaration can be used to indicate that 
string literals and comments use an encoding different from ASCII]{2.3}
For compatibility with older versions, Python only warns if it finds
8-bit characters; those warnings should be corrected by either declaring
an explicit encoding, or using escape sequences if those bytes are binary
data, instead of characters.


The run-time character set depends on the I/O devices connected to the
program but is generally a superset of \ASCII.

\strong{Future compatibility note:} It may be tempting to assume that the
character set for 8-bit characters is ISO Latin-1 (an \ASCII{}
superset that covers most western languages that use the Latin
alphabet), but it is possible that in the future Unicode text editors
will become common.  These generally use the UTF-8 encoding, which is
also an \ASCII{} superset, but with very different use for the
characters with ordinals 128-255.  While there is no consensus on this
subject yet, it is unwise to assume either Latin-1 or UTF-8, even
though the current implementation appears to favor Latin-1.  This
applies both to the source character set and the run-time character
set.


\section{Line structure\label{line-structure}}

A Python program is divided into a number of \emph{logical lines}.
\index{line structure}


\subsection{Logical lines\label{logical}}

The end of
a logical line is represented by the token NEWLINE.  Statements cannot
cross logical line boundaries except where NEWLINE is allowed by the
syntax (e.g., between statements in compound statements).
A logical line is constructed from one or more \emph{physical lines}
by following the explicit or implicit \emph{line joining} rules.
\index{logical line}
\index{physical line}
\index{line joining}
\index{NEWLINE token}


\subsection{Physical lines\label{physical}}

A physical line is a sequence of characters terminated by an end-of-line
sequence.  In source files, any of the standard platform line
termination sequences can be used - the \UNIX{} form using \ASCII{} LF
(linefeed), the Windows form using the \ASCII{} sequence CR LF (return
followed by linefeed), or the Macintosh form using the \ASCII{} CR
(return) character.  All of these forms can be used equally, regardless
of platform.

When embedding Python, source code strings should be passed to Python
APIs using the standard C conventions for newline characters (the
\code{\e n} character, representing \ASCII{} LF, is the line
terminator).


\subsection{Comments\label{comments}}

A comment starts with a hash character (\code{\#}) that is not part of
a string literal, and ends at the end of the physical line.  A comment
signifies the end of the logical line unless the implicit line joining
rules are invoked.
Comments are ignored by the syntax; they are not tokens.
\index{comment}
\index{hash character}


\subsection{Encoding declarations\label{encodings}}
\index{source character set}
\index{encodings}

If a comment in the first or second line of the Python script matches
the regular expression \regexp{coding[=:]\e s*([-\e w.]+)}, this comment is
processed as an encoding declaration; the first group of this
expression names the encoding of the source code file. The recommended
forms of this expression are

\begin{verbatim}
# -*- coding: <encoding-name> -*-
\end{verbatim}

which is recognized also by GNU Emacs, and

\begin{verbatim}
# vim:fileencoding=<encoding-name>
\end{verbatim}

which is recognized by Bram Moolenaar's VIM. In addition, if the first
bytes of the file are the UTF-8 byte-order mark
(\code{'\e xef\e xbb\e xbf'}), the declared file encoding is UTF-8
(this is supported, among others, by Microsoft's \program{notepad}).

If an encoding is declared, the encoding name must be recognized by
Python. % XXX there should be a list of supported encodings.
The encoding is used for all lexical analysis, in particular to find
the end of a string, and to interpret the contents of Unicode literals.
String literals are converted to Unicode for syntactical analysis,
then converted back to their original encoding before interpretation
starts. The encoding declaration must appear on a line of its own.

\subsection{Explicit line joining\label{explicit-joining}}

Two or more physical lines may be joined into logical lines using
backslash characters (\code{\e}), as follows: when a physical line ends
in a backslash that is not part of a string literal or comment, it is
joined with the following forming a single logical line, deleting the
backslash and the following end-of-line character.  For example:
\index{physical line}
\index{line joining}
\index{line continuation}
\index{backslash character}
%
\begin{verbatim}
if 1900 < year < 2100 and 1 <= month <= 12 \
   and 1 <= day <= 31 and 0 <= hour < 24 \
   and 0 <= minute < 60 and 0 <= second < 60:   # Looks like a valid date
        return 1
\end{verbatim}

A line ending in a backslash cannot carry a comment.  A backslash does
not continue a comment.  A backslash does not continue a token except
for string literals (i.e., tokens other than string literals cannot be
split across physical lines using a backslash).  A backslash is
illegal elsewhere on a line outside a string literal.


\subsection{Implicit line joining\label{implicit-joining}}

Expressions in parentheses, square brackets or curly braces can be
split over more than one physical line without using backslashes.
For example:

\begin{verbatim}
month_names = ['Januari', 'Februari', 'Maart',      # These are the
               'April',   'Mei',      'Juni',       # Dutch names
               'Juli',    'Augustus', 'September',  # for the months
               'Oktober', 'November', 'December']   # of the year
\end{verbatim}

Implicitly continued lines can carry comments.  The indentation of the
continuation lines is not important.  Blank continuation lines are
allowed.  There is no NEWLINE token between implicit continuation
lines.  Implicitly continued lines can also occur within triple-quoted
strings (see below); in that case they cannot carry comments.


\subsection{Blank lines \label{blank-lines}}

\index{blank line}
A logical line that contains only spaces, tabs, formfeeds and possibly
a comment, is ignored (i.e., no NEWLINE token is generated).  During
interactive input of statements, handling of a blank line may differ
depending on the implementation of the read-eval-print loop.  In the
standard implementation, an entirely blank logical line (i.e.\ one
containing not even whitespace or a comment) terminates a multi-line
statement.


\subsection{Indentation\label{indentation}}

Leading whitespace (spaces and tabs) at the beginning of a logical
line is used to compute the indentation level of the line, which in
turn is used to determine the grouping of statements.
\index{indentation}
\index{whitespace}
\index{leading whitespace}
\index{space}
\index{tab}
\index{grouping}
\index{statement grouping}

First, tabs are replaced (from left to right) by one to eight spaces
such that the total number of characters up to and including the
replacement is a multiple of
eight (this is intended to be the same rule as used by \UNIX).  The
total number of spaces preceding the first non-blank character then
determines the line's indentation.  Indentation cannot be split over
multiple physical lines using backslashes; the whitespace up to the
first backslash determines the indentation.

\strong{Cross-platform compatibility note:} because of the nature of
text editors on non-UNIX platforms, it is unwise to use a mixture of
spaces and tabs for the indentation in a single source file.  It
should also be noted that different platforms may explicitly limit the
maximum indentation level.

A formfeed character may be present at the start of the line; it will
be ignored for the indentation calculations above.  Formfeed
characters occurring elsewhere in the leading whitespace have an
undefined effect (for instance, they may reset the space count to
zero).

The indentation levels of consecutive lines are used to generate
INDENT and DEDENT tokens, using a stack, as follows.
\index{INDENT token}
\index{DEDENT token}

Before the first line of the file is read, a single zero is pushed on
the stack; this will never be popped off again.  The numbers pushed on
the stack will always be strictly increasing from bottom to top.  At
the beginning of each logical line, the line's indentation level is
compared to the top of the stack.  If it is equal, nothing happens.
If it is larger, it is pushed on the stack, and one INDENT token is
generated.  If it is smaller, it \emph{must} be one of the numbers
occurring on the stack; all numbers on the stack that are larger are
popped off, and for each number popped off a DEDENT token is
generated.  At the end of the file, a DEDENT token is generated for
each number remaining on the stack that is larger than zero.

Here is an example of a correctly (though confusingly) indented piece
of Python code:

\begin{verbatim}
def perm(l):
        # Compute the list of all permutations of l
    if len(l) <= 1:
                  return [l]
    r = []
    for i in range(len(l)):
             s = l[:i] + l[i+1:]
             p = perm(s)
             for x in p:
              r.append(l[i:i+1] + x)
    return r
\end{verbatim}

The following example shows various indentation errors:

\begin{verbatim}
 def perm(l):                       # error: first line indented
for i in range(len(l)):             # error: not indented
    s = l[:i] + l[i+1:]
        p = perm(l[:i] + l[i+1:])   # error: unexpected indent
        for x in p:
                r.append(l[i:i+1] + x)
            return r                # error: inconsistent dedent
\end{verbatim}

(Actually, the first three errors are detected by the parser; only the
last error is found by the lexical analyzer --- the indentation of
\code{return r} does not match a level popped off the stack.)


\subsection{Whitespace between tokens\label{whitespace}}

Except at the beginning of a logical line or in string literals, the
whitespace characters space, tab and formfeed can be used
interchangeably to separate tokens.  Whitespace is needed between two
tokens only if their concatenation could otherwise be interpreted as a
different token (e.g., ab is one token, but a b is two tokens).


\section{Other tokens\label{other-tokens}}

Besides NEWLINE, INDENT and DEDENT, the following categories of tokens
exist: \emph{identifiers}, \emph{keywords}, \emph{literals},
\emph{operators}, and \emph{delimiters}.
Whitespace characters (other than line terminators, discussed earlier)
are not tokens, but serve to delimit tokens.
Where
ambiguity exists, a token comprises the longest possible string that
forms a legal token, when read from left to right.


\section{Identifiers and keywords\label{identifiers}}

Identifiers (also referred to as \emph{names}) are described by the following
lexical definitions:
\index{identifier}
\index{name}

\begin{productionlist}
  \production{identifier}
             {(\token{letter}|"_") (\token{letter} | \token{digit} | "_")*}
  \production{letter}
             {\token{lowercase} | \token{uppercase}}
  \production{lowercase}
             {"a"..."z"}
  \production{uppercase}
             {"A"..."Z"}
  \production{digit}
             {"0"..."9"}
\end{productionlist}

Identifiers are unlimited in length.  Case is significant.


\subsection{Keywords\label{keywords}}

The following identifiers are used as reserved words, or
\emph{keywords} of the language, and cannot be used as ordinary
identifiers.  They must be spelled exactly as written here:%
\index{keyword}%
\index{reserved word}

\begin{verbatim}
and       del       from      not       while    
as        elif      global    or        with     
assert    else      if        pass      yield    
break     except    import    print              
class     exec      in        raise              
continue  finally   is        return             
def       for       lambda    try 
\end{verbatim}

% When adding keywords, use reswords.py for reformatting

\versionchanged[\constant{None} became a constant and is now
recognized by the compiler as a name for the built-in object
\constant{None}.  Although it is not a keyword, you cannot assign
a different object to it]{2.4}

\versionchanged[Both \keyword{as} and \keyword{with} are only recognized
when the \code{with_statement} future feature has been enabled.
It will always be enabled in Python 2.6.  See section~\ref{with} for
details.  Note that using \keyword{as} and \keyword{with} as identifiers
will always issue a warning, even when the \code{with_statement} future
directive is not in effect]{2.5}


\subsection{Reserved classes of identifiers\label{id-classes}}

Certain classes of identifiers (besides keywords) have special
meanings.  These classes are identified by the patterns of leading and
trailing underscore characters:

\begin{description}

\item[\code{_*}]
  Not imported by \samp{from \var{module} import *}.  The special
  identifier \samp{_} is used in the interactive interpreter to store
  the result of the last evaluation; it is stored in the
  \module{__builtin__} module.  When not in interactive mode, \samp{_}
  has no special meaning and is not defined.
  See section~\ref{import}, ``The \keyword{import} statement.''

  \note{The name \samp{_} is often used in conjunction with
  internationalization; refer to the documentation for the
  \ulink{\module{gettext} module}{../lib/module-gettext.html} for more
  information on this convention.}

\item[\code{__*__}]
  System-defined names.  These names are defined by the interpreter
  and its implementation (including the standard library);
  applications should not expect to define additional names using this
  convention.  The set of names of this class defined by Python may be
  extended in future versions.
  See section~\ref{specialnames}, ``Special method names.''

\item[\code{__*}]
  Class-private names.  Names in this category, when used within the
  context of a class definition, are re-written to use a mangled form
  to help avoid name clashes between ``private'' attributes of base
  and derived classes.
  See section~\ref{atom-identifiers}, ``Identifiers (Names).''

\end{description}


\section{Literals\label{literals}}

Literals are notations for constant values of some built-in types.
\index{literal}
\index{constant}


\subsection{String literals\label{strings}}

String literals are described by the following lexical definitions:
\index{string literal}

\index{ASCII@\ASCII}
\begin{productionlist}
  \production{stringliteral}
             {[\token{stringprefix}](\token{shortstring} | \token{longstring})}
  \production{stringprefix}
             {"r" | "u" | "ur" | "R" | "U" | "UR" | "Ur" | "uR"}
  \production{shortstring}
             {"'" \token{shortstringitem}* "'"
              | '"' \token{shortstringitem}* '"'}
  \production{longstring}
             {"'''" \token{longstringitem}* "'''"}
  \productioncont{| '"""' \token{longstringitem}* '"""'}
  \production{shortstringitem}
             {\token{shortstringchar} | \token{escapeseq}}
  \production{longstringitem}
             {\token{longstringchar} | \token{escapeseq}}
  \production{shortstringchar}
             {<any source character except "\e" or newline or the quote>}
  \production{longstringchar}
             {<any source character except "\e">}
  \production{escapeseq}
             {"\e" <any ASCII character>}
\end{productionlist}

One syntactic restriction not indicated by these productions is that
whitespace is not allowed between the \grammartoken{stringprefix} and
the rest of the string literal. The source character set is defined
by the encoding declaration; it is \ASCII{} if no encoding declaration
is given in the source file; see section~\ref{encodings}.

\index{triple-quoted string}
\index{Unicode Consortium}
\index{string!Unicode}
In plain English: String literals can be enclosed in matching single
quotes (\code{'}) or double quotes (\code{"}).  They can also be
enclosed in matching groups of three single or double quotes (these
are generally referred to as \emph{triple-quoted strings}).  The
backslash (\code{\e}) character is used to escape characters that
otherwise have a special meaning, such as newline, backslash itself,
or the quote character.  String literals may optionally be prefixed
with a letter \character{r} or \character{R}; such strings are called
\dfn{raw strings}\index{raw string} and use different rules for interpreting
backslash escape sequences.  A prefix of \character{u} or \character{U}
makes the string a Unicode string.  Unicode strings use the Unicode character
set as defined by the Unicode Consortium and ISO~10646.  Some additional
escape sequences, described below, are available in Unicode strings.
The two prefix characters may be combined; in this case, \character{u} must
appear before \character{r}.

In triple-quoted strings,
unescaped newlines and quotes are allowed (and are retained), except
that three unescaped quotes in a row terminate the string.  (A
``quote'' is the character used to open the string, i.e. either
\code{'} or \code{"}.)

Unless an \character{r} or \character{R} prefix is present, escape
sequences in strings are interpreted according to rules similar
to those used by Standard C.  The recognized escape sequences are:
\index{physical line}
\index{escape sequence}
\index{Standard C}
\index{C}

\begin{tableiii}{l|l|c}{code}{Escape Sequence}{Meaning}{Notes}
\lineiii{\e\var{newline}} {Ignored}{}
\lineiii{\e\e}	{Backslash (\code{\e})}{}
\lineiii{\e'}	{Single quote (\code{'})}{}
\lineiii{\e"}	{Double quote (\code{"})}{}
\lineiii{\e a}	{\ASCII{} Bell (BEL)}{}
\lineiii{\e b}	{\ASCII{} Backspace (BS)}{}
\lineiii{\e f}	{\ASCII{} Formfeed (FF)}{}
\lineiii{\e n}	{\ASCII{} Linefeed (LF)}{}
\lineiii{\e N\{\var{name}\}}
        {Character named \var{name} in the Unicode database (Unicode only)}{}
\lineiii{\e r}	{\ASCII{} Carriage Return (CR)}{}
\lineiii{\e t}	{\ASCII{} Horizontal Tab (TAB)}{}
\lineiii{\e u\var{xxxx}}
        {Character with 16-bit hex value \var{xxxx} (Unicode only)}{(1)}
\lineiii{\e U\var{xxxxxxxx}}
        {Character with 32-bit hex value \var{xxxxxxxx} (Unicode only)}{(2)}
\lineiii{\e v}	{\ASCII{} Vertical Tab (VT)}{}
\lineiii{\e\var{ooo}} {Character with octal value \var{ooo}}{(3,5)}
\lineiii{\e x\var{hh}} {Character with hex value \var{hh}}{(4,5)}
\end{tableiii}
\index{ASCII@\ASCII}

\noindent
Notes:

\begin{itemize}
\item[(1)]
  Individual code units which form parts of a surrogate pair can be
  encoded using this escape sequence.
\item[(2)]
  Any Unicode character can be encoded this way, but characters
  outside the Basic Multilingual Plane (BMP) will be encoded using a
  surrogate pair if Python is compiled to use 16-bit code units (the
  default).  Individual code units which form parts of a surrogate
  pair can be encoded using this escape sequence.
\item[(3)]
  As in Standard C, up to three octal digits are accepted.
\item[(4)]
  Unlike in Standard C, exactly two hex digits are required.
\item[(5)]
  In a string literal, hexadecimal and octal escapes denote the
  byte with the given value; it is not necessary that the byte
  encodes a character in the source character set. In a Unicode
  literal, these escapes denote a Unicode character with the given
  value.
\end{itemize}


Unlike Standard \index{unrecognized escape sequence}C,
all unrecognized escape sequences are left in the string unchanged,
i.e., \emph{the backslash is left in the string}.  (This behavior is
useful when debugging: if an escape sequence is mistyped, the
resulting output is more easily recognized as broken.)  It is also
important to note that the escape sequences marked as ``(Unicode
only)'' in the table above fall into the category of unrecognized
escapes for non-Unicode string literals.

When an \character{r} or \character{R} prefix is present, a character
following a backslash is included in the string without change, and \emph{all
backslashes are left in the string}.  For example, the string literal
\code{r"\e n"} consists of two characters: a backslash and a lowercase
\character{n}.  String quotes can be escaped with a backslash, but the
backslash remains in the string; for example, \code{r"\e""} is a valid string
literal consisting of two characters: a backslash and a double quote;
\code{r"\e"} is not a valid string literal (even a raw string cannot
end in an odd number of backslashes).  Specifically, \emph{a raw
string cannot end in a single backslash} (since the backslash would
escape the following quote character).  Note also that a single
backslash followed by a newline is interpreted as those two characters
as part of the string, \emph{not} as a line continuation.

When an \character{r} or \character{R} prefix is used in conjunction
with a \character{u} or \character{U} prefix, then the \code{\e uXXXX}
and \code{\e UXXXXXXXX} escape sequences are processed while 
\emph{all other backslashes are left in the string}.
For example, the string literal
\code{ur"\e{}u0062\e n"} consists of three Unicode characters: `LATIN
SMALL LETTER B', `REVERSE SOLIDUS', and `LATIN SMALL LETTER N'.
Backslashes can be escaped with a preceding backslash; however, both
remain in the string.  As a result, \code{\e uXXXX} escape sequences
are only recognized when there are an odd number of backslashes.

\subsection{String literal concatenation\label{string-catenation}}

Multiple adjacent string literals (delimited by whitespace), possibly
using different quoting conventions, are allowed, and their meaning is
the same as their concatenation.  Thus, \code{"hello" 'world'} is
equivalent to \code{"helloworld"}.  This feature can be used to reduce
the number of backslashes needed, to split long strings conveniently
across long lines, or even to add comments to parts of strings, for
example:

\begin{verbatim}
re.compile("[A-Za-z_]"       # letter or underscore
           "[A-Za-z0-9_]*"   # letter, digit or underscore
          )
\end{verbatim}

Note that this feature is defined at the syntactical level, but
implemented at compile time.  The `+' operator must be used to
concatenate string expressions at run time.  Also note that literal
concatenation can use different quoting styles for each component
(even mixing raw strings and triple quoted strings).


\subsection{Numeric literals\label{numbers}}

There are four types of numeric literals: plain integers, long
integers, floating point numbers, and imaginary numbers.  There are no
complex literals (complex numbers can be formed by adding a real
number and an imaginary number).
\index{number}
\index{numeric literal}
\index{integer literal}
\index{plain integer literal}
\index{long integer literal}
\index{floating point literal}
\index{hexadecimal literal}
\index{octal literal}
\index{decimal literal}
\index{imaginary literal}
\index{complex!literal}

Note that numeric literals do not include a sign; a phrase like
\code{-1} is actually an expression composed of the unary operator
`\code{-}' and the literal \code{1}.


\subsection{Integer and long integer literals\label{integers}}

Integer and long integer literals are described by the following
lexical definitions:

\begin{productionlist}
  \production{longinteger}
             {\token{integer} ("l" | "L")}
  \production{integer}
             {\token{decimalinteger} | \token{octinteger} | \token{hexinteger}}
  \production{decimalinteger}
             {\token{nonzerodigit} \token{digit}* | "0"}
  \production{octinteger}
             {"0" \token{octdigit}+}
  \production{hexinteger}
             {"0" ("x" | "X") \token{hexdigit}+}
  \production{nonzerodigit}
             {"1"..."9"}
  \production{octdigit}
             {"0"..."7"}
  \production{hexdigit}
             {\token{digit} | "a"..."f" | "A"..."F"}
\end{productionlist}

Although both lower case \character{l} and upper case \character{L} are
allowed as suffix for long integers, it is strongly recommended to always
use \character{L}, since the letter \character{l} looks too much like the
digit \character{1}.

Plain integer literals that are above the largest representable plain
integer (e.g., 2147483647 when using 32-bit arithmetic) are accepted
as if they were long integers instead.\footnote{In versions of Python
prior to 2.4, octal and hexadecimal literals in the range just above
the largest representable plain integer but below the largest unsigned
32-bit number (on a machine using 32-bit arithmetic), 4294967296, were
taken as the negative plain integer obtained by subtracting 4294967296
from their unsigned value.}  There is no limit for long integer
literals apart from what can be stored in available memory.

Some examples of plain integer literals (first row) and long integer
literals (second and third rows):

\begin{verbatim}
7     2147483647                        0177
3L    79228162514264337593543950336L    0377L   0x100000000L
      79228162514264337593543950336             0xdeadbeef						    
\end{verbatim}


\subsection{Floating point literals\label{floating}}

Floating point literals are described by the following lexical
definitions:

\begin{productionlist}
  \production{floatnumber}
             {\token{pointfloat} | \token{exponentfloat}}
  \production{pointfloat}
             {[\token{intpart}] \token{fraction} | \token{intpart} "."}
  \production{exponentfloat}
             {(\token{intpart} | \token{pointfloat})
              \token{exponent}}
  \production{intpart}
             {\token{digit}+}
  \production{fraction}
             {"." \token{digit}+}
  \production{exponent}
             {("e" | "E") ["+" | "-"] \token{digit}+}
\end{productionlist}

Note that the integer and exponent parts of floating point numbers
can look like octal integers, but are interpreted using radix 10.  For
example, \samp{077e010} is legal, and denotes the same number
as \samp{77e10}.
The allowed range of floating point literals is
implementation-dependent.
Some examples of floating point literals:

\begin{verbatim}
3.14    10.    .001    1e100    3.14e-10    0e0
\end{verbatim}

Note that numeric literals do not include a sign; a phrase like
\code{-1} is actually an expression composed of the unary operator
\code{-} and the literal \code{1}.


\subsection{Imaginary literals\label{imaginary}}

Imaginary literals are described by the following lexical definitions:

\begin{productionlist}
  \production{imagnumber}{(\token{floatnumber} | \token{intpart}) ("j" | "J")}
\end{productionlist}

An imaginary literal yields a complex number with a real part of
0.0.  Complex numbers are represented as a pair of floating point
numbers and have the same restrictions on their range.  To create a
complex number with a nonzero real part, add a floating point number
to it, e.g., \code{(3+4j)}.  Some examples of imaginary literals:

\begin{verbatim}
3.14j   10.j    10j     .001j   1e100j  3.14e-10j 
\end{verbatim}


\section{Operators\label{operators}}

The following tokens are operators:
\index{operators}

\begin{verbatim}
+       -       *       **      /       //      %
<<      >>      &       |       ^       ~
<       >       <=      >=      ==      !=      <>
\end{verbatim}

The comparison operators \code{<>} and \code{!=} are alternate
spellings of the same operator.  \code{!=} is the preferred spelling;
\code{<>} is obsolescent.


\section{Delimiters\label{delimiters}}

The following tokens serve as delimiters in the grammar:
\index{delimiters}

\begin{verbatim}
(       )       [       ]       {       }      @
,       :       .       `       =       ;
+=      -=      *=      /=      //=     %=
&=      |=      ^=      >>=     <<=     **=
\end{verbatim}

The period can also occur in floating-point and imaginary literals.  A
sequence of three periods has a special meaning as an ellipsis in slices.
The second half of the list, the augmented assignment operators, serve
lexically as delimiters, but also perform an operation.

The following printing \ASCII{} characters have special meaning as part
of other tokens or are otherwise significant to the lexical analyzer:

\begin{verbatim}
'       "       #       \
\end{verbatim}

The following printing \ASCII{} characters are not used in Python.  Their
occurrence outside string literals and comments is an unconditional
error:
\index{ASCII@\ASCII}

\begin{verbatim}
$       ?
\end{verbatim}
		% Lexical analysis
\chapter{Data model\label{datamodel}}


\section{Objects, values and types\label{objects}}

\dfn{Objects} are Python's abstraction for data.  All data in a Python
program is represented by objects or by relations between objects.
(In a sense, and in conformance to Von Neumann's model of a
``stored program computer,'' code is also represented by objects.)
\index{object}
\index{data}

Every object has an identity, a type and a value.  An object's
\emph{identity} never changes once it has been created; you may think
of it as the object's address in memory.  The `\keyword{is}' operator
compares the identity of two objects; the
\function{id()}\bifuncindex{id} function returns an integer
representing its identity (currently implemented as its address).
An object's \dfn{type} is
also unchangeable.\footnote{Since Python 2.2, a gradual merging of
types and classes has been started that makes this and a few other
assertions made in this manual not 100\% accurate and complete:
for example, it \emph{is} now possible in some cases to change an
object's type, under certain controlled conditions.  Until this manual
undergoes extensive revision, it must now be taken as authoritative
only regarding ``classic classes'', that are still the default, for
compatibility purposes, in Python 2.2 and 2.3.  For more information,
see \url{http://www.python.org/doc/newstyle/}.}
An object's type determines the operations that the object
supports (e.g., ``does it have a length?'') and also defines the
possible values for objects of that type.  The
\function{type()}\bifuncindex{type} function returns an object's type
(which is an object itself).  The \emph{value} of some
objects can change.  Objects whose value can change are said to be
\emph{mutable}; objects whose value is unchangeable once they are
created are called \emph{immutable}.
(The value of an immutable container object that contains a reference
to a mutable object can change when the latter's value is changed;
however the container is still considered immutable, because the
collection of objects it contains cannot be changed.  So, immutability
is not strictly the same as having an unchangeable value, it is more
subtle.)
An object's mutability is determined by its type; for instance,
numbers, strings and tuples are immutable, while dictionaries and
lists are mutable.
\index{identity of an object}
\index{value of an object}
\index{type of an object}
\index{mutable object}
\index{immutable object}

Objects are never explicitly destroyed; however, when they become
unreachable they may be garbage-collected.  An implementation is
allowed to postpone garbage collection or omit it altogether --- it is
a matter of implementation quality how garbage collection is
implemented, as long as no objects are collected that are still
reachable.  (Implementation note: the current implementation uses a
reference-counting scheme with (optional) delayed detection of
cyclically linked garbage, which collects most objects as soon as they
become unreachable, but is not guaranteed to collect garbage
containing circular references.  See the
\citetitle[../lib/module-gc.html]{Python Library Reference} for
information on controlling the collection of cyclic garbage.)
\index{garbage collection}
\index{reference counting}
\index{unreachable object}

Note that the use of the implementation's tracing or debugging
facilities may keep objects alive that would normally be collectable.
Also note that catching an exception with a
`\keyword{try}...\keyword{except}' statement may keep objects alive.

Some objects contain references to ``external'' resources such as open
files or windows.  It is understood that these resources are freed
when the object is garbage-collected, but since garbage collection is
not guaranteed to happen, such objects also provide an explicit way to
release the external resource, usually a \method{close()} method.
Programs are strongly recommended to explicitly close such
objects.  The `\keyword{try}...\keyword{finally}' statement provides
a convenient way to do this.

Some objects contain references to other objects; these are called
\emph{containers}.  Examples of containers are tuples, lists and
dictionaries.  The references are part of a container's value.  In
most cases, when we talk about the value of a container, we imply the
values, not the identities of the contained objects; however, when we
talk about the mutability of a container, only the identities of
the immediately contained objects are implied.  So, if an immutable
container (like a tuple)
contains a reference to a mutable object, its value changes
if that mutable object is changed.
\index{container}

Types affect almost all aspects of object behavior.  Even the importance
of object identity is affected in some sense: for immutable types,
operations that compute new values may actually return a reference to
any existing object with the same type and value, while for mutable
objects this is not allowed.  E.g., after
\samp{a = 1; b = 1},
\code{a} and \code{b} may or may not refer to the same object with the
value one, depending on the implementation, but after
\samp{c = []; d = []}, \code{c} and \code{d}
are guaranteed to refer to two different, unique, newly created empty
lists.
(Note that \samp{c = d = []} assigns the same object to both
\code{c} and \code{d}.)


\section{The standard type hierarchy\label{types}}

Below is a list of the types that are built into Python.  Extension
modules (written in C, Java, or other languages, depending on
the implementation) can define additional types.  Future versions of
Python may add types to the type hierarchy (e.g., rational
numbers, efficiently stored arrays of integers, etc.).
\index{type}
\indexii{data}{type}
\indexii{type}{hierarchy}
\indexii{extension}{module}
\indexii{C}{language}

Some of the type descriptions below contain a paragraph listing
`special attributes.'  These are attributes that provide access to the
implementation and are not intended for general use.  Their definition
may change in the future.
\index{attribute}
\indexii{special}{attribute}
\indexiii{generic}{special}{attribute}

\begin{description}

\item[None]
This type has a single value.  There is a single object with this value.
This object is accessed through the built-in name \code{None}.
It is used to signify the absence of a value in many situations, e.g.,
it is returned from functions that don't explicitly return anything.
Its truth value is false.
\obindex{None}

\item[NotImplemented]
This type has a single value.  There is a single object with this value.
This object is accessed through the built-in name \code{NotImplemented}.
Numeric methods and rich comparison methods may return this value if
they do not implement the operation for the operands provided.  (The
interpreter will then try the reflected operation, or some other
fallback, depending on the operator.)  Its truth value is true.
\obindex{NotImplemented}

\item[Ellipsis]
This type has a single value.  There is a single object with this value.
This object is accessed through the built-in name \code{Ellipsis}.
It is used to indicate the presence of the \samp{...} syntax in a
slice.  Its truth value is true.
\obindex{Ellipsis}

\item[Numbers]
These are created by numeric literals and returned as results by
arithmetic operators and arithmetic built-in functions.  Numeric
objects are immutable; once created their value never changes.  Python
numbers are of course strongly related to mathematical numbers, but
subject to the limitations of numerical representation in computers.
\obindex{numeric}

Python distinguishes between integers, floating point numbers, and
complex numbers:

\begin{description}
\item[Integers]
These represent elements from the mathematical set of integers
(positive and negative).
\obindex{integer}

There are three types of integers:

\begin{description}

\item[Plain integers]
These represent numbers in the range -2147483648 through 2147483647.
(The range may be larger on machines with a larger natural word
size, but not smaller.)
When the result of an operation would fall outside this range, the
result is normally returned as a long integer (in some cases, the
exception \exception{OverflowError} is raised instead).
For the purpose of shift and mask operations, integers are assumed to
have a binary, 2's complement notation using 32 or more bits, and
hiding no bits from the user (i.e., all 4294967296 different bit
patterns correspond to different values).
\obindex{plain integer}
\withsubitem{(built-in exception)}{\ttindex{OverflowError}}

\item[Long integers]
These represent numbers in an unlimited range, subject to available
(virtual) memory only.  For the purpose of shift and mask operations,
a binary representation is assumed, and negative numbers are
represented in a variant of 2's complement which gives the illusion of
an infinite string of sign bits extending to the left.
\obindex{long integer}

\item[Booleans]
These represent the truth values False and True.  The two objects
representing the values False and True are the only Boolean objects.
The Boolean type is a subtype of plain integers, and Boolean values
behave like the values 0 and 1, respectively, in almost all contexts,
the exception being that when converted to a string, the strings
\code{"False"} or \code{"True"} are returned, respectively.
\obindex{Boolean}
\ttindex{False}
\ttindex{True}

\end{description} % Integers

The rules for integer representation are intended to give the most
meaningful interpretation of shift and mask operations involving
negative integers and the least surprises when switching between the
plain and long integer domains.  Any operation except left shift,
if it yields a result in the plain integer domain without causing
overflow, will yield the same result in the long integer domain or
when using mixed operands.
\indexii{integer}{representation}

\item[Floating point numbers]
These represent machine-level double precision floating point numbers.  
You are at the mercy of the underlying machine architecture (and
C or Java implementation) for the accepted range and handling of overflow.
Python does not support single-precision floating point numbers; the
savings in processor and memory usage that are usually the reason for using
these is dwarfed by the overhead of using objects in Python, so there
is no reason to complicate the language with two kinds of floating
point numbers.
\obindex{floating point}
\indexii{floating point}{number}
\indexii{C}{language}
\indexii{Java}{language}

\item[Complex numbers]
These represent complex numbers as a pair of machine-level double
precision floating point numbers.  The same caveats apply as for
floating point numbers.  The real and imaginary parts of a complex
number \code{z} can be retrieved through the read-only attributes
\code{z.real} and \code{z.imag}.
\obindex{complex}
\indexii{complex}{number}

\end{description} % Numbers


\item[Sequences]
These represent finite ordered sets indexed by non-negative numbers.
The built-in function \function{len()}\bifuncindex{len} returns the
number of items of a sequence.
When the length of a sequence is \var{n}, the
index set contains the numbers 0, 1, \ldots, \var{n}-1.  Item
\var{i} of sequence \var{a} is selected by \code{\var{a}[\var{i}]}.
\obindex{sequence}
\index{index operation}
\index{item selection}
\index{subscription}

Sequences also support slicing: \code{\var{a}[\var{i}:\var{j}]}
selects all items with index \var{k} such that \var{i} \code{<=}
\var{k} \code{<} \var{j}.  When used as an expression, a slice is a
sequence of the same type.  This implies that the index set is
renumbered so that it starts at 0.
\index{slicing}

Some sequences also support ``extended slicing'' with a third ``step''
parameter: \code{\var{a}[\var{i}:\var{j}:\var{k}]} selects all items
of \var{a} with index \var{x} where \code{\var{x} = \var{i} +
\var{n}*\var{k}}, \var{n} \code{>=} \code{0} and \var{i} \code{<=}
\var{x} \code{<} \var{j}.
\index{extended slicing}

Sequences are distinguished according to their mutability:

\begin{description}

\item[Immutable sequences]
An object of an immutable sequence type cannot change once it is
created.  (If the object contains references to other objects,
these other objects may be mutable and may be changed; however,
the collection of objects directly referenced by an immutable object
cannot change.)
\obindex{immutable sequence}
\obindex{immutable}

The following types are immutable sequences:

\begin{description}

\item[Strings]
The items of a string are characters.  There is no separate
character type; a character is represented by a string of one item.
Characters represent (at least) 8-bit bytes.  The built-in
functions \function{chr()}\bifuncindex{chr} and
\function{ord()}\bifuncindex{ord} convert between characters and
nonnegative integers representing the byte values.  Bytes with the
values 0-127 usually represent the corresponding \ASCII{} values, but
the interpretation of values is up to the program.  The string
data type is also used to represent arrays of bytes, e.g., to hold data
read from a file.
\obindex{string}
\index{character}
\index{byte}
\index{ASCII@\ASCII}

(On systems whose native character set is not \ASCII, strings may use
EBCDIC in their internal representation, provided the functions
\function{chr()} and \function{ord()} implement a mapping between \ASCII{} and
EBCDIC, and string comparison preserves the \ASCII{} order.
Or perhaps someone can propose a better rule?)
\index{ASCII@\ASCII}
\index{EBCDIC}
\index{character set}
\indexii{string}{comparison}
\bifuncindex{chr}
\bifuncindex{ord}

\item[Unicode]
The items of a Unicode object are Unicode code units.  A Unicode code
unit is represented by a Unicode object of one item and can hold
either a 16-bit or 32-bit value representing a Unicode ordinal (the
maximum value for the ordinal is given in \code{sys.maxunicode}, and
depends on how Python is configured at compile time).  Surrogate pairs
may be present in the Unicode object, and will be reported as two
separate items.  The built-in functions
\function{unichr()}\bifuncindex{unichr} and
\function{ord()}\bifuncindex{ord} convert between code units and
nonnegative integers representing the Unicode ordinals as defined in
the Unicode Standard 3.0. Conversion from and to other encodings are
possible through the Unicode method \method{encode()} and the built-in
function \function{unicode()}.\bifuncindex{unicode}
\obindex{unicode}
\index{character}
\index{integer}
\index{Unicode}

\item[Tuples]
The items of a tuple are arbitrary Python objects.
Tuples of two or more items are formed by comma-separated lists
of expressions.  A tuple of one item (a `singleton') can be formed
by affixing a comma to an expression (an expression by itself does
not create a tuple, since parentheses must be usable for grouping of
expressions).  An empty tuple can be formed by an empty pair of
parentheses.
\obindex{tuple}
\indexii{singleton}{tuple}
\indexii{empty}{tuple}

\end{description} % Immutable sequences

\item[Mutable sequences]
Mutable sequences can be changed after they are created.  The
subscription and slicing notations can be used as the target of
assignment and \keyword{del} (delete) statements.
\obindex{mutable sequence}
\obindex{mutable}
\indexii{assignment}{statement}
\index{delete}
\stindex{del}
\index{subscription}
\index{slicing}

There is currently a single intrinsic mutable sequence type:

\begin{description}

\item[Lists]
The items of a list are arbitrary Python objects.  Lists are formed
by placing a comma-separated list of expressions in square brackets.
(Note that there are no special cases needed to form lists of length 0
or 1.)
\obindex{list}

\end{description} % Mutable sequences

The extension module \module{array}\refstmodindex{array} provides an
additional example of a mutable sequence type.


\end{description} % Sequences


\item[Set types]
These represent unordered, finite sets of unique, immutable objects.
As such, they cannot be indexed by any subscript. However, they can be
iterated over, and the built-in function \function{len()} returns the
number of items in a set. Common uses for sets are
fast membership testing, removing duplicates from a sequence, and
computing mathematical operations such as intersection, union, difference,
and symmetric difference.
\bifuncindex{len}
\obindex{set type}

For set elements, the same immutability rules apply as for dictionary
keys. Note that numeric types obey the normal rules for numeric
comparison: if two numbers compare equal (e.g., \code{1} and
\code{1.0}), only one of them can be contained in a set.

There are currently two intrinsic set types:

\begin{description}

\item[Sets]
These\obindex{set} represent a mutable set. They are created by the
built-in \function{set()} constructor and can be modified afterwards
by several methods, such as \method{add()}.

\item[Frozen sets]
These\obindex{frozenset} represent an immutable set. They are created by
the built-in \function{frozenset()} constructor. As a frozenset is
immutable and hashable, it can be used again as an element of another set,
or as a dictionary key.

\end{description} % Set types


\item[Mappings]
These represent finite sets of objects indexed by arbitrary index sets.
The subscript notation \code{a[k]} selects the item indexed
by \code{k} from the mapping \code{a}; this can be used in
expressions and as the target of assignments or \keyword{del} statements.
The built-in function \function{len()} returns the number of items
in a mapping.
\bifuncindex{len}
\index{subscription}
\obindex{mapping}

There is currently a single intrinsic mapping type:

\begin{description}

\item[Dictionaries]
These\obindex{dictionary} represent finite sets of objects indexed by
nearly arbitrary values.  The only types of values not acceptable as
keys are values containing lists or dictionaries or other mutable
types that are compared by value rather than by object identity, the
reason being that the efficient implementation of dictionaries
requires a key's hash value to remain constant.
Numeric types used for keys obey the normal rules for numeric
comparison: if two numbers compare equal (e.g., \code{1} and
\code{1.0}) then they can be used interchangeably to index the same
dictionary entry.

Dictionaries are mutable; they can be created by the
\code{\{...\}} notation (see section~\ref{dict}, ``Dictionary
Displays'').

The extension modules \module{dbm}\refstmodindex{dbm},
\module{gdbm}\refstmodindex{gdbm}, and
\module{bsddb}\refstmodindex{bsddb} provide additional examples of
mapping types.

\end{description} % Mapping types

\item[Callable types]
These\obindex{callable} are the types to which the function call
operation (see section~\ref{calls}, ``Calls'') can be applied:
\indexii{function}{call}
\index{invocation}
\indexii{function}{argument}

\begin{description}

\item[User-defined functions]
A user-defined function object is created by a function definition
(see section~\ref{function}, ``Function definitions'').  It should be
called with an argument
list containing the same number of items as the function's formal
parameter list.
\indexii{user-defined}{function}
\obindex{function}
\obindex{user-defined function}

Special attributes: 

\begin{tableiii}{lll}{member}{Attribute}{Meaning}{}
  \lineiii{func_doc}{The function's documentation string, or
    \code{None} if unavailable}{Writable}

  \lineiii{__doc__}{Another way of spelling
    \member{func_doc}}{Writable}

  \lineiii{func_name}{The function's name}{Writable}

  \lineiii{__name__}{Another way of spelling
    \member{func_name}}{Writable}

  \lineiii{__module__}{The name of the module the function was defined
    in, or \code{None} if unavailable.}{Writable}

  \lineiii{func_defaults}{A tuple containing default argument values
    for those arguments that have defaults, or \code{None} if no
    arguments have a default value}{Writable}

  \lineiii{func_code}{The code object representing the compiled
    function body.}{Writable}

  \lineiii{func_globals}{A reference to the dictionary that holds the
    function's global variables --- the global namespace of the module
    in which the function was defined.}{Read-only}

  \lineiii{func_dict}{The namespace supporting arbitrary function
    attributes.}{Writable}

  \lineiii{func_closure}{\code{None} or a tuple of cells that contain
    bindings for the function's free variables.}{Read-only}
\end{tableiii}

Most of the attributes labelled ``Writable'' check the type of the
assigned value.

\versionchanged[\code{func_name} is now writable]{2.4}

Function objects also support getting and setting arbitrary
attributes, which can be used, for example, to attach metadata to
functions.  Regular attribute dot-notation is used to get and set such
attributes. \emph{Note that the current implementation only supports
function attributes on user-defined functions.  Function attributes on
built-in functions may be supported in the future.}

Additional information about a function's definition can be retrieved
from its code object; see the description of internal types below.

\withsubitem{(function attribute)}{
  \ttindex{func_doc}
  \ttindex{__doc__}
  \ttindex{__name__}
  \ttindex{__module__}
  \ttindex{__dict__}
  \ttindex{func_defaults}
  \ttindex{func_closure}
  \ttindex{func_code}
  \ttindex{func_globals}
  \ttindex{func_dict}}
\indexii{global}{namespace}

\item[User-defined methods]
A user-defined method object combines a class, a class instance (or
\code{None}) and any callable object (normally a user-defined
function).
\obindex{method}
\obindex{user-defined method}
\indexii{user-defined}{method}

Special read-only attributes: \member{im_self} is the class instance
object, \member{im_func} is the function object;
\member{im_class} is the class of \member{im_self} for bound methods
or the class that asked for the method for unbound methods;
\member{__doc__} is the method's documentation (same as
\code{im_func.__doc__}); \member{__name__} is the method name (same as
\code{im_func.__name__}); \member{__module__} is the name of the
module the method was defined in, or \code{None} if unavailable.
\versionchanged[\member{im_self} used to refer to the class that
                defined the method]{2.2}
\withsubitem{(method attribute)}{
  \ttindex{__doc__}
  \ttindex{__name__}
  \ttindex{__module__}
  \ttindex{im_func}
  \ttindex{im_self}}

Methods also support accessing (but not setting) the arbitrary
function attributes on the underlying function object.

User-defined method objects may be created when getting an attribute
of a class (perhaps via an instance of that class), if that attribute
is a user-defined function object, an unbound user-defined method object,
or a class method object.
When the attribute is a user-defined method object, a new
method object is only created if the class from which it is being
retrieved is the same as, or a derived class of, the class stored
in the original method object; otherwise, the original method object
is used as it is.

When a user-defined method object is created by retrieving
a user-defined function object from a class, its \member{im_self}
attribute is \code{None} and the method object is said to be unbound.
When one is created by retrieving a user-defined function object
from a class via one of its instances, its \member{im_self} attribute
is the instance, and the method object is said to be bound.
In either case, the new method's \member{im_class} attribute
is the class from which the retrieval takes place, and
its \member{im_func} attribute is the original function object.
\withsubitem{(method attribute)}{
  \ttindex{im_class}\ttindex{im_func}\ttindex{im_self}}

When a user-defined method object is created by retrieving another
method object from a class or instance, the behaviour is the same
as for a function object, except that the \member{im_func} attribute
of the new instance is not the original method object but its
\member{im_func} attribute.
\withsubitem{(method attribute)}{
  \ttindex{im_func}}

When a user-defined method object is created by retrieving a
class method object from a class or instance, its \member{im_self}
attribute is the class itself (the same as the \member{im_class}
attribute), and its \member{im_func} attribute is the function
object underlying the class method.
\withsubitem{(method attribute)}{
  \ttindex{im_class}\ttindex{im_func}\ttindex{im_self}}

When an unbound user-defined method object is called, the underlying
function (\member{im_func}) is called, with the restriction that the
first argument must be an instance of the proper class
(\member{im_class}) or of a derived class thereof.

When a bound user-defined method object is called, the underlying
function (\member{im_func}) is called, inserting the class instance
(\member{im_self}) in front of the argument list.  For instance, when
\class{C} is a class which contains a definition for a function
\method{f()}, and \code{x} is an instance of \class{C}, calling
\code{x.f(1)} is equivalent to calling \code{C.f(x, 1)}.

When a user-defined method object is derived from a class method object,
the ``class instance'' stored in \member{im_self} will actually be the
class itself, so that calling either \code{x.f(1)} or \code{C.f(1)} is
equivalent to calling \code{f(C,1)} where \code{f} is the underlying
function.

Note that the transformation from function object to (unbound or
bound) method object happens each time the attribute is retrieved from
the class or instance.  In some cases, a fruitful optimization is to
assign the attribute to a local variable and call that local variable.
Also notice that this transformation only happens for user-defined
functions; other callable objects (and all non-callable objects) are
retrieved without transformation.  It is also important to note that
user-defined functions which are attributes of a class instance are
not converted to bound methods; this \emph{only} happens when the
function is an attribute of the class.

\item[Generator functions\index{generator!function}\index{generator!iterator}]
A function or method which uses the \keyword{yield} statement (see
section~\ref{yield}, ``The \keyword{yield} statement'') is called a
\dfn{generator function}.  Such a function, when called, always
returns an iterator object which can be used to execute the body of
the function:  calling the iterator's \method{next()} method will
cause the function to execute until it provides a value using the
\keyword{yield} statement.  When the function executes a
\keyword{return} statement or falls off the end, a
\exception{StopIteration} exception is raised and the iterator will
have reached the end of the set of values to be returned.

\item[Built-in functions]
A built-in function object is a wrapper around a C function.  Examples
of built-in functions are \function{len()} and \function{math.sin()}
(\module{math} is a standard built-in module).
The number and type of the arguments are
determined by the C function.
Special read-only attributes: \member{__doc__} is the function's
documentation string, or \code{None} if unavailable; \member{__name__}
is the function's name; \member{__self__} is set to \code{None} (but see
the next item); \member{__module__} is the name of the module the
function was defined in or \code{None} if unavailable.
\obindex{built-in function}
\obindex{function}
\indexii{C}{language}

\item[Built-in methods]
This is really a different disguise of a built-in function, this time
containing an object passed to the C function as an implicit extra
argument.  An example of a built-in method is
\code{\var{alist}.append()}, assuming
\var{alist} is a list object.
In this case, the special read-only attribute \member{__self__} is set
to the object denoted by \var{list}.
\obindex{built-in method}
\obindex{method}
\indexii{built-in}{method}

\item[Class Types]
Class types, or ``new-style classes,'' are callable.  These objects
normally act as factories for new instances of themselves, but
variations are possible for class types that override
\method{__new__()}.  The arguments of the call are passed to
\method{__new__()} and, in the typical case, to \method{__init__()} to
initialize the new instance.

\item[Classic Classes]
Class objects are described below.  When a class object is called,
a new class instance (also described below) is created and
returned.  This implies a call to the class's \method{__init__()} method
if it has one.  Any arguments are passed on to the \method{__init__()}
method.  If there is no \method{__init__()} method, the class must be called
without arguments.
\withsubitem{(object method)}{\ttindex{__init__()}}
\obindex{class}
\obindex{class instance}
\obindex{instance}
\indexii{class object}{call}

\item[Class instances]
Class instances are described below.  Class instances are callable
only when the class has a \method{__call__()} method; \code{x(arguments)}
is a shorthand for \code{x.__call__(arguments)}.

\end{description}

\item[Modules]
Modules are imported by the \keyword{import} statement (see
section~\ref{import}, ``The \keyword{import} statement'').%
\stindex{import}\obindex{module}
A module object has a namespace implemented by a dictionary object
(this is the dictionary referenced by the func_globals attribute of
functions defined in the module).  Attribute references are translated
to lookups in this dictionary, e.g., \code{m.x} is equivalent to
\code{m.__dict__["x"]}.
A module object does not contain the code object used to
initialize the module (since it isn't needed once the initialization
is done).

Attribute assignment updates the module's namespace dictionary,
e.g., \samp{m.x = 1} is equivalent to \samp{m.__dict__["x"] = 1}.

Special read-only attribute: \member{__dict__} is the module's
namespace as a dictionary object.
\withsubitem{(module attribute)}{\ttindex{__dict__}}

Predefined (writable) attributes: \member{__name__}
is the module's name; \member{__doc__} is the
module's documentation string, or
\code{None} if unavailable; \member{__file__} is the pathname of the
file from which the module was loaded, if it was loaded from a file.
The \member{__file__} attribute is not present for C{} modules that are
statically linked into the interpreter; for extension modules loaded
dynamically from a shared library, it is the pathname of the shared
library file.
\withsubitem{(module attribute)}{
  \ttindex{__name__}
  \ttindex{__doc__}
  \ttindex{__file__}}
\indexii{module}{namespace}

\item[Classes]
Class objects are created by class definitions (see
section~\ref{class}, ``Class definitions'').
A class has a namespace implemented by a dictionary object.
Class attribute references are translated to
lookups in this dictionary,
e.g., \samp{C.x} is translated to \samp{C.__dict__["x"]}.
When the attribute name is not found
there, the attribute search continues in the base classes.  The search
is depth-first, left-to-right in the order of occurrence in the
base class list.

When a class attribute reference (for class \class{C}, say)
would yield a user-defined function object or
an unbound user-defined method object whose associated class is either
\class{C} or one of its base classes, it is transformed into an unbound
user-defined method object whose \member{im_class} attribute is~\class{C}.
When it would yield a class method object, it is transformed into
a bound user-defined method object whose \member{im_class} and
\member{im_self} attributes are both~\class{C}.  When it would yield
a static method object, it is transformed into the object wrapped
by the static method object. See section~\ref{descriptors} for another
way in which attributes retrieved from a class may differ from those
actually contained in its \member{__dict__}.
\obindex{class}
\obindex{class instance}
\obindex{instance}
\indexii{class object}{call}
\index{container}
\obindex{dictionary}
\indexii{class}{attribute}

Class attribute assignments update the class's dictionary, never the
dictionary of a base class.
\indexiii{class}{attribute}{assignment}

A class object can be called (see above) to yield a class instance (see
below).
\indexii{class object}{call}

Special attributes: \member{__name__} is the class name;
\member{__module__} is the module name in which the class was defined;
\member{__dict__} is the dictionary containing the class's namespace;
\member{__bases__} is a tuple (possibly empty or a singleton)
containing the base classes, in the order of their occurrence in the
base class list; \member{__doc__} is the class's documentation string,
or None if undefined.
\withsubitem{(class attribute)}{
  \ttindex{__name__}
  \ttindex{__module__}
  \ttindex{__dict__}
  \ttindex{__bases__}
  \ttindex{__doc__}}

\item[Class instances]
A class instance is created by calling a class object (see above).
A class instance has a namespace implemented as a dictionary which
is the first place in which
attribute references are searched.  When an attribute is not found
there, and the instance's class has an attribute by that name,
the search continues with the class attributes.  If a class attribute
is found that is a user-defined function object or an unbound
user-defined method object whose associated class is the class
(call it~\class{C}) of the instance for which the attribute reference
was initiated or one of its bases,
it is transformed into a bound user-defined method object whose
\member{im_class} attribute is~\class{C} and whose \member{im_self} attribute
is the instance. Static method and class method objects are also
transformed, as if they had been retrieved from class~\class{C};
see above under ``Classes''. See section~\ref{descriptors} for
another way in which attributes of a class retrieved via its
instances may differ from the objects actually stored in the
class's \member{__dict__}.
If no class attribute is found, and the object's class has a
\method{__getattr__()} method, that is called to satisfy the lookup.
\obindex{class instance}
\obindex{instance}
\indexii{class}{instance}
\indexii{class instance}{attribute}

Attribute assignments and deletions update the instance's dictionary,
never a class's dictionary.  If the class has a \method{__setattr__()} or
\method{__delattr__()} method, this is called instead of updating the
instance dictionary directly.
\indexiii{class instance}{attribute}{assignment}

Class instances can pretend to be numbers, sequences, or mappings if
they have methods with certain special names.  See
section~\ref{specialnames}, ``Special method names.''
\obindex{numeric}
\obindex{sequence}
\obindex{mapping}

Special attributes: \member{__dict__} is the attribute
dictionary; \member{__class__} is the instance's class.
\withsubitem{(instance attribute)}{
  \ttindex{__dict__}
  \ttindex{__class__}}

\item[Files]
A file\obindex{file} object represents an open file.  File objects are
created by the \function{open()}\bifuncindex{open} built-in function,
and also by
\withsubitem{(in module os)}{\ttindex{popen()}}\function{os.popen()},
\function{os.fdopen()}, and the
\method{makefile()}\withsubitem{(socket method)}{\ttindex{makefile()}}
method of socket objects (and perhaps by other functions or methods
provided by extension modules).  The objects
\ttindex{sys.stdin}\code{sys.stdin},
\ttindex{sys.stdout}\code{sys.stdout} and
\ttindex{sys.stderr}\code{sys.stderr} are initialized to file objects
corresponding to the interpreter's standard\index{stdio} input, output
and error streams.  See the \citetitle[../lib/lib.html]{Python Library
Reference} for complete documentation of file objects.
\withsubitem{(in module sys)}{
  \ttindex{stdin}
  \ttindex{stdout}
  \ttindex{stderr}}


\item[Internal types]
A few types used internally by the interpreter are exposed to the user.
Their definitions may change with future versions of the interpreter,
but they are mentioned here for completeness.
\index{internal type}
\index{types, internal}

\begin{description}

\item[Code objects]
Code objects represent \emph{byte-compiled} executable Python code, or 
\emph{bytecode}.
The difference between a code
object and a function object is that the function object contains an
explicit reference to the function's globals (the module in which it
was defined), while a code object contains no context; 
also the default argument values are stored in the function object,
not in the code object (because they represent values calculated at
run-time).  Unlike function objects, code objects are immutable and
contain no references (directly or indirectly) to mutable objects.
\index{bytecode}
\obindex{code}

Special read-only attributes: \member{co_name} gives the function
name; \member{co_argcount} is the number of positional arguments
(including arguments with default values); \member{co_nlocals} is the
number of local variables used by the function (including arguments);
\member{co_varnames} is a tuple containing the names of the local
variables (starting with the argument names); \member{co_cellvars} is
a tuple containing the names of local variables that are referenced by
nested functions; \member{co_freevars} is a tuple containing the names
of free variables; \member{co_code} is a string representing the
sequence of bytecode instructions;
\member{co_consts} is a tuple containing the literals used by the
bytecode; \member{co_names} is a tuple containing the names used by
the bytecode; \member{co_filename} is the filename from which the code
was compiled; \member{co_firstlineno} is the first line number of the
function; \member{co_lnotab} is a string encoding the mapping from
byte code offsets to line numbers (for details see the source code of
the interpreter); \member{co_stacksize} is the required stack size
(including local variables); \member{co_flags} is an integer encoding
a number of flags for the interpreter.

\withsubitem{(code object attribute)}{
  \ttindex{co_argcount}
  \ttindex{co_code}
  \ttindex{co_consts}
  \ttindex{co_filename}
  \ttindex{co_firstlineno}
  \ttindex{co_flags}
  \ttindex{co_lnotab}
  \ttindex{co_name}
  \ttindex{co_names}
  \ttindex{co_nlocals}
  \ttindex{co_stacksize}
  \ttindex{co_varnames}
  \ttindex{co_cellvars}
  \ttindex{co_freevars}}

The following flag bits are defined for \member{co_flags}: bit
\code{0x04} is set if the function uses the \samp{*arguments} syntax
to accept an arbitrary number of positional arguments; bit
\code{0x08} is set if the function uses the \samp{**keywords} syntax
to accept arbitrary keyword arguments; bit \code{0x20} is set if the
function is a generator.
\obindex{generator}

Future feature declarations (\samp{from __future__ import division})
also use bits in \member{co_flags} to indicate whether a code object
was compiled with a particular feature enabled: bit \code{0x2000} is
set if the function was compiled with future division enabled; bits
\code{0x10} and \code{0x1000} were used in earlier versions of Python.

Other bits in \member{co_flags} are reserved for internal use.

If\index{documentation string} a code object represents a function,
the first item in
\member{co_consts} is the documentation string of the function, or
\code{None} if undefined.

\item[Frame objects]
Frame objects represent execution frames.  They may occur in traceback
objects (see below).
\obindex{frame}

Special read-only attributes: \member{f_back} is to the previous
stack frame (towards the caller), or \code{None} if this is the bottom
stack frame; \member{f_code} is the code object being executed in this
frame; \member{f_locals} is the dictionary used to look up local
variables; \member{f_globals} is used for global variables;
\member{f_builtins} is used for built-in (intrinsic) names;
\member{f_restricted} is a flag indicating whether the function is
executing in restricted execution mode; \member{f_lasti} gives the
precise instruction (this is an index into the bytecode string of
the code object).
\withsubitem{(frame attribute)}{
  \ttindex{f_back}
  \ttindex{f_code}
  \ttindex{f_globals}
  \ttindex{f_locals}
  \ttindex{f_lasti}
  \ttindex{f_builtins}
  \ttindex{f_restricted}}

Special writable attributes: \member{f_trace}, if not \code{None}, is
a function called at the start of each source code line (this is used
by the debugger); \member{f_exc_type}, \member{f_exc_value},
\member{f_exc_traceback} represent the last exception raised in the
parent frame provided another exception was ever raised in the current
frame (in all other cases they are None); \member{f_lineno} is the
current line number of the frame --- writing to this from within a
trace function jumps to the given line (only for the bottom-most
frame).  A debugger can implement a Jump command (aka Set Next
Statement) by writing to f_lineno.
\withsubitem{(frame attribute)}{
  \ttindex{f_trace}
  \ttindex{f_exc_type}
  \ttindex{f_exc_value}
  \ttindex{f_exc_traceback}
  \ttindex{f_lineno}}

\item[Traceback objects] \label{traceback}
Traceback objects represent a stack trace of an exception.  A
traceback object is created when an exception occurs.  When the search
for an exception handler unwinds the execution stack, at each unwound
level a traceback object is inserted in front of the current
traceback.  When an exception handler is entered, the stack trace is
made available to the program.
(See section~\ref{try}, ``The \code{try} statement.'')
It is accessible as \code{sys.exc_traceback}, and also as the third
item of the tuple returned by \code{sys.exc_info()}.  The latter is
the preferred interface, since it works correctly when the program is
using multiple threads.
When the program contains no suitable handler, the stack trace is written
(nicely formatted) to the standard error stream; if the interpreter is
interactive, it is also made available to the user as
\code{sys.last_traceback}.
\obindex{traceback}
\indexii{stack}{trace}
\indexii{exception}{handler}
\indexii{execution}{stack}
\withsubitem{(in module sys)}{
  \ttindex{exc_info}
  \ttindex{exc_traceback}
  \ttindex{last_traceback}}
\ttindex{sys.exc_info}
\ttindex{sys.exc_traceback}
\ttindex{sys.last_traceback}

Special read-only attributes: \member{tb_next} is the next level in the
stack trace (towards the frame where the exception occurred), or
\code{None} if there is no next level; \member{tb_frame} points to the
execution frame of the current level; \member{tb_lineno} gives the line
number where the exception occurred; \member{tb_lasti} indicates the
precise instruction.  The line number and last instruction in the
traceback may differ from the line number of its frame object if the
exception occurred in a \keyword{try} statement with no matching
except clause or with a finally clause.
\withsubitem{(traceback attribute)}{
  \ttindex{tb_next}
  \ttindex{tb_frame}
  \ttindex{tb_lineno}
  \ttindex{tb_lasti}}
\stindex{try}

\item[Slice objects]
Slice objects are used to represent slices when \emph{extended slice
syntax} is used.  This is a slice using two colons, or multiple slices
or ellipses separated by commas, e.g., \code{a[i:j:step]}, \code{a[i:j,
k:l]}, or \code{a[..., i:j]}.  They are also created by the built-in
\function{slice()}\bifuncindex{slice} function.

Special read-only attributes: \member{start} is the lower bound;
\member{stop} is the upper bound; \member{step} is the step value; each is
\code{None} if omitted. These attributes can have any type.
\withsubitem{(slice object attribute)}{
  \ttindex{start}
  \ttindex{stop}
  \ttindex{step}}

Slice objects support one method:

\begin{methoddesc}[slice]{indices}{self, length}
This method takes a single integer argument \var{length} and computes
information about the extended slice that the slice object would
describe if applied to a sequence of \var{length} items.  It returns a
tuple of three integers; respectively these are the \var{start} and
\var{stop} indices and the \var{step} or stride length of the slice.
Missing or out-of-bounds indices are handled in a manner consistent
with regular slices.
\versionadded{2.3}
\end{methoddesc}

\item[Static method objects]
Static method objects provide a way of defeating the transformation
of function objects to method objects described above. A static method
object is a wrapper around any other object, usually a user-defined
method object. When a static method object is retrieved from a class
or a class instance, the object actually returned is the wrapped object,
which is not subject to any further transformation. Static method
objects are not themselves callable, although the objects they
wrap usually are. Static method objects are created by the built-in
\function{staticmethod()} constructor.

\item[Class method objects]
A class method object, like a static method object, is a wrapper
around another object that alters the way in which that object
is retrieved from classes and class instances. The behaviour of
class method objects upon such retrieval is described above,
under ``User-defined methods''. Class method objects are created
by the built-in \function{classmethod()} constructor.

\end{description} % Internal types

\end{description} % Types

%=========================================================================
\section{New-style and classic classes}

Classes and instances come in two flavors: old-style or classic, and new-style.  

Up to Python 2.1, old-style classes were the only flavour available to the
user.  The concept of (old-style) class is unrelated to the concept of type: if
\var{x} is an instance of an old-style class, then \code{x.__class__}
designates the class of \var{x}, but \code{type(x)} is always \code{<type
'instance'>}.  This reflects the fact that all old-style instances,
independently of their class, are implemented with a single built-in type,
called \code{instance}.

New-style classes were introduced in Python 2.2 to unify classes and types.  A
new-style class neither more nor less than a user-defined type.  If \var{x} is
an instance of a new-style class, then \code{type(x)} is the same as
\code{x.__class__}.

The major motivation for introducing new-style classes is to provide a unified
object model with a full meta-model.  It also has a number of immediate
benefits, like the ability to subclass most built-in types, or the introduction
of "descriptors", which enable computed properties.

For compatibility reasons, classes are still old-style by default.  New-style
classes are created by specifying another new-style class (i.e.\ a type) as a
parent class, or the "top-level type" \class{object} if no other parent is
needed.  The behaviour of new-style classes differs from that of old-style
classes in a number of important details in addition to what \function{type}
returns.  Some of these changes are fundamental to the new object model, like
the way special methods are invoked.  Others are "fixes" that could not be
implemented before for compatibility concerns, like the method resolution order
in case of multiple inheritance.

This manual is not up-to-date with respect to new-style classes.  For now,
please see \url{http://www.python.org/doc/newstyle/} for more information.

The plan is to eventually drop old-style classes, leaving only the semantics of
new-style classes.  This change will probably only be feasible in Python 3.0.
\index{class}{new-style}
\index{class}{classic}
\index{class}{old-style}

%=========================================================================
\section{Special method names\label{specialnames}}

A class can implement certain operations that are invoked by special
syntax (such as arithmetic operations or subscripting and slicing) by
defining methods with special names.\indexii{operator}{overloading}
This is Python's approach to \dfn{operator overloading}, allowing
classes to define their own behavior with respect to language
operators.  For instance, if a class defines
a method named \method{__getitem__()}, and \code{x} is an instance of
this class, then \code{x[i]} is equivalent\footnote{This, and other
statements, are only roughly true for instances of new-style
classes.} to
\code{x.__getitem__(i)}.  Except where mentioned, attempts to execute
an operation raise an exception when no appropriate method is defined.
\withsubitem{(mapping object method)}{\ttindex{__getitem__()}}

When implementing a class that emulates any built-in type, it is
important that the emulation only be implemented to the degree that it
makes sense for the object being modelled.  For example, some
sequences may work well with retrieval of individual elements, but
extracting a slice may not make sense.  (One example of this is the
\class{NodeList} interface in the W3C's Document Object Model.)


\subsection{Basic customization\label{customization}}

\begin{methoddesc}[object]{__new__}{cls\optional{, \moreargs}}
Called to create a new instance of class \var{cls}.  \method{__new__()}
is a static method (special-cased so you need not declare it as such)
that takes the class of which an instance was requested as its first
argument.  The remaining arguments are those passed to the object
constructor expression (the call to the class).  The return value of
\method{__new__()} should be the new object instance (usually an
instance of \var{cls}).

Typical implementations create a new instance of the class by invoking
the superclass's \method{__new__()} method using
\samp{super(\var{currentclass}, \var{cls}).__new__(\var{cls}[, ...])}
with appropriate arguments and then modifying the newly-created instance
as necessary before returning it.

If \method{__new__()} returns an instance of \var{cls}, then the new
instance's \method{__init__()} method will be invoked like
\samp{__init__(\var{self}[, ...])}, where \var{self} is the new instance
and the remaining arguments are the same as were passed to
\method{__new__()}.

If \method{__new__()} does not return an instance of \var{cls}, then the
new instance's \method{__init__()} method will not be invoked.

\method{__new__()} is intended mainly to allow subclasses of
immutable types (like int, str, or tuple) to customize instance
creation.
\end{methoddesc}

\begin{methoddesc}[object]{__init__}{self\optional{, \moreargs}}
Called\indexii{class}{constructor} when the instance is created.  The
arguments are those passed to the class constructor expression.  If a
base class has an \method{__init__()} method, the derived class's
\method{__init__()} method, if any, must explicitly call it to ensure proper
initialization of the base class part of the instance; for example:
\samp{BaseClass.__init__(\var{self}, [\var{args}...])}.  As a special
constraint on constructors, no value may be returned; doing so will
cause a \exception{TypeError} to be raised at runtime.
\end{methoddesc}


\begin{methoddesc}[object]{__del__}{self}
Called when the instance is about to be destroyed.  This is also
called a destructor\index{destructor}.  If a base class
has a \method{__del__()} method, the derived class's \method{__del__()}
method, if any,
must explicitly call it to ensure proper deletion of the base class
part of the instance.  Note that it is possible (though not recommended!)
for the \method{__del__()}
method to postpone destruction of the instance by creating a new
reference to it.  It may then be called at a later time when this new
reference is deleted.  It is not guaranteed that
\method{__del__()} methods are called for objects that still exist when
the interpreter exits.
\stindex{del}

\begin{notice}
\samp{del x} doesn't directly call
\code{x.__del__()} --- the former decrements the reference count for
\code{x} by one, and the latter is only called when \code{x}'s reference
count reaches zero.  Some common situations that may prevent the
reference count of an object from going to zero include: circular
references between objects (e.g., a doubly-linked list or a tree data
structure with parent and child pointers); a reference to the object
on the stack frame of a function that caught an exception (the
traceback stored in \code{sys.exc_traceback} keeps the stack frame
alive); or a reference to the object on the stack frame that raised an
unhandled exception in interactive mode (the traceback stored in
\code{sys.last_traceback} keeps the stack frame alive).  The first
situation can only be remedied by explicitly breaking the cycles; the
latter two situations can be resolved by storing \code{None} in
\code{sys.exc_traceback} or \code{sys.last_traceback}.  Circular
references which are garbage are detected when the option cycle
detector is enabled (it's on by default), but can only be cleaned up
if there are no Python-level \method{__del__()} methods involved.
Refer to the documentation for the \ulink{\module{gc}
module}{../lib/module-gc.html} for more information about how
\method{__del__()} methods are handled by the cycle detector,
particularly the description of the \code{garbage} value.
\end{notice}

\begin{notice}[warning]
Due to the precarious circumstances under which
\method{__del__()} methods are invoked, exceptions that occur during their
execution are ignored, and a warning is printed to \code{sys.stderr}
instead.  Also, when \method{__del__()} is invoked in response to a module
being deleted (e.g., when execution of the program is done), other
globals referenced by the \method{__del__()} method may already have been
deleted.  For this reason, \method{__del__()} methods should do the
absolute minimum needed to maintain external invariants.  Starting with
version 1.5, Python guarantees that globals whose name begins with a single
underscore are deleted from their module before other globals are deleted;
if no other references to such globals exist, this may help in assuring that
imported modules are still available at the time when the
\method{__del__()} method is called.
\end{notice}
\end{methoddesc}

\begin{methoddesc}[object]{__repr__}{self}
Called by the \function{repr()}\bifuncindex{repr} built-in function
and by string conversions (reverse quotes) to compute the ``official''
string representation of an object.  If at all possible, this should
look like a valid Python expression that could be used to recreate an
object with the same value (given an appropriate environment).  If
this is not possible, a string of the form \samp{<\var{...some useful
description...}>} should be returned.  The return value must be a
string object.
If a class defines \method{__repr__()} but not \method{__str__()},
then \method{__repr__()} is also used when an ``informal'' string
representation of instances of that class is required.		     

This is typically used for debugging, so it is important that the
representation is information-rich and unambiguous.
\indexii{string}{conversion}
\indexii{reverse}{quotes}
\indexii{backward}{quotes}
\index{back-quotes}
\end{methoddesc}

\begin{methoddesc}[object]{__str__}{self}
Called by the \function{str()}\bifuncindex{str} built-in function and
by the \keyword{print}\stindex{print} statement to compute the
``informal'' string representation of an object.  This differs from
\method{__repr__()} in that it does not have to be a valid Python
expression: a more convenient or concise representation may be used
instead.  The return value must be a string object.
\end{methoddesc}

\begin{methoddesc}[object]{__lt__}{self, other}
\methodline[object]{__le__}{self, other}
\methodline[object]{__eq__}{self, other}
\methodline[object]{__ne__}{self, other}
\methodline[object]{__gt__}{self, other}
\methodline[object]{__ge__}{self, other}
\versionadded{2.1}
These are the so-called ``rich comparison'' methods, and are called
for comparison operators in preference to \method{__cmp__()} below.
The correspondence between operator symbols and method names is as
follows:
\code{\var{x}<\var{y}} calls \code{\var{x}.__lt__(\var{y})},
\code{\var{x}<=\var{y}} calls \code{\var{x}.__le__(\var{y})},
\code{\var{x}==\var{y}} calls \code{\var{x}.__eq__(\var{y})},
\code{\var{x}!=\var{y}} and \code{\var{x}<>\var{y}} call
\code{\var{x}.__ne__(\var{y})},
\code{\var{x}>\var{y}} calls \code{\var{x}.__gt__(\var{y})}, and
\code{\var{x}>=\var{y}} calls \code{\var{x}.__ge__(\var{y})}.

A rich comparison method may return the singleton \code{NotImplemented} if it
does not implement the operation for a given pair of arguments.
By convention, \code{False} and \code{True} are returned for a successful
comparison. However, these methods can return any value, so if the
comparison operator is used in a Boolean context (e.g., in the condition
of an \code{if} statement), Python will call \function{bool()} on the
value to determine if the result is true or false.

There are no implied relationships among the comparison operators.
The truth of \code{\var{x}==\var{y}} does not imply that \code{\var{x}!=\var{y}}
is false.  Accordingly, when defining \method{__eq__()}, one should also
define \method{__ne__()} so that the operators will behave as expected.

There are no reflected (swapped-argument) versions of these methods
(to be used when the left argument does not support the operation but
the right argument does); rather, \method{__lt__()} and
\method{__gt__()} are each other's reflection, \method{__le__()} and
\method{__ge__()} are each other's reflection, and \method{__eq__()}
and \method{__ne__()} are their own reflection.

Arguments to rich comparison methods are never coerced.
\end{methoddesc}

\begin{methoddesc}[object]{__cmp__}{self, other}
Called by comparison operations if rich comparison (see above) is not
defined.  Should return a negative integer if \code{self < other},
zero if \code{self == other}, a positive integer if \code{self >
other}.  If no \method{__cmp__()}, \method{__eq__()} or
\method{__ne__()} operation is defined, class instances are compared
by object identity (``address'').  See also the description of
\method{__hash__()} for some important notes on creating objects which
support custom comparison operations and are usable as dictionary
keys.
(Note: the restriction that exceptions are not propagated by
\method{__cmp__()} has been removed since Python 1.5.)
\bifuncindex{cmp}
\index{comparisons}
\end{methoddesc}

\begin{methoddesc}[object]{__rcmp__}{self, other}
  \versionchanged[No longer supported]{2.1}
\end{methoddesc}

\begin{methoddesc}[object]{__hash__}{self}
Called for the key object for dictionary \obindex{dictionary}
operations, and by the built-in function
\function{hash()}\bifuncindex{hash}.  Should return a 32-bit integer
usable as a hash value
for dictionary operations.  The only required property is that objects
which compare equal have the same hash value; it is advised to somehow
mix together (e.g., using exclusive or) the hash values for the
components of the object that also play a part in comparison of
objects.  If a class does not define a \method{__cmp__()} method it should
not define a \method{__hash__()} operation either; if it defines
\method{__cmp__()} or \method{__eq__()} but not \method{__hash__()},
its instances will not be usable as dictionary keys.  If a class
defines mutable objects and implements a \method{__cmp__()} or
\method{__eq__()} method, it should not implement \method{__hash__()},
since the dictionary implementation requires that a key's hash value
is immutable (if the object's hash value changes, it will be in the
wrong hash bucket).

\versionchanged[\method{__hash__()} may now also return a long
integer object; the 32-bit integer is then derived from the hash
of that object]{2.5}

\withsubitem{(object method)}{\ttindex{__cmp__()}}
\end{methoddesc}

\begin{methoddesc}[object]{__nonzero__}{self}
Called to implement truth value testing, and the built-in operation
\code{bool()}; should return \code{False} or \code{True}, or their
integer equivalents \code{0} or \code{1}.
When this method is not defined, \method{__len__()} is
called, if it is defined (see below).  If a class defines neither
\method{__len__()} nor \method{__nonzero__()}, all its instances are
considered true.
\withsubitem{(mapping object method)}{\ttindex{__len__()}}
\end{methoddesc}

\begin{methoddesc}[object]{__unicode__}{self}
Called to implement \function{unicode()}\bifuncindex{unicode} builtin;
should return a Unicode object. When this method is not defined, string
conversion is attempted, and the result of string conversion is converted
to Unicode using the system default encoding.
\end{methoddesc}


\subsection{Customizing attribute access\label{attribute-access}}

The following methods can be defined to customize the meaning of
attribute access (use of, assignment to, or deletion of \code{x.name})
for class instances.

\begin{methoddesc}[object]{__getattr__}{self, name}
Called when an attribute lookup has not found the attribute in the
usual places (i.e. it is not an instance attribute nor is it found in
the class tree for \code{self}).  \code{name} is the attribute name.
This method should return the (computed) attribute value or raise an
\exception{AttributeError} exception.

Note that if the attribute is found through the normal mechanism,
\method{__getattr__()} is not called.  (This is an intentional
asymmetry between \method{__getattr__()} and \method{__setattr__()}.)
This is done both for efficiency reasons and because otherwise
\method{__setattr__()} would have no way to access other attributes of
the instance.  Note that at least for instance variables, you can fake
total control by not inserting any values in the instance attribute
dictionary (but instead inserting them in another object).  See the
\method{__getattribute__()} method below for a way to actually get
total control in new-style classes.
\withsubitem{(object method)}{\ttindex{__setattr__()}}
\end{methoddesc}

\begin{methoddesc}[object]{__setattr__}{self, name, value}
Called when an attribute assignment is attempted.  This is called
instead of the normal mechanism (i.e.\ store the value in the instance
dictionary).  \var{name} is the attribute name, \var{value} is the
value to be assigned to it.

If \method{__setattr__()} wants to assign to an instance attribute, it 
should not simply execute \samp{self.\var{name} = value} --- this
would cause a recursive call to itself.  Instead, it should insert the
value in the dictionary of instance attributes, e.g.,
\samp{self.__dict__[\var{name}] = value}.  For new-style classes,
rather than accessing the instance dictionary, it should call the base
class method with the same name, for example,
\samp{object.__setattr__(self, name, value)}.
\withsubitem{(instance attribute)}{\ttindex{__dict__}}
\end{methoddesc}

\begin{methoddesc}[object]{__delattr__}{self, name}
Like \method{__setattr__()} but for attribute deletion instead of
assignment.  This should only be implemented if \samp{del
obj.\var{name}} is meaningful for the object.
\end{methoddesc}

\subsubsection{More attribute access for new-style classes \label{new-style-attribute-access}}

The following methods only apply to new-style classes.

\begin{methoddesc}[object]{__getattribute__}{self, name}
Called unconditionally to implement attribute accesses for instances
of the class. If the class also defines \method{__getattr__()}, the latter 
will not be called unless \method{__getattribute__()} either calls it 
explicitly or raises an \exception{AttributeError}.
This method should return the (computed) attribute
value or raise an \exception{AttributeError} exception.
In order to avoid infinite recursion in this method, its
implementation should always call the base class method with the same
name to access any attributes it needs, for example,
\samp{object.__getattribute__(self, name)}.
\end{methoddesc}

\subsubsection{Implementing Descriptors \label{descriptors}}

The following methods only apply when an instance of the class
containing the method (a so-called \emph{descriptor} class) appears in
the class dictionary of another new-style class, known as the
\emph{owner} class. In the examples below, ``the attribute'' refers to
the attribute whose name is the key of the property in the owner
class' \code{__dict__}.  Descriptors can only be implemented as
new-style classes themselves.

\begin{methoddesc}[object]{__get__}{self, instance, owner}
Called to get the attribute of the owner class (class attribute access)
or of an instance of that class (instance attribute access).
\var{owner} is always the owner class, while \var{instance} is the
instance that the attribute was accessed through, or \code{None} when
the attribute is accessed through the \var{owner}.  This method should
return the (computed) attribute value or raise an
\exception{AttributeError} exception.
\end{methoddesc}

\begin{methoddesc}[object]{__set__}{self, instance, value}
Called to set the attribute on an instance \var{instance} of the owner
class to a new value, \var{value}.
\end{methoddesc}

\begin{methoddesc}[object]{__delete__}{self, instance}
Called to delete the attribute on an instance \var{instance} of the
owner class.
\end{methoddesc}


\subsubsection{Invoking Descriptors \label{descriptor-invocation}}

In general, a descriptor is an object attribute with ``binding behavior'',
one whose attribute access has been overridden by methods in the descriptor
protocol:  \method{__get__()}, \method{__set__()}, and \method{__delete__()}.
If any of those methods are defined for an object, it is said to be a
descriptor.

The default behavior for attribute access is to get, set, or delete the
attribute from an object's dictionary. For instance, \code{a.x} has a
lookup chain starting with \code{a.__dict__['x']}, then
\code{type(a).__dict__['x']}, and continuing 
through the base classes of \code{type(a)} excluding metaclasses.

However, if the looked-up value is an object defining one of the descriptor
methods, then Python may override the default behavior and invoke the
descriptor method instead.  Where this occurs in the precedence chain depends
on which descriptor methods were defined and how they were called.  Note that
descriptors are only invoked for new style objects or classes
(ones that subclass \class{object()} or \class{type()}).

The starting point for descriptor invocation is a binding, \code{a.x}.
How the arguments are assembled depends on \code{a}:

\begin{itemize}
                      
  \item[Direct Call] The simplest and least common call is when user code
    directly invokes a descriptor method:    \code{x.__get__(a)}.

  \item[Instance Binding]  If binding to a new-style object instance,
    \code{a.x} is transformed into the call:
    \code{type(a).__dict__['x'].__get__(a, type(a))}.
                     
  \item[Class Binding]  If binding to a new-style class, \code{A.x}
    is transformed into the call: \code{A.__dict__['x'].__get__(None, A)}.

  \item[Super Binding] If \code{a} is an instance of \class{super},
    then the binding \code{super(B, obj).m()} searches
    \code{obj.__class__.__mro__} for the base class \code{A} immediately
    preceding \code{B} and then invokes the descriptor with the call:
    \code{A.__dict__['m'].__get__(obj, A)}.
                     
\end{itemize}

For instance bindings, the precedence of descriptor invocation depends
on the which descriptor methods are defined.  Data descriptors define
both \method{__get__()} and \method{__set__()}.  Non-data descriptors have
just the \method{__get__()} method.  Data descriptors always override
a redefinition in an instance dictionary.  In contrast, non-data
descriptors can be overridden by instances.

Python methods (including \function{staticmethod()} and \function{classmethod()})
are implemented as non-data descriptors.  Accordingly, instances can
redefine and override methods.  This allows individual instances to acquire
behaviors that differ from other instances of the same class.                     

The \function{property()} function is implemented as a data descriptor.
Accordingly, instances cannot override the behavior of a property.


\subsubsection{__slots__\label{slots}}

By default, instances of both old and new-style classes have a dictionary
for attribute storage.  This wastes space for objects having very few instance
variables.  The space consumption can become acute when creating large numbers
of instances.

The default can be overridden by defining \var{__slots__} in a new-style class
definition.  The \var{__slots__} declaration takes a sequence of instance
variables and reserves just enough space in each instance to hold a value
for each variable.  Space is saved because \var{__dict__} is not created for
each instance.
    
\begin{datadesc}{__slots__}
This class variable can be assigned a string, iterable, or sequence of strings
with variable names used by instances.  If defined in a new-style class,
\var{__slots__} reserves space for the declared variables
and prevents the automatic creation of \var{__dict__} and \var{__weakref__}
for each instance.
\versionadded{2.2}                     
\end{datadesc}

\noindent
Notes on using \var{__slots__}

\begin{itemize}

\item Without a \var{__dict__} variable, instances cannot be assigned new
variables not listed in the \var{__slots__} definition.  Attempts to assign
to an unlisted variable name raises \exception{AttributeError}. If dynamic
assignment of new variables is desired, then add \code{'__dict__'} to the
sequence of strings in the \var{__slots__} declaration.                                     
\versionchanged[Previously, adding \code{'__dict__'} to the \var{__slots__}
declaration would not enable the assignment of new attributes not
specifically listed in the sequence of instance variable names]{2.3}                     

\item Without a \var{__weakref__} variable for each instance, classes
defining \var{__slots__} do not support weak references to its instances.
If weak reference support is needed, then add \code{'__weakref__'} to the
sequence of strings in the \var{__slots__} declaration.                    
\versionchanged[Previously, adding \code{'__weakref__'} to the \var{__slots__}
declaration would not enable support for weak references]{2.3}                                            

\item \var{__slots__} are implemented at the class level by creating
descriptors (\ref{descriptors}) for each variable name.  As a result,
class attributes cannot be used to set default values for instance
variables defined by \var{__slots__}; otherwise, the class attribute would
overwrite the descriptor assignment. 

\item If a class defines a slot also defined in a base class, the instance
variable defined by the base class slot is inaccessible (except by retrieving
its descriptor directly from the base class). This renders the meaning of the
program undefined.  In the future, a check may be added to prevent this.

\item The action of a \var{__slots__} declaration is limited to the class
where it is defined.  As a result, subclasses will have a \var{__dict__}
unless they also define  \var{__slots__}.                     

\item \var{__slots__} do not work for classes derived from ``variable-length''
built-in types such as \class{long}, \class{str} and \class{tuple}. 

\item Any non-string iterable may be assigned to \var{__slots__}.
Mappings may also be used; however, in the future, special meaning may
be assigned to the values corresponding to each key.                      

\end{itemize}


\subsection{Customizing class creation\label{metaclasses}}

By default, new-style classes are constructed using \function{type()}.
A class definition is read into a separate namespace and the value
of class name is bound to the result of \code{type(name, bases, dict)}.

When the class definition is read, if \var{__metaclass__} is defined
then the callable assigned to it will be called instead of \function{type()}.
The allows classes or functions to be written which monitor or alter the class
creation process:

\begin{itemize}
\item Modifying the class dictionary prior to the class being created.
\item Returning an instance of another class -- essentially performing
the role of a factory function.
\end{itemize}

\begin{datadesc}{__metaclass__}
This variable can be any callable accepting arguments for \code{name},
\code{bases}, and \code{dict}.  Upon class creation, the callable is
used instead of the built-in \function{type()}.
\versionadded{2.2}                     
\end{datadesc}

The appropriate metaclass is determined by the following precedence rules:

\begin{itemize}

\item If \code{dict['__metaclass__']} exists, it is used.

\item Otherwise, if there is at least one base class, its metaclass is used
(this looks for a \var{__class__} attribute first and if not found, uses its
type).

\item Otherwise, if a global variable named __metaclass__ exists, it is used.

\item Otherwise, the old-style, classic metaclass (types.ClassType) is used.

\end{itemize}      

The potential uses for metaclasses are boundless. Some ideas that have
been explored including logging, interface checking, automatic delegation,
automatic property creation, proxies, frameworks, and automatic resource
locking/synchronization.


\subsection{Emulating callable objects\label{callable-types}}

\begin{methoddesc}[object]{__call__}{self\optional{, args...}}
Called when the instance is ``called'' as a function; if this method
is defined, \code{\var{x}(arg1, arg2, ...)} is a shorthand for
\code{\var{x}.__call__(arg1, arg2, ...)}.
\indexii{call}{instance}
\end{methoddesc}


\subsection{Emulating container types\label{sequence-types}}

The following methods can be defined to implement container
objects.  Containers usually are sequences (such as lists or tuples)
or mappings (like dictionaries), but can represent other containers as
well.  The first set of methods is used either to emulate a
sequence or to emulate a mapping; the difference is that for a
sequence, the allowable keys should be the integers \var{k} for which
\code{0 <= \var{k} < \var{N}} where \var{N} is the length of the
sequence, or slice objects, which define a range of items. (For backwards
compatibility, the method \method{__getslice__()} (see below) can also be
defined to handle simple, but not extended slices.) It is also recommended
that mappings provide the methods \method{keys()}, \method{values()},
\method{items()}, \method{has_key()}, \method{get()}, \method{clear()},
\method{setdefault()}, \method{iterkeys()}, \method{itervalues()},
\method{iteritems()}, \method{pop()}, \method{popitem()},		     
\method{copy()}, and \method{update()} behaving similar to those for
Python's standard dictionary objects.  The \module{UserDict} module
provides a \class{DictMixin} class to help create those methods
from a base set of \method{__getitem__()}, \method{__setitem__()},
\method{__delitem__()}, and \method{keys()}.		     
Mutable sequences should provide
methods \method{append()}, \method{count()}, \method{index()},
\method{extend()},		     
\method{insert()}, \method{pop()}, \method{remove()}, \method{reverse()}
and \method{sort()}, like Python standard list objects.  Finally,
sequence types should implement addition (meaning concatenation) and
multiplication (meaning repetition) by defining the methods
\method{__add__()}, \method{__radd__()}, \method{__iadd__()},
\method{__mul__()}, \method{__rmul__()} and \method{__imul__()} described
below; they should not define \method{__coerce__()} or other numerical
operators.  It is recommended that both mappings and sequences
implement the \method{__contains__()} method to allow efficient use of
the \code{in} operator; for mappings, \code{in} should be equivalent
of \method{has_key()}; for sequences, it should search through the
values.  It is further recommended that both mappings and sequences
implement the \method{__iter__()} method to allow efficient iteration
through the container; for mappings, \method{__iter__()} should be
the same as \method{iterkeys()}; for sequences, it should iterate
through the values.
\withsubitem{(mapping object method)}{
  \ttindex{keys()}
  \ttindex{values()}
  \ttindex{items()}
  \ttindex{iterkeys()}
  \ttindex{itervalues()}
  \ttindex{iteritems()}    
  \ttindex{has_key()}
  \ttindex{get()}
  \ttindex{setdefault()}
  \ttindex{pop()}      
  \ttindex{popitem()}    
  \ttindex{clear()}
  \ttindex{copy()}
  \ttindex{update()}
  \ttindex{__contains__()}}
\withsubitem{(sequence object method)}{
  \ttindex{append()}
  \ttindex{count()}
  \ttindex{extend()}    
  \ttindex{index()}
  \ttindex{insert()}
  \ttindex{pop()}
  \ttindex{remove()}
  \ttindex{reverse()}
  \ttindex{sort()}
  \ttindex{__add__()}
  \ttindex{__radd__()}
  \ttindex{__iadd__()}
  \ttindex{__mul__()}
  \ttindex{__rmul__()}
  \ttindex{__imul__()}
  \ttindex{__contains__()}
  \ttindex{__iter__()}}		     
\withsubitem{(numeric object method)}{\ttindex{__coerce__()}}

\begin{methoddesc}[container object]{__len__}{self}
Called to implement the built-in function
\function{len()}\bifuncindex{len}.  Should return the length of the
object, an integer \code{>=} 0.  Also, an object that doesn't define a
\method{__nonzero__()} method and whose \method{__len__()} method
returns zero is considered to be false in a Boolean context.
\withsubitem{(object method)}{\ttindex{__nonzero__()}}
\end{methoddesc}

\begin{methoddesc}[container object]{__getitem__}{self, key}
Called to implement evaluation of \code{\var{self}[\var{key}]}.
For sequence types, the accepted keys should be integers and slice
objects.\obindex{slice}  Note that
the special interpretation of negative indexes (if the class wishes to
emulate a sequence type) is up to the \method{__getitem__()} method.
If \var{key} is of an inappropriate type, \exception{TypeError} may be
raised; if of a value outside the set of indexes for the sequence
(after any special interpretation of negative values),
\exception{IndexError} should be raised.
For mapping types, if \var{key} is missing (not in the container),
\exception{KeyError} should be raised.                     
\note{\keyword{for} loops expect that an
\exception{IndexError} will be raised for illegal indexes to allow
proper detection of the end of the sequence.}
\end{methoddesc}

\begin{methoddesc}[container object]{__setitem__}{self, key, value}
Called to implement assignment to \code{\var{self}[\var{key}]}.  Same
note as for \method{__getitem__()}.  This should only be implemented
for mappings if the objects support changes to the values for keys, or
if new keys can be added, or for sequences if elements can be
replaced.  The same exceptions should be raised for improper
\var{key} values as for the \method{__getitem__()} method.
\end{methoddesc}

\begin{methoddesc}[container object]{__delitem__}{self, key}
Called to implement deletion of \code{\var{self}[\var{key}]}.  Same
note as for \method{__getitem__()}.  This should only be implemented
for mappings if the objects support removal of keys, or for sequences
if elements can be removed from the sequence.  The same exceptions
should be raised for improper \var{key} values as for the
\method{__getitem__()} method.
\end{methoddesc}

\begin{methoddesc}[container object]{__iter__}{self}
This method is called when an iterator is required for a container.
This method should return a new iterator object that can iterate over
all the objects in the container.  For mappings, it should iterate
over the keys of the container, and should also be made available as
the method \method{iterkeys()}.

Iterator objects also need to implement this method; they are required
to return themselves.  For more information on iterator objects, see
``\ulink{Iterator Types}{../lib/typeiter.html}'' in the
\citetitle[../lib/lib.html]{Python Library Reference}.
\end{methoddesc}

The membership test operators (\keyword{in} and \keyword{not in}) are
normally implemented as an iteration through a sequence.  However,
container objects can supply the following special method with a more
efficient implementation, which also does not require the object be a
sequence.

\begin{methoddesc}[container object]{__contains__}{self, item}
Called to implement membership test operators.  Should return true if
\var{item} is in \var{self}, false otherwise.  For mapping objects,
this should consider the keys of the mapping rather than the values or
the key-item pairs.
\end{methoddesc}


\subsection{Additional methods for emulation of sequence types
  \label{sequence-methods}}

The following optional methods can be defined to further emulate sequence
objects.  Immutable sequences methods should at most only define
\method{__getslice__()}; mutable sequences might define all three
methods.

\begin{methoddesc}[sequence object]{__getslice__}{self, i, j}
\deprecated{2.0}{Support slice objects as parameters to the
\method{__getitem__()} method.}
Called to implement evaluation of \code{\var{self}[\var{i}:\var{j}]}.
The returned object should be of the same type as \var{self}.  Note
that missing \var{i} or \var{j} in the slice expression are replaced
by zero or \code{sys.maxint}, respectively.  If negative indexes are
used in the slice, the length of the sequence is added to that index.
If the instance does not implement the \method{__len__()} method, an
\exception{AttributeError} is raised.
No guarantee is made that indexes adjusted this way are not still
negative.  Indexes which are greater than the length of the sequence
are not modified.
If no \method{__getslice__()} is found, a slice
object is created instead, and passed to \method{__getitem__()} instead.
\end{methoddesc}

\begin{methoddesc}[sequence object]{__setslice__}{self, i, j, sequence}
Called to implement assignment to \code{\var{self}[\var{i}:\var{j}]}.
Same notes for \var{i} and \var{j} as for \method{__getslice__()}.

This method is deprecated. If no \method{__setslice__()} is found,
or for extended slicing of the form
\code{\var{self}[\var{i}:\var{j}:\var{k}]}, a
slice object is created, and passed to \method{__setitem__()},
instead of \method{__setslice__()} being called.
\end{methoddesc}

\begin{methoddesc}[sequence object]{__delslice__}{self, i, j}
Called to implement deletion of \code{\var{self}[\var{i}:\var{j}]}.
Same notes for \var{i} and \var{j} as for \method{__getslice__()}.
This method is deprecated. If no \method{__delslice__()} is found,
or for extended slicing of the form
\code{\var{self}[\var{i}:\var{j}:\var{k}]}, a
slice object is created, and passed to \method{__delitem__()},
instead of \method{__delslice__()} being called.
\end{methoddesc}

Notice that these methods are only invoked when a single slice with a
single colon is used, and the slice method is available.  For slice
operations involving extended slice notation, or in absence of the
slice methods, \method{__getitem__()}, \method{__setitem__()} or
\method{__delitem__()} is called with a slice object as argument.

The following example demonstrate how to make your program or module
compatible with earlier versions of Python (assuming that methods
\method{__getitem__()}, \method{__setitem__()} and \method{__delitem__()}
support slice objects as arguments):

\begin{verbatim}
class MyClass:
    ...
    def __getitem__(self, index):
        ...
    def __setitem__(self, index, value):
        ...
    def __delitem__(self, index):
        ...

    if sys.version_info < (2, 0):
        # They won't be defined if version is at least 2.0 final

        def __getslice__(self, i, j):
            return self[max(0, i):max(0, j):]
        def __setslice__(self, i, j, seq):
            self[max(0, i):max(0, j):] = seq
        def __delslice__(self, i, j):
            del self[max(0, i):max(0, j):]
    ...
\end{verbatim}

Note the calls to \function{max()}; these are necessary because of
the handling of negative indices before the
\method{__*slice__()} methods are called.  When negative indexes are
used, the \method{__*item__()} methods receive them as provided, but
the \method{__*slice__()} methods get a ``cooked'' form of the index
values.  For each negative index value, the length of the sequence is
added to the index before calling the method (which may still result
in a negative index); this is the customary handling of negative
indexes by the built-in sequence types, and the \method{__*item__()}
methods are expected to do this as well.  However, since they should
already be doing that, negative indexes cannot be passed in; they must
be constrained to the bounds of the sequence before being passed to
the \method{__*item__()} methods.
Calling \code{max(0, i)} conveniently returns the proper value.


\subsection{Emulating numeric types\label{numeric-types}}

The following methods can be defined to emulate numeric objects.
Methods corresponding to operations that are not supported by the
particular kind of number implemented (e.g., bitwise operations for
non-integral numbers) should be left undefined.

\begin{methoddesc}[numeric object]{__add__}{self, other}
\methodline[numeric object]{__sub__}{self, other}
\methodline[numeric object]{__mul__}{self, other}
\methodline[numeric object]{__floordiv__}{self, other}
\methodline[numeric object]{__mod__}{self, other}
\methodline[numeric object]{__divmod__}{self, other}
\methodline[numeric object]{__pow__}{self, other\optional{, modulo}}
\methodline[numeric object]{__lshift__}{self, other}
\methodline[numeric object]{__rshift__}{self, other}
\methodline[numeric object]{__and__}{self, other}
\methodline[numeric object]{__xor__}{self, other}
\methodline[numeric object]{__or__}{self, other}
These methods are
called to implement the binary arithmetic operations (\code{+},
\code{-}, \code{*}, \code{//}, \code{\%},
\function{divmod()}\bifuncindex{divmod},
\function{pow()}\bifuncindex{pow}, \code{**}, \code{<<},
\code{>>}, \code{\&}, \code{\^}, \code{|}).  For instance, to
evaluate the expression \var{x}\code{+}\var{y}, where \var{x} is an
instance of a class that has an \method{__add__()} method,
\code{\var{x}.__add__(\var{y})} is called.  The \method{__divmod__()}
method should be the equivalent to using \method{__floordiv__()} and
\method{__mod__()}; it should not be related to \method{__truediv__()}
(described below).  Note that
\method{__pow__()} should be defined to accept an optional third
argument if the ternary version of the built-in
\function{pow()}\bifuncindex{pow} function is to be supported.

If one of those methods does not support the operation with the
supplied arguments, it should return \code{NotImplemented}.
\end{methoddesc}

\begin{methoddesc}[numeric object]{__div__}{self, other}
\methodline[numeric object]{__truediv__}{self, other}
The division operator (\code{/}) is implemented by these methods.  The
\method{__truediv__()} method is used when \code{__future__.division}
is in effect, otherwise \method{__div__()} is used.  If only one of
these two methods is defined, the object will not support division in
the alternate context; \exception{TypeError} will be raised instead.
\end{methoddesc}

\begin{methoddesc}[numeric object]{__radd__}{self, other}
\methodline[numeric object]{__rsub__}{self, other}
\methodline[numeric object]{__rmul__}{self, other}
\methodline[numeric object]{__rdiv__}{self, other}
\methodline[numeric object]{__rtruediv__}{self, other}
\methodline[numeric object]{__rfloordiv__}{self, other}	     
\methodline[numeric object]{__rmod__}{self, other}
\methodline[numeric object]{__rdivmod__}{self, other}
\methodline[numeric object]{__rpow__}{self, other}
\methodline[numeric object]{__rlshift__}{self, other}
\methodline[numeric object]{__rrshift__}{self, other}
\methodline[numeric object]{__rand__}{self, other}
\methodline[numeric object]{__rxor__}{self, other}
\methodline[numeric object]{__ror__}{self, other}
These methods are
called to implement the binary arithmetic operations (\code{+},
\code{-}, \code{*}, \code{/}, \code{\%},
\function{divmod()}\bifuncindex{divmod},
\function{pow()}\bifuncindex{pow}, \code{**}, \code{<<},
\code{>>}, \code{\&}, \code{\^}, \code{|}) with reflected
(swapped) operands.  These functions are only called if the left
operand does not support the corresponding operation and the
operands are of different types.\footnote{
    For operands of the same type, it is assumed that if the
    non-reflected method (such as \method{__add__()}) fails the
    operation is not supported, which is why the reflected method
    is not called.} 
For instance, to evaluate the expression \var{x}\code{-}\var{y},
where \var{y} is an instance of a class that has an
\method{__rsub__()} method, \code{\var{y}.__rsub__(\var{x})}
is called if \code{\var{x}.__sub__(\var{y})} returns
\var{NotImplemented}.

Note that ternary
\function{pow()}\bifuncindex{pow} will not try calling
\method{__rpow__()} (the coercion rules would become too
complicated).

\note{If the right operand's type is a subclass of the left operand's
      type and that subclass provides the reflected method for the
      operation, this method will be called before the left operand's
      non-reflected method.  This behavior allows subclasses to
      override their ancestors' operations.}
\end{methoddesc}

\begin{methoddesc}[numeric object]{__iadd__}{self, other}
\methodline[numeric object]{__isub__}{self, other}
\methodline[numeric object]{__imul__}{self, other}
\methodline[numeric object]{__idiv__}{self, other}
\methodline[numeric object]{__itruediv__}{self, other}
\methodline[numeric object]{__ifloordiv__}{self, other}
\methodline[numeric object]{__imod__}{self, other}		     
\methodline[numeric object]{__ipow__}{self, other\optional{, modulo}}
\methodline[numeric object]{__ilshift__}{self, other}
\methodline[numeric object]{__irshift__}{self, other}
\methodline[numeric object]{__iand__}{self, other}
\methodline[numeric object]{__ixor__}{self, other}
\methodline[numeric object]{__ior__}{self, other}
These methods are called to implement the augmented arithmetic
operations (\code{+=}, \code{-=}, \code{*=}, \code{/=}, \code{//=},
\code{\%=}, \code{**=}, \code{<<=}, \code{>>=}, \code{\&=},
\code{\textasciicircum=}, \code{|=}).  These methods should attempt to do the
operation in-place (modifying \var{self}) and return the result (which
could be, but does not have to be, \var{self}).  If a specific method
is not defined, the augmented operation falls back to the normal
methods.  For instance, to evaluate the expression
\var{x}\code{+=}\var{y}, where \var{x} is an instance of a class that
has an \method{__iadd__()} method, \code{\var{x}.__iadd__(\var{y})} is
called.  If \var{x} is an instance of a class that does not define a
\method{__iadd__()} method, \code{\var{x}.__add__(\var{y})} and
\code{\var{y}.__radd__(\var{x})} are considered, as with the
evaluation of \var{x}\code{+}\var{y}.
\end{methoddesc}

\begin{methoddesc}[numeric object]{__neg__}{self}
\methodline[numeric object]{__pos__}{self}
\methodline[numeric object]{__abs__}{self}
\methodline[numeric object]{__invert__}{self}
Called to implement the unary arithmetic operations (\code{-},
\code{+}, \function{abs()}\bifuncindex{abs} and \code{\~{}}).
\end{methoddesc}

\begin{methoddesc}[numeric object]{__complex__}{self}
\methodline[numeric object]{__int__}{self}
\methodline[numeric object]{__long__}{self}
\methodline[numeric object]{__float__}{self}
Called to implement the built-in functions
\function{complex()}\bifuncindex{complex},
\function{int()}\bifuncindex{int}, \function{long()}\bifuncindex{long},
and \function{float()}\bifuncindex{float}.  Should return a value of
the appropriate type.
\end{methoddesc}

\begin{methoddesc}[numeric object]{__oct__}{self}
\methodline[numeric object]{__hex__}{self}
Called to implement the built-in functions
\function{oct()}\bifuncindex{oct} and
\function{hex()}\bifuncindex{hex}.  Should return a string value.
\end{methoddesc}

\begin{methoddesc}[numeric object]{__index__}{self}
Called to implement \function{operator.index()}.  Also called whenever
Python needs an integer object (such as in slicing).  Must return an
integer (int or long).
\versionadded{2.5}
\end{methoddesc}

\begin{methoddesc}[numeric object]{__coerce__}{self, other}
Called to implement ``mixed-mode'' numeric arithmetic.  Should either
return a 2-tuple containing \var{self} and \var{other} converted to
a common numeric type, or \code{None} if conversion is impossible.  When
the common type would be the type of \code{other}, it is sufficient to
return \code{None}, since the interpreter will also ask the other
object to attempt a coercion (but sometimes, if the implementation of
the other type cannot be changed, it is useful to do the conversion to
the other type here).  A return value of \code{NotImplemented} is
equivalent to returning \code{None}.
\end{methoddesc}

\subsection{Coercion rules\label{coercion-rules}}

This section used to document the rules for coercion.  As the language
has evolved, the coercion rules have become hard to document
precisely; documenting what one version of one particular
implementation does is undesirable.  Instead, here are some informal
guidelines regarding coercion.  In Python 3.0, coercion will not be
supported.

\begin{itemize}

\item

If the left operand of a \% operator is a string or Unicode object, no
coercion takes place and the string formatting operation is invoked
instead.

\item

It is no longer recommended to define a coercion operation.
Mixed-mode operations on types that don't define coercion pass the
original arguments to the operation.

\item

New-style classes (those derived from \class{object}) never invoke the
\method{__coerce__()} method in response to a binary operator; the only
time \method{__coerce__()} is invoked is when the built-in function
\function{coerce()} is called.

\item

For most intents and purposes, an operator that returns
\code{NotImplemented} is treated the same as one that is not
implemented at all.

\item

Below, \method{__op__()} and \method{__rop__()} are used to signify
the generic method names corresponding to an operator;
\method{__iop__()} is used for the corresponding in-place operator.  For
example, for the operator `\code{+}', \method{__add__()} and
\method{__radd__()} are used for the left and right variant of the
binary operator, and \method{__iadd__()} for the in-place variant.

\item

For objects \var{x} and \var{y}, first \code{\var{x}.__op__(\var{y})}
is tried.  If this is not implemented or returns \code{NotImplemented},
\code{\var{y}.__rop__(\var{x})} is tried.  If this is also not
implemented or returns \code{NotImplemented}, a \exception{TypeError}
exception is raised.  But see the following exception:

\item

Exception to the previous item: if the left operand is an instance of
a built-in type or a new-style class, and the right operand is an instance
of a proper subclass of that type or class and overrides the base's
\method{__rop__()} method, the right operand's \method{__rop__()} method
is tried \emph{before} the left operand's \method{__op__()} method.

This is done so that a subclass can completely override binary operators.
Otherwise, the left operand's \method{__op__()} method would always
accept the right operand: when an instance of a given class is expected,
an instance of a subclass of that class is always acceptable.

\item

When either operand type defines a coercion, this coercion is called
before that type's \method{__op__()} or \method{__rop__()} method is
called, but no sooner.  If the coercion returns an object of a
different type for the operand whose coercion is invoked, part of the
process is redone using the new object.

\item

When an in-place operator (like `\code{+=}') is used, if the left
operand implements \method{__iop__()}, it is invoked without any
coercion.  When the operation falls back to \method{__op__()} and/or
\method{__rop__()}, the normal coercion rules apply.

\item

In \var{x}\code{+}\var{y}, if \var{x} is a sequence that implements
sequence concatenation, sequence concatenation is invoked.

\item

In \var{x}\code{*}\var{y}, if one operator is a sequence that
implements sequence repetition, and the other is an integer
(\class{int} or \class{long}), sequence repetition is invoked.

\item

Rich comparisons (implemented by methods \method{__eq__()} and so on)
never use coercion.  Three-way comparison (implemented by
\method{__cmp__()}) does use coercion under the same conditions as
other binary operations use it.

\item

In the current implementation, the built-in numeric types \class{int},
\class{long} and \class{float} do not use coercion; the type
\class{complex} however does use it.  The difference can become
apparent when subclassing these types.  Over time, the type
\class{complex} may be fixed to avoid coercion.  All these types
implement a \method{__coerce__()} method, for use by the built-in
\function{coerce()} function.

\end{itemize}

\subsection{With Statement Context Managers\label{context-managers}}

\versionadded{2.5}

A \dfn{context manager} is an object that defines the runtime
context to be established when executing a \keyword{with}
statement. The context manager handles the entry into,
and the exit from, the desired runtime context for the execution
of the block of code.  Context managers are normally invoked using
the \keyword{with} statement (described in section~\ref{with}), but
can also be used by directly invoking their methods.

\stindex{with}
\index{context manager}

Typical uses of context managers include saving and
restoring various kinds of global state, locking and unlocking
resources, closing opened files, etc.

For more information on context managers, see
``\ulink{Context Types}{../lib/typecontextmanager.html}'' in the
\citetitle[../lib/lib.html]{Python Library Reference}.

\begin{methoddesc}[context manager]{__enter__}{self}
Enter the runtime context related to this object. The \keyword{with}
statement will bind this method's return value to the target(s)
specified in the \keyword{as} clause of the statement, if any.
\end{methoddesc}

\begin{methoddesc}[context manager]{__exit__}
{self, exc_type, exc_value, traceback}
Exit the runtime context related to this object. The parameters
describe the exception that caused the context to be exited. If
the context was exited without an exception, all three arguments
will be \constant{None}.

If an exception is supplied, and the method wishes to suppress the
exception (i.e., prevent it from being propagated), it should return a
true value. Otherwise, the exception will be processed normally upon
exit from this method.

Note that \method{__exit__} methods should not reraise the passed-in
exception; this is the caller's responsibility.
\end{methoddesc}

\begin{seealso}
  \seepep{0343}{The "with" statement}
         {The specification, background, and examples for the
          Python \keyword{with} statement.}
\end{seealso}

		% Data model
\chapter{Execution model \label{execmodel}}
\index{execution model}


\section{Naming and binding \label{naming}}
\indexii{code}{block}
\index{namespace}
\index{scope}

\dfn{Names}\index{name} refer to objects.  Names are introduced by
name binding operations.  Each occurrence of a name in the program
text refers to the \dfn{binding}\indexii{binding}{name} of that name
established in the innermost function block containing the use.

A \dfn{block}\index{block} is a piece of Python program text that is
executed as a unit.  The following are blocks: a module, a function
body, and a class definition.  Each command typed interactively is a
block.  A script file (a file given as standard input to the
interpreter or specified on the interpreter command line the first
argument) is a code block.  A script command (a command specified on
the interpreter command line with the `\strong{-c}' option) is a code
block.  The file read by the built-in function \function{execfile()}
is a code block.  The string argument passed to the built-in function
\function{eval()} and to the \keyword{exec} statement is a code block.
The expression read and evaluated by the built-in function
\function{input()} is a code block.

A code block is executed in an \dfn{execution
frame}\indexii{execution}{frame}.  A frame contains some
administrative information (used for debugging) and determines where
and how execution continues after the code block's execution has
completed.

A \dfn{scope}\index{scope} defines the visibility of a name within a
block.  If a local variable is defined in a block, its scope includes
that block.  If the definition occurs in a function block, the scope
extends to any blocks contained within the defining one, unless a
contained block introduces a different binding for the name.  The
scope of names defined in a class block is limited to the class block;
it does not extend to the code blocks of methods.

When a name is used in a code block, it is resolved using the nearest
enclosing scope.  The set of all such scopes visible to a code block
is called the block's \dfn{environment}\index{environment}.  

If a name is bound in a block, it is a local variable of that block.
If a name is bound at the module level, it is a global variable.  (The
variables of the module code block are local and global.)  If a
variable is used in a code block but not defined there, it is a
\dfn{free variable}\indexii{free}{variable}.

When a name is not found at all, a
\exception{NameError}\withsubitem{(built-in
exception)}{\ttindex{NameError}} exception is raised.  If the name
refers to a local variable that has not been bound, a
\exception{UnboundLocalError}\ttindex{UnboundLocalError} exception is
raised.  \exception{UnboundLocalError} is a subclass of
\exception{NameError}.

The following constructs bind names: formal parameters to functions,
\keyword{import} statements, class and function definitions (these
bind the class or function name in the defining block), and targets
that are identifiers if occurring in an assignment, \keyword{for} loop
header, or in the second position of an \keyword{except} clause
header.  The \keyword{import} statement of the form ``\samp{from
\ldots import *}''\stindex{from} binds all names defined in the
imported module, except those beginning with an underscore.  This form
may only be used at the module level.

A target occurring in a \keyword{del} statement is also considered bound
for this purpose (though the actual semantics are to unbind the
name).  It is illegal to unbind a name that is referenced by an
enclosing scope; the compiler will report a \exception{SyntaxError}.

Each assignment or import statement occurs within a block defined by a
class or function definition or at the module level (the top-level
code block).

If a name binding operation occurs anywhere within a code block, all
uses of the name within the block are treated as references to the
current block.  This can lead to errors when a name is used within a
block before it is bound.
This rule is subtle.  Python lacks declarations and allows
name binding operations to occur anywhere within a code block.  The
local variables of a code block can be determined by scanning the
entire text of the block for name binding operations.

If the global statement occurs within a block, all uses of the name
specified in the statement refer to the binding of that name in the
top-level namespace.  Names are resolved in the top-level namespace by
searching the global namespace, i.e. the namespace of the module
containing the code block, and the builtin namespace, the namespace of
the module \module{__builtin__}.  The global namespace is searched
first.  If the name is not found there, the builtin namespace is
searched.  The global statement must precede all uses of the name.

The built-in namespace associated with the execution of a code block
is actually found by looking up the name \code{__builtins__} in its
global namespace; this should be a dictionary or a module (in the
latter case the module's dictionary is used).  By default, when in the
\module{__main__} module, \code{__builtins__} is the built-in module
\module{__builtin__} (note: no `s'); when in any other module,
\code{__builtins__} is an alias for the dictionary of the
\module{__builtin__} module itself.  \code{__builtins__} can be set
to a user-created dictionary to create a weak form of restricted
execution\indexii{restricted}{execution}.

\begin{notice}
  Users should not touch \code{__builtins__}; it is strictly an
  implementation detail.  Users wanting to override values in the
  built-in namespace should \keyword{import} the \module{__builtin__}
  (no `s') module and modify its attributes appropriately.
\end{notice}

The namespace for a module is automatically created the first time a
module is imported.  The main module for a script is always called
\module{__main__}\refbimodindex{__main__}.

The global statement has the same scope as a name binding operation
in the same block.  If the nearest enclosing scope for a free variable
contains a global statement, the free variable is treated as a global.

A class definition is an executable statement that may use and define
names.  These references follow the normal rules for name resolution.
The namespace of the class definition becomes the attribute dictionary
of the class.  Names defined at the class scope are not visible in
methods. 

\subsection{Interaction with dynamic features \label{dynamic-features}}

There are several cases where Python statements are illegal when
used in conjunction with nested scopes that contain free
variables.

If a variable is referenced in an enclosing scope, it is illegal
to delete the name.  An error will be reported at compile time.

If the wild card form of import --- \samp{import *} --- is used in a
function and the function contains or is a nested block with free
variables, the compiler will raise a \exception{SyntaxError}.

If \keyword{exec} is used in a function and the function contains or
is a nested block with free variables, the compiler will raise a
\exception{SyntaxError} unless the exec explicitly specifies the local
namespace for the \keyword{exec}.  (In other words, \samp{exec obj}
would be illegal, but \samp{exec obj in ns} would be legal.)

The \function{eval()}, \function{execfile()}, and \function{input()}
functions and the \keyword{exec} statement do not have access to the
full environment for resolving names.  Names may be resolved in the
local and global namespaces of the caller.  Free variables are not
resolved in the nearest enclosing namespace, but in the global
namespace.\footnote{This limitation occurs because the code that is
    executed by these operations is not available at the time the
    module is compiled.}
The \keyword{exec} statement and the \function{eval()} and
\function{execfile()} functions have optional arguments to override
the global and local namespace.  If only one namespace is specified,
it is used for both.

\section{Exceptions \label{exceptions}}
\index{exception}

Exceptions are a means of breaking out of the normal flow of control
of a code block in order to handle errors or other exceptional
conditions.  An exception is
\emph{raised}\index{raise an exception} at the point where the error
is detected; it may be \emph{handled}\index{handle an exception} by
the surrounding code block or by any code block that directly or
indirectly invoked the code block where the error occurred.
\index{exception handler}
\index{errors}
\index{error handling}

The Python interpreter raises an exception when it detects a run-time
error (such as division by zero).  A Python program can also
explicitly raise an exception with the \keyword{raise} statement.
Exception handlers are specified with the \keyword{try} ... \keyword{except}
statement.  The \keyword{try} ... \keyword{finally} statement
specifies cleanup code which does not handle the exception, but is
executed whether an exception occurred or not in the preceding code.

Python uses the ``termination''\index{termination model} model of
error handling: an exception handler can find out what happened and
continue execution at an outer level, but it cannot repair the cause
of the error and retry the failing operation (except by re-entering
the offending piece of code from the top).

When an exception is not handled at all, the interpreter terminates
execution of the program, or returns to its interactive main loop.  In
either case, it prints a stack backtrace, except when the exception is 
\exception{SystemExit}\withsubitem{(built-in
exception)}{\ttindex{SystemExit}}.

Exceptions are identified by class instances.  The \keyword{except}
clause is selected depending on the class of the instance: it must
reference the class of the instance or a base class thereof.  The
instance can be received by the handler and can carry additional
information about the exceptional condition.

Exceptions can also be identified by strings, in which case the
\keyword{except} clause is selected by object identity.  An arbitrary
value can be raised along with the identifying string which can be
passed to the handler.

\deprecated{2.5}{String exceptions should not be used in new code.
They will not be supported in a future version of Python.  Old code
should be rewritten to use class exceptions instead.}

\begin{notice}[warning]
Messages to exceptions are not part of the Python API.  Their contents may
change from one version of Python to the next without warning and should not
be relied on by code which will run under multiple versions of the
interpreter.
\end{notice}

See also the description of the \keyword{try} statement in
section~\ref{try} and \keyword{raise} statement in
section~\ref{raise}.
		% Execution model
\chapter{Expressions\label{expressions}}
\index{expression}

This chapter explains the meaning of the elements of expressions in
Python.

\strong{Syntax Notes:} In this and the following chapters, extended
BNF\index{BNF} notation will be used to describe syntax, not lexical
analysis.  When (one alternative of) a syntax rule has the form

\begin{productionlist}[*]
  \production{name}{\token{othername}}
\end{productionlist}

and no semantics are given, the semantics of this form of \code{name}
are the same as for \code{othername}.
\index{syntax}


\section{Arithmetic conversions\label{conversions}}
\indexii{arithmetic}{conversion}

When a description of an arithmetic operator below uses the phrase
``the numeric arguments are converted to a common type,'' the
arguments are coerced using the coercion rules listed at
~\ref{coercion-rules}.  If both arguments are standard numeric types,
the following coercions are applied:

\begin{itemize}
\item	If either argument is a complex number, the other is converted
	to complex;
\item	otherwise, if either argument is a floating point number,
	the other is converted to floating point;
\item	otherwise, if either argument is a long integer,
	the other is converted to long integer;
\item	otherwise, both must be plain integers and no conversion
	is necessary.
\end{itemize}

Some additional rules apply for certain operators (e.g., a string left
argument to the `\%' operator). Extensions can define their own
coercions.


\section{Atoms\label{atoms}}
\index{atom}

Atoms are the most basic elements of expressions.  The simplest atoms
are identifiers or literals.  Forms enclosed in
reverse quotes or in parentheses, brackets or braces are also
categorized syntactically as atoms.  The syntax for atoms is:

\begin{productionlist}
  \production{atom}
             {\token{identifier} | \token{literal} | \token{enclosure}}
  \production{enclosure}
             {\token{parenth_form} | \token{list_display}}
  \productioncont{| \token{generator_expression} | \token{dict_display}}
  \productioncont{| \token{string_conversion} | \token{yield_atom}}
\end{productionlist}


\subsection{Identifiers (Names)\label{atom-identifiers}}
\index{name}
\index{identifier}

An identifier occurring as an atom is a name.  See
section \ref{identifiers} for lexical definition and
section~\ref{naming} for documentation of naming and binding.

When the name is bound to an object, evaluation of the atom yields
that object.  When a name is not bound, an attempt to evaluate it
raises a \exception{NameError} exception.
\exindex{NameError}

\strong{Private name mangling:}
\indexii{name}{mangling}%
\indexii{private}{names}%
When an identifier that textually occurs in a class definition begins
with two or more underscore characters and does not end in two or more
underscores, it is considered a \dfn{private name} of that class.
Private names are transformed to a longer form before code is
generated for them.  The transformation inserts the class name in
front of the name, with leading underscores removed, and a single
underscore inserted in front of the class name.  For example, the
identifier \code{__spam} occurring in a class named \code{Ham} will be
transformed to \code{_Ham__spam}.  This transformation is independent
of the syntactical context in which the identifier is used.  If the
transformed name is extremely long (longer than 255 characters),
implementation defined truncation may happen.  If the class name
consists only of underscores, no transformation is done.


\subsection{Literals\label{atom-literals}}
\index{literal}

Python supports string literals and various numeric literals:

\begin{productionlist}
  \production{literal}
             {\token{stringliteral} | \token{integer} | \token{longinteger}}
  \productioncont{| \token{floatnumber} | \token{imagnumber}}
\end{productionlist}

Evaluation of a literal yields an object of the given type (string,
integer, long integer, floating point number, complex number) with the
given value.  The value may be approximated in the case of floating
point and imaginary (complex) literals.  See section \ref{literals}
for details.

All literals correspond to immutable data types, and hence the
object's identity is less important than its value.  Multiple
evaluations of literals with the same value (either the same
occurrence in the program text or a different occurrence) may obtain
the same object or a different object with the same value.
\indexiii{immutable}{data}{type}
\indexii{immutable}{object}


\subsection{Parenthesized forms\label{parenthesized}}
\index{parenthesized form}

A parenthesized form is an optional expression list enclosed in
parentheses:

\begin{productionlist}
  \production{parenth_form}
             {"(" [\token{expression_list}] ")"}
\end{productionlist}

A parenthesized expression list yields whatever that expression list
yields: if the list contains at least one comma, it yields a tuple;
otherwise, it yields the single expression that makes up the
expression list.

An empty pair of parentheses yields an empty tuple object.  Since
tuples are immutable, the rules for literals apply (i.e., two
occurrences of the empty tuple may or may not yield the same object).
\indexii{empty}{tuple}

Note that tuples are not formed by the parentheses, but rather by use
of the comma operator.  The exception is the empty tuple, for which
parentheses \emph{are} required --- allowing unparenthesized ``nothing''
in expressions would cause ambiguities and allow common typos to
pass uncaught.
\index{comma}
\indexii{tuple}{display}


\subsection{List displays\label{lists}}
\indexii{list}{display}
\indexii{list}{comprehensions}

A list display is a possibly empty series of expressions enclosed in
square brackets:

\begin{productionlist}
  \production{list_display}
             {"[" [\token{expression_list} | \token{list_comprehension}] "]"}
  \production{list_comprehension}
             {\token{expression} \token{list_for}}
  \production{list_for}
             {"for" \token{target_list} "in" \token{old_expression_list}
              [\token{list_iter}]}
  \production{old_expression_list}
             {\token{old_expression}
              [("," \token{old_expression})+ [","]]}
  \production{list_iter}
             {\token{list_for} | \token{list_if}}
  \production{list_if}
             {"if" \token{old_expression} [\token{list_iter}]}
\end{productionlist}

A list display yields a new list object.  Its contents are specified
by providing either a list of expressions or a list comprehension.
\indexii{list}{comprehensions}
When a comma-separated list of expressions is supplied, its elements are
evaluated from left to right and placed into the list object in that
order.  When a list comprehension is supplied, it consists of a
single expression followed by at least one \keyword{for} clause and zero or
more \keyword{for} or \keyword{if} clauses.  In this
case, the elements of the new list are those that would be produced
by considering each of the \keyword{for} or \keyword{if} clauses a block,
nesting from
left to right, and evaluating the expression to produce a list element
each time the innermost block is reached\footnote{In Python 2.3, a
list comprehension "leaks" the control variables of each
\samp{for} it contains into the containing scope.  However, this
behavior is deprecated, and relying on it will not work once this
bug is fixed in a future release}.
\obindex{list}
\indexii{empty}{list}


\subsection{Generator expressions\label{genexpr}}
\indexii{generator}{expression}

A generator expression is a compact generator notation in parentheses:

\begin{productionlist}
  \production{generator_expression}
             {"(" \token{expression} \token{genexpr_for} ")"}
  \production{genexpr_for}
             {"for" \token{target_list} "in" \token{or_test}
              [\token{genexpr_iter}]}
  \production{genexpr_iter}
             {\token{genexpr_for} | \token{genexpr_if}}
  \production{genexpr_if}
             {"if" \token{old_expression} [\token{genexpr_iter}]}
\end{productionlist}

A generator expression yields a new generator object.
\obindex{generator}
It consists of a single expression followed by at least one
\keyword{for} clause and zero or more \keyword{for} or \keyword{if}
clauses.  The iterating values of the new generator are those that
would be produced by considering each of the \keyword{for} or
\keyword{if} clauses a block, nesting from left to right, and
evaluating the expression to yield a value that is reached the
innermost block for each iteration.

Variables used in the generator expression are evaluated lazily
when the \method{next()} method is called for generator object
(in the same fashion as normal generators). However, the leftmost
\keyword{for} clause is immediately evaluated so that error produced
by it can be seen before any other possible error in the code that
handles the generator expression.
Subsequent \keyword{for} clauses cannot be evaluated immediately since
they may depend on the previous \keyword{for} loop.
For example: \samp{(x*y for x in range(10) for y in bar(x))}.

The parentheses can be omitted on calls with only one argument.
See section \ref{calls} for the detail.


\subsection{Dictionary displays\label{dict}}
\indexii{dictionary}{display}

A dictionary display is a possibly empty series of key/datum pairs
enclosed in curly braces:
\index{key}
\index{datum}
\index{key/datum pair}

\begin{productionlist}
  \production{dict_display}
             {"\{" [\token{key_datum_list}] "\}"}
  \production{key_datum_list}
             {\token{key_datum} ("," \token{key_datum})* [","]}
  \production{key_datum}
             {\token{expression} ":" \token{expression}}
\end{productionlist}

A dictionary display yields a new dictionary object.
\obindex{dictionary}

The key/datum pairs are evaluated from left to right to define the
entries of the dictionary: each key object is used as a key into the
dictionary to store the corresponding datum.

Restrictions on the types of the key values are listed earlier in
section \ref{types}.  (To summarize, the key type should be hashable,
which excludes all mutable objects.)  Clashes between duplicate keys
are not detected; the last datum (textually rightmost in the display)
stored for a given key value prevails.
\indexii{immutable}{object}


\subsection{String conversions\label{string-conversions}}
\indexii{string}{conversion}
\indexii{reverse}{quotes}
\indexii{backward}{quotes}
\index{back-quotes}

A string conversion is an expression list enclosed in reverse (a.k.a.
backward) quotes:

\begin{productionlist}
  \production{string_conversion}
             {"`" \token{expression_list} "`"}
\end{productionlist}

A string conversion evaluates the contained expression list and
converts the resulting object into a string according to rules
specific to its type.

If the object is a string, a number, \code{None}, or a tuple, list or
dictionary containing only objects whose type is one of these, the
resulting string is a valid Python expression which can be passed to
the built-in function \function{eval()} to yield an expression with the
same value (or an approximation, if floating point numbers are
involved).

(In particular, converting a string adds quotes around it and converts
``funny'' characters to escape sequences that are safe to print.)

Recursive objects (for example, lists or dictionaries that contain a
reference to themselves, directly or indirectly) use \samp{...} to
indicate a recursive reference, and the result cannot be passed to
\function{eval()} to get an equal value (\exception{SyntaxError} will
be raised instead).
\obindex{recursive}

The built-in function \function{repr()} performs exactly the same
conversion in its argument as enclosing it in parentheses and reverse
quotes does.  The built-in function \function{str()} performs a
similar but more user-friendly conversion.
\bifuncindex{repr}
\bifuncindex{str}


\subsection{Yield expressions\label{yieldexpr}}
\kwindex{yield}
\indexii{yield}{expression}
\indexii{generator}{function}

\begin{productionlist}
  \production{yield_atom}
             {"(" \token{yield_expression} ")"}
  \production{yield_expression}
             {"yield" [\token{expression_list}]}
\end{productionlist}

\versionadded{2.5}

The \keyword{yield} expression is only used when defining a generator
function, and can only be used in the body of a function definition.
Using a \keyword{yield} expression in a function definition is
sufficient to cause that definition to create a generator function
instead of a normal function.

When a generator function is called, it returns an iterator known as a
generator.  That generator then controls the execution of a generator
function.  The execution starts when one of the generator's methods is
called.  At that time, the execution proceeds to the first
\keyword{yield} expression, where it is suspended again, returning the
value of \grammartoken{expression_list} to generator's caller.  By
suspended we mean that all local state is retained, including the
current bindings of local variables, the instruction pointer, and the
internal evaluation stack.  When the execution is resumed by calling
one of the generator's methods, the function can proceed exactly as
if the \keyword{yield} expression was just another external call.
The value of the \keyword{yield} expression after resuming depends on
the method which resumed the execution.

\index{coroutine}

All of this makes generator functions quite similar to coroutines; they
yield multiple times, they have more than one entry point and their
execution can be suspended.  The only difference is that a generator
function cannot control where should the execution continue after it
yields; the control is always transfered to the generator's caller.

\obindex{generator}

The following generator's methods can be used to control the execution
of a generator function:

\exindex{StopIteration}

\begin{methoddesc}[generator]{next}{}
  Starts the execution of a generator function or resumes it at the
  last executed \keyword{yield} expression.  When a generator function
  is resumed with a \method{next()} method, the current \keyword{yield}
  expression always evaluates to \constant{None}.  The execution then
  continues to the next \keyword{yield} expression, where the generator
  is suspended again, and the value of the
  \grammartoken{expression_list} is returned to \method{next()}'s
  caller. If the generator exits without yielding another value, a
  \exception{StopIteration} exception is raised.
\end{methoddesc}

\begin{methoddesc}[generator]{send}{value}
  Resumes the execution and ``sends'' a value into the generator
  function.  The \code{value} argument becomes the result of the
  current \keyword{yield} expression.  The \method{send()} method
  returns the next value yielded by the generator, or raises
  \exception{StopIteration} if the generator exits without yielding
  another value.
  When \method{send()} is called to start the generator, it must be
  called with \constant{None} as the argument, because there is no
  \keyword{yield} expression that could receieve the value.
\end{methoddesc}

\begin{methoddesc}[generator]{throw}
                  {type\optional{, value\optional{, traceback}}}
  Raises an exception of type \code{type} at the point where generator
  was paused, and returns the next value yielded by the generator
  function.  If the generator exits without yielding another value, a
  \exception{StopIteration} exception is raised.  If the generator
  function does not catch the passed-in exception, or raises a
  different exception, then that exception propagates to the caller.
\end{methoddesc}

\exindex{GeneratorExit}

\begin{methoddesc}[generator]{close}{}
  Raises a \exception{GeneratorExit} at the point where the generator
  function was paused.  If the generator function then raises
  \exception{StopIteration} (by exiting normally, or due to already
  being closed) or \exception{GeneratorExit} (by not catching the
  exception), close returns to its caller.  If the generator yields a
  value, a \exception{RuntimeError} is raised.  If the generator raises
  any other exception, it is propagated to the caller.  \method{close}
  does nothing if the generator has already exited due to an exception
  or normal exit.
\end{methoddesc}

Here is a simple example that demonstrates the behavior of generators
and generator functions:

\begin{verbatim}
>>> def echo(value=None):
...     print "Execution starts when 'next()' is called for the first time."
...     try:
...         while True:
...             try:
...                 value = (yield value)
...             except GeneratorExit:
...                 # never catch GeneratorExit
...                 raise
...             except Exception, e:
...                 value = e
...     finally:
...         print "Don't forget to clean up when 'close()' is called."
...
>>> generator = echo(1)
>>> print generator.next()
Execution starts when 'next()' is called for the first time.
1
>>> print generator.next()
None
>>> print generator.send(2)
2
>>> generator.throw(TypeError, "spam")
TypeError('spam',)
>>> generator.close()
Don't forget to clean up when 'close()' is called.
\end{verbatim}

\begin{seealso}
  \seepep{0342}{Coroutines via Enhanced Generators}
         {The proposal to enhance the API and syntax of generators,
          making them usable as simple coroutines.}
\end{seealso}


\section{Primaries\label{primaries}}
\index{primary}

Primaries represent the most tightly bound operations of the language.
Their syntax is:

\begin{productionlist}
  \production{primary}
             {\token{atom} | \token{attributeref}
              | \token{subscription} | \token{slicing} | \token{call}}
\end{productionlist}


\subsection{Attribute references\label{attribute-references}}
\indexii{attribute}{reference}

An attribute reference is a primary followed by a period and a name:

\begin{productionlist}
  \production{attributeref}
             {\token{primary} "." \token{identifier}}
\end{productionlist}

The primary must evaluate to an object of a type that supports
attribute references, e.g., a module, list, or an instance.  This
object is then asked to produce the attribute whose name is the
identifier.  If this attribute is not available, the exception
\exception{AttributeError}\exindex{AttributeError} is raised.
Otherwise, the type and value of the object produced is determined by
the object.  Multiple evaluations of the same attribute reference may
yield different objects.
\obindex{module}
\obindex{list}


\subsection{Subscriptions\label{subscriptions}}
\index{subscription}

A subscription selects an item of a sequence (string, tuple or list)
or mapping (dictionary) object:
\obindex{sequence}
\obindex{mapping}
\obindex{string}
\obindex{tuple}
\obindex{list}
\obindex{dictionary}
\indexii{sequence}{item}

\begin{productionlist}
  \production{subscription}
             {\token{primary} "[" \token{expression_list} "]"}
\end{productionlist}

The primary must evaluate to an object of a sequence or mapping type.

If the primary is a mapping, the expression list must evaluate to an
object whose value is one of the keys of the mapping, and the
subscription selects the value in the mapping that corresponds to that
key.  (The expression list is a tuple except if it has exactly one
item.)

If the primary is a sequence, the expression (list) must evaluate to a
plain integer.  If this value is negative, the length of the sequence
is added to it (so that, e.g., \code{x[-1]} selects the last item of
\code{x}.)  The resulting value must be a nonnegative integer less
than the number of items in the sequence, and the subscription selects
the item whose index is that value (counting from zero).

A string's items are characters.  A character is not a separate data
type but a string of exactly one character.
\index{character}
\indexii{string}{item}


\subsection{Slicings\label{slicings}}
\index{slicing}
\index{slice}

A slicing selects a range of items in a sequence object (e.g., a
string, tuple or list).  Slicings may be used as expressions or as
targets in assignment or \keyword{del} statements.  The syntax for a
slicing:
\obindex{sequence}
\obindex{string}
\obindex{tuple}
\obindex{list}

\begin{productionlist}
  \production{slicing}
             {\token{simple_slicing} | \token{extended_slicing}}
  \production{simple_slicing}
             {\token{primary} "[" \token{short_slice} "]"}
  \production{extended_slicing}
             {\token{primary} "[" \token{slice_list} "]" }
  \production{slice_list}
             {\token{slice_item} ("," \token{slice_item})* [","]}
  \production{slice_item}
             {\token{expression} | \token{proper_slice} | \token{ellipsis}}
  \production{proper_slice}
             {\token{short_slice} | \token{long_slice}}
  \production{short_slice}
             {[\token{lower_bound}] ":" [\token{upper_bound}]}
  \production{long_slice}
             {\token{short_slice} ":" [\token{stride}]}
  \production{lower_bound}
             {\token{expression}}
  \production{upper_bound}
             {\token{expression}}
  \production{stride}
             {\token{expression}}
  \production{ellipsis}
             {"..."}
\end{productionlist}

There is ambiguity in the formal syntax here: anything that looks like
an expression list also looks like a slice list, so any subscription
can be interpreted as a slicing.  Rather than further complicating the
syntax, this is disambiguated by defining that in this case the
interpretation as a subscription takes priority over the
interpretation as a slicing (this is the case if the slice list
contains no proper slice nor ellipses).  Similarly, when the slice
list has exactly one short slice and no trailing comma, the
interpretation as a simple slicing takes priority over that as an
extended slicing.\indexii{extended}{slicing}

The semantics for a simple slicing are as follows.  The primary must
evaluate to a sequence object.  The lower and upper bound expressions,
if present, must evaluate to plain integers; defaults are zero and the
\code{sys.maxint}, respectively.  If either bound is negative, the
sequence's length is added to it.  The slicing now selects all items
with index \var{k} such that
\code{\var{i} <= \var{k} < \var{j}} where \var{i}
and \var{j} are the specified lower and upper bounds.  This may be an
empty sequence.  It is not an error if \var{i} or \var{j} lie outside the
range of valid indexes (such items don't exist so they aren't
selected).

The semantics for an extended slicing are as follows.  The primary
must evaluate to a mapping object, and it is indexed with a key that
is constructed from the slice list, as follows.  If the slice list
contains at least one comma, the key is a tuple containing the
conversion of the slice items; otherwise, the conversion of the lone
slice item is the key.  The conversion of a slice item that is an
expression is that expression.  The conversion of an ellipsis slice
item is the built-in \code{Ellipsis} object.  The conversion of a
proper slice is a slice object (see section \ref{types}) whose
\member{start}, \member{stop} and \member{step} attributes are the
values of the expressions given as lower bound, upper bound and
stride, respectively, substituting \code{None} for missing
expressions.
\withsubitem{(slice object attribute)}{\ttindex{start}
  \ttindex{stop}\ttindex{step}}


\subsection{Calls\label{calls}}
\index{call}

A call calls a callable object (e.g., a function) with a possibly empty
series of arguments:
\obindex{callable}

\begin{productionlist}
  \production{call}
             {\token{primary} "(" [\token{argument_list} [","]}
  \productioncont{            | \token{expression} \token{genexpr_for}] ")"}
  \production{argument_list}
             {\token{positional_arguments} ["," \token{keyword_arguments}]}
  \productioncont{                     ["," "*" \token{expression}]}
  \productioncont{                     ["," "**" \token{expression}]}
  \productioncont{| \token{keyword_arguments} ["," "*" \token{expression}]}
  \productioncont{                    ["," "**" \token{expression}]}
  \productioncont{| "*" \token{expression} ["," "**" \token{expression}]}
  \productioncont{| "**" \token{expression}}
  \production{positional_arguments}
             {\token{expression} ("," \token{expression})*}
  \production{keyword_arguments}
             {\token{keyword_item} ("," \token{keyword_item})*}
  \production{keyword_item}
             {\token{identifier} "=" \token{expression}}
\end{productionlist}

A trailing comma may be present after the positional and keyword
arguments but does not affect the semantics.

The primary must evaluate to a callable object (user-defined
functions, built-in functions, methods of built-in objects, class
objects, methods of class instances, and certain class instances
themselves are callable; extensions may define additional callable
object types).  All argument expressions are evaluated before the call
is attempted.  Please refer to section \ref{function} for the syntax
of formal parameter lists.

If keyword arguments are present, they are first converted to
positional arguments, as follows.  First, a list of unfilled slots is
created for the formal parameters.  If there are N positional
arguments, they are placed in the first N slots.  Next, for each
keyword argument, the identifier is used to determine the
corresponding slot (if the identifier is the same as the first formal
parameter name, the first slot is used, and so on).  If the slot is
already filled, a \exception{TypeError} exception is raised.
Otherwise, the value of the argument is placed in the slot, filling it
(even if the expression is \code{None}, it fills the slot).  When all
arguments have been processed, the slots that are still unfilled are
filled with the corresponding default value from the function
definition.  (Default values are calculated, once, when the function
is defined; thus, a mutable object such as a list or dictionary used
as default value will be shared by all calls that don't specify an
argument value for the corresponding slot; this should usually be
avoided.)  If there are any unfilled slots for which no default value
is specified, a \exception{TypeError} exception is raised.  Otherwise,
the list of filled slots is used as the argument list for the call.

If there are more positional arguments than there are formal parameter
slots, a \exception{TypeError} exception is raised, unless a formal
parameter using the syntax \samp{*identifier} is present; in this
case, that formal parameter receives a tuple containing the excess
positional arguments (or an empty tuple if there were no excess
positional arguments).

If any keyword argument does not correspond to a formal parameter
name, a \exception{TypeError} exception is raised, unless a formal
parameter using the syntax \samp{**identifier} is present; in this
case, that formal parameter receives a dictionary containing the
excess keyword arguments (using the keywords as keys and the argument
values as corresponding values), or a (new) empty dictionary if there
were no excess keyword arguments.

If the syntax \samp{*expression} appears in the function call,
\samp{expression} must evaluate to a sequence.  Elements from this
sequence are treated as if they were additional positional arguments;
if there are postional arguments \var{x1},...,\var{xN} , and
\samp{expression} evaluates to a sequence \var{y1},...,\var{yM}, this
is equivalent to a call with M+N positional arguments
\var{x1},...,\var{xN},\var{y1},...,\var{yM}.

A consequence of this is that although the \samp{*expression} syntax
appears \emph{after} any keyword arguments, it is processed
\emph{before} the keyword arguments (and the
\samp{**expression} argument, if any -- see below).  So:

\begin{verbatim}
>>> def f(a, b):
...  print a, b
...
>>> f(b=1, *(2,))
2 1
>>> f(a=1, *(2,))
Traceback (most recent call last):
  File "<stdin>", line 1, in ?
TypeError: f() got multiple values for keyword argument 'a'
>>> f(1, *(2,))
1 2
\end{verbatim}

It is unusual for both keyword arguments and the
\samp{*expression} syntax to be used in the same call, so in practice
this confusion does not arise.

If the syntax \samp{**expression} appears in the function call,
\samp{expression} must evaluate to a (subclass of) dictionary, the
contents of which are treated as additional keyword arguments.  In the
case of a keyword appearing in both \samp{expression} and as an
explicit keyword argument, a \exception{TypeError} exception is
raised.

Formal parameters using the syntax \samp{*identifier} or
\samp{**identifier} cannot be used as positional argument slots or
as keyword argument names.  Formal parameters using the syntax
\samp{(sublist)} cannot be used as keyword argument names; the
outermost sublist corresponds to a single unnamed argument slot, and
the argument value is assigned to the sublist using the usual tuple
assignment rules after all other parameter processing is done.

A call always returns some value, possibly \code{None}, unless it
raises an exception.  How this value is computed depends on the type
of the callable object.

If it is---

\begin{description}

\item[a user-defined function:] The code block for the function is
executed, passing it the argument list.  The first thing the code
block will do is bind the formal parameters to the arguments; this is
described in section \ref{function}.  When the code block executes a
\keyword{return} statement, this specifies the return value of the
function call.
\indexii{function}{call}
\indexiii{user-defined}{function}{call}
\obindex{user-defined function}
\obindex{function}

\item[a built-in function or method:] The result is up to the
interpreter; see the \citetitle[../lib/built-in-funcs.html]{Python
Library Reference} for the descriptions of built-in functions and
methods.
\indexii{function}{call}
\indexii{built-in function}{call}
\indexii{method}{call}
\indexii{built-in method}{call}
\obindex{built-in method}
\obindex{built-in function}
\obindex{method}
\obindex{function}

\item[a class object:] A new instance of that class is returned.
\obindex{class}
\indexii{class object}{call}

\item[a class instance method:] The corresponding user-defined
function is called, with an argument list that is one longer than the
argument list of the call: the instance becomes the first argument.
\obindex{class instance}
\obindex{instance}
\indexii{class instance}{call}

\item[a class instance:] The class must define a \method{__call__()}
method; the effect is then the same as if that method was called.
\indexii{instance}{call}
\withsubitem{(object method)}{\ttindex{__call__()}}

\end{description}


\section{The power operator\label{power}}

The power operator binds more tightly than unary operators on its
left; it binds less tightly than unary operators on its right.  The
syntax is:

\begin{productionlist}
  \production{power}
             {\token{primary} ["**" \token{u_expr}]}
\end{productionlist}

Thus, in an unparenthesized sequence of power and unary operators, the
operators are evaluated from right to left (this does not constrain
the evaluation order for the operands).

The power operator has the same semantics as the built-in
\function{pow()} function, when called with two arguments: it yields
its left argument raised to the power of its right argument.  The
numeric arguments are first converted to a common type.  The result
type is that of the arguments after coercion.

With mixed operand types, the coercion rules for binary arithmetic
operators apply. For int and long int operands, the result has the
same type as the operands (after coercion) unless the second argument
is negative; in that case, all arguments are converted to float and a
float result is delivered. For example, \code{10**2} returns \code{100},
but \code{10**-2} returns \code{0.01}. (This last feature was added in
Python 2.2. In Python 2.1 and before, if both arguments were of integer
types and the second argument was negative, an exception was raised).

Raising \code{0.0} to a negative power results in a
\exception{ZeroDivisionError}.  Raising a negative number to a
fractional power results in a \exception{ValueError}.


\section{Unary arithmetic operations \label{unary}}
\indexiii{unary}{arithmetic}{operation}
\indexiii{unary}{bit-wise}{operation}

All unary arithmetic (and bit-wise) operations have the same priority:

\begin{productionlist}
  \production{u_expr}
             {\token{power} | "-" \token{u_expr}
              | "+" \token{u_expr} | "{\~}" \token{u_expr}}
\end{productionlist}

The unary \code{-} (minus) operator yields the negation of its
numeric argument.
\index{negation}
\index{minus}

The unary \code{+} (plus) operator yields its numeric argument
unchanged.
\index{plus}

The unary \code{\~} (invert) operator yields the bit-wise inversion
of its plain or long integer argument.  The bit-wise inversion of
\code{x} is defined as \code{-(x+1)}.  It only applies to integral
numbers.
\index{inversion}

In all three cases, if the argument does not have the proper type,
a \exception{TypeError} exception is raised.
\exindex{TypeError}


\section{Binary arithmetic operations\label{binary}}
\indexiii{binary}{arithmetic}{operation}

The binary arithmetic operations have the conventional priority
levels.  Note that some of these operations also apply to certain
non-numeric types.  Apart from the power operator, there are only two
levels, one for multiplicative operators and one for additive
operators:

\begin{productionlist}
  \production{m_expr}
             {\token{u_expr} | \token{m_expr} "*" \token{u_expr}
              | \token{m_expr} "//" \token{u_expr}
              | \token{m_expr} "/" \token{u_expr}}
  \productioncont{| \token{m_expr} "\%" \token{u_expr}}
  \production{a_expr}
             {\token{m_expr} | \token{a_expr} "+" \token{m_expr}
              | \token{a_expr} "-" \token{m_expr}}
\end{productionlist}

The \code{*} (multiplication) operator yields the product of its
arguments.  The arguments must either both be numbers, or one argument
must be an integer (plain or long) and the other must be a sequence.
In the former case, the numbers are converted to a common type and
then multiplied together.  In the latter case, sequence repetition is
performed; a negative repetition factor yields an empty sequence.
\index{multiplication}

The \code{/} (division) and \code{//} (floor division) operators yield
the quotient of their arguments.  The numeric arguments are first
converted to a common type.  Plain or long integer division yields an
integer of the same type; the result is that of mathematical division
with the `floor' function applied to the result.  Division by zero
raises the
\exception{ZeroDivisionError} exception.
\exindex{ZeroDivisionError}
\index{division}

The \code{\%} (modulo) operator yields the remainder from the
division of the first argument by the second.  The numeric arguments
are first converted to a common type.  A zero right argument raises
the \exception{ZeroDivisionError} exception.  The arguments may be floating
point numbers, e.g., \code{3.14\%0.7} equals \code{0.34} (since
\code{3.14} equals \code{4*0.7 + 0.34}.)  The modulo operator always
yields a result with the same sign as its second operand (or zero);
the absolute value of the result is strictly smaller than the absolute
value of the second operand\footnote{
    While \code{abs(x\%y) < abs(y)} is true mathematically, for
    floats it may not be true numerically due to roundoff.  For
    example, and assuming a platform on which a Python float is an
    IEEE 754 double-precision number, in order that \code{-1e-100 \% 1e100}
    have the same sign as \code{1e100}, the computed result is
    \code{-1e-100 + 1e100}, which is numerically exactly equal
    to \code{1e100}.  Function \function{fmod()} in the \module{math}
    module returns a result whose sign matches the sign of the
    first argument instead, and so returns \code{-1e-100} in this case.
    Which approach is more appropriate depends on the application.
}.
\index{modulo}

The integer division and modulo operators are connected by the
following identity: \code{x == (x/y)*y + (x\%y)}.  Integer division and
modulo are also connected with the built-in function \function{divmod()}:
\code{divmod(x, y) == (x/y, x\%y)}.  These identities don't hold for
floating point numbers; there similar identities hold
approximately where \code{x/y} is replaced by \code{floor(x/y)} or
\code{floor(x/y) - 1}\footnote{
    If x is very close to an exact integer multiple of y, it's
    possible for \code{floor(x/y)} to be one larger than
    \code{(x-x\%y)/y} due to rounding.  In such cases, Python returns
    the latter result, in order to preserve that \code{divmod(x,y)[0]
    * y + x \%{} y} be very close to \code{x}.
}.

In addition to performing the modulo operation on numbers, the \code{\%}
operator is also overloaded by string and unicode objects to perform
string formatting (also known as interpolation). The syntax for string
formatting is described in the
\citetitle[../lib/typesseq-strings.html]{Python Library Reference},
section ``Sequence Types''.

\deprecated{2.3}{The floor division operator, the modulo operator,
and the \function{divmod()} function are no longer defined for complex
numbers.  Instead, convert to a floating point number using the
\function{abs()} function if appropriate.}

The \code{+} (addition) operator yields the sum of its arguments.
The arguments must either both be numbers or both sequences of the
same type.  In the former case, the numbers are converted to a common
type and then added together.  In the latter case, the sequences are
concatenated.
\index{addition}

The \code{-} (subtraction) operator yields the difference of its
arguments.  The numeric arguments are first converted to a common
type.
\index{subtraction}


\section{Shifting operations\label{shifting}}
\indexii{shifting}{operation}

The shifting operations have lower priority than the arithmetic
operations:

\begin{productionlist}
  \production{shift_expr}
             {\token{a_expr}
              | \token{shift_expr} ( "<<" | ">>" ) \token{a_expr}}
\end{productionlist}

These operators accept plain or long integers as arguments.  The
arguments are converted to a common type.  They shift the first
argument to the left or right by the number of bits given by the
second argument.

A right shift by \var{n} bits is defined as division by
\code{pow(2,\var{n})}.  A left shift by \var{n} bits is defined as
multiplication with \code{pow(2,\var{n})}; for plain integers there is
no overflow check so in that case the operation drops bits and flips
the sign if the result is not less than \code{pow(2,31)} in absolute
value.  Negative shift counts raise a \exception{ValueError}
exception.
\exindex{ValueError}


\section{Binary bit-wise operations\label{bitwise}}
\indexiii{binary}{bit-wise}{operation}

Each of the three bitwise operations has a different priority level:

\begin{productionlist}
  \production{and_expr}
             {\token{shift_expr} | \token{and_expr} "\&" \token{shift_expr}}
  \production{xor_expr}
             {\token{and_expr} | \token{xor_expr} "\textasciicircum" \token{and_expr}}
  \production{or_expr}
             {\token{xor_expr} | \token{or_expr} "|" \token{xor_expr}}
\end{productionlist}

The \code{\&} operator yields the bitwise AND of its arguments, which
must be plain or long integers.  The arguments are converted to a
common type.
\indexii{bit-wise}{and}

The \code{\^} operator yields the bitwise XOR (exclusive OR) of its
arguments, which must be plain or long integers.  The arguments are
converted to a common type.
\indexii{bit-wise}{xor}
\indexii{exclusive}{or}

The \code{|} operator yields the bitwise (inclusive) OR of its
arguments, which must be plain or long integers.  The arguments are
converted to a common type.
\indexii{bit-wise}{or}
\indexii{inclusive}{or}


\section{Comparisons\label{comparisons}}
\index{comparison}

Unlike C, all comparison operations in Python have the same priority,
which is lower than that of any arithmetic, shifting or bitwise
operation.  Also unlike C, expressions like \code{a < b < c} have the
interpretation that is conventional in mathematics:
\indexii{C}{language}

\begin{productionlist}
  \production{comparison}
             {\token{or_expr} ( \token{comp_operator} \token{or_expr} )*}
  \production{comp_operator}
             {"<" | ">" | "==" | ">=" | "<=" | "<>" | "!="}
  \productioncont{| "is" ["not"] | ["not"] "in"}
\end{productionlist}

Comparisons yield boolean values: \code{True} or \code{False}.

Comparisons can be chained arbitrarily, e.g., \code{x < y <= z} is
equivalent to \code{x < y and y <= z}, except that \code{y} is
evaluated only once (but in both cases \code{z} is not evaluated at all
when \code{x < y} is found to be false).
\indexii{chaining}{comparisons}

Formally, if \var{a}, \var{b}, \var{c}, \ldots, \var{y}, \var{z} are
expressions and \var{opa}, \var{opb}, \ldots, \var{opy} are comparison
operators, then \var{a opa b opb c} \ldots \var{y opy z} is equivalent
to \var{a opa b} \keyword{and} \var{b opb c} \keyword{and} \ldots
\var{y opy z}, except that each expression is evaluated at most once.

Note that \var{a opa b opb c} doesn't imply any kind of comparison
between \var{a} and \var{c}, so that, e.g., \code{x < y > z} is
perfectly legal (though perhaps not pretty).

The forms \code{<>} and \code{!=} are equivalent; for consistency with
C, \code{!=} is preferred; where \code{!=} is mentioned below
\code{<>} is also accepted.  The \code{<>} spelling is considered
obsolescent.

The operators \code{<}, \code{>}, \code{==}, \code{>=}, \code{<=}, and
\code{!=} compare
the values of two objects.  The objects need not have the same type.
If both are numbers, they are converted to a common type.  Otherwise,
objects of different types \emph{always} compare unequal, and are
ordered consistently but arbitrarily.  You can control comparison
behavior of objects of non-builtin types by defining a \code{__cmp__}
method or rich comparison methods like \code{__gt__}, described in
section~\ref{specialnames}.

(This unusual definition of comparison was used to simplify the
definition of operations like sorting and the \keyword{in} and
\keyword{not in} operators.  In the future, the comparison rules for
objects of different types are likely to change.)

Comparison of objects of the same type depends on the type:

\begin{itemize}

\item
Numbers are compared arithmetically.

\item
Strings are compared lexicographically using the numeric equivalents
(the result of the built-in function \function{ord()}) of their
characters.  Unicode and 8-bit strings are fully interoperable in this
behavior.

\item
Tuples and lists are compared lexicographically using comparison of
corresponding elements.  This means that to compare equal, each
element must compare equal and the two sequences must be of the same
type and have the same length.

If not equal, the sequences are ordered the same as their first
differing elements.  For example, \code{cmp([1,2,x], [1,2,y])} returns
the same as \code{cmp(x,y)}.  If the corresponding element does not
exist, the shorter sequence is ordered first (for example,
\code{[1,2] < [1,2,3]}).

\item
Mappings (dictionaries) compare equal if and only if their sorted
(key, value) lists compare equal.\footnote{The implementation computes
   this efficiently, without constructing lists or sorting.}
Outcomes other than equality are resolved consistently, but are not
otherwise defined.\footnote{Earlier versions of Python used
  lexicographic comparison of the sorted (key, value) lists, but this
  was very expensive for the common case of comparing for equality.  An
  even earlier version of Python compared dictionaries by identity only,
  but this caused surprises because people expected to be able to test
  a dictionary for emptiness by comparing it to \code{\{\}}.}

\item
Most other objects of builtin types compare unequal unless they are
the same object;
the choice whether one object is considered smaller or larger than
another one is made arbitrarily but consistently within one
execution of a program.

\end{itemize}

The operators \keyword{in} and \keyword{not in} test for set
membership.  \code{\var{x} in \var{s}} evaluates to true if \var{x}
is a member of the set \var{s}, and false otherwise.  \code{\var{x}
not in \var{s}} returns the negation of \code{\var{x} in \var{s}}.
The set membership test has traditionally been bound to sequences; an
object is a member of a set if the set is a sequence and contains an
element equal to that object.  However, it is possible for an object
to support membership tests without being a sequence.  In particular,
dictionaries support membership testing as a nicer way of spelling
\code{\var{key} in \var{dict}}; other mapping types may follow suit.

For the list and tuple types, \code{\var{x} in \var{y}} is true if and
only if there exists an index \var{i} such that
\code{\var{x} == \var{y}[\var{i}]} is true.

For the Unicode and string types, \code{\var{x} in \var{y}} is true if
and only if \var{x} is a substring of \var{y}.  An equivalent test is
\code{y.find(x) != -1}.  Note, \var{x} and \var{y} need not be the
same type; consequently, \code{u'ab' in 'abc'} will return \code{True}.
Empty strings are always considered to be a substring of any other string,
so \code{"" in "abc"} will return \code{True}.
\versionchanged[Previously, \var{x} was required to be a string of
length \code{1}]{2.3}

For user-defined classes which define the \method{__contains__()} method,
\code{\var{x} in \var{y}} is true if and only if
\code{\var{y}.__contains__(\var{x})} is true.

For user-defined classes which do not define \method{__contains__()} and
do define \method{__getitem__()}, \code{\var{x} in \var{y}} is true if
and only if there is a non-negative integer index \var{i} such that
\code{\var{x} == \var{y}[\var{i}]}, and all lower integer indices
do not raise \exception{IndexError} exception. (If any other exception
is raised, it is as if \keyword{in} raised that exception).

The operator \keyword{not in} is defined to have the inverse true value
of \keyword{in}.
\opindex{in}
\opindex{not in}
\indexii{membership}{test}
\obindex{sequence}

The operators \keyword{is} and \keyword{is not} test for object identity:
\code{\var{x} is \var{y}} is true if and only if \var{x} and \var{y}
are the same object.  \code{\var{x} is not \var{y}} yields the inverse
truth value.
\opindex{is}
\opindex{is not}
\indexii{identity}{test}


\section{Boolean operations\label{Booleans}}
\indexii{Conditional}{expression}
\indexii{Boolean}{operation}

Boolean operations have the lowest priority of all Python operations:

\begin{productionlist}
  \production{expression}
             {\token{conditional_expression} | \token{lambda_form}}
  \production{old_expression}
             {\token{or_test} | \token{old_lambda_form}}
  \production{conditional_expression}
             {\token{or_test} ["if" \token{or_test} "else" \token{expression}]}
  \production{or_test}
             {\token{and_test} | \token{or_test} "or" \token{and_test}}
  \production{and_test}
             {\token{not_test} | \token{and_test} "and" \token{not_test}}
  \production{not_test}
             {\token{comparison} | "not" \token{not_test}}
\end{productionlist}

In the context of Boolean operations, and also when expressions are
used by control flow statements, the following values are interpreted
as false: \code{False}, \code{None}, numeric zero of all types, and empty
strings and containers (including strings, tuples, lists, dictionaries,
sets and frozensets).  All other values are interpreted as true.

The operator \keyword{not} yields \code{True} if its argument is false,
\code{False} otherwise.
\opindex{not}

The expression \code{\var{x} if \var{C} else \var{y}} first evaluates
\var{C} (\emph{not} \var{x}); if \var{C} is true, \var{x} is evaluated and
its value is returned; otherwise, \var{y} is evaluated and its value is
returned.  \versionadded{2.5}

The expression \code{\var{x} and \var{y}} first evaluates \var{x}; if
\var{x} is false, its value is returned; otherwise, \var{y} is
evaluated and the resulting value is returned.
\opindex{and}

The expression \code{\var{x} or \var{y}} first evaluates \var{x}; if
\var{x} is true, its value is returned; otherwise, \var{y} is
evaluated and the resulting value is returned.
\opindex{or}

(Note that neither \keyword{and} nor \keyword{or} restrict the value
and type they return to \code{False} and \code{True}, but rather return the
last evaluated argument.
This is sometimes useful, e.g., if \code{s} is a string that should be
replaced by a default value if it is empty, the expression
\code{s or 'foo'} yields the desired value.  Because \keyword{not} has to
invent a value anyway, it does not bother to return a value of the
same type as its argument, so e.g., \code{not 'foo'} yields \code{False},
not \code{''}.)

\section{Lambdas\label{lambdas}}
\indexii{lambda}{expression}
\indexii{lambda}{form}
\indexii{anonymous}{function}

\begin{productionlist}
  \production{lambda_form}
             {"lambda" [\token{parameter_list}]: \token{expression}}
  \production{old_lambda_form}
             {"lambda" [\token{parameter_list}]: \token{old_expression}}
\end{productionlist}

Lambda forms (lambda expressions) have the same syntactic position as
expressions.  They are a shorthand to create anonymous functions; the
expression \code{lambda \var{arguments}: \var{expression}}
yields a function object.  The unnamed object behaves like a function
object defined with

\begin{verbatim}
def name(arguments):
    return expression
\end{verbatim}

See section \ref{function} for the syntax of parameter lists.  Note
that functions created with lambda forms cannot contain statements.
\label{lambda}

\section{Expression lists\label{exprlists}}
\indexii{expression}{list}

\begin{productionlist}
  \production{expression_list}
             {\token{expression} ( "," \token{expression} )* [","]}
\end{productionlist}

An expression list containing at least one comma yields a
tuple.  The length of the tuple is the number of expressions in the
list.  The expressions are evaluated from left to right.
\obindex{tuple}

The trailing comma is required only to create a single tuple (a.k.a. a
\emph{singleton}); it is optional in all other cases.  A single
expression without a trailing comma doesn't create a
tuple, but rather yields the value of that expression.
(To create an empty tuple, use an empty pair of parentheses:
\code{()}.)
\indexii{trailing}{comma}

\section{Evaluation order\label{evalorder}}
\indexii{evaluation}{order}

Python evaluates expressions from left to right. Notice that while
evaluating an assignment, the right-hand side is evaluated before
the left-hand side.

In the following lines, expressions will be evaluated in the
arithmetic order of their suffixes:

\begin{verbatim}
expr1, expr2, expr3, expr4
(expr1, expr2, expr3, expr4)
{expr1: expr2, expr3: expr4}
expr1 + expr2 * (expr3 - expr4)
func(expr1, expr2, *expr3, **expr4)
expr3, expr4 = expr1, expr2
\end{verbatim}

\section{Summary\label{summary}}

The following table summarizes the operator
precedences\indexii{operator}{precedence} in Python, from lowest
precedence (least binding) to highest precedence (most binding).
Operators in the same box have the same precedence.  Unless the syntax
is explicitly given, operators are binary.  Operators in the same box
group left to right (except for comparisons, including tests, which all
have the same precedence and chain from left to right --- see section
\ref{comparisons} -- and exponentiation, which groups from right to left).

\begin{tableii}{c|l}{textrm}{Operator}{Description}
    \lineii{\keyword{lambda}}			{Lambda expression}
  \hline
    \lineii{\keyword{or}}			{Boolean OR}
  \hline
    \lineii{\keyword{and}}			{Boolean AND}
  \hline
    \lineii{\keyword{not} \var{x}}		{Boolean NOT}
  \hline
    \lineii{\keyword{in}, \keyword{not} \keyword{in}}{Membership tests}
    \lineii{\keyword{is}, \keyword{is not}}{Identity tests}
    \lineii{\code{<}, \code{<=}, \code{>}, \code{>=},
            \code{<>}, \code{!=}, \code{==}}
	   {Comparisons}
  \hline
    \lineii{\code{|}}				{Bitwise OR}
  \hline
    \lineii{\code{\^}}				{Bitwise XOR}
  \hline
    \lineii{\code{\&}}				{Bitwise AND}
  \hline
    \lineii{\code{<<}, \code{>>}}		{Shifts}
  \hline
    \lineii{\code{+}, \code{-}}{Addition and subtraction}
  \hline
    \lineii{\code{*}, \code{/}, \code{\%}}
           {Multiplication, division, remainder}
  \hline
    \lineii{\code{+\var{x}}, \code{-\var{x}}}	{Positive, negative}
    \lineii{\code{\~\var{x}}}			{Bitwise not}
  \hline
    \lineii{\code{**}}				{Exponentiation}
  \hline
    \lineii{\code{\var{x}.\var{attribute}}}	{Attribute reference}
    \lineii{\code{\var{x}[\var{index}]}}	{Subscription}
    \lineii{\code{\var{x}[\var{index}:\var{index}]}}	{Slicing}
    \lineii{\code{\var{f}(\var{arguments}...)}}	{Function call}
  \hline
    \lineii{\code{(\var{expressions}\ldots)}}	{Binding or tuple display}
    \lineii{\code{[\var{expressions}\ldots]}}	{List display}
    \lineii{\code{\{\var{key}:\var{datum}\ldots\}}}{Dictionary display}
    \lineii{\code{`\var{expressions}\ldots`}}	{String conversion}
\end{tableii}
		% Expressions and conditions
\chapter{Simple statements \label{simple}}
\indexii{simple}{statement}

Simple statements are comprised within a single logical line.
Several simple statements may occur on a single line separated
by semicolons.  The syntax for simple statements is:

\begin{productionlist}
  \production{simple_stmt}{\token{expression_stmt}}
  \productioncont{| \token{assert_stmt}}
  \productioncont{| \token{assignment_stmt}}
  \productioncont{| \token{augmented_assignment_stmt}}
  \productioncont{| \token{pass_stmt}}
  \productioncont{| \token{del_stmt}}
  \productioncont{| \token{print_stmt}}
  \productioncont{| \token{return_stmt}}
  \productioncont{| \token{yield_stmt}}
  \productioncont{| \token{raise_stmt}}
  \productioncont{| \token{break_stmt}}
  \productioncont{| \token{continue_stmt}}
  \productioncont{| \token{import_stmt}}
  \productioncont{| \token{global_stmt}}
  \productioncont{| \token{exec_stmt}}
\end{productionlist}


\section{Expression statements \label{exprstmts}}
\indexii{expression}{statement}

Expression statements are used (mostly interactively) to compute and
write a value, or (usually) to call a procedure (a function that
returns no meaningful result; in Python, procedures return the value
\code{None}).  Other uses of expression statements are allowed and
occasionally useful.  The syntax for an expression statement is:

\begin{productionlist}
  \production{expression_stmt}
             {\token{expression_list}}
\end{productionlist}

An expression statement evaluates the expression list (which may be a
single expression).
\indexii{expression}{list}

In interactive mode, if the value is not \code{None}, it is converted
to a string using the built-in \function{repr()}\bifuncindex{repr}
function and the resulting string is written to standard output (see
section~\ref{print}) on a line by itself.  (Expression statements
yielding \code{None} are not written, so that procedure calls do not
cause any output.)
\obindex{None}
\indexii{string}{conversion}
\index{output}
\indexii{standard}{output}
\indexii{writing}{values}
\indexii{procedure}{call}


\section{Assert statements \label{assert}}

Assert statements\stindex{assert} are a convenient way to insert
debugging assertions\indexii{debugging}{assertions} into a program:

\begin{productionlist}
  \production{assert_stmt}
             {"assert" \token{expression} ["," \token{expression}]}
\end{productionlist}

The simple form, \samp{assert expression}, is equivalent to

\begin{verbatim}
if __debug__:
   if not expression: raise AssertionError
\end{verbatim}

The extended form, \samp{assert expression1, expression2}, is
equivalent to

\begin{verbatim}
if __debug__:
   if not expression1: raise AssertionError, expression2
\end{verbatim}

These equivalences assume that \code{__debug__}\ttindex{__debug__} and
\exception{AssertionError}\exindex{AssertionError} refer to the built-in
variables with those names.  In the current implementation, the
built-in variable \code{__debug__} is \code{True} under normal
circumstances, \code{False} when optimization is requested (command line
option -O).  The current code generator emits no code for an assert
statement when optimization is requested at compile time.  Note that it
is unnecessary to include the source code for the expression that failed
in the error message;
it will be displayed as part of the stack trace.

Assignments to \code{__debug__} are illegal.  The value for the
built-in variable is determined when the interpreter starts.


\section{Assignment statements \label{assignment}}

Assignment statements\indexii{assignment}{statement} are used to
(re)bind names to values and to modify attributes or items of mutable
objects:
\indexii{binding}{name}
\indexii{rebinding}{name}
\obindex{mutable}
\indexii{attribute}{assignment}

\begin{productionlist}
  \production{assignment_stmt}
             {(\token{target_list} "=")+
              (\token{expression_list} | \token{yield_expression})}
  \production{target_list}
             {\token{target} ("," \token{target})* [","]}
  \production{target}
             {\token{identifier}}
  \productioncont{| "(" \token{target_list} ")"}
  \productioncont{| "[" \token{target_list} "]"}
  \productioncont{| \token{attributeref}}
  \productioncont{| \token{subscription}}
  \productioncont{| \token{slicing}}
\end{productionlist}

(See section~\ref{primaries} for the syntax definitions for the last
three symbols.)

An assignment statement evaluates the expression list (remember that
this can be a single expression or a comma-separated list, the latter
yielding a tuple) and assigns the single resulting object to each of
the target lists, from left to right.
\indexii{expression}{list}

Assignment is defined recursively depending on the form of the target
(list).  When a target is part of a mutable object (an attribute
reference, subscription or slicing), the mutable object must
ultimately perform the assignment and decide about its validity, and
may raise an exception if the assignment is unacceptable.  The rules
observed by various types and the exceptions raised are given with the
definition of the object types (see section~\ref{types}).
\index{target}
\indexii{target}{list}

Assignment of an object to a target list is recursively defined as
follows.
\indexiii{target}{list}{assignment}

\begin{itemize}
\item
If the target list is a single target: The object is assigned to that
target.

\item
If the target list is a comma-separated list of targets: The object
must be a sequence with the same number of items as there are
targets in the target list, and the items are assigned, from left to
right, to the corresponding targets.  (This rule is relaxed as of
Python 1.5; in earlier versions, the object had to be a tuple.  Since
strings are sequences, an assignment like \samp{a, b = "xy"} is
now legal as long as the string has the right length.)

\end{itemize}

Assignment of an object to a single target is recursively defined as
follows.

\begin{itemize} % nested

\item
If the target is an identifier (name):

\begin{itemize}

\item
If the name does not occur in a \keyword{global} statement in the current
code block: the name is bound to the object in the current local
namespace.
\stindex{global}

\item
Otherwise: the name is bound to the object in the current global
namespace.

\end{itemize} % nested

The name is rebound if it was already bound.  This may cause the
reference count for the object previously bound to the name to reach
zero, causing the object to be deallocated and its
destructor\index{destructor} (if it has one) to be called.

\item
If the target is a target list enclosed in parentheses or in square
brackets: The object must be a sequence with the same number of items
as there are targets in the target list, and its items are assigned,
from left to right, to the corresponding targets.

\item
If the target is an attribute reference: The primary expression in the
reference is evaluated.  It should yield an object with assignable
attributes; if this is not the case, \exception{TypeError} is raised.  That
object is then asked to assign the assigned object to the given
attribute; if it cannot perform the assignment, it raises an exception
(usually but not necessarily \exception{AttributeError}).
\indexii{attribute}{assignment}

\item
If the target is a subscription: The primary expression in the
reference is evaluated.  It should yield either a mutable sequence
object (such as a list) or a mapping object (such as a dictionary). Next,
the subscript expression is evaluated.
\indexii{subscription}{assignment}
\obindex{mutable}

If the primary is a mutable sequence object (such as a list), the subscript
must yield a plain integer.  If it is negative, the sequence's length
is added to it.  The resulting value must be a nonnegative integer
less than the sequence's length, and the sequence is asked to assign
the assigned object to its item with that index.  If the index is out
of range, \exception{IndexError} is raised (assignment to a subscripted
sequence cannot add new items to a list).
\obindex{sequence}
\obindex{list}

If the primary is a mapping object (such as a dictionary), the subscript must
have a type compatible with the mapping's key type, and the mapping is
then asked to create a key/datum pair which maps the subscript to
the assigned object.  This can either replace an existing key/value
pair with the same key value, or insert a new key/value pair (if no
key with the same value existed).
\obindex{mapping}
\obindex{dictionary}

\item
If the target is a slicing: The primary expression in the reference is
evaluated.  It should yield a mutable sequence object (such as a list).  The
assigned object should be a sequence object of the same type.  Next,
the lower and upper bound expressions are evaluated, insofar they are
present; defaults are zero and the sequence's length.  The bounds
should evaluate to (small) integers.  If either bound is negative, the
sequence's length is added to it.  The resulting bounds are clipped to
lie between zero and the sequence's length, inclusive.  Finally, the
sequence object is asked to replace the slice with the items of the
assigned sequence.  The length of the slice may be different from the
length of the assigned sequence, thus changing the length of the
target sequence, if the object allows it.
\indexii{slicing}{assignment}

\end{itemize}
        
(In the current implementation, the syntax for targets is taken
to be the same as for expressions, and invalid syntax is rejected
during the code generation phase, causing less detailed error
messages.)

WARNING: Although the definition of assignment implies that overlaps
between the left-hand side and the right-hand side are `safe' (for example
\samp{a, b = b, a} swaps two variables), overlaps \emph{within} the
collection of assigned-to variables are not safe!  For instance, the
following program prints \samp{[0, 2]}:

\begin{verbatim}
x = [0, 1]
i = 0
i, x[i] = 1, 2
print x
\end{verbatim}


\subsection{Augmented assignment statements \label{augassign}}

Augmented assignment is the combination, in a single statement, of a binary
operation and an assignment statement:
\indexii{augmented}{assignment}
\index{statement!assignment, augmented}

\begin{productionlist}
  \production{augmented_assignment_stmt}
             {\token{target} \token{augop}
              (\token{expression_list} | \token{yield_expression})}
  \production{augop}
             {"+=" | "-=" | "*=" | "/=" | "\%=" | "**="}
  \productioncont{| ">>=" | "<<=" | "\&=" | "\textasciicircum=" | "|="}
\end{productionlist}

(See section~\ref{primaries} for the syntax definitions for the last
three symbols.)

An augmented assignment evaluates the target (which, unlike normal
assignment statements, cannot be an unpacking) and the expression
list, performs the binary operation specific to the type of assignment
on the two operands, and assigns the result to the original
target.  The target is only evaluated once.

An augmented assignment expression like \code{x += 1} can be rewritten as
\code{x = x + 1} to achieve a similar, but not exactly equal effect. In the
augmented version, \code{x} is only evaluated once. Also, when possible, the
actual operation is performed \emph{in-place}, meaning that rather than
creating a new object and assigning that to the target, the old object is
modified instead.

With the exception of assigning to tuples and multiple targets in a single
statement, the assignment done by augmented assignment statements is handled
the same way as normal assignments. Similarly, with the exception of the
possible \emph{in-place} behavior, the binary operation performed by
augmented assignment is the same as the normal binary operations.

For targets which are attribute references, the initial value is
retrieved with a \method{getattr()} and the result is assigned with a
\method{setattr()}.  Notice that the two methods do not necessarily
refer to the same variable.  When \method{getattr()} refers to a class
variable, \method{setattr()} still writes to an instance variable.
For example:

\begin{verbatim}
class A:
    x = 3    # class variable
a = A()
a.x += 1     # writes a.x as 4 leaving A.x as 3
\end{verbatim}


\section{The \keyword{pass} statement \label{pass}}
\stindex{pass}

\begin{productionlist}
  \production{pass_stmt}
             {"pass"}
\end{productionlist}

\keyword{pass} is a null operation --- when it is executed, nothing
happens.  It is useful as a placeholder when a statement is
required syntactically, but no code needs to be executed, for example:
\indexii{null}{operation}

\begin{verbatim}
def f(arg): pass    # a function that does nothing (yet)

class C: pass       # a class with no methods (yet)
\end{verbatim}


\section{The \keyword{del} statement \label{del}}
\stindex{del}

\begin{productionlist}
  \production{del_stmt}
             {"del" \token{target_list}}
\end{productionlist}

Deletion is recursively defined very similar to the way assignment is
defined. Rather that spelling it out in full details, here are some
hints.
\indexii{deletion}{target}
\indexiii{deletion}{target}{list}

Deletion of a target list recursively deletes each target, from left
to right.

Deletion of a name removes the binding of that name 
from the local or global namespace, depending on whether the name
occurs in a \keyword{global} statement in the same code block.  If the
name is unbound, a \exception{NameError} exception will be raised.
\stindex{global}
\indexii{unbinding}{name}

It is illegal to delete a name from the local namespace if it occurs
as a free variable\indexii{free}{variable} in a nested block.

Deletion of attribute references, subscriptions and slicings
is passed to the primary object involved; deletion of a slicing
is in general equivalent to assignment of an empty slice of the
right type (but even this is determined by the sliced object).
\indexii{attribute}{deletion}


\section{The \keyword{print} statement \label{print}}
\stindex{print}

\begin{productionlist}
  \production{print_stmt}
             {"print" ([\token{expression} ("," \token{expression})* [","]}
  \productioncont{| ">>" \token{expression}
                  [("," \token{expression})+ [","])}
\end{productionlist}

\keyword{print} evaluates each expression in turn and writes the
resulting object to standard output (see below).  If an object is not
a string, it is first converted to a string using the rules for string
conversions.  The (resulting or original) string is then written.  A
space is written before each object is (converted and) written, unless
the output system believes it is positioned at the beginning of a
line.  This is the case (1) when no characters have yet been written
to standard output, (2) when the last character written to standard
output is \character{\e n}, or (3) when the last write operation on
standard output was not a \keyword{print} statement.  (In some cases
it may be functional to write an empty string to standard output for
this reason.)  \note{Objects which act like file objects but which are
not the built-in file objects often do not properly emulate this
aspect of the file object's behavior, so it is best not to rely on
this.}
\index{output}
\indexii{writing}{values}

A \character{\e n} character is written at the end, unless the
\keyword{print} statement ends with a comma.  This is the only action
if the statement contains just the keyword \keyword{print}.
\indexii{trailing}{comma}
\indexii{newline}{suppression}

Standard output is defined as the file object named \code{stdout}
in the built-in module \module{sys}.  If no such object exists, or if
it does not have a \method{write()} method, a \exception{RuntimeError}
exception is raised.
\indexii{standard}{output}
\refbimodindex{sys}
\withsubitem{(in module sys)}{\ttindex{stdout}}
\exindex{RuntimeError}

\keyword{print} also has an extended\index{extended print statement}
form, defined by the second portion of the syntax described above.
This form is sometimes referred to as ``\keyword{print} chevron.''
In this form, the first expression after the \code{>>} must
evaluate to a ``file-like'' object, specifically an object that has a
\method{write()} method as described above.  With this extended form,
the subsequent expressions are printed to this file object.  If the
first expression evaluates to \code{None}, then \code{sys.stdout} is
used as the file for output.


\section{The \keyword{return} statement \label{return}}
\stindex{return}

\begin{productionlist}
  \production{return_stmt}
             {"return" [\token{expression_list}]}
\end{productionlist}

\keyword{return} may only occur syntactically nested in a function
definition, not within a nested class definition.
\indexii{function}{definition}
\indexii{class}{definition}

If an expression list is present, it is evaluated, else \code{None}
is substituted.

\keyword{return} leaves the current function call with the expression
list (or \code{None}) as return value.

When \keyword{return} passes control out of a \keyword{try} statement
with a \keyword{finally} clause, that \keyword{finally} clause is executed
before really leaving the function.
\kwindex{finally}

In a generator function, the \keyword{return} statement is not allowed
to include an \grammartoken{expression_list}.  In that context, a bare
\keyword{return} indicates that the generator is done and will cause
\exception{StopIteration} to be raised.


\section{The \keyword{yield} statement \label{yield}}
\stindex{yield}

\begin{productionlist}
  \production{yield_stmt}
             {\token{yield_expression}}
\end{productionlist}

\index{generator!function}
\index{generator!iterator}
\index{function!generator}
\exindex{StopIteration}

The \keyword{yield} statement is only used when defining a generator
function, and is only used in the body of the generator function.
Using a \keyword{yield} statement in a function definition is
sufficient to cause that definition to create a generator function
instead of a normal function.

When a generator function is called, it returns an iterator known as a
generator iterator, or more commonly, a generator.  The body of the
generator function is executed by calling the generator's
\method{next()} method repeatedly until it raises an exception.

When a \keyword{yield} statement is executed, the state of the
generator is frozen and the value of \grammartoken{expression_list} is
returned to \method{next()}'s caller.  By ``frozen'' we mean that all
local state is retained, including the current bindings of local
variables, the instruction pointer, and the internal evaluation stack:
enough information is saved so that the next time \method{next()} is
invoked, the function can proceed exactly as if the \keyword{yield}
statement were just another external call.

As of Python version 2.5, the \keyword{yield} statement is now
allowed in the \keyword{try} clause of a \keyword{try} ...\ 
\keyword{finally} construct.  If the generator is not resumed before
it is finalized (by reaching a zero reference count or by being garbage
collected), the generator-iterator's \method{close()} method will be
called, allowing any pending \keyword{finally} clauses to execute.

\begin{notice}
In Python 2.2, the \keyword{yield} statement is only allowed
when the \code{generators} feature has been enabled.  It will always
be enabled in Python 2.3.  This \code{__future__} import statement can
be used to enable the feature:

\begin{verbatim}
from __future__ import generators
\end{verbatim}
\end{notice}


\begin{seealso}
  \seepep{0255}{Simple Generators}
         {The proposal for adding generators and the \keyword{yield}
          statement to Python.}

  \seepep{0342}{Coroutines via Enhanced Generators}
         {The proposal that, among other generator enhancements,
          proposed allowing \keyword{yield} to appear inside a
          \keyword{try} ... \keyword{finally} block.}
\end{seealso}


\section{The \keyword{raise} statement \label{raise}}
\stindex{raise}

\begin{productionlist}
  \production{raise_stmt}
             {"raise" [\token{expression} ["," \token{expression}
              ["," \token{expression}]]]}
\end{productionlist}

If no expressions are present, \keyword{raise} re-raises the last
exception that was active in the current scope.  If no exception is
active in the current scope, a \exception{TypeError} exception is
raised indicating that this is an error (if running under IDLE, a
\exception{Queue.Empty} exception is raised instead).
\index{exception}
\indexii{raising}{exception}

Otherwise, \keyword{raise} evaluates the expressions to get three
objects, using \code{None} as the value of omitted expressions.  The
first two objects are used to determine the \emph{type} and
\emph{value} of the exception.

If the first object is an instance, the type of the exception is the
class of the instance, the instance itself is the value, and the
second object must be \code{None}.

If the first object is a class, it becomes the type of the exception.
The second object is used to determine the exception value: If it is
an instance of the class, the instance becomes the exception value.
If the second object is a tuple, it is used as the argument list for
the class constructor; if it is \code{None}, an empty argument list is
used, and any other object is treated as a single argument to the
constructor.  The instance so created by calling the constructor is
used as the exception value.

If a third object is present and not \code{None}, it must be a
traceback\obindex{traceback} object (see section~\ref{traceback}), and
it is substituted instead of the current location as the place where
the exception occurred.  If the third object is present and not a
traceback object or \code{None}, a \exception{TypeError} exception is
raised.  The three-expression form of \keyword{raise} is useful to
re-raise an exception transparently in an except clause, but
\keyword{raise} with no expressions should be preferred if the
exception to be re-raised was the most recently active exception in
the current scope.

Additional information on exceptions can be found in
section~\ref{exceptions}, and information about handling exceptions is
in section~\ref{try}.


\section{The \keyword{break} statement \label{break}}
\stindex{break}

\begin{productionlist}
  \production{break_stmt}
             {"break"}
\end{productionlist}

\keyword{break} may only occur syntactically nested in a \keyword{for}
or \keyword{while} loop, but not nested in a function or class definition
within that loop.
\stindex{for}
\stindex{while}
\indexii{loop}{statement}

It terminates the nearest enclosing loop, skipping the optional
\keyword{else} clause if the loop has one.
\kwindex{else}

If a \keyword{for} loop is terminated by \keyword{break}, the loop control
target keeps its current value.
\indexii{loop control}{target}

When \keyword{break} passes control out of a \keyword{try} statement
with a \keyword{finally} clause, that \keyword{finally} clause is executed
before really leaving the loop.
\kwindex{finally}


\section{The \keyword{continue} statement \label{continue}}
\stindex{continue}

\begin{productionlist}
  \production{continue_stmt}
             {"continue"}
\end{productionlist}

\keyword{continue} may only occur syntactically nested in a \keyword{for} or
\keyword{while} loop, but not nested in a function or class definition or
\keyword{finally} statement within that loop.\footnote{It may
occur within an \keyword{except} or \keyword{else} clause.  The
restriction on occurring in the \keyword{try} clause is implementor's
laziness and will eventually be lifted.}
It continues with the next cycle of the nearest enclosing loop.
\stindex{for}
\stindex{while}
\indexii{loop}{statement}
\kwindex{finally}


\section{The \keyword{import} statement \label{import}}
\stindex{import}
\index{module!importing}
\indexii{name}{binding}
\kwindex{from}

\begin{productionlist}
  \production{import_stmt}
             {"import" \token{module} ["as" \token{name}]
                ( "," \token{module} ["as" \token{name}] )*}
  \productioncont{| "from" \token{relative_module} "import" \token{identifier}
                    ["as" \token{name}]}
  \productioncont{  ( "," \token{identifier} ["as" \token{name}] )*}
  \productioncont{| "from" \token{relative_module} "import" "("
                    \token{identifier} ["as" \token{name}]}
  \productioncont{  ( "," \token{identifier} ["as" \token{name}] )* [","] ")"}
  \productioncont{| "from" \token{module} "import" "*"}
  \production{module}
             {(\token{identifier} ".")* \token{identifier}}
  \production{relative_module}
             {"."* \token{module} | "."+}
  \production{name}
             {\token{identifier}}
\end{productionlist}

Import statements are executed in two steps: (1) find a module, and
initialize it if necessary; (2) define a name or names in the local
namespace (of the scope where the \keyword{import} statement occurs).
The first form (without \keyword{from}) repeats these steps for each
identifier in the list.  The form with \keyword{from} performs step
(1) once, and then performs step (2) repeatedly.

In this context, to ``initialize'' a built-in or extension module means to
call an initialization function that the module must provide for the purpose
(in the reference implementation, the function's name is obtained by
prepending string ``init'' to the module's name); to ``initialize'' a
Python-coded module means to execute the module's body.
  
The system maintains a table of modules that have been or are being
initialized,
indexed by module name.  This table is
accessible as \code{sys.modules}.  When a module name is found in
this table, step (1) is finished.  If not, a search for a module
definition is started.  When a module is found, it is loaded.  Details
of the module searching and loading process are implementation and
platform specific.  It generally involves searching for a ``built-in''
module with the given name and then searching a list of locations
given as \code{sys.path}.
\withsubitem{(in module sys)}{\ttindex{modules}}
\ttindex{sys.modules}
\indexii{module}{name}
\indexii{built-in}{module}
\indexii{user-defined}{module}
\refbimodindex{sys}
\indexii{filename}{extension}
\indexiii{module}{search}{path}

If a built-in module is found,\indexii{module}{initialization} its
built-in initialization code is executed and step (1) is finished.  If
no matching file is found,
\exception{ImportError}\exindex{ImportError} is raised.
\index{code block}If a file is found, it is parsed,
yielding an executable code block.  If a syntax error occurs,
\exception{SyntaxError}\exindex{SyntaxError} is raised.  Otherwise, an
empty module of the given name is created and inserted in the module
table, and then the code block is executed in the context of this
module.  Exceptions during this execution terminate step (1).

When step (1) finishes without raising an exception, step (2) can
begin.

The first form of \keyword{import} statement binds the module name in
the local namespace to the module object, and then goes on to import
the next identifier, if any.  If the module name is followed by
\keyword{as}, the name following \keyword{as} is used as the local
name for the module. 

The \keyword{from} form does not bind the module name: it goes through the
list of identifiers, looks each one of them up in the module found in step
(1), and binds the name in the local namespace to the object thus found. 
As with the first form of \keyword{import}, an alternate local name can be
supplied by specifying "\keyword{as} localname".  If a name is not found,
\exception{ImportError} is raised.  If the list of identifiers is replaced
by a star (\character{*}), all public names defined in the module are
bound in the local namespace of the \keyword{import} statement..
\indexii{name}{binding}
\exindex{ImportError}

The \emph{public names} defined by a module are determined by checking
the module's namespace for a variable named \code{__all__}; if
defined, it must be a sequence of strings which are names defined or
imported by that module.  The names given in \code{__all__} are all
considered public and are required to exist.  If \code{__all__} is not
defined, the set of public names includes all names found in the
module's namespace which do not begin with an underscore character
(\character{_}).  \code{__all__} should contain the entire public API.
It is intended to avoid accidentally exporting items that are not part
of the API (such as library modules which were imported and used within
the module).
\withsubitem{(optional module attribute)}{\ttindex{__all__}}

The \keyword{from} form with \samp{*} may only occur in a module
scope.  If the wild card form of import --- \samp{import *} --- is
used in a function and the function contains or is a nested block with
free variables, the compiler will raise a \exception{SyntaxError}.

\kwindex{from}
\stindex{from}

\strong{Hierarchical module names:}\indexiii{hierarchical}{module}{names}
when the module names contains one or more dots, the module search
path is carried out differently.  The sequence of identifiers up to
the last dot is used to find a ``package''\index{packages}; the final
identifier is then searched inside the package.  A package is
generally a subdirectory of a directory on \code{sys.path} that has a
file \file{__init__.py}.\ttindex{__init__.py}
%
[XXX Can't be bothered to spell this out right now; see the URL
\url{http://www.python.org/doc/essays/packages.html} for more details, also
about how the module search works from inside a package.]

The built-in function \function{__import__()} is provided to support
applications that determine which modules need to be loaded
dynamically; refer to \ulink{Built-in
Functions}{../lib/built-in-funcs.html} in the
\citetitle[../lib/lib.html]{Python Library Reference} for additional
information.
\bifuncindex{__import__}

\subsection{Future statements \label{future}}

A \dfn{future statement}\indexii{future}{statement} is a directive to
the compiler that a particular module should be compiled using syntax
or semantics that will be available in a specified future release of
Python.  The future statement is intended to ease migration to future
versions of Python that introduce incompatible changes to the
language.  It allows use of the new features on a per-module basis
before the release in which the feature becomes standard.

\begin{productionlist}[*]
  \production{future_statement}
             {"from" "__future__" "import" feature ["as" name]}
  \productioncont{  ("," feature ["as" name])*}
  \productioncont{| "from" "__future__" "import" "(" feature ["as" name]}
  \productioncont{  ("," feature ["as" name])* [","] ")"}
  \production{feature}{identifier}
  \production{name}{identifier}
\end{productionlist}

A future statement must appear near the top of the module.  The only
lines that can appear before a future statement are:

\begin{itemize}

\item the module docstring (if any),
\item comments,
\item blank lines, and
\item other future statements.

\end{itemize}

The features recognized by Python 2.5 are \samp{absolute_import},
\samp{division}, \samp{generators}, \samp{nested_scopes} and
\samp{with_statement}.  \samp{generators} and \samp{nested_scopes} 
are redundant in Python version 2.3 and above because they are always
enabled. 

A future statement is recognized and treated specially at compile
time: Changes to the semantics of core constructs are often
implemented by generating different code.  It may even be the case
that a new feature introduces new incompatible syntax (such as a new
reserved word), in which case the compiler may need to parse the
module differently.  Such decisions cannot be pushed off until
runtime.

For any given release, the compiler knows which feature names have been
defined, and raises a compile-time error if a future statement contains
a feature not known to it.

The direct runtime semantics are the same as for any import statement:
there is a standard module \module{__future__}, described later, and
it will be imported in the usual way at the time the future statement
is executed.

The interesting runtime semantics depend on the specific feature
enabled by the future statement.

Note that there is nothing special about the statement:

\begin{verbatim}
import __future__ [as name]
\end{verbatim}

That is not a future statement; it's an ordinary import statement with
no special semantics or syntax restrictions.

Code compiled by an \keyword{exec} statement or calls to the builtin functions
\function{compile()} and \function{execfile()} that occur in a module
\module{M} containing a future statement will, by default, use the new 
syntax or semantics associated with the future statement.  This can,
starting with Python 2.2 be controlled by optional arguments to
\function{compile()} --- see the documentation of that function in the
\citetitle[../lib/built-in-funcs.html]{Python Library Reference} for
details.

A future statement typed at an interactive interpreter prompt will
take effect for the rest of the interpreter session.  If an
interpreter is started with the \programopt{-i} option, is passed a
script name to execute, and the script includes a future statement, it
will be in effect in the interactive session started after the script
is executed.

\section{The \keyword{global} statement \label{global}}
\stindex{global}

\begin{productionlist}
  \production{global_stmt}
             {"global" \token{identifier} ("," \token{identifier})*}
\end{productionlist}

The \keyword{global} statement is a declaration which holds for the
entire current code block.  It means that the listed identifiers are to be
interpreted as globals.  It would be impossible to assign to a global
variable without \keyword{global}, although free variables may refer
to globals without being declared global.
\indexiii{global}{name}{binding}

Names listed in a \keyword{global} statement must not be used in the same
code block textually preceding that \keyword{global} statement.

Names listed in a \keyword{global} statement must not be defined as formal
parameters or in a \keyword{for} loop control target, \keyword{class}
definition, function definition, or \keyword{import} statement.

(The current implementation does not enforce the latter two
restrictions, but programs should not abuse this freedom, as future
implementations may enforce them or silently change the meaning of the
program.)

\strong{Programmer's note:}
the \keyword{global} is a directive to the parser.  It
applies only to code parsed at the same time as the \keyword{global}
statement.  In particular, a \keyword{global} statement contained in an
\keyword{exec} statement does not affect the code block \emph{containing}
the \keyword{exec} statement, and code contained in an \keyword{exec}
statement is unaffected by \keyword{global} statements in the code
containing the \keyword{exec} statement.  The same applies to the
\function{eval()}, \function{execfile()} and \function{compile()} functions.
\stindex{exec}
\bifuncindex{eval}
\bifuncindex{execfile}
\bifuncindex{compile}


\section{The \keyword{exec} statement \label{exec}}
\stindex{exec}

\begin{productionlist}
  \production{exec_stmt}
             {"exec" \token{or_expr}
              ["in" \token{expression} ["," \token{expression}]]}
\end{productionlist}

This statement supports dynamic execution of Python code.  The first
expression should evaluate to either a string, an open file object, or
a code object.  If it is a string, the string is parsed as a suite of
Python statements which is then executed (unless a syntax error
occurs).  If it is an open file, the file is parsed until \EOF{} and
executed.  If it is a code object, it is simply executed.  In all
cases, the code that's executed is expected to be valid as file
input (see section~\ref{file-input}, ``File input'').  Be aware that
the \keyword{return} and \keyword{yield} statements may not be used
outside of function definitions even within the context of code passed
to the \keyword{exec} statement.

In all cases, if the optional parts are omitted, the code is executed
in the current scope.  If only the first expression after \keyword{in}
is specified, it should be a dictionary, which will be used for both
the global and the local variables.  If two expressions are given,
they are used for the global and local variables, respectively.
If provided, \var{locals} can be any mapping object.
\versionchanged[formerly \var{locals} was required to be a dictionary]{2.4}

As a side effect, an implementation may insert additional keys into
the dictionaries given besides those corresponding to variable names
set by the executed code.  For example, the current implementation
may add a reference to the dictionary of the built-in module
\module{__builtin__} under the key \code{__builtins__} (!).
\ttindex{__builtins__}
\refbimodindex{__builtin__}

\strong{Programmer's hints:}
dynamic evaluation of expressions is supported by the built-in
function \function{eval()}.  The built-in functions
\function{globals()} and \function{locals()} return the current global
and local dictionary, respectively, which may be useful to pass around
for use by \keyword{exec}.
\bifuncindex{eval}
\bifuncindex{globals}
\bifuncindex{locals}

  



		% Simple statements
\chapter{Compound statements\label{compound}}
\indexii{compound}{statement}

Compound statements contain (groups of) other statements; they affect
or control the execution of those other statements in some way.  In
general, compound statements span multiple lines, although in simple
incarnations a whole compound statement may be contained in one line.

The \keyword{if}, \keyword{while} and \keyword{for} statements implement
traditional control flow constructs.  \keyword{try} specifies exception
handlers and/or cleanup code for a group of statements.  Function and
class definitions are also syntactically compound statements.

Compound statements consist of one or more `clauses.'  A clause
consists of a header and a `suite.'  The clause headers of a
particular compound statement are all at the same indentation level.
Each clause header begins with a uniquely identifying keyword and ends
with a colon.  A suite is a group of statements controlled by a
clause.  A suite can be one or more semicolon-separated simple
statements on the same line as the header, following the header's
colon, or it can be one or more indented statements on subsequent
lines.  Only the latter form of suite can contain nested compound
statements; the following is illegal, mostly because it wouldn't be
clear to which \keyword{if} clause a following \keyword{else} clause would
belong:
\index{clause}
\index{suite}

\begin{verbatim}
if test1: if test2: print x
\end{verbatim}

Also note that the semicolon binds tighter than the colon in this
context, so that in the following example, either all or none of the
\keyword{print} statements are executed:

\begin{verbatim}
if x < y < z: print x; print y; print z
\end{verbatim}

Summarizing:

\begin{productionlist}
  \production{compound_stmt}
             {\token{if_stmt}}
  \productioncont{| \token{while_stmt}}
  \productioncont{| \token{for_stmt}}
  \productioncont{| \token{try_stmt}}
  \productioncont{| \token{with_stmt}}
  \productioncont{| \token{funcdef}}
  \productioncont{| \token{classdef}}
  \production{suite}
             {\token{stmt_list} NEWLINE
              | NEWLINE INDENT \token{statement}+ DEDENT}
  \production{statement}
             {\token{stmt_list} NEWLINE | \token{compound_stmt}}
  \production{stmt_list}
             {\token{simple_stmt} (";" \token{simple_stmt})* [";"]}
\end{productionlist}

Note that statements always end in a
\code{NEWLINE}\index{NEWLINE token} possibly followed by a
\code{DEDENT}.\index{DEDENT token} Also note that optional
continuation clauses always begin with a keyword that cannot start a
statement, thus there are no ambiguities (the `dangling
\keyword{else}' problem is solved in Python by requiring nested
\keyword{if} statements to be indented).
\indexii{dangling}{else}

The formatting of the grammar rules in the following sections places
each clause on a separate line for clarity.


\section{The \keyword{if} statement\label{if}}
\stindex{if}

The \keyword{if} statement is used for conditional execution:

\begin{productionlist}
  \production{if_stmt}
             {"if" \token{expression} ":" \token{suite}}
  \productioncont{( "elif" \token{expression} ":" \token{suite} )*}
  \productioncont{["else" ":" \token{suite}]}
\end{productionlist}

It selects exactly one of the suites by evaluating the expressions one
by one until one is found to be true (see section~\ref{Booleans} for
the definition of true and false); then that suite is executed (and no
other part of the \keyword{if} statement is executed or evaluated).  If
all expressions are false, the suite of the \keyword{else} clause, if
present, is executed.
\kwindex{elif}
\kwindex{else}


\section{The \keyword{while} statement\label{while}}
\stindex{while}
\indexii{loop}{statement}

The \keyword{while} statement is used for repeated execution as long
as an expression is true:

\begin{productionlist}
  \production{while_stmt}
             {"while" \token{expression} ":" \token{suite}}
  \productioncont{["else" ":" \token{suite}]}
\end{productionlist}

This repeatedly tests the expression and, if it is true, executes the
first suite; if the expression is false (which may be the first time it
is tested) the suite of the \keyword{else} clause, if present, is
executed and the loop terminates.
\kwindex{else}

A \keyword{break} statement executed in the first suite terminates the
loop without executing the \keyword{else} clause's suite.  A
\keyword{continue} statement executed in the first suite skips the rest
of the suite and goes back to testing the expression.
\stindex{break}
\stindex{continue}


\section{The \keyword{for} statement\label{for}}
\stindex{for}
\indexii{loop}{statement}

The \keyword{for} statement is used to iterate over the elements of a
sequence (such as a string, tuple or list) or other iterable object:
\obindex{sequence}

\begin{productionlist}
  \production{for_stmt}
             {"for" \token{target_list} "in" \token{expression_list}
              ":" \token{suite}}
  \productioncont{["else" ":" \token{suite}]}
\end{productionlist}

The expression list is evaluated once; it should yield an iterable
object.  An iterator is created for the result of the
{}\code{expression_list}.  The suite is then executed once for each
item provided by the iterator, in the
order of ascending indices.  Each item in turn is assigned to the
target list using the standard rules for assignments, and then the
suite is executed.  When the items are exhausted (which is immediately
when the sequence is empty), the suite in the \keyword{else} clause, if
present, is executed, and the loop terminates.
\kwindex{in}
\kwindex{else}
\indexii{target}{list}

A \keyword{break} statement executed in the first suite terminates the
loop without executing the \keyword{else} clause's suite.  A
\keyword{continue} statement executed in the first suite skips the rest
of the suite and continues with the next item, or with the \keyword{else}
clause if there was no next item.
\stindex{break}
\stindex{continue}

The suite may assign to the variable(s) in the target list; this does
not affect the next item assigned to it.

The target list is not deleted when the loop is finished, but if the
sequence is empty, it will not have been assigned to at all by the
loop.  Hint: the built-in function \function{range()} returns a
sequence of integers suitable to emulate the effect of Pascal's
\code{for i := a to b do};
e.g., \code{range(3)} returns the list \code{[0, 1, 2]}.
\bifuncindex{range}
\indexii{Pascal}{language}

\warning{There is a subtlety when the sequence is being modified
by the loop (this can only occur for mutable sequences, i.e. lists).
An internal counter is used to keep track of which item is used next,
and this is incremented on each iteration.  When this counter has
reached the length of the sequence the loop terminates.  This means that
if the suite deletes the current (or a previous) item from the
sequence, the next item will be skipped (since it gets the index of
the current item which has already been treated).  Likewise, if the
suite inserts an item in the sequence before the current item, the
current item will be treated again the next time through the loop.
This can lead to nasty bugs that can be avoided by making a temporary
copy using a slice of the whole sequence, e.g.,
\index{loop!over mutable sequence}
\index{mutable sequence!loop over}}

\begin{verbatim}
for x in a[:]:
    if x < 0: a.remove(x)
\end{verbatim}


\section{The \keyword{try} statement\label{try}}
\stindex{try}

The \keyword{try} statement specifies exception handlers and/or cleanup
code for a group of statements:

\begin{productionlist}
  \production{try_stmt} {try1_stmt | try2_stmt}
  \production{try1_stmt}
             {"try" ":" \token{suite}}
  \productioncont{("except" [\token{expression}
                             ["," \token{target}]] ":" \token{suite})+}
  \productioncont{["else" ":" \token{suite}]}
  \productioncont{["finally" ":" \token{suite}]}
  \production{try2_stmt}
             {"try" ":" \token{suite}}
  \productioncont{"finally" ":" \token{suite}}
\end{productionlist}

\versionchanged[In previous versions of Python,
\keyword{try}...\keyword{except}...\keyword{finally} did not work.
\keyword{try}...\keyword{except} had to be nested in
\keyword{try}...\keyword{finally}]{2.5}

The \keyword{except} clause(s) specify one or more exception handlers.
When no exception occurs in the
\keyword{try} clause, no exception handler is executed.  When an
exception occurs in the \keyword{try} suite, a search for an exception
handler is started.  This search inspects the except clauses in turn until
one is found that matches the exception.  An expression-less except
clause, if present, must be last; it matches any exception.  For an
except clause with an expression, that expression is evaluated, and the
clause matches the exception if the resulting object is ``compatible''
with the exception.  An object is compatible with an exception if it
is the class or a base class of the exception object, a tuple
containing an item compatible with the exception, or, in the
(deprecated) case of string exceptions, is the raised string itself
(note that the object identities must match, i.e. it must be the same
string object, not just a string with the same value).
\kwindex{except}

If no except clause matches the exception, the search for an exception
handler continues in the surrounding code and on the invocation stack.
\footnote{The exception is propogated to the invocation stack only if
there is no \keyword{finally} clause that negates the exception.}

If the evaluation of an expression in the header of an except clause
raises an exception, the original search for a handler is canceled
and a search starts for the new exception in the surrounding code and
on the call stack (it is treated as if the entire \keyword{try} statement
raised the exception).

When a matching except clause is found, the exception is assigned to
the target specified in that except clause, if present, and the except
clause's suite is executed.  All except clauses must have an
executable block.  When the end of this block is reached, execution
continues normally after the entire try statement.  (This means that
if two nested handlers exist for the same exception, and the exception
occurs in the try clause of the inner handler, the outer handler will
not handle the exception.)

Before an except clause's suite is executed, details about the
exception are assigned to three variables in the
\module{sys}\refbimodindex{sys} module: \code{sys.exc_type} receives
the object identifying the exception; \code{sys.exc_value} receives
the exception's parameter; \code{sys.exc_traceback} receives a
traceback object\obindex{traceback} (see section~\ref{traceback})
identifying the point in the program where the exception occurred.
These details are also available through the \function{sys.exc_info()}
function, which returns a tuple \code{(\var{exc_type}, \var{exc_value},
\var{exc_traceback})}.  Use of the corresponding variables is
deprecated in favor of this function, since their use is unsafe in a
threaded program.  As of Python 1.5, the variables are restored to
their previous values (before the call) when returning from a function
that handled an exception.
\withsubitem{(in module sys)}{\ttindex{exc_type}
  \ttindex{exc_value}\ttindex{exc_traceback}}

The optional \keyword{else} clause is executed if and when control
flows off the end of the \keyword{try} clause.\footnote{
  Currently, control ``flows off the end'' except in the case of an
  exception or the execution of a \keyword{return},
  \keyword{continue}, or \keyword{break} statement.
} Exceptions in the \keyword{else} clause are not handled by the
preceding \keyword{except} clauses.
\kwindex{else}
\stindex{return}
\stindex{break}
\stindex{continue}

If \keyword{finally} is present, it specifies a `cleanup' handler.  The
\keyword{try} clause is executed, including any \keyword{except} and
\keyword{else} clauses.  If an exception occurs in any of the clauses
and is not handled, the exception is temporarily saved. The
\keyword{finally} clause is executed.  If there is a saved exception,
it is re-raised at the end of the \keyword{finally} clause.
If the \keyword{finally} clause raises another exception or
executes a \keyword{return} or \keyword{break} statement, the saved
exception is lost.  The exception information is not available to the
program during execution of the \keyword{finally} clause.
\kwindex{finally}

When a \keyword{return}, \keyword{break} or \keyword{continue} statement is
executed in the \keyword{try} suite of a \keyword{try}...\keyword{finally}
statement, the \keyword{finally} clause is also executed `on the way out.' A
\keyword{continue} statement is illegal in the \keyword{finally} clause.
(The reason is a problem with the current implementation --- this
restriction may be lifted in the future).
\stindex{return}
\stindex{break}
\stindex{continue}

Additional information on exceptions can be found in
section~\ref{exceptions}, and information on using the \keyword{raise}
statement to generate exceptions may be found in section~\ref{raise}.


\section{The \keyword{with} statement\label{with}}
\stindex{with}

\versionadded{2.5}

The \keyword{with} statement is used to wrap the execution of a block
with methods defined by a context manager (see
section~\ref{context-managers}). This allows common
\keyword{try}...\keyword{except}...\keyword{finally} usage patterns to
be encapsulated for convenient reuse.

\begin{productionlist}
  \production{with_stmt}
  {"with" \token{expression} ["as" \token{target}] ":" \token{suite}}
\end{productionlist}

The execution of the \keyword{with} statement proceeds as follows:

\begin{enumerate}

\item The context expression is evaluated to obtain a context manager.

\item The context manager's \method{__enter__()} method is invoked.

\item If a target was included in the \keyword{with}
statement, the return value from \method{__enter__()} is assigned to it.

\note{The \keyword{with} statement guarantees that if the
\method{__enter__()} method returns without an error, then
\method{__exit__()} will always be called. Thus, if an error occurs
during the assignment to the target list, it will be treated the same as
an error occurring within the suite would be. See step 5 below.}

\item The suite is executed.

\item The context manager's \method{__exit__()} method is invoked. If
an exception caused the suite to be exited, its type, value, and
traceback are passed as arguments to \method{__exit__()}. Otherwise,
three \constant{None} arguments are supplied.

If the suite was exited due to an exception, and the return
value from the \method{__exit__()} method was false, the exception is
reraised. If the return value was true, the exception is suppressed, and
execution continues with the statement following the \keyword{with}
statement.

If the suite was exited for any reason other than an exception, the
return value from \method{__exit__()} is ignored, and execution proceeds
at the normal location for the kind of exit that was taken.

\end{enumerate}

\begin{notice}
In Python 2.5, the \keyword{with} statement is only allowed
when the \code{with_statement} feature has been enabled.  It will always
be enabled in Python 2.6.  This \code{__future__} import statement can
be used to enable the feature:

\begin{verbatim}
from __future__ import with_statement
\end{verbatim}
\end{notice}

\begin{seealso}
  \seepep{0343}{The "with" statement}
         {The specification, background, and examples for the
          Python \keyword{with} statement.}
\end{seealso}

\section{Function definitions\label{function}}
\indexii{function}{definition}
\stindex{def}

A function definition defines a user-defined function object (see
section~\ref{types}):
\obindex{user-defined function}
\obindex{function}

\begin{productionlist}
  \production{funcdef}
             {[\token{decorators}] "def" \token{funcname} "(" [\token{parameter_list}] ")"
              ":" \token{suite}}
  \production{decorators}
             {\token{decorator}+}
  \production{decorator}
             {"@" \token{dotted_name} ["(" [\token{argument_list} [","]] ")"] NEWLINE}
  \production{dotted_name}
             {\token{identifier} ("." \token{identifier})*}
  \production{parameter_list}
                 {(\token{defparameter} ",")*}
  \productioncont{(~~"*" \token{identifier} [, "**" \token{identifier}]}
  \productioncont{ | "**" \token{identifier}}
  \productioncont{ | \token{defparameter} [","] )}
  \production{defparameter}
             {\token{parameter} ["=" \token{expression}]}
  \production{sublist}
             {\token{parameter} ("," \token{parameter})* [","]}
  \production{parameter}
             {\token{identifier} | "(" \token{sublist} ")"}
  \production{funcname}
             {\token{identifier}}
\end{productionlist}

A function definition is an executable statement.  Its execution binds
the function name in the current local namespace to a function object
(a wrapper around the executable code for the function).  This
function object contains a reference to the current global namespace
as the global namespace to be used when the function is called.
\indexii{function}{name}
\indexii{name}{binding}

The function definition does not execute the function body; this gets
executed only when the function is called.

A function definition may be wrapped by one or more decorator expressions.
Decorator expressions are evaluated when the function is defined, in the scope
that contains the function definition.  The result must be a callable,
which is invoked with the function object as the only argument.
The returned value is bound to the function name instead of the function
object.  Multiple decorators are applied in nested fashion.
For example, the following code:

\begin{verbatim}
@f1(arg)
@f2
def func(): pass
\end{verbatim}

is equivalent to:

\begin{verbatim}
def func(): pass
func = f1(arg)(f2(func))
\end{verbatim}

When one or more top-level parameters have the form \var{parameter}
\code{=} \var{expression}, the function is said to have ``default
parameter values.''  For a parameter with a
default value, the corresponding argument may be omitted from a call,
in which case the parameter's default value is substituted.  If a
parameter has a default value, all following parameters must also have
a default value --- this is a syntactic restriction that is not
expressed by the grammar.
\indexiii{default}{parameter}{value}

\strong{Default parameter values are evaluated when the function
definition is executed.}  This means that the expression is evaluated
once, when the function is defined, and that that same
``pre-computed'' value is used for each call.  This is especially
important to understand when a default parameter is a mutable object,
such as a list or a dictionary: if the function modifies the object
(e.g. by appending an item to a list), the default value is in effect
modified.  This is generally not what was intended.  A way around this 
is to use \code{None} as the default, and explicitly test for it in
the body of the function, e.g.:

\begin{verbatim}
def whats_on_the_telly(penguin=None):
    if penguin is None:
        penguin = []
    penguin.append("property of the zoo")
    return penguin
\end{verbatim}

Function call semantics are described in more detail in
section~\ref{calls}.
A function call always assigns values to all parameters mentioned in
the parameter list, either from position arguments, from keyword
arguments, or from default values.  If the form ``\code{*identifier}''
is present, it is initialized to a tuple receiving any excess
positional parameters, defaulting to the empty tuple.  If the form
``\code{**identifier}'' is present, it is initialized to a new
dictionary receiving any excess keyword arguments, defaulting to a
new empty dictionary.

It is also possible to create anonymous functions (functions not bound
to a name), for immediate use in expressions.  This uses lambda forms,
described in section~\ref{lambda}.  Note that the lambda form is
merely a shorthand for a simplified function definition; a function
defined in a ``\keyword{def}'' statement can be passed around or
assigned to another name just like a function defined by a lambda
form.  The ``\keyword{def}'' form is actually more powerful since it
allows the execution of multiple statements.
\indexii{lambda}{form}

\strong{Programmer's note:} Functions are first-class objects.  A
``\code{def}'' form executed inside a function definition defines a
local function that can be returned or passed around.  Free variables
used in the nested function can access the local variables of the
function containing the def.  See section~\ref{naming} for details.


\section{Class definitions\label{class}}
\indexii{class}{definition}
\stindex{class}

A class definition defines a class object (see section~\ref{types}):
\obindex{class}

\begin{productionlist}
  \production{classdef}
             {"class" \token{classname} [\token{inheritance}] ":"
              \token{suite}}
  \production{inheritance}
             {"(" [\token{expression_list}] ")"}
  \production{classname}
             {\token{identifier}}
\end{productionlist}

A class definition is an executable statement.  It first evaluates the
inheritance list, if present.  Each item in the inheritance list
should evaluate to a class object or class type which allows
subclassing.  The class's suite is then executed
in a new execution frame (see section~\ref{naming}), using a newly
created local namespace and the original global namespace.
(Usually, the suite contains only function definitions.)  When the
class's suite finishes execution, its execution frame is discarded but
its local namespace is saved.  A class object is then created using
the inheritance list for the base classes and the saved local
namespace for the attribute dictionary.  The class name is bound to this
class object in the original local namespace.
\index{inheritance}
\indexii{class}{name}
\indexii{name}{binding}
\indexii{execution}{frame}

\strong{Programmer's note:} Variables defined in the class definition
are class variables; they are shared by all instances.  To define
instance variables, they must be given a value in the
\method{__init__()} method or in another method.  Both class and
instance variables are accessible through the notation
``\code{self.name}'', and an instance variable hides a class variable
with the same name when accessed in this way.  Class variables with
immutable values can be used as defaults for instance variables.
For new-style classes, descriptors can be used to create instance
variables with different implementation details.
		% Compound statements
\chapter{Top-level components\label{top-level}}

The Python interpreter can get its input from a number of sources:
from a script passed to it as standard input or as program argument,
typed in interactively, from a module source file, etc.  This chapter
gives the syntax used in these cases.
\index{interpreter}


\section{Complete Python programs\label{programs}}
\index{program}

While a language specification need not prescribe how the language
interpreter is invoked, it is useful to have a notion of a complete
Python program.  A complete Python program is executed in a minimally
initialized environment: all built-in and standard modules are
available, but none have been initialized, except for \module{sys}
(various system services), \module{__builtin__} (built-in functions,
exceptions and \code{None}) and \module{__main__}.  The latter is used
to provide the local and global namespace for execution of the
complete program.
\refbimodindex{sys}
\refbimodindex{__main__}
\refbimodindex{__builtin__}

The syntax for a complete Python program is that for file input,
described in the next section.

The interpreter may also be invoked in interactive mode; in this case,
it does not read and execute a complete program but reads and executes
one statement (possibly compound) at a time.  The initial environment
is identical to that of a complete program; each statement is executed
in the namespace of \module{__main__}.
\index{interactive mode}
\refbimodindex{__main__}

Under \UNIX, a complete program can be passed to the interpreter in
three forms: with the \programopt{-c} \var{string} command line option, as a
file passed as the first command line argument, or as standard input.
If the file or standard input is a tty device, the interpreter enters
interactive mode; otherwise, it executes the file as a complete
program.
\index{UNIX}
\index{command line}
\index{standard input}


\section{File input\label{file-input}}

All input read from non-interactive files has the same form:

\begin{productionlist}
  \production{file_input}
             {(NEWLINE | \token{statement})*}
\end{productionlist}

This syntax is used in the following situations:

\begin{itemize}

\item when parsing a complete Python program (from a file or from a string);

\item when parsing a module;

\item when parsing a string passed to the \keyword{exec} statement;

\end{itemize}


\section{Interactive input\label{interactive}}

Input in interactive mode is parsed using the following grammar:

\begin{productionlist}
  \production{interactive_input}
             {[\token{stmt_list}] NEWLINE | \token{compound_stmt} NEWLINE}
\end{productionlist}

Note that a (top-level) compound statement must be followed by a blank
line in interactive mode; this is needed to help the parser detect the
end of the input.


\section{Expression input\label{expression-input}}
\index{input}

There are two forms of expression input.  Both ignore leading
whitespace.
The string argument to \function{eval()} must have the following form:
\bifuncindex{eval}

\begin{productionlist}
  \production{eval_input}
             {\token{expression_list} NEWLINE*}
\end{productionlist}

The input line read by \function{input()} must have the following form:
\bifuncindex{input}

\begin{productionlist}
  \production{input_input}
             {\token{expression_list} NEWLINE}
\end{productionlist}

Note: to read `raw' input line without interpretation, you can use the
built-in function \function{raw_input()} or the \method{readline()} method
of file objects.
\obindex{file}
\index{input!raw}
\index{raw input}
\bifuncindex{raw_input}
\withsubitem{(file method)}{\ttindex{readline()}}
		% Top-level components

\appendix

\chapter{History and License}
\section{History of the software}

Python was created in the early 1990s by Guido van Rossum at Stichting
Mathematisch Centrum (CWI, see \url{http://www.cwi.nl/}) in the Netherlands
as a successor of a language called ABC.  Guido remains Python's
principal author, although it includes many contributions from others.

In 1995, Guido continued his work on Python at the Corporation for
National Research Initiatives (CNRI, see \url{http://www.cnri.reston.va.us/})
in Reston, Virginia where he released several versions of the
software.

In May 2000, Guido and the Python core development team moved to
BeOpen.com to form the BeOpen PythonLabs team.  In October of the same
year, the PythonLabs team moved to Digital Creations (now Zope
Corporation; see \url{http://www.zope.com/}).  In 2001, the Python
Software Foundation (PSF, see \url{http://www.python.org/psf/}) was
formed, a non-profit organization created specifically to own
Python-related Intellectual Property.  Zope Corporation is a
sponsoring member of the PSF.

All Python releases are Open Source (see
\url{http://www.opensource.org/} for the Open Source Definition).
Historically, most, but not all, Python releases have also been
GPL-compatible; the table below summarizes the various releases.

\begin{tablev}{c|c|c|c|c}{textrm}%
  {Release}{Derived from}{Year}{Owner}{GPL compatible?}
  \linev{0.9.0 thru 1.2}{n/a}{1991-1995}{CWI}{yes}
  \linev{1.3 thru 1.5.2}{1.2}{1995-1999}{CNRI}{yes}
  \linev{1.6}{1.5.2}{2000}{CNRI}{no}
  \linev{2.0}{1.6}{2000}{BeOpen.com}{no}
  \linev{1.6.1}{1.6}{2001}{CNRI}{no}
  \linev{2.1}{2.0+1.6.1}{2001}{PSF}{no}
  \linev{2.0.1}{2.0+1.6.1}{2001}{PSF}{yes}
  \linev{2.1.1}{2.1+2.0.1}{2001}{PSF}{yes}
  \linev{2.2}{2.1.1}{2001}{PSF}{yes}
  \linev{2.1.2}{2.1.1}{2002}{PSF}{yes}
  \linev{2.1.3}{2.1.2}{2002}{PSF}{yes}
  \linev{2.2.1}{2.2}{2002}{PSF}{yes}
  \linev{2.2.2}{2.2.1}{2002}{PSF}{yes}
  \linev{2.2.3}{2.2.2}{2002-2003}{PSF}{yes}
  \linev{2.3}{2.2.2}{2002-2003}{PSF}{yes}
  \linev{2.3.1}{2.3}{2002-2003}{PSF}{yes}
  \linev{2.3.2}{2.3.1}{2003}{PSF}{yes}
  \linev{2.3.3}{2.3.2}{2003}{PSF}{yes}
  \linev{2.3.4}{2.3.3}{2004}{PSF}{yes}
  \linev{2.3.5}{2.3.4}{2005}{PSF}{yes}
  \linev{2.4}{2.3}{2004}{PSF}{yes}
  \linev{2.4.1}{2.4}{2005}{PSF}{yes}
  \linev{2.4.2}{2.4.1}{2005}{PSF}{yes}
  \linev{2.4.3}{2.4.2}{2006}{PSF}{yes}
  \linev{2.4.4}{2.4.3}{2006}{PSF}{yes}
  \linev{2.5}{2.4}{2006}{PSF}{yes}
  \linev{2.5.1}{2.5}{2007}{PSF}{yes}
  \linev{2.5.2}{2.5.1}{2008}{PSF}{yes}
\end{tablev}

\note{GPL-compatible doesn't mean that we're distributing
Python under the GPL.  All Python licenses, unlike the GPL, let you
distribute a modified version without making your changes open source.
The GPL-compatible licenses make it possible to combine Python with
other software that is released under the GPL; the others don't.}

Thanks to the many outside volunteers who have worked under Guido's
direction to make these releases possible.


\section{Terms and conditions for accessing or otherwise using Python}

\centerline{\strong{PSF LICENSE AGREEMENT FOR PYTHON \version}}

\begin{enumerate}
\item
This LICENSE AGREEMENT is between the Python Software Foundation
(``PSF''), and the Individual or Organization (``Licensee'') accessing
and otherwise using Python \version{} software in source or binary
form and its associated documentation.

\item
Subject to the terms and conditions of this License Agreement, PSF
hereby grants Licensee a nonexclusive, royalty-free, world-wide
license to reproduce, analyze, test, perform and/or display publicly,
prepare derivative works, distribute, and otherwise use Python
\version{} alone or in any derivative version, provided, however, that
PSF's License Agreement and PSF's notice of copyright, i.e.,
``Copyright \copyright{} 2001-2008 Python Software Foundation; All
Rights Reserved'' are retained in Python \version{} alone or in any
derivative version prepared by Licensee.

\item
In the event Licensee prepares a derivative work that is based on
or incorporates Python \version{} or any part thereof, and wants to
make the derivative work available to others as provided herein, then
Licensee hereby agrees to include in any such work a brief summary of
the changes made to Python \version.

\item
PSF is making Python \version{} available to Licensee on an ``AS IS''
basis.  PSF MAKES NO REPRESENTATIONS OR WARRANTIES, EXPRESS OR
IMPLIED.  BY WAY OF EXAMPLE, BUT NOT LIMITATION, PSF MAKES NO AND
DISCLAIMS ANY REPRESENTATION OR WARRANTY OF MERCHANTABILITY OR FITNESS
FOR ANY PARTICULAR PURPOSE OR THAT THE USE OF PYTHON \version{} WILL
NOT INFRINGE ANY THIRD PARTY RIGHTS.

\item
PSF SHALL NOT BE LIABLE TO LICENSEE OR ANY OTHER USERS OF PYTHON
\version{} FOR ANY INCIDENTAL, SPECIAL, OR CONSEQUENTIAL DAMAGES OR
LOSS AS A RESULT OF MODIFYING, DISTRIBUTING, OR OTHERWISE USING PYTHON
\version, OR ANY DERIVATIVE THEREOF, EVEN IF ADVISED OF THE
POSSIBILITY THEREOF.

\item
This License Agreement will automatically terminate upon a material
breach of its terms and conditions.

\item
Nothing in this License Agreement shall be deemed to create any
relationship of agency, partnership, or joint venture between PSF and
Licensee.  This License Agreement does not grant permission to use PSF
trademarks or trade name in a trademark sense to endorse or promote
products or services of Licensee, or any third party.

\item
By copying, installing or otherwise using Python \version, Licensee
agrees to be bound by the terms and conditions of this License
Agreement.
\end{enumerate}


\centerline{\strong{BEOPEN.COM LICENSE AGREEMENT FOR PYTHON 2.0}}

\centerline{\strong{BEOPEN PYTHON OPEN SOURCE LICENSE AGREEMENT VERSION 1}}

\begin{enumerate}
\item
This LICENSE AGREEMENT is between BeOpen.com (``BeOpen''), having an
office at 160 Saratoga Avenue, Santa Clara, CA 95051, and the
Individual or Organization (``Licensee'') accessing and otherwise
using this software in source or binary form and its associated
documentation (``the Software'').

\item
Subject to the terms and conditions of this BeOpen Python License
Agreement, BeOpen hereby grants Licensee a non-exclusive,
royalty-free, world-wide license to reproduce, analyze, test, perform
and/or display publicly, prepare derivative works, distribute, and
otherwise use the Software alone or in any derivative version,
provided, however, that the BeOpen Python License is retained in the
Software, alone or in any derivative version prepared by Licensee.

\item
BeOpen is making the Software available to Licensee on an ``AS IS''
basis.  BEOPEN MAKES NO REPRESENTATIONS OR WARRANTIES, EXPRESS OR
IMPLIED.  BY WAY OF EXAMPLE, BUT NOT LIMITATION, BEOPEN MAKES NO AND
DISCLAIMS ANY REPRESENTATION OR WARRANTY OF MERCHANTABILITY OR FITNESS
FOR ANY PARTICULAR PURPOSE OR THAT THE USE OF THE SOFTWARE WILL NOT
INFRINGE ANY THIRD PARTY RIGHTS.

\item
BEOPEN SHALL NOT BE LIABLE TO LICENSEE OR ANY OTHER USERS OF THE
SOFTWARE FOR ANY INCIDENTAL, SPECIAL, OR CONSEQUENTIAL DAMAGES OR LOSS
AS A RESULT OF USING, MODIFYING OR DISTRIBUTING THE SOFTWARE, OR ANY
DERIVATIVE THEREOF, EVEN IF ADVISED OF THE POSSIBILITY THEREOF.

\item
This License Agreement will automatically terminate upon a material
breach of its terms and conditions.

\item
This License Agreement shall be governed by and interpreted in all
respects by the law of the State of California, excluding conflict of
law provisions.  Nothing in this License Agreement shall be deemed to
create any relationship of agency, partnership, or joint venture
between BeOpen and Licensee.  This License Agreement does not grant
permission to use BeOpen trademarks or trade names in a trademark
sense to endorse or promote products or services of Licensee, or any
third party.  As an exception, the ``BeOpen Python'' logos available
at http://www.pythonlabs.com/logos.html may be used according to the
permissions granted on that web page.

\item
By copying, installing or otherwise using the software, Licensee
agrees to be bound by the terms and conditions of this License
Agreement.
\end{enumerate}


\centerline{\strong{CNRI LICENSE AGREEMENT FOR PYTHON 1.6.1}}

\begin{enumerate}
\item
This LICENSE AGREEMENT is between the Corporation for National
Research Initiatives, having an office at 1895 Preston White Drive,
Reston, VA 20191 (``CNRI''), and the Individual or Organization
(``Licensee'') accessing and otherwise using Python 1.6.1 software in
source or binary form and its associated documentation.

\item
Subject to the terms and conditions of this License Agreement, CNRI
hereby grants Licensee a nonexclusive, royalty-free, world-wide
license to reproduce, analyze, test, perform and/or display publicly,
prepare derivative works, distribute, and otherwise use Python 1.6.1
alone or in any derivative version, provided, however, that CNRI's
License Agreement and CNRI's notice of copyright, i.e., ``Copyright
\copyright{} 1995-2001 Corporation for National Research Initiatives;
All Rights Reserved'' are retained in Python 1.6.1 alone or in any
derivative version prepared by Licensee.  Alternately, in lieu of
CNRI's License Agreement, Licensee may substitute the following text
(omitting the quotes): ``Python 1.6.1 is made available subject to the
terms and conditions in CNRI's License Agreement.  This Agreement
together with Python 1.6.1 may be located on the Internet using the
following unique, persistent identifier (known as a handle):
1895.22/1013.  This Agreement may also be obtained from a proxy server
on the Internet using the following URL:
\url{http://hdl.handle.net/1895.22/1013}.''

\item
In the event Licensee prepares a derivative work that is based on
or incorporates Python 1.6.1 or any part thereof, and wants to make
the derivative work available to others as provided herein, then
Licensee hereby agrees to include in any such work a brief summary of
the changes made to Python 1.6.1.

\item
CNRI is making Python 1.6.1 available to Licensee on an ``AS IS''
basis.  CNRI MAKES NO REPRESENTATIONS OR WARRANTIES, EXPRESS OR
IMPLIED.  BY WAY OF EXAMPLE, BUT NOT LIMITATION, CNRI MAKES NO AND
DISCLAIMS ANY REPRESENTATION OR WARRANTY OF MERCHANTABILITY OR FITNESS
FOR ANY PARTICULAR PURPOSE OR THAT THE USE OF PYTHON 1.6.1 WILL NOT
INFRINGE ANY THIRD PARTY RIGHTS.

\item
CNRI SHALL NOT BE LIABLE TO LICENSEE OR ANY OTHER USERS OF PYTHON
1.6.1 FOR ANY INCIDENTAL, SPECIAL, OR CONSEQUENTIAL DAMAGES OR LOSS AS
A RESULT OF MODIFYING, DISTRIBUTING, OR OTHERWISE USING PYTHON 1.6.1,
OR ANY DERIVATIVE THEREOF, EVEN IF ADVISED OF THE POSSIBILITY THEREOF.

\item
This License Agreement will automatically terminate upon a material
breach of its terms and conditions.

\item
This License Agreement shall be governed by the federal
intellectual property law of the United States, including without
limitation the federal copyright law, and, to the extent such
U.S. federal law does not apply, by the law of the Commonwealth of
Virginia, excluding Virginia's conflict of law provisions.
Notwithstanding the foregoing, with regard to derivative works based
on Python 1.6.1 that incorporate non-separable material that was
previously distributed under the GNU General Public License (GPL), the
law of the Commonwealth of Virginia shall govern this License
Agreement only as to issues arising under or with respect to
Paragraphs 4, 5, and 7 of this License Agreement.  Nothing in this
License Agreement shall be deemed to create any relationship of
agency, partnership, or joint venture between CNRI and Licensee.  This
License Agreement does not grant permission to use CNRI trademarks or
trade name in a trademark sense to endorse or promote products or
services of Licensee, or any third party.

\item
By clicking on the ``ACCEPT'' button where indicated, or by copying,
installing or otherwise using Python 1.6.1, Licensee agrees to be
bound by the terms and conditions of this License Agreement.
\end{enumerate}

\centerline{ACCEPT}



\centerline{\strong{CWI LICENSE AGREEMENT FOR PYTHON 0.9.0 THROUGH 1.2}}

Copyright \copyright{} 1991 - 1995, Stichting Mathematisch Centrum
Amsterdam, The Netherlands.  All rights reserved.

Permission to use, copy, modify, and distribute this software and its
documentation for any purpose and without fee is hereby granted,
provided that the above copyright notice appear in all copies and that
both that copyright notice and this permission notice appear in
supporting documentation, and that the name of Stichting Mathematisch
Centrum or CWI not be used in advertising or publicity pertaining to
distribution of the software without specific, written prior
permission.

STICHTING MATHEMATISCH CENTRUM DISCLAIMS ALL WARRANTIES WITH REGARD TO
THIS SOFTWARE, INCLUDING ALL IMPLIED WARRANTIES OF MERCHANTABILITY AND
FITNESS, IN NO EVENT SHALL STICHTING MATHEMATISCH CENTRUM BE LIABLE
FOR ANY SPECIAL, INDIRECT OR CONSEQUENTIAL DAMAGES OR ANY DAMAGES
WHATSOEVER RESULTING FROM LOSS OF USE, DATA OR PROFITS, WHETHER IN AN
ACTION OF CONTRACT, NEGLIGENCE OR OTHER TORTIOUS ACTION, ARISING OUT
OF OR IN CONNECTION WITH THE USE OR PERFORMANCE OF THIS SOFTWARE.


\section{Licenses and Acknowledgements for Incorporated Software}

This section is an incomplete, but growing list of licenses and
acknowledgements for third-party software incorporated in the
Python distribution.


\subsection{Mersenne Twister}

The \module{_random} module includes code based on a download from
\url{http://www.math.keio.ac.jp/~matumoto/MT2002/emt19937ar.html}.
The following are the verbatim comments from the original code:

\begin{verbatim}
A C-program for MT19937, with initialization improved 2002/1/26.
Coded by Takuji Nishimura and Makoto Matsumoto.

Before using, initialize the state by using init_genrand(seed)
or init_by_array(init_key, key_length).

Copyright (C) 1997 - 2002, Makoto Matsumoto and Takuji Nishimura,
All rights reserved.

Redistribution and use in source and binary forms, with or without
modification, are permitted provided that the following conditions
are met:

 1. Redistributions of source code must retain the above copyright
    notice, this list of conditions and the following disclaimer.

 2. Redistributions in binary form must reproduce the above copyright
    notice, this list of conditions and the following disclaimer in the
    documentation and/or other materials provided with the distribution.

 3. The names of its contributors may not be used to endorse or promote
    products derived from this software without specific prior written
    permission.

THIS SOFTWARE IS PROVIDED BY THE COPYRIGHT HOLDERS AND CONTRIBUTORS
"AS IS" AND ANY EXPRESS OR IMPLIED WARRANTIES, INCLUDING, BUT NOT
LIMITED TO, THE IMPLIED WARRANTIES OF MERCHANTABILITY AND FITNESS FOR
A PARTICULAR PURPOSE ARE DISCLAIMED.  IN NO EVENT SHALL THE COPYRIGHT OWNER OR
CONTRIBUTORS BE LIABLE FOR ANY DIRECT, INDIRECT, INCIDENTAL, SPECIAL,
EXEMPLARY, OR CONSEQUENTIAL DAMAGES (INCLUDING, BUT NOT LIMITED TO,
PROCUREMENT OF SUBSTITUTE GOODS OR SERVICES; LOSS OF USE, DATA, OR
PROFITS; OR BUSINESS INTERRUPTION) HOWEVER CAUSED AND ON ANY THEORY OF
LIABILITY, WHETHER IN CONTRACT, STRICT LIABILITY, OR TORT (INCLUDING
NEGLIGENCE OR OTHERWISE) ARISING IN ANY WAY OUT OF THE USE OF THIS
SOFTWARE, EVEN IF ADVISED OF THE POSSIBILITY OF SUCH DAMAGE.


Any feedback is very welcome.
http://www.math.keio.ac.jp/matumoto/emt.html
email: matumoto@math.keio.ac.jp
\end{verbatim}



\subsection{Sockets}

The \module{socket} module uses the functions, \function{getaddrinfo},
and \function{getnameinfo}, which are coded in separate source files
from the WIDE Project, \url{http://www.wide.ad.jp/about/index.html}.

\begin{verbatim}      
Copyright (C) 1995, 1996, 1997, and 1998 WIDE Project.
All rights reserved.
 
Redistribution and use in source and binary forms, with or without
modification, are permitted provided that the following conditions
are met:
1. Redistributions of source code must retain the above copyright
   notice, this list of conditions and the following disclaimer.
2. Redistributions in binary form must reproduce the above copyright
   notice, this list of conditions and the following disclaimer in the
   documentation and/or other materials provided with the distribution.
3. Neither the name of the project nor the names of its contributors
   may be used to endorse or promote products derived from this software
   without specific prior written permission.

THIS SOFTWARE IS PROVIDED BY THE PROJECT AND CONTRIBUTORS ``AS IS'' AND
GAI_ANY EXPRESS OR IMPLIED WARRANTIES, INCLUDING, BUT NOT LIMITED TO, THE
IMPLIED WARRANTIES OF MERCHANTABILITY AND FITNESS FOR A PARTICULAR PURPOSE
ARE DISCLAIMED.  IN NO EVENT SHALL THE PROJECT OR CONTRIBUTORS BE LIABLE
FOR GAI_ANY DIRECT, INDIRECT, INCIDENTAL, SPECIAL, EXEMPLARY, OR CONSEQUENTIAL
DAMAGES (INCLUDING, BUT NOT LIMITED TO, PROCUREMENT OF SUBSTITUTE GOODS
OR SERVICES; LOSS OF USE, DATA, OR PROFITS; OR BUSINESS INTERRUPTION)
HOWEVER CAUSED AND ON GAI_ANY THEORY OF LIABILITY, WHETHER IN CONTRACT, STRICT
LIABILITY, OR TORT (INCLUDING NEGLIGENCE OR OTHERWISE) ARISING IN GAI_ANY WAY
OUT OF THE USE OF THIS SOFTWARE, EVEN IF ADVISED OF THE POSSIBILITY OF
SUCH DAMAGE.
\end{verbatim}



\subsection{Floating point exception control}

The source for the \module{fpectl} module includes the following notice:

\begin{verbatim}
     ---------------------------------------------------------------------  
    /                       Copyright (c) 1996.                           \ 
   |          The Regents of the University of California.                 |
   |                        All rights reserved.                           |
   |                                                                       |
   |   Permission to use, copy, modify, and distribute this software for   |
   |   any purpose without fee is hereby granted, provided that this en-   |
   |   tire notice is included in all copies of any software which is or   |
   |   includes  a  copy  or  modification  of  this software and in all   |
   |   copies of the supporting documentation for such software.           |
   |                                                                       |
   |   This  work was produced at the University of California, Lawrence   |
   |   Livermore National Laboratory under  contract  no.  W-7405-ENG-48   |
   |   between  the  U.S.  Department  of  Energy and The Regents of the   |
   |   University of California for the operation of UC LLNL.              |
   |                                                                       |
   |                              DISCLAIMER                               |
   |                                                                       |
   |   This  software was prepared as an account of work sponsored by an   |
   |   agency of the United States Government. Neither the United States   |
   |   Government  nor the University of California nor any of their em-   |
   |   ployees, makes any warranty, express or implied, or  assumes  any   |
   |   liability  or  responsibility  for the accuracy, completeness, or   |
   |   usefulness of any information,  apparatus,  product,  or  process   |
   |   disclosed,   or  represents  that  its  use  would  not  infringe   |
   |   privately-owned rights. Reference herein to any specific  commer-   |
   |   cial  products,  process,  or  service  by trade name, trademark,   |
   |   manufacturer, or otherwise, does not  necessarily  constitute  or   |
   |   imply  its endorsement, recommendation, or favoring by the United   |
   |   States Government or the University of California. The views  and   |
   |   opinions  of authors expressed herein do not necessarily state or   |
   |   reflect those of the United States Government or  the  University   |
   |   of  California,  and shall not be used for advertising or product   |
    \  endorsement purposes.                                              / 
     ---------------------------------------------------------------------
\end{verbatim}



\subsection{MD5 message digest algorithm}

The source code for the \module{md5} module contains the following notice:

\begin{verbatim}
  Copyright (C) 1999, 2002 Aladdin Enterprises.  All rights reserved.

  This software is provided 'as-is', without any express or implied
  warranty.  In no event will the authors be held liable for any damages
  arising from the use of this software.

  Permission is granted to anyone to use this software for any purpose,
  including commercial applications, and to alter it and redistribute it
  freely, subject to the following restrictions:

  1. The origin of this software must not be misrepresented; you must not
     claim that you wrote the original software. If you use this software
     in a product, an acknowledgment in the product documentation would be
     appreciated but is not required.
  2. Altered source versions must be plainly marked as such, and must not be
     misrepresented as being the original software.
  3. This notice may not be removed or altered from any source distribution.

  L. Peter Deutsch
  ghost@aladdin.com

  Independent implementation of MD5 (RFC 1321).

  This code implements the MD5 Algorithm defined in RFC 1321, whose
  text is available at
	http://www.ietf.org/rfc/rfc1321.txt
  The code is derived from the text of the RFC, including the test suite
  (section A.5) but excluding the rest of Appendix A.  It does not include
  any code or documentation that is identified in the RFC as being
  copyrighted.

  The original and principal author of md5.h is L. Peter Deutsch
  <ghost@aladdin.com>.  Other authors are noted in the change history
  that follows (in reverse chronological order):

  2002-04-13 lpd Removed support for non-ANSI compilers; removed
	references to Ghostscript; clarified derivation from RFC 1321;
	now handles byte order either statically or dynamically.
  1999-11-04 lpd Edited comments slightly for automatic TOC extraction.
  1999-10-18 lpd Fixed typo in header comment (ansi2knr rather than md5);
	added conditionalization for C++ compilation from Martin
	Purschke <purschke@bnl.gov>.
  1999-05-03 lpd Original version.
\end{verbatim}



\subsection{Asynchronous socket services}

The \module{asynchat} and \module{asyncore} modules contain the
following notice:

\begin{verbatim}      
 Copyright 1996 by Sam Rushing

                         All Rights Reserved

 Permission to use, copy, modify, and distribute this software and
 its documentation for any purpose and without fee is hereby
 granted, provided that the above copyright notice appear in all
 copies and that both that copyright notice and this permission
 notice appear in supporting documentation, and that the name of Sam
 Rushing not be used in advertising or publicity pertaining to
 distribution of the software without specific, written prior
 permission.

 SAM RUSHING DISCLAIMS ALL WARRANTIES WITH REGARD TO THIS SOFTWARE,
 INCLUDING ALL IMPLIED WARRANTIES OF MERCHANTABILITY AND FITNESS, IN
 NO EVENT SHALL SAM RUSHING BE LIABLE FOR ANY SPECIAL, INDIRECT OR
 CONSEQUENTIAL DAMAGES OR ANY DAMAGES WHATSOEVER RESULTING FROM LOSS
 OF USE, DATA OR PROFITS, WHETHER IN AN ACTION OF CONTRACT,
 NEGLIGENCE OR OTHER TORTIOUS ACTION, ARISING OUT OF OR IN
 CONNECTION WITH THE USE OR PERFORMANCE OF THIS SOFTWARE.
\end{verbatim}


\subsection{Cookie management}

The \module{Cookie} module contains the following notice:

\begin{verbatim}
 Copyright 2000 by Timothy O'Malley <timo@alum.mit.edu>

                All Rights Reserved

 Permission to use, copy, modify, and distribute this software
 and its documentation for any purpose and without fee is hereby
 granted, provided that the above copyright notice appear in all
 copies and that both that copyright notice and this permission
 notice appear in supporting documentation, and that the name of
 Timothy O'Malley  not be used in advertising or publicity
 pertaining to distribution of the software without specific, written
 prior permission.

 Timothy O'Malley DISCLAIMS ALL WARRANTIES WITH REGARD TO THIS
 SOFTWARE, INCLUDING ALL IMPLIED WARRANTIES OF MERCHANTABILITY
 AND FITNESS, IN NO EVENT SHALL Timothy O'Malley BE LIABLE FOR
 ANY SPECIAL, INDIRECT OR CONSEQUENTIAL DAMAGES OR ANY DAMAGES
 WHATSOEVER RESULTING FROM LOSS OF USE, DATA OR PROFITS,
 WHETHER IN AN ACTION OF CONTRACT, NEGLIGENCE OR OTHER TORTIOUS
 ACTION, ARISING OUT OF OR IN CONNECTION WITH THE USE OR
 PERFORMANCE OF THIS SOFTWARE.
\end{verbatim}      



\subsection{Profiling}

The \module{profile} and \module{pstats} modules contain
the following notice:

\begin{verbatim}
 Copyright 1994, by InfoSeek Corporation, all rights reserved.
 Written by James Roskind

 Permission to use, copy, modify, and distribute this Python software
 and its associated documentation for any purpose (subject to the
 restriction in the following sentence) without fee is hereby granted,
 provided that the above copyright notice appears in all copies, and
 that both that copyright notice and this permission notice appear in
 supporting documentation, and that the name of InfoSeek not be used in
 advertising or publicity pertaining to distribution of the software
 without specific, written prior permission.  This permission is
 explicitly restricted to the copying and modification of the software
 to remain in Python, compiled Python, or other languages (such as C)
 wherein the modified or derived code is exclusively imported into a
 Python module.

 INFOSEEK CORPORATION DISCLAIMS ALL WARRANTIES WITH REGARD TO THIS
 SOFTWARE, INCLUDING ALL IMPLIED WARRANTIES OF MERCHANTABILITY AND
 FITNESS. IN NO EVENT SHALL INFOSEEK CORPORATION BE LIABLE FOR ANY
 SPECIAL, INDIRECT OR CONSEQUENTIAL DAMAGES OR ANY DAMAGES WHATSOEVER
 RESULTING FROM LOSS OF USE, DATA OR PROFITS, WHETHER IN AN ACTION OF
 CONTRACT, NEGLIGENCE OR OTHER TORTIOUS ACTION, ARISING OUT OF OR IN
 CONNECTION WITH THE USE OR PERFORMANCE OF THIS SOFTWARE.
\end{verbatim}



\subsection{Execution tracing}

The \module{trace} module contains the following notice:

\begin{verbatim}
 portions copyright 2001, Autonomous Zones Industries, Inc., all rights...
 err...  reserved and offered to the public under the terms of the
 Python 2.2 license.
 Author: Zooko O'Whielacronx
 http://zooko.com/
 mailto:zooko@zooko.com

 Copyright 2000, Mojam Media, Inc., all rights reserved.
 Author: Skip Montanaro

 Copyright 1999, Bioreason, Inc., all rights reserved.
 Author: Andrew Dalke

 Copyright 1995-1997, Automatrix, Inc., all rights reserved.
 Author: Skip Montanaro

 Copyright 1991-1995, Stichting Mathematisch Centrum, all rights reserved.


 Permission to use, copy, modify, and distribute this Python software and
 its associated documentation for any purpose without fee is hereby
 granted, provided that the above copyright notice appears in all copies,
 and that both that copyright notice and this permission notice appear in
 supporting documentation, and that the name of neither Automatrix,
 Bioreason or Mojam Media be used in advertising or publicity pertaining to
 distribution of the software without specific, written prior permission.
\end{verbatim} 



\subsection{UUencode and UUdecode functions}

The \module{uu} module contains the following notice:

\begin{verbatim}
 Copyright 1994 by Lance Ellinghouse
 Cathedral City, California Republic, United States of America.
                        All Rights Reserved
 Permission to use, copy, modify, and distribute this software and its
 documentation for any purpose and without fee is hereby granted,
 provided that the above copyright notice appear in all copies and that
 both that copyright notice and this permission notice appear in
 supporting documentation, and that the name of Lance Ellinghouse
 not be used in advertising or publicity pertaining to distribution
 of the software without specific, written prior permission.
 LANCE ELLINGHOUSE DISCLAIMS ALL WARRANTIES WITH REGARD TO
 THIS SOFTWARE, INCLUDING ALL IMPLIED WARRANTIES OF MERCHANTABILITY AND
 FITNESS, IN NO EVENT SHALL LANCE ELLINGHOUSE CENTRUM BE LIABLE
 FOR ANY SPECIAL, INDIRECT OR CONSEQUENTIAL DAMAGES OR ANY DAMAGES
 WHATSOEVER RESULTING FROM LOSS OF USE, DATA OR PROFITS, WHETHER IN AN
 ACTION OF CONTRACT, NEGLIGENCE OR OTHER TORTIOUS ACTION, ARISING OUT
 OF OR IN CONNECTION WITH THE USE OR PERFORMANCE OF THIS SOFTWARE.

 Modified by Jack Jansen, CWI, July 1995:
 - Use binascii module to do the actual line-by-line conversion
   between ascii and binary. This results in a 1000-fold speedup. The C
   version is still 5 times faster, though.
 - Arguments more compliant with python standard
\end{verbatim}



\subsection{XML Remote Procedure Calls}

The \module{xmlrpclib} module contains the following notice:

\begin{verbatim}
     The XML-RPC client interface is

 Copyright (c) 1999-2002 by Secret Labs AB
 Copyright (c) 1999-2002 by Fredrik Lundh

 By obtaining, using, and/or copying this software and/or its
 associated documentation, you agree that you have read, understood,
 and will comply with the following terms and conditions:

 Permission to use, copy, modify, and distribute this software and
 its associated documentation for any purpose and without fee is
 hereby granted, provided that the above copyright notice appears in
 all copies, and that both that copyright notice and this permission
 notice appear in supporting documentation, and that the name of
 Secret Labs AB or the author not be used in advertising or publicity
 pertaining to distribution of the software without specific, written
 prior permission.

 SECRET LABS AB AND THE AUTHOR DISCLAIMS ALL WARRANTIES WITH REGARD
 TO THIS SOFTWARE, INCLUDING ALL IMPLIED WARRANTIES OF MERCHANT-
 ABILITY AND FITNESS.  IN NO EVENT SHALL SECRET LABS AB OR THE AUTHOR
 BE LIABLE FOR ANY SPECIAL, INDIRECT OR CONSEQUENTIAL DAMAGES OR ANY
 DAMAGES WHATSOEVER RESULTING FROM LOSS OF USE, DATA OR PROFITS,
 WHETHER IN AN ACTION OF CONTRACT, NEGLIGENCE OR OTHER TORTIOUS
 ACTION, ARISING OUT OF OR IN CONNECTION WITH THE USE OR PERFORMANCE
 OF THIS SOFTWARE.
\end{verbatim}


\documentclass{manual}

\title{Python Reference Manual}

\author{Guido van Rossum\\
	Fred L. Drake, Jr., editor}
\authoraddress{
	\strong{Python Software Foundation}\\
	Email: \email{docs@python.org}
}

\date{23rd December, 2008}			% XXX update before final release!
\input{patchlevel}		% include Python version information


\makeindex

\begin{document}

\maketitle

\ifhtml
\chapter*{Front Matter\label{front}}
\fi

Copyright \copyright{} 2001-2008 Python Software Foundation.
All rights reserved.

Copyright \copyright{} 2000 BeOpen.com.
All rights reserved.

Copyright \copyright{} 1995-2000 Corporation for National Research Initiatives.
All rights reserved.

Copyright \copyright{} 1991-1995 Stichting Mathematisch Centrum.
All rights reserved.

See the end of this document for complete license and permissions
information.


\begin{abstract}

\noindent
Python is an interpreted, object-oriented, high-level programming
language with dynamic semantics.  Its high-level built in data
structures, combined with dynamic typing and dynamic binding, make it
very attractive for rapid application development, as well as for use
as a scripting or glue language to connect existing components
together.  Python's simple, easy to learn syntax emphasizes
readability and therefore reduces the cost of program
maintenance.  Python supports modules and packages, which encourages
program modularity and code reuse.  The Python interpreter and the
extensive standard library are available in source or binary form
without charge for all major platforms, and can be freely distributed.

This reference manual describes the syntax and ``core semantics'' of
the language.  It is terse, but attempts to be exact and complete.
The semantics of non-essential built-in object types and of the
built-in functions and modules are described in the
\citetitle[../lib/lib.html]{Python Library Reference}.  For an
informal introduction to the language, see the
\citetitle[../tut/tut.html]{Python Tutorial}.  For C or
\Cpp{} programmers, two additional manuals exist:
\citetitle[../ext/ext.html]{Extending and Embedding the Python
Interpreter} describes the high-level picture of how to write a Python
extension module, and the \citetitle[../api/api.html]{Python/C API
Reference Manual} describes the interfaces available to
C/\Cpp{} programmers in detail.

\end{abstract}

\tableofcontents

\chapter{Introduction\label{introduction}}

This reference manual describes the Python programming language.
It is not intended as a tutorial.

While I am trying to be as precise as possible, I chose to use English
rather than formal specifications for everything except syntax and
lexical analysis.  This should make the document more understandable
to the average reader, but will leave room for ambiguities.
Consequently, if you were coming from Mars and tried to re-implement
Python from this document alone, you might have to guess things and in
fact you would probably end up implementing quite a different language.
On the other hand, if you are using
Python and wonder what the precise rules about a particular area of
the language are, you should definitely be able to find them here.
If you would like to see a more formal definition of the language,
maybe you could volunteer your time --- or invent a cloning machine
:-).

It is dangerous to add too many implementation details to a language
reference document --- the implementation may change, and other
implementations of the same language may work differently.  On the
other hand, there is currently only one Python implementation in
widespread use (although alternate implementations exist), and
its particular quirks are sometimes worth being mentioned, especially
where the implementation imposes additional limitations.  Therefore,
you'll find short ``implementation notes'' sprinkled throughout the
text.

Every Python implementation comes with a number of built-in and
standard modules.  These are not documented here, but in the separate
\citetitle[../lib/lib.html]{Python Library Reference} document.  A few
built-in modules are mentioned when they interact in a significant way
with the language definition.


\section{Alternate Implementations\label{implementations}}

Though there is one Python implementation which is by far the most
popular, there are some alternate implementations which are of
particular interest to different audiences.

Known implementations include:

\begin{itemize}
\item[CPython]
This is the original and most-maintained implementation of Python,
written in C.  New language features generally appear here first.

\item[Jython]
Python implemented in Java.  This implementation can be used as a
scripting language for Java applications, or can be used to create
applications using the Java class libraries.  It is also often used to
create tests for Java libraries.  More information can be found at
\ulink{the Jython website}{http://www.jython.org/}.

\item[Python for .NET]
This implementation actually uses the CPython implementation, but is a
managed .NET application and makes .NET libraries available.  This was
created by Brian Lloyd.  For more information, see the \ulink{Python
for .NET home page}{http://www.zope.org/Members/Brian/PythonNet}.

\item[IronPython]
An alternate Python for\ .NET.  Unlike Python.NET, this is a complete
Python implementation that generates IL, and compiles Python code
directly to\ .NET assemblies.  It was created by Jim Hugunin, the
original creator of Jython.  For more information, see \ulink{the
IronPython website}{http://workspaces.gotdotnet.com/ironpython}.

\item[PyPy]
An implementation of Python written in Python; even the bytecode
interpreter is written in Python.  This is executed using CPython as
the underlying interpreter.  One of the goals of the project is to
encourage experimentation with the language itself by making it easier
to modify the interpreter (since it is written in Python).  Additional
information is available on \ulink{the PyPy project's home
page}{http://codespeak.net/pypy/}.
\end{itemize}

Each of these implementations varies in some way from the language as
documented in this manual, or introduces specific information beyond
what's covered in the standard Python documentation.  Please refer to
the implementation-specific documentation to determine what else you
need to know about the specific implementation you're using.


\section{Notation\label{notation}}

The descriptions of lexical analysis and syntax use a modified BNF
grammar notation.  This uses the following style of definition:
\index{BNF}
\index{grammar}
\index{syntax}
\index{notation}

\begin{productionlist}[*]
  \production{name}{\token{lc_letter} (\token{lc_letter} | "_")*}
  \production{lc_letter}{"a"..."z"}
\end{productionlist}

The first line says that a \code{name} is an \code{lc_letter} followed by
a sequence of zero or more \code{lc_letter}s and underscores.  An
\code{lc_letter} in turn is any of the single characters \character{a}
through \character{z}.  (This rule is actually adhered to for the
names defined in lexical and grammar rules in this document.)

Each rule begins with a name (which is the name defined by the rule)
and \code{::=}.  A vertical bar (\code{|}) is used to separate
alternatives; it is the least binding operator in this notation.  A
star (\code{*}) means zero or more repetitions of the preceding item;
likewise, a plus (\code{+}) means one or more repetitions, and a
phrase enclosed in square brackets (\code{[ ]}) means zero or one
occurrences (in other words, the enclosed phrase is optional).  The
\code{*} and \code{+} operators bind as tightly as possible;
parentheses are used for grouping.  Literal strings are enclosed in
quotes.  White space is only meaningful to separate tokens.
Rules are normally contained on a single line; rules with many
alternatives may be formatted alternatively with each line after the
first beginning with a vertical bar.

In lexical definitions (as the example above), two more conventions
are used: Two literal characters separated by three dots mean a choice
of any single character in the given (inclusive) range of \ASCII{}
characters.  A phrase between angular brackets (\code{<...>}) gives an
informal description of the symbol defined; e.g., this could be used
to describe the notion of `control character' if needed.
\index{lexical definitions}
\index{ASCII@\ASCII}

Even though the notation used is almost the same, there is a big
difference between the meaning of lexical and syntactic definitions:
a lexical definition operates on the individual characters of the
input source, while a syntax definition operates on the stream of
tokens generated by the lexical analysis.  All uses of BNF in the next
chapter (``Lexical Analysis'') are lexical definitions; uses in
subsequent chapters are syntactic definitions.
		% Introduction
\chapter{Lexical analysis\label{lexical}}

A Python program is read by a \emph{parser}.  Input to the parser is a
stream of \emph{tokens}, generated by the \emph{lexical analyzer}.  This
chapter describes how the lexical analyzer breaks a file into tokens.
\index{lexical analysis}
\index{parser}
\index{token}

Python uses the 7-bit \ASCII{} character set for program text.
\versionadded[An encoding declaration can be used to indicate that 
string literals and comments use an encoding different from ASCII]{2.3}
For compatibility with older versions, Python only warns if it finds
8-bit characters; those warnings should be corrected by either declaring
an explicit encoding, or using escape sequences if those bytes are binary
data, instead of characters.


The run-time character set depends on the I/O devices connected to the
program but is generally a superset of \ASCII.

\strong{Future compatibility note:} It may be tempting to assume that the
character set for 8-bit characters is ISO Latin-1 (an \ASCII{}
superset that covers most western languages that use the Latin
alphabet), but it is possible that in the future Unicode text editors
will become common.  These generally use the UTF-8 encoding, which is
also an \ASCII{} superset, but with very different use for the
characters with ordinals 128-255.  While there is no consensus on this
subject yet, it is unwise to assume either Latin-1 or UTF-8, even
though the current implementation appears to favor Latin-1.  This
applies both to the source character set and the run-time character
set.


\section{Line structure\label{line-structure}}

A Python program is divided into a number of \emph{logical lines}.
\index{line structure}


\subsection{Logical lines\label{logical}}

The end of
a logical line is represented by the token NEWLINE.  Statements cannot
cross logical line boundaries except where NEWLINE is allowed by the
syntax (e.g., between statements in compound statements).
A logical line is constructed from one or more \emph{physical lines}
by following the explicit or implicit \emph{line joining} rules.
\index{logical line}
\index{physical line}
\index{line joining}
\index{NEWLINE token}


\subsection{Physical lines\label{physical}}

A physical line is a sequence of characters terminated by an end-of-line
sequence.  In source files, any of the standard platform line
termination sequences can be used - the \UNIX{} form using \ASCII{} LF
(linefeed), the Windows form using the \ASCII{} sequence CR LF (return
followed by linefeed), or the Macintosh form using the \ASCII{} CR
(return) character.  All of these forms can be used equally, regardless
of platform.

When embedding Python, source code strings should be passed to Python
APIs using the standard C conventions for newline characters (the
\code{\e n} character, representing \ASCII{} LF, is the line
terminator).


\subsection{Comments\label{comments}}

A comment starts with a hash character (\code{\#}) that is not part of
a string literal, and ends at the end of the physical line.  A comment
signifies the end of the logical line unless the implicit line joining
rules are invoked.
Comments are ignored by the syntax; they are not tokens.
\index{comment}
\index{hash character}


\subsection{Encoding declarations\label{encodings}}
\index{source character set}
\index{encodings}

If a comment in the first or second line of the Python script matches
the regular expression \regexp{coding[=:]\e s*([-\e w.]+)}, this comment is
processed as an encoding declaration; the first group of this
expression names the encoding of the source code file. The recommended
forms of this expression are

\begin{verbatim}
# -*- coding: <encoding-name> -*-
\end{verbatim}

which is recognized also by GNU Emacs, and

\begin{verbatim}
# vim:fileencoding=<encoding-name>
\end{verbatim}

which is recognized by Bram Moolenaar's VIM. In addition, if the first
bytes of the file are the UTF-8 byte-order mark
(\code{'\e xef\e xbb\e xbf'}), the declared file encoding is UTF-8
(this is supported, among others, by Microsoft's \program{notepad}).

If an encoding is declared, the encoding name must be recognized by
Python. % XXX there should be a list of supported encodings.
The encoding is used for all lexical analysis, in particular to find
the end of a string, and to interpret the contents of Unicode literals.
String literals are converted to Unicode for syntactical analysis,
then converted back to their original encoding before interpretation
starts. The encoding declaration must appear on a line of its own.

\subsection{Explicit line joining\label{explicit-joining}}

Two or more physical lines may be joined into logical lines using
backslash characters (\code{\e}), as follows: when a physical line ends
in a backslash that is not part of a string literal or comment, it is
joined with the following forming a single logical line, deleting the
backslash and the following end-of-line character.  For example:
\index{physical line}
\index{line joining}
\index{line continuation}
\index{backslash character}
%
\begin{verbatim}
if 1900 < year < 2100 and 1 <= month <= 12 \
   and 1 <= day <= 31 and 0 <= hour < 24 \
   and 0 <= minute < 60 and 0 <= second < 60:   # Looks like a valid date
        return 1
\end{verbatim}

A line ending in a backslash cannot carry a comment.  A backslash does
not continue a comment.  A backslash does not continue a token except
for string literals (i.e., tokens other than string literals cannot be
split across physical lines using a backslash).  A backslash is
illegal elsewhere on a line outside a string literal.


\subsection{Implicit line joining\label{implicit-joining}}

Expressions in parentheses, square brackets or curly braces can be
split over more than one physical line without using backslashes.
For example:

\begin{verbatim}
month_names = ['Januari', 'Februari', 'Maart',      # These are the
               'April',   'Mei',      'Juni',       # Dutch names
               'Juli',    'Augustus', 'September',  # for the months
               'Oktober', 'November', 'December']   # of the year
\end{verbatim}

Implicitly continued lines can carry comments.  The indentation of the
continuation lines is not important.  Blank continuation lines are
allowed.  There is no NEWLINE token between implicit continuation
lines.  Implicitly continued lines can also occur within triple-quoted
strings (see below); in that case they cannot carry comments.


\subsection{Blank lines \label{blank-lines}}

\index{blank line}
A logical line that contains only spaces, tabs, formfeeds and possibly
a comment, is ignored (i.e., no NEWLINE token is generated).  During
interactive input of statements, handling of a blank line may differ
depending on the implementation of the read-eval-print loop.  In the
standard implementation, an entirely blank logical line (i.e.\ one
containing not even whitespace or a comment) terminates a multi-line
statement.


\subsection{Indentation\label{indentation}}

Leading whitespace (spaces and tabs) at the beginning of a logical
line is used to compute the indentation level of the line, which in
turn is used to determine the grouping of statements.
\index{indentation}
\index{whitespace}
\index{leading whitespace}
\index{space}
\index{tab}
\index{grouping}
\index{statement grouping}

First, tabs are replaced (from left to right) by one to eight spaces
such that the total number of characters up to and including the
replacement is a multiple of
eight (this is intended to be the same rule as used by \UNIX).  The
total number of spaces preceding the first non-blank character then
determines the line's indentation.  Indentation cannot be split over
multiple physical lines using backslashes; the whitespace up to the
first backslash determines the indentation.

\strong{Cross-platform compatibility note:} because of the nature of
text editors on non-UNIX platforms, it is unwise to use a mixture of
spaces and tabs for the indentation in a single source file.  It
should also be noted that different platforms may explicitly limit the
maximum indentation level.

A formfeed character may be present at the start of the line; it will
be ignored for the indentation calculations above.  Formfeed
characters occurring elsewhere in the leading whitespace have an
undefined effect (for instance, they may reset the space count to
zero).

The indentation levels of consecutive lines are used to generate
INDENT and DEDENT tokens, using a stack, as follows.
\index{INDENT token}
\index{DEDENT token}

Before the first line of the file is read, a single zero is pushed on
the stack; this will never be popped off again.  The numbers pushed on
the stack will always be strictly increasing from bottom to top.  At
the beginning of each logical line, the line's indentation level is
compared to the top of the stack.  If it is equal, nothing happens.
If it is larger, it is pushed on the stack, and one INDENT token is
generated.  If it is smaller, it \emph{must} be one of the numbers
occurring on the stack; all numbers on the stack that are larger are
popped off, and for each number popped off a DEDENT token is
generated.  At the end of the file, a DEDENT token is generated for
each number remaining on the stack that is larger than zero.

Here is an example of a correctly (though confusingly) indented piece
of Python code:

\begin{verbatim}
def perm(l):
        # Compute the list of all permutations of l
    if len(l) <= 1:
                  return [l]
    r = []
    for i in range(len(l)):
             s = l[:i] + l[i+1:]
             p = perm(s)
             for x in p:
              r.append(l[i:i+1] + x)
    return r
\end{verbatim}

The following example shows various indentation errors:

\begin{verbatim}
 def perm(l):                       # error: first line indented
for i in range(len(l)):             # error: not indented
    s = l[:i] + l[i+1:]
        p = perm(l[:i] + l[i+1:])   # error: unexpected indent
        for x in p:
                r.append(l[i:i+1] + x)
            return r                # error: inconsistent dedent
\end{verbatim}

(Actually, the first three errors are detected by the parser; only the
last error is found by the lexical analyzer --- the indentation of
\code{return r} does not match a level popped off the stack.)


\subsection{Whitespace between tokens\label{whitespace}}

Except at the beginning of a logical line or in string literals, the
whitespace characters space, tab and formfeed can be used
interchangeably to separate tokens.  Whitespace is needed between two
tokens only if their concatenation could otherwise be interpreted as a
different token (e.g., ab is one token, but a b is two tokens).


\section{Other tokens\label{other-tokens}}

Besides NEWLINE, INDENT and DEDENT, the following categories of tokens
exist: \emph{identifiers}, \emph{keywords}, \emph{literals},
\emph{operators}, and \emph{delimiters}.
Whitespace characters (other than line terminators, discussed earlier)
are not tokens, but serve to delimit tokens.
Where
ambiguity exists, a token comprises the longest possible string that
forms a legal token, when read from left to right.


\section{Identifiers and keywords\label{identifiers}}

Identifiers (also referred to as \emph{names}) are described by the following
lexical definitions:
\index{identifier}
\index{name}

\begin{productionlist}
  \production{identifier}
             {(\token{letter}|"_") (\token{letter} | \token{digit} | "_")*}
  \production{letter}
             {\token{lowercase} | \token{uppercase}}
  \production{lowercase}
             {"a"..."z"}
  \production{uppercase}
             {"A"..."Z"}
  \production{digit}
             {"0"..."9"}
\end{productionlist}

Identifiers are unlimited in length.  Case is significant.


\subsection{Keywords\label{keywords}}

The following identifiers are used as reserved words, or
\emph{keywords} of the language, and cannot be used as ordinary
identifiers.  They must be spelled exactly as written here:%
\index{keyword}%
\index{reserved word}

\begin{verbatim}
and       del       from      not       while    
as        elif      global    or        with     
assert    else      if        pass      yield    
break     except    import    print              
class     exec      in        raise              
continue  finally   is        return             
def       for       lambda    try 
\end{verbatim}

% When adding keywords, use reswords.py for reformatting

\versionchanged[\constant{None} became a constant and is now
recognized by the compiler as a name for the built-in object
\constant{None}.  Although it is not a keyword, you cannot assign
a different object to it]{2.4}

\versionchanged[Both \keyword{as} and \keyword{with} are only recognized
when the \code{with_statement} future feature has been enabled.
It will always be enabled in Python 2.6.  See section~\ref{with} for
details.  Note that using \keyword{as} and \keyword{with} as identifiers
will always issue a warning, even when the \code{with_statement} future
directive is not in effect]{2.5}


\subsection{Reserved classes of identifiers\label{id-classes}}

Certain classes of identifiers (besides keywords) have special
meanings.  These classes are identified by the patterns of leading and
trailing underscore characters:

\begin{description}

\item[\code{_*}]
  Not imported by \samp{from \var{module} import *}.  The special
  identifier \samp{_} is used in the interactive interpreter to store
  the result of the last evaluation; it is stored in the
  \module{__builtin__} module.  When not in interactive mode, \samp{_}
  has no special meaning and is not defined.
  See section~\ref{import}, ``The \keyword{import} statement.''

  \note{The name \samp{_} is often used in conjunction with
  internationalization; refer to the documentation for the
  \ulink{\module{gettext} module}{../lib/module-gettext.html} for more
  information on this convention.}

\item[\code{__*__}]
  System-defined names.  These names are defined by the interpreter
  and its implementation (including the standard library);
  applications should not expect to define additional names using this
  convention.  The set of names of this class defined by Python may be
  extended in future versions.
  See section~\ref{specialnames}, ``Special method names.''

\item[\code{__*}]
  Class-private names.  Names in this category, when used within the
  context of a class definition, are re-written to use a mangled form
  to help avoid name clashes between ``private'' attributes of base
  and derived classes.
  See section~\ref{atom-identifiers}, ``Identifiers (Names).''

\end{description}


\section{Literals\label{literals}}

Literals are notations for constant values of some built-in types.
\index{literal}
\index{constant}


\subsection{String literals\label{strings}}

String literals are described by the following lexical definitions:
\index{string literal}

\index{ASCII@\ASCII}
\begin{productionlist}
  \production{stringliteral}
             {[\token{stringprefix}](\token{shortstring} | \token{longstring})}
  \production{stringprefix}
             {"r" | "u" | "ur" | "R" | "U" | "UR" | "Ur" | "uR"}
  \production{shortstring}
             {"'" \token{shortstringitem}* "'"
              | '"' \token{shortstringitem}* '"'}
  \production{longstring}
             {"'''" \token{longstringitem}* "'''"}
  \productioncont{| '"""' \token{longstringitem}* '"""'}
  \production{shortstringitem}
             {\token{shortstringchar} | \token{escapeseq}}
  \production{longstringitem}
             {\token{longstringchar} | \token{escapeseq}}
  \production{shortstringchar}
             {<any source character except "\e" or newline or the quote>}
  \production{longstringchar}
             {<any source character except "\e">}
  \production{escapeseq}
             {"\e" <any ASCII character>}
\end{productionlist}

One syntactic restriction not indicated by these productions is that
whitespace is not allowed between the \grammartoken{stringprefix} and
the rest of the string literal. The source character set is defined
by the encoding declaration; it is \ASCII{} if no encoding declaration
is given in the source file; see section~\ref{encodings}.

\index{triple-quoted string}
\index{Unicode Consortium}
\index{string!Unicode}
In plain English: String literals can be enclosed in matching single
quotes (\code{'}) or double quotes (\code{"}).  They can also be
enclosed in matching groups of three single or double quotes (these
are generally referred to as \emph{triple-quoted strings}).  The
backslash (\code{\e}) character is used to escape characters that
otherwise have a special meaning, such as newline, backslash itself,
or the quote character.  String literals may optionally be prefixed
with a letter \character{r} or \character{R}; such strings are called
\dfn{raw strings}\index{raw string} and use different rules for interpreting
backslash escape sequences.  A prefix of \character{u} or \character{U}
makes the string a Unicode string.  Unicode strings use the Unicode character
set as defined by the Unicode Consortium and ISO~10646.  Some additional
escape sequences, described below, are available in Unicode strings.
The two prefix characters may be combined; in this case, \character{u} must
appear before \character{r}.

In triple-quoted strings,
unescaped newlines and quotes are allowed (and are retained), except
that three unescaped quotes in a row terminate the string.  (A
``quote'' is the character used to open the string, i.e. either
\code{'} or \code{"}.)

Unless an \character{r} or \character{R} prefix is present, escape
sequences in strings are interpreted according to rules similar
to those used by Standard C.  The recognized escape sequences are:
\index{physical line}
\index{escape sequence}
\index{Standard C}
\index{C}

\begin{tableiii}{l|l|c}{code}{Escape Sequence}{Meaning}{Notes}
\lineiii{\e\var{newline}} {Ignored}{}
\lineiii{\e\e}	{Backslash (\code{\e})}{}
\lineiii{\e'}	{Single quote (\code{'})}{}
\lineiii{\e"}	{Double quote (\code{"})}{}
\lineiii{\e a}	{\ASCII{} Bell (BEL)}{}
\lineiii{\e b}	{\ASCII{} Backspace (BS)}{}
\lineiii{\e f}	{\ASCII{} Formfeed (FF)}{}
\lineiii{\e n}	{\ASCII{} Linefeed (LF)}{}
\lineiii{\e N\{\var{name}\}}
        {Character named \var{name} in the Unicode database (Unicode only)}{}
\lineiii{\e r}	{\ASCII{} Carriage Return (CR)}{}
\lineiii{\e t}	{\ASCII{} Horizontal Tab (TAB)}{}
\lineiii{\e u\var{xxxx}}
        {Character with 16-bit hex value \var{xxxx} (Unicode only)}{(1)}
\lineiii{\e U\var{xxxxxxxx}}
        {Character with 32-bit hex value \var{xxxxxxxx} (Unicode only)}{(2)}
\lineiii{\e v}	{\ASCII{} Vertical Tab (VT)}{}
\lineiii{\e\var{ooo}} {Character with octal value \var{ooo}}{(3,5)}
\lineiii{\e x\var{hh}} {Character with hex value \var{hh}}{(4,5)}
\end{tableiii}
\index{ASCII@\ASCII}

\noindent
Notes:

\begin{itemize}
\item[(1)]
  Individual code units which form parts of a surrogate pair can be
  encoded using this escape sequence.
\item[(2)]
  Any Unicode character can be encoded this way, but characters
  outside the Basic Multilingual Plane (BMP) will be encoded using a
  surrogate pair if Python is compiled to use 16-bit code units (the
  default).  Individual code units which form parts of a surrogate
  pair can be encoded using this escape sequence.
\item[(3)]
  As in Standard C, up to three octal digits are accepted.
\item[(4)]
  Unlike in Standard C, exactly two hex digits are required.
\item[(5)]
  In a string literal, hexadecimal and octal escapes denote the
  byte with the given value; it is not necessary that the byte
  encodes a character in the source character set. In a Unicode
  literal, these escapes denote a Unicode character with the given
  value.
\end{itemize}


Unlike Standard \index{unrecognized escape sequence}C,
all unrecognized escape sequences are left in the string unchanged,
i.e., \emph{the backslash is left in the string}.  (This behavior is
useful when debugging: if an escape sequence is mistyped, the
resulting output is more easily recognized as broken.)  It is also
important to note that the escape sequences marked as ``(Unicode
only)'' in the table above fall into the category of unrecognized
escapes for non-Unicode string literals.

When an \character{r} or \character{R} prefix is present, a character
following a backslash is included in the string without change, and \emph{all
backslashes are left in the string}.  For example, the string literal
\code{r"\e n"} consists of two characters: a backslash and a lowercase
\character{n}.  String quotes can be escaped with a backslash, but the
backslash remains in the string; for example, \code{r"\e""} is a valid string
literal consisting of two characters: a backslash and a double quote;
\code{r"\e"} is not a valid string literal (even a raw string cannot
end in an odd number of backslashes).  Specifically, \emph{a raw
string cannot end in a single backslash} (since the backslash would
escape the following quote character).  Note also that a single
backslash followed by a newline is interpreted as those two characters
as part of the string, \emph{not} as a line continuation.

When an \character{r} or \character{R} prefix is used in conjunction
with a \character{u} or \character{U} prefix, then the \code{\e uXXXX}
and \code{\e UXXXXXXXX} escape sequences are processed while 
\emph{all other backslashes are left in the string}.
For example, the string literal
\code{ur"\e{}u0062\e n"} consists of three Unicode characters: `LATIN
SMALL LETTER B', `REVERSE SOLIDUS', and `LATIN SMALL LETTER N'.
Backslashes can be escaped with a preceding backslash; however, both
remain in the string.  As a result, \code{\e uXXXX} escape sequences
are only recognized when there are an odd number of backslashes.

\subsection{String literal concatenation\label{string-catenation}}

Multiple adjacent string literals (delimited by whitespace), possibly
using different quoting conventions, are allowed, and their meaning is
the same as their concatenation.  Thus, \code{"hello" 'world'} is
equivalent to \code{"helloworld"}.  This feature can be used to reduce
the number of backslashes needed, to split long strings conveniently
across long lines, or even to add comments to parts of strings, for
example:

\begin{verbatim}
re.compile("[A-Za-z_]"       # letter or underscore
           "[A-Za-z0-9_]*"   # letter, digit or underscore
          )
\end{verbatim}

Note that this feature is defined at the syntactical level, but
implemented at compile time.  The `+' operator must be used to
concatenate string expressions at run time.  Also note that literal
concatenation can use different quoting styles for each component
(even mixing raw strings and triple quoted strings).


\subsection{Numeric literals\label{numbers}}

There are four types of numeric literals: plain integers, long
integers, floating point numbers, and imaginary numbers.  There are no
complex literals (complex numbers can be formed by adding a real
number and an imaginary number).
\index{number}
\index{numeric literal}
\index{integer literal}
\index{plain integer literal}
\index{long integer literal}
\index{floating point literal}
\index{hexadecimal literal}
\index{octal literal}
\index{decimal literal}
\index{imaginary literal}
\index{complex!literal}

Note that numeric literals do not include a sign; a phrase like
\code{-1} is actually an expression composed of the unary operator
`\code{-}' and the literal \code{1}.


\subsection{Integer and long integer literals\label{integers}}

Integer and long integer literals are described by the following
lexical definitions:

\begin{productionlist}
  \production{longinteger}
             {\token{integer} ("l" | "L")}
  \production{integer}
             {\token{decimalinteger} | \token{octinteger} | \token{hexinteger}}
  \production{decimalinteger}
             {\token{nonzerodigit} \token{digit}* | "0"}
  \production{octinteger}
             {"0" \token{octdigit}+}
  \production{hexinteger}
             {"0" ("x" | "X") \token{hexdigit}+}
  \production{nonzerodigit}
             {"1"..."9"}
  \production{octdigit}
             {"0"..."7"}
  \production{hexdigit}
             {\token{digit} | "a"..."f" | "A"..."F"}
\end{productionlist}

Although both lower case \character{l} and upper case \character{L} are
allowed as suffix for long integers, it is strongly recommended to always
use \character{L}, since the letter \character{l} looks too much like the
digit \character{1}.

Plain integer literals that are above the largest representable plain
integer (e.g., 2147483647 when using 32-bit arithmetic) are accepted
as if they were long integers instead.\footnote{In versions of Python
prior to 2.4, octal and hexadecimal literals in the range just above
the largest representable plain integer but below the largest unsigned
32-bit number (on a machine using 32-bit arithmetic), 4294967296, were
taken as the negative plain integer obtained by subtracting 4294967296
from their unsigned value.}  There is no limit for long integer
literals apart from what can be stored in available memory.

Some examples of plain integer literals (first row) and long integer
literals (second and third rows):

\begin{verbatim}
7     2147483647                        0177
3L    79228162514264337593543950336L    0377L   0x100000000L
      79228162514264337593543950336             0xdeadbeef						    
\end{verbatim}


\subsection{Floating point literals\label{floating}}

Floating point literals are described by the following lexical
definitions:

\begin{productionlist}
  \production{floatnumber}
             {\token{pointfloat} | \token{exponentfloat}}
  \production{pointfloat}
             {[\token{intpart}] \token{fraction} | \token{intpart} "."}
  \production{exponentfloat}
             {(\token{intpart} | \token{pointfloat})
              \token{exponent}}
  \production{intpart}
             {\token{digit}+}
  \production{fraction}
             {"." \token{digit}+}
  \production{exponent}
             {("e" | "E") ["+" | "-"] \token{digit}+}
\end{productionlist}

Note that the integer and exponent parts of floating point numbers
can look like octal integers, but are interpreted using radix 10.  For
example, \samp{077e010} is legal, and denotes the same number
as \samp{77e10}.
The allowed range of floating point literals is
implementation-dependent.
Some examples of floating point literals:

\begin{verbatim}
3.14    10.    .001    1e100    3.14e-10    0e0
\end{verbatim}

Note that numeric literals do not include a sign; a phrase like
\code{-1} is actually an expression composed of the unary operator
\code{-} and the literal \code{1}.


\subsection{Imaginary literals\label{imaginary}}

Imaginary literals are described by the following lexical definitions:

\begin{productionlist}
  \production{imagnumber}{(\token{floatnumber} | \token{intpart}) ("j" | "J")}
\end{productionlist}

An imaginary literal yields a complex number with a real part of
0.0.  Complex numbers are represented as a pair of floating point
numbers and have the same restrictions on their range.  To create a
complex number with a nonzero real part, add a floating point number
to it, e.g., \code{(3+4j)}.  Some examples of imaginary literals:

\begin{verbatim}
3.14j   10.j    10j     .001j   1e100j  3.14e-10j 
\end{verbatim}


\section{Operators\label{operators}}

The following tokens are operators:
\index{operators}

\begin{verbatim}
+       -       *       **      /       //      %
<<      >>      &       |       ^       ~
<       >       <=      >=      ==      !=      <>
\end{verbatim}

The comparison operators \code{<>} and \code{!=} are alternate
spellings of the same operator.  \code{!=} is the preferred spelling;
\code{<>} is obsolescent.


\section{Delimiters\label{delimiters}}

The following tokens serve as delimiters in the grammar:
\index{delimiters}

\begin{verbatim}
(       )       [       ]       {       }      @
,       :       .       `       =       ;
+=      -=      *=      /=      //=     %=
&=      |=      ^=      >>=     <<=     **=
\end{verbatim}

The period can also occur in floating-point and imaginary literals.  A
sequence of three periods has a special meaning as an ellipsis in slices.
The second half of the list, the augmented assignment operators, serve
lexically as delimiters, but also perform an operation.

The following printing \ASCII{} characters have special meaning as part
of other tokens or are otherwise significant to the lexical analyzer:

\begin{verbatim}
'       "       #       \
\end{verbatim}

The following printing \ASCII{} characters are not used in Python.  Their
occurrence outside string literals and comments is an unconditional
error:
\index{ASCII@\ASCII}

\begin{verbatim}
$       ?
\end{verbatim}
		% Lexical analysis
\chapter{Data model\label{datamodel}}


\section{Objects, values and types\label{objects}}

\dfn{Objects} are Python's abstraction for data.  All data in a Python
program is represented by objects or by relations between objects.
(In a sense, and in conformance to Von Neumann's model of a
``stored program computer,'' code is also represented by objects.)
\index{object}
\index{data}

Every object has an identity, a type and a value.  An object's
\emph{identity} never changes once it has been created; you may think
of it as the object's address in memory.  The `\keyword{is}' operator
compares the identity of two objects; the
\function{id()}\bifuncindex{id} function returns an integer
representing its identity (currently implemented as its address).
An object's \dfn{type} is
also unchangeable.\footnote{Since Python 2.2, a gradual merging of
types and classes has been started that makes this and a few other
assertions made in this manual not 100\% accurate and complete:
for example, it \emph{is} now possible in some cases to change an
object's type, under certain controlled conditions.  Until this manual
undergoes extensive revision, it must now be taken as authoritative
only regarding ``classic classes'', that are still the default, for
compatibility purposes, in Python 2.2 and 2.3.  For more information,
see \url{http://www.python.org/doc/newstyle/}.}
An object's type determines the operations that the object
supports (e.g., ``does it have a length?'') and also defines the
possible values for objects of that type.  The
\function{type()}\bifuncindex{type} function returns an object's type
(which is an object itself).  The \emph{value} of some
objects can change.  Objects whose value can change are said to be
\emph{mutable}; objects whose value is unchangeable once they are
created are called \emph{immutable}.
(The value of an immutable container object that contains a reference
to a mutable object can change when the latter's value is changed;
however the container is still considered immutable, because the
collection of objects it contains cannot be changed.  So, immutability
is not strictly the same as having an unchangeable value, it is more
subtle.)
An object's mutability is determined by its type; for instance,
numbers, strings and tuples are immutable, while dictionaries and
lists are mutable.
\index{identity of an object}
\index{value of an object}
\index{type of an object}
\index{mutable object}
\index{immutable object}

Objects are never explicitly destroyed; however, when they become
unreachable they may be garbage-collected.  An implementation is
allowed to postpone garbage collection or omit it altogether --- it is
a matter of implementation quality how garbage collection is
implemented, as long as no objects are collected that are still
reachable.  (Implementation note: the current implementation uses a
reference-counting scheme with (optional) delayed detection of
cyclically linked garbage, which collects most objects as soon as they
become unreachable, but is not guaranteed to collect garbage
containing circular references.  See the
\citetitle[../lib/module-gc.html]{Python Library Reference} for
information on controlling the collection of cyclic garbage.)
\index{garbage collection}
\index{reference counting}
\index{unreachable object}

Note that the use of the implementation's tracing or debugging
facilities may keep objects alive that would normally be collectable.
Also note that catching an exception with a
`\keyword{try}...\keyword{except}' statement may keep objects alive.

Some objects contain references to ``external'' resources such as open
files or windows.  It is understood that these resources are freed
when the object is garbage-collected, but since garbage collection is
not guaranteed to happen, such objects also provide an explicit way to
release the external resource, usually a \method{close()} method.
Programs are strongly recommended to explicitly close such
objects.  The `\keyword{try}...\keyword{finally}' statement provides
a convenient way to do this.

Some objects contain references to other objects; these are called
\emph{containers}.  Examples of containers are tuples, lists and
dictionaries.  The references are part of a container's value.  In
most cases, when we talk about the value of a container, we imply the
values, not the identities of the contained objects; however, when we
talk about the mutability of a container, only the identities of
the immediately contained objects are implied.  So, if an immutable
container (like a tuple)
contains a reference to a mutable object, its value changes
if that mutable object is changed.
\index{container}

Types affect almost all aspects of object behavior.  Even the importance
of object identity is affected in some sense: for immutable types,
operations that compute new values may actually return a reference to
any existing object with the same type and value, while for mutable
objects this is not allowed.  E.g., after
\samp{a = 1; b = 1},
\code{a} and \code{b} may or may not refer to the same object with the
value one, depending on the implementation, but after
\samp{c = []; d = []}, \code{c} and \code{d}
are guaranteed to refer to two different, unique, newly created empty
lists.
(Note that \samp{c = d = []} assigns the same object to both
\code{c} and \code{d}.)


\section{The standard type hierarchy\label{types}}

Below is a list of the types that are built into Python.  Extension
modules (written in C, Java, or other languages, depending on
the implementation) can define additional types.  Future versions of
Python may add types to the type hierarchy (e.g., rational
numbers, efficiently stored arrays of integers, etc.).
\index{type}
\indexii{data}{type}
\indexii{type}{hierarchy}
\indexii{extension}{module}
\indexii{C}{language}

Some of the type descriptions below contain a paragraph listing
`special attributes.'  These are attributes that provide access to the
implementation and are not intended for general use.  Their definition
may change in the future.
\index{attribute}
\indexii{special}{attribute}
\indexiii{generic}{special}{attribute}

\begin{description}

\item[None]
This type has a single value.  There is a single object with this value.
This object is accessed through the built-in name \code{None}.
It is used to signify the absence of a value in many situations, e.g.,
it is returned from functions that don't explicitly return anything.
Its truth value is false.
\obindex{None}

\item[NotImplemented]
This type has a single value.  There is a single object with this value.
This object is accessed through the built-in name \code{NotImplemented}.
Numeric methods and rich comparison methods may return this value if
they do not implement the operation for the operands provided.  (The
interpreter will then try the reflected operation, or some other
fallback, depending on the operator.)  Its truth value is true.
\obindex{NotImplemented}

\item[Ellipsis]
This type has a single value.  There is a single object with this value.
This object is accessed through the built-in name \code{Ellipsis}.
It is used to indicate the presence of the \samp{...} syntax in a
slice.  Its truth value is true.
\obindex{Ellipsis}

\item[Numbers]
These are created by numeric literals and returned as results by
arithmetic operators and arithmetic built-in functions.  Numeric
objects are immutable; once created their value never changes.  Python
numbers are of course strongly related to mathematical numbers, but
subject to the limitations of numerical representation in computers.
\obindex{numeric}

Python distinguishes between integers, floating point numbers, and
complex numbers:

\begin{description}
\item[Integers]
These represent elements from the mathematical set of integers
(positive and negative).
\obindex{integer}

There are three types of integers:

\begin{description}

\item[Plain integers]
These represent numbers in the range -2147483648 through 2147483647.
(The range may be larger on machines with a larger natural word
size, but not smaller.)
When the result of an operation would fall outside this range, the
result is normally returned as a long integer (in some cases, the
exception \exception{OverflowError} is raised instead).
For the purpose of shift and mask operations, integers are assumed to
have a binary, 2's complement notation using 32 or more bits, and
hiding no bits from the user (i.e., all 4294967296 different bit
patterns correspond to different values).
\obindex{plain integer}
\withsubitem{(built-in exception)}{\ttindex{OverflowError}}

\item[Long integers]
These represent numbers in an unlimited range, subject to available
(virtual) memory only.  For the purpose of shift and mask operations,
a binary representation is assumed, and negative numbers are
represented in a variant of 2's complement which gives the illusion of
an infinite string of sign bits extending to the left.
\obindex{long integer}

\item[Booleans]
These represent the truth values False and True.  The two objects
representing the values False and True are the only Boolean objects.
The Boolean type is a subtype of plain integers, and Boolean values
behave like the values 0 and 1, respectively, in almost all contexts,
the exception being that when converted to a string, the strings
\code{"False"} or \code{"True"} are returned, respectively.
\obindex{Boolean}
\ttindex{False}
\ttindex{True}

\end{description} % Integers

The rules for integer representation are intended to give the most
meaningful interpretation of shift and mask operations involving
negative integers and the least surprises when switching between the
plain and long integer domains.  Any operation except left shift,
if it yields a result in the plain integer domain without causing
overflow, will yield the same result in the long integer domain or
when using mixed operands.
\indexii{integer}{representation}

\item[Floating point numbers]
These represent machine-level double precision floating point numbers.  
You are at the mercy of the underlying machine architecture (and
C or Java implementation) for the accepted range and handling of overflow.
Python does not support single-precision floating point numbers; the
savings in processor and memory usage that are usually the reason for using
these is dwarfed by the overhead of using objects in Python, so there
is no reason to complicate the language with two kinds of floating
point numbers.
\obindex{floating point}
\indexii{floating point}{number}
\indexii{C}{language}
\indexii{Java}{language}

\item[Complex numbers]
These represent complex numbers as a pair of machine-level double
precision floating point numbers.  The same caveats apply as for
floating point numbers.  The real and imaginary parts of a complex
number \code{z} can be retrieved through the read-only attributes
\code{z.real} and \code{z.imag}.
\obindex{complex}
\indexii{complex}{number}

\end{description} % Numbers


\item[Sequences]
These represent finite ordered sets indexed by non-negative numbers.
The built-in function \function{len()}\bifuncindex{len} returns the
number of items of a sequence.
When the length of a sequence is \var{n}, the
index set contains the numbers 0, 1, \ldots, \var{n}-1.  Item
\var{i} of sequence \var{a} is selected by \code{\var{a}[\var{i}]}.
\obindex{sequence}
\index{index operation}
\index{item selection}
\index{subscription}

Sequences also support slicing: \code{\var{a}[\var{i}:\var{j}]}
selects all items with index \var{k} such that \var{i} \code{<=}
\var{k} \code{<} \var{j}.  When used as an expression, a slice is a
sequence of the same type.  This implies that the index set is
renumbered so that it starts at 0.
\index{slicing}

Some sequences also support ``extended slicing'' with a third ``step''
parameter: \code{\var{a}[\var{i}:\var{j}:\var{k}]} selects all items
of \var{a} with index \var{x} where \code{\var{x} = \var{i} +
\var{n}*\var{k}}, \var{n} \code{>=} \code{0} and \var{i} \code{<=}
\var{x} \code{<} \var{j}.
\index{extended slicing}

Sequences are distinguished according to their mutability:

\begin{description}

\item[Immutable sequences]
An object of an immutable sequence type cannot change once it is
created.  (If the object contains references to other objects,
these other objects may be mutable and may be changed; however,
the collection of objects directly referenced by an immutable object
cannot change.)
\obindex{immutable sequence}
\obindex{immutable}

The following types are immutable sequences:

\begin{description}

\item[Strings]
The items of a string are characters.  There is no separate
character type; a character is represented by a string of one item.
Characters represent (at least) 8-bit bytes.  The built-in
functions \function{chr()}\bifuncindex{chr} and
\function{ord()}\bifuncindex{ord} convert between characters and
nonnegative integers representing the byte values.  Bytes with the
values 0-127 usually represent the corresponding \ASCII{} values, but
the interpretation of values is up to the program.  The string
data type is also used to represent arrays of bytes, e.g., to hold data
read from a file.
\obindex{string}
\index{character}
\index{byte}
\index{ASCII@\ASCII}

(On systems whose native character set is not \ASCII, strings may use
EBCDIC in their internal representation, provided the functions
\function{chr()} and \function{ord()} implement a mapping between \ASCII{} and
EBCDIC, and string comparison preserves the \ASCII{} order.
Or perhaps someone can propose a better rule?)
\index{ASCII@\ASCII}
\index{EBCDIC}
\index{character set}
\indexii{string}{comparison}
\bifuncindex{chr}
\bifuncindex{ord}

\item[Unicode]
The items of a Unicode object are Unicode code units.  A Unicode code
unit is represented by a Unicode object of one item and can hold
either a 16-bit or 32-bit value representing a Unicode ordinal (the
maximum value for the ordinal is given in \code{sys.maxunicode}, and
depends on how Python is configured at compile time).  Surrogate pairs
may be present in the Unicode object, and will be reported as two
separate items.  The built-in functions
\function{unichr()}\bifuncindex{unichr} and
\function{ord()}\bifuncindex{ord} convert between code units and
nonnegative integers representing the Unicode ordinals as defined in
the Unicode Standard 3.0. Conversion from and to other encodings are
possible through the Unicode method \method{encode()} and the built-in
function \function{unicode()}.\bifuncindex{unicode}
\obindex{unicode}
\index{character}
\index{integer}
\index{Unicode}

\item[Tuples]
The items of a tuple are arbitrary Python objects.
Tuples of two or more items are formed by comma-separated lists
of expressions.  A tuple of one item (a `singleton') can be formed
by affixing a comma to an expression (an expression by itself does
not create a tuple, since parentheses must be usable for grouping of
expressions).  An empty tuple can be formed by an empty pair of
parentheses.
\obindex{tuple}
\indexii{singleton}{tuple}
\indexii{empty}{tuple}

\end{description} % Immutable sequences

\item[Mutable sequences]
Mutable sequences can be changed after they are created.  The
subscription and slicing notations can be used as the target of
assignment and \keyword{del} (delete) statements.
\obindex{mutable sequence}
\obindex{mutable}
\indexii{assignment}{statement}
\index{delete}
\stindex{del}
\index{subscription}
\index{slicing}

There is currently a single intrinsic mutable sequence type:

\begin{description}

\item[Lists]
The items of a list are arbitrary Python objects.  Lists are formed
by placing a comma-separated list of expressions in square brackets.
(Note that there are no special cases needed to form lists of length 0
or 1.)
\obindex{list}

\end{description} % Mutable sequences

The extension module \module{array}\refstmodindex{array} provides an
additional example of a mutable sequence type.


\end{description} % Sequences


\item[Set types]
These represent unordered, finite sets of unique, immutable objects.
As such, they cannot be indexed by any subscript. However, they can be
iterated over, and the built-in function \function{len()} returns the
number of items in a set. Common uses for sets are
fast membership testing, removing duplicates from a sequence, and
computing mathematical operations such as intersection, union, difference,
and symmetric difference.
\bifuncindex{len}
\obindex{set type}

For set elements, the same immutability rules apply as for dictionary
keys. Note that numeric types obey the normal rules for numeric
comparison: if two numbers compare equal (e.g., \code{1} and
\code{1.0}), only one of them can be contained in a set.

There are currently two intrinsic set types:

\begin{description}

\item[Sets]
These\obindex{set} represent a mutable set. They are created by the
built-in \function{set()} constructor and can be modified afterwards
by several methods, such as \method{add()}.

\item[Frozen sets]
These\obindex{frozenset} represent an immutable set. They are created by
the built-in \function{frozenset()} constructor. As a frozenset is
immutable and hashable, it can be used again as an element of another set,
or as a dictionary key.

\end{description} % Set types


\item[Mappings]
These represent finite sets of objects indexed by arbitrary index sets.
The subscript notation \code{a[k]} selects the item indexed
by \code{k} from the mapping \code{a}; this can be used in
expressions and as the target of assignments or \keyword{del} statements.
The built-in function \function{len()} returns the number of items
in a mapping.
\bifuncindex{len}
\index{subscription}
\obindex{mapping}

There is currently a single intrinsic mapping type:

\begin{description}

\item[Dictionaries]
These\obindex{dictionary} represent finite sets of objects indexed by
nearly arbitrary values.  The only types of values not acceptable as
keys are values containing lists or dictionaries or other mutable
types that are compared by value rather than by object identity, the
reason being that the efficient implementation of dictionaries
requires a key's hash value to remain constant.
Numeric types used for keys obey the normal rules for numeric
comparison: if two numbers compare equal (e.g., \code{1} and
\code{1.0}) then they can be used interchangeably to index the same
dictionary entry.

Dictionaries are mutable; they can be created by the
\code{\{...\}} notation (see section~\ref{dict}, ``Dictionary
Displays'').

The extension modules \module{dbm}\refstmodindex{dbm},
\module{gdbm}\refstmodindex{gdbm}, and
\module{bsddb}\refstmodindex{bsddb} provide additional examples of
mapping types.

\end{description} % Mapping types

\item[Callable types]
These\obindex{callable} are the types to which the function call
operation (see section~\ref{calls}, ``Calls'') can be applied:
\indexii{function}{call}
\index{invocation}
\indexii{function}{argument}

\begin{description}

\item[User-defined functions]
A user-defined function object is created by a function definition
(see section~\ref{function}, ``Function definitions'').  It should be
called with an argument
list containing the same number of items as the function's formal
parameter list.
\indexii{user-defined}{function}
\obindex{function}
\obindex{user-defined function}

Special attributes: 

\begin{tableiii}{lll}{member}{Attribute}{Meaning}{}
  \lineiii{func_doc}{The function's documentation string, or
    \code{None} if unavailable}{Writable}

  \lineiii{__doc__}{Another way of spelling
    \member{func_doc}}{Writable}

  \lineiii{func_name}{The function's name}{Writable}

  \lineiii{__name__}{Another way of spelling
    \member{func_name}}{Writable}

  \lineiii{__module__}{The name of the module the function was defined
    in, or \code{None} if unavailable.}{Writable}

  \lineiii{func_defaults}{A tuple containing default argument values
    for those arguments that have defaults, or \code{None} if no
    arguments have a default value}{Writable}

  \lineiii{func_code}{The code object representing the compiled
    function body.}{Writable}

  \lineiii{func_globals}{A reference to the dictionary that holds the
    function's global variables --- the global namespace of the module
    in which the function was defined.}{Read-only}

  \lineiii{func_dict}{The namespace supporting arbitrary function
    attributes.}{Writable}

  \lineiii{func_closure}{\code{None} or a tuple of cells that contain
    bindings for the function's free variables.}{Read-only}
\end{tableiii}

Most of the attributes labelled ``Writable'' check the type of the
assigned value.

\versionchanged[\code{func_name} is now writable]{2.4}

Function objects also support getting and setting arbitrary
attributes, which can be used, for example, to attach metadata to
functions.  Regular attribute dot-notation is used to get and set such
attributes. \emph{Note that the current implementation only supports
function attributes on user-defined functions.  Function attributes on
built-in functions may be supported in the future.}

Additional information about a function's definition can be retrieved
from its code object; see the description of internal types below.

\withsubitem{(function attribute)}{
  \ttindex{func_doc}
  \ttindex{__doc__}
  \ttindex{__name__}
  \ttindex{__module__}
  \ttindex{__dict__}
  \ttindex{func_defaults}
  \ttindex{func_closure}
  \ttindex{func_code}
  \ttindex{func_globals}
  \ttindex{func_dict}}
\indexii{global}{namespace}

\item[User-defined methods]
A user-defined method object combines a class, a class instance (or
\code{None}) and any callable object (normally a user-defined
function).
\obindex{method}
\obindex{user-defined method}
\indexii{user-defined}{method}

Special read-only attributes: \member{im_self} is the class instance
object, \member{im_func} is the function object;
\member{im_class} is the class of \member{im_self} for bound methods
or the class that asked for the method for unbound methods;
\member{__doc__} is the method's documentation (same as
\code{im_func.__doc__}); \member{__name__} is the method name (same as
\code{im_func.__name__}); \member{__module__} is the name of the
module the method was defined in, or \code{None} if unavailable.
\versionchanged[\member{im_self} used to refer to the class that
                defined the method]{2.2}
\withsubitem{(method attribute)}{
  \ttindex{__doc__}
  \ttindex{__name__}
  \ttindex{__module__}
  \ttindex{im_func}
  \ttindex{im_self}}

Methods also support accessing (but not setting) the arbitrary
function attributes on the underlying function object.

User-defined method objects may be created when getting an attribute
of a class (perhaps via an instance of that class), if that attribute
is a user-defined function object, an unbound user-defined method object,
or a class method object.
When the attribute is a user-defined method object, a new
method object is only created if the class from which it is being
retrieved is the same as, or a derived class of, the class stored
in the original method object; otherwise, the original method object
is used as it is.

When a user-defined method object is created by retrieving
a user-defined function object from a class, its \member{im_self}
attribute is \code{None} and the method object is said to be unbound.
When one is created by retrieving a user-defined function object
from a class via one of its instances, its \member{im_self} attribute
is the instance, and the method object is said to be bound.
In either case, the new method's \member{im_class} attribute
is the class from which the retrieval takes place, and
its \member{im_func} attribute is the original function object.
\withsubitem{(method attribute)}{
  \ttindex{im_class}\ttindex{im_func}\ttindex{im_self}}

When a user-defined method object is created by retrieving another
method object from a class or instance, the behaviour is the same
as for a function object, except that the \member{im_func} attribute
of the new instance is not the original method object but its
\member{im_func} attribute.
\withsubitem{(method attribute)}{
  \ttindex{im_func}}

When a user-defined method object is created by retrieving a
class method object from a class or instance, its \member{im_self}
attribute is the class itself (the same as the \member{im_class}
attribute), and its \member{im_func} attribute is the function
object underlying the class method.
\withsubitem{(method attribute)}{
  \ttindex{im_class}\ttindex{im_func}\ttindex{im_self}}

When an unbound user-defined method object is called, the underlying
function (\member{im_func}) is called, with the restriction that the
first argument must be an instance of the proper class
(\member{im_class}) or of a derived class thereof.

When a bound user-defined method object is called, the underlying
function (\member{im_func}) is called, inserting the class instance
(\member{im_self}) in front of the argument list.  For instance, when
\class{C} is a class which contains a definition for a function
\method{f()}, and \code{x} is an instance of \class{C}, calling
\code{x.f(1)} is equivalent to calling \code{C.f(x, 1)}.

When a user-defined method object is derived from a class method object,
the ``class instance'' stored in \member{im_self} will actually be the
class itself, so that calling either \code{x.f(1)} or \code{C.f(1)} is
equivalent to calling \code{f(C,1)} where \code{f} is the underlying
function.

Note that the transformation from function object to (unbound or
bound) method object happens each time the attribute is retrieved from
the class or instance.  In some cases, a fruitful optimization is to
assign the attribute to a local variable and call that local variable.
Also notice that this transformation only happens for user-defined
functions; other callable objects (and all non-callable objects) are
retrieved without transformation.  It is also important to note that
user-defined functions which are attributes of a class instance are
not converted to bound methods; this \emph{only} happens when the
function is an attribute of the class.

\item[Generator functions\index{generator!function}\index{generator!iterator}]
A function or method which uses the \keyword{yield} statement (see
section~\ref{yield}, ``The \keyword{yield} statement'') is called a
\dfn{generator function}.  Such a function, when called, always
returns an iterator object which can be used to execute the body of
the function:  calling the iterator's \method{next()} method will
cause the function to execute until it provides a value using the
\keyword{yield} statement.  When the function executes a
\keyword{return} statement or falls off the end, a
\exception{StopIteration} exception is raised and the iterator will
have reached the end of the set of values to be returned.

\item[Built-in functions]
A built-in function object is a wrapper around a C function.  Examples
of built-in functions are \function{len()} and \function{math.sin()}
(\module{math} is a standard built-in module).
The number and type of the arguments are
determined by the C function.
Special read-only attributes: \member{__doc__} is the function's
documentation string, or \code{None} if unavailable; \member{__name__}
is the function's name; \member{__self__} is set to \code{None} (but see
the next item); \member{__module__} is the name of the module the
function was defined in or \code{None} if unavailable.
\obindex{built-in function}
\obindex{function}
\indexii{C}{language}

\item[Built-in methods]
This is really a different disguise of a built-in function, this time
containing an object passed to the C function as an implicit extra
argument.  An example of a built-in method is
\code{\var{alist}.append()}, assuming
\var{alist} is a list object.
In this case, the special read-only attribute \member{__self__} is set
to the object denoted by \var{list}.
\obindex{built-in method}
\obindex{method}
\indexii{built-in}{method}

\item[Class Types]
Class types, or ``new-style classes,'' are callable.  These objects
normally act as factories for new instances of themselves, but
variations are possible for class types that override
\method{__new__()}.  The arguments of the call are passed to
\method{__new__()} and, in the typical case, to \method{__init__()} to
initialize the new instance.

\item[Classic Classes]
Class objects are described below.  When a class object is called,
a new class instance (also described below) is created and
returned.  This implies a call to the class's \method{__init__()} method
if it has one.  Any arguments are passed on to the \method{__init__()}
method.  If there is no \method{__init__()} method, the class must be called
without arguments.
\withsubitem{(object method)}{\ttindex{__init__()}}
\obindex{class}
\obindex{class instance}
\obindex{instance}
\indexii{class object}{call}

\item[Class instances]
Class instances are described below.  Class instances are callable
only when the class has a \method{__call__()} method; \code{x(arguments)}
is a shorthand for \code{x.__call__(arguments)}.

\end{description}

\item[Modules]
Modules are imported by the \keyword{import} statement (see
section~\ref{import}, ``The \keyword{import} statement'').%
\stindex{import}\obindex{module}
A module object has a namespace implemented by a dictionary object
(this is the dictionary referenced by the func_globals attribute of
functions defined in the module).  Attribute references are translated
to lookups in this dictionary, e.g., \code{m.x} is equivalent to
\code{m.__dict__["x"]}.
A module object does not contain the code object used to
initialize the module (since it isn't needed once the initialization
is done).

Attribute assignment updates the module's namespace dictionary,
e.g., \samp{m.x = 1} is equivalent to \samp{m.__dict__["x"] = 1}.

Special read-only attribute: \member{__dict__} is the module's
namespace as a dictionary object.
\withsubitem{(module attribute)}{\ttindex{__dict__}}

Predefined (writable) attributes: \member{__name__}
is the module's name; \member{__doc__} is the
module's documentation string, or
\code{None} if unavailable; \member{__file__} is the pathname of the
file from which the module was loaded, if it was loaded from a file.
The \member{__file__} attribute is not present for C{} modules that are
statically linked into the interpreter; for extension modules loaded
dynamically from a shared library, it is the pathname of the shared
library file.
\withsubitem{(module attribute)}{
  \ttindex{__name__}
  \ttindex{__doc__}
  \ttindex{__file__}}
\indexii{module}{namespace}

\item[Classes]
Class objects are created by class definitions (see
section~\ref{class}, ``Class definitions'').
A class has a namespace implemented by a dictionary object.
Class attribute references are translated to
lookups in this dictionary,
e.g., \samp{C.x} is translated to \samp{C.__dict__["x"]}.
When the attribute name is not found
there, the attribute search continues in the base classes.  The search
is depth-first, left-to-right in the order of occurrence in the
base class list.

When a class attribute reference (for class \class{C}, say)
would yield a user-defined function object or
an unbound user-defined method object whose associated class is either
\class{C} or one of its base classes, it is transformed into an unbound
user-defined method object whose \member{im_class} attribute is~\class{C}.
When it would yield a class method object, it is transformed into
a bound user-defined method object whose \member{im_class} and
\member{im_self} attributes are both~\class{C}.  When it would yield
a static method object, it is transformed into the object wrapped
by the static method object. See section~\ref{descriptors} for another
way in which attributes retrieved from a class may differ from those
actually contained in its \member{__dict__}.
\obindex{class}
\obindex{class instance}
\obindex{instance}
\indexii{class object}{call}
\index{container}
\obindex{dictionary}
\indexii{class}{attribute}

Class attribute assignments update the class's dictionary, never the
dictionary of a base class.
\indexiii{class}{attribute}{assignment}

A class object can be called (see above) to yield a class instance (see
below).
\indexii{class object}{call}

Special attributes: \member{__name__} is the class name;
\member{__module__} is the module name in which the class was defined;
\member{__dict__} is the dictionary containing the class's namespace;
\member{__bases__} is a tuple (possibly empty or a singleton)
containing the base classes, in the order of their occurrence in the
base class list; \member{__doc__} is the class's documentation string,
or None if undefined.
\withsubitem{(class attribute)}{
  \ttindex{__name__}
  \ttindex{__module__}
  \ttindex{__dict__}
  \ttindex{__bases__}
  \ttindex{__doc__}}

\item[Class instances]
A class instance is created by calling a class object (see above).
A class instance has a namespace implemented as a dictionary which
is the first place in which
attribute references are searched.  When an attribute is not found
there, and the instance's class has an attribute by that name,
the search continues with the class attributes.  If a class attribute
is found that is a user-defined function object or an unbound
user-defined method object whose associated class is the class
(call it~\class{C}) of the instance for which the attribute reference
was initiated or one of its bases,
it is transformed into a bound user-defined method object whose
\member{im_class} attribute is~\class{C} and whose \member{im_self} attribute
is the instance. Static method and class method objects are also
transformed, as if they had been retrieved from class~\class{C};
see above under ``Classes''. See section~\ref{descriptors} for
another way in which attributes of a class retrieved via its
instances may differ from the objects actually stored in the
class's \member{__dict__}.
If no class attribute is found, and the object's class has a
\method{__getattr__()} method, that is called to satisfy the lookup.
\obindex{class instance}
\obindex{instance}
\indexii{class}{instance}
\indexii{class instance}{attribute}

Attribute assignments and deletions update the instance's dictionary,
never a class's dictionary.  If the class has a \method{__setattr__()} or
\method{__delattr__()} method, this is called instead of updating the
instance dictionary directly.
\indexiii{class instance}{attribute}{assignment}

Class instances can pretend to be numbers, sequences, or mappings if
they have methods with certain special names.  See
section~\ref{specialnames}, ``Special method names.''
\obindex{numeric}
\obindex{sequence}
\obindex{mapping}

Special attributes: \member{__dict__} is the attribute
dictionary; \member{__class__} is the instance's class.
\withsubitem{(instance attribute)}{
  \ttindex{__dict__}
  \ttindex{__class__}}

\item[Files]
A file\obindex{file} object represents an open file.  File objects are
created by the \function{open()}\bifuncindex{open} built-in function,
and also by
\withsubitem{(in module os)}{\ttindex{popen()}}\function{os.popen()},
\function{os.fdopen()}, and the
\method{makefile()}\withsubitem{(socket method)}{\ttindex{makefile()}}
method of socket objects (and perhaps by other functions or methods
provided by extension modules).  The objects
\ttindex{sys.stdin}\code{sys.stdin},
\ttindex{sys.stdout}\code{sys.stdout} and
\ttindex{sys.stderr}\code{sys.stderr} are initialized to file objects
corresponding to the interpreter's standard\index{stdio} input, output
and error streams.  See the \citetitle[../lib/lib.html]{Python Library
Reference} for complete documentation of file objects.
\withsubitem{(in module sys)}{
  \ttindex{stdin}
  \ttindex{stdout}
  \ttindex{stderr}}


\item[Internal types]
A few types used internally by the interpreter are exposed to the user.
Their definitions may change with future versions of the interpreter,
but they are mentioned here for completeness.
\index{internal type}
\index{types, internal}

\begin{description}

\item[Code objects]
Code objects represent \emph{byte-compiled} executable Python code, or 
\emph{bytecode}.
The difference between a code
object and a function object is that the function object contains an
explicit reference to the function's globals (the module in which it
was defined), while a code object contains no context; 
also the default argument values are stored in the function object,
not in the code object (because they represent values calculated at
run-time).  Unlike function objects, code objects are immutable and
contain no references (directly or indirectly) to mutable objects.
\index{bytecode}
\obindex{code}

Special read-only attributes: \member{co_name} gives the function
name; \member{co_argcount} is the number of positional arguments
(including arguments with default values); \member{co_nlocals} is the
number of local variables used by the function (including arguments);
\member{co_varnames} is a tuple containing the names of the local
variables (starting with the argument names); \member{co_cellvars} is
a tuple containing the names of local variables that are referenced by
nested functions; \member{co_freevars} is a tuple containing the names
of free variables; \member{co_code} is a string representing the
sequence of bytecode instructions;
\member{co_consts} is a tuple containing the literals used by the
bytecode; \member{co_names} is a tuple containing the names used by
the bytecode; \member{co_filename} is the filename from which the code
was compiled; \member{co_firstlineno} is the first line number of the
function; \member{co_lnotab} is a string encoding the mapping from
byte code offsets to line numbers (for details see the source code of
the interpreter); \member{co_stacksize} is the required stack size
(including local variables); \member{co_flags} is an integer encoding
a number of flags for the interpreter.

\withsubitem{(code object attribute)}{
  \ttindex{co_argcount}
  \ttindex{co_code}
  \ttindex{co_consts}
  \ttindex{co_filename}
  \ttindex{co_firstlineno}
  \ttindex{co_flags}
  \ttindex{co_lnotab}
  \ttindex{co_name}
  \ttindex{co_names}
  \ttindex{co_nlocals}
  \ttindex{co_stacksize}
  \ttindex{co_varnames}
  \ttindex{co_cellvars}
  \ttindex{co_freevars}}

The following flag bits are defined for \member{co_flags}: bit
\code{0x04} is set if the function uses the \samp{*arguments} syntax
to accept an arbitrary number of positional arguments; bit
\code{0x08} is set if the function uses the \samp{**keywords} syntax
to accept arbitrary keyword arguments; bit \code{0x20} is set if the
function is a generator.
\obindex{generator}

Future feature declarations (\samp{from __future__ import division})
also use bits in \member{co_flags} to indicate whether a code object
was compiled with a particular feature enabled: bit \code{0x2000} is
set if the function was compiled with future division enabled; bits
\code{0x10} and \code{0x1000} were used in earlier versions of Python.

Other bits in \member{co_flags} are reserved for internal use.

If\index{documentation string} a code object represents a function,
the first item in
\member{co_consts} is the documentation string of the function, or
\code{None} if undefined.

\item[Frame objects]
Frame objects represent execution frames.  They may occur in traceback
objects (see below).
\obindex{frame}

Special read-only attributes: \member{f_back} is to the previous
stack frame (towards the caller), or \code{None} if this is the bottom
stack frame; \member{f_code} is the code object being executed in this
frame; \member{f_locals} is the dictionary used to look up local
variables; \member{f_globals} is used for global variables;
\member{f_builtins} is used for built-in (intrinsic) names;
\member{f_restricted} is a flag indicating whether the function is
executing in restricted execution mode; \member{f_lasti} gives the
precise instruction (this is an index into the bytecode string of
the code object).
\withsubitem{(frame attribute)}{
  \ttindex{f_back}
  \ttindex{f_code}
  \ttindex{f_globals}
  \ttindex{f_locals}
  \ttindex{f_lasti}
  \ttindex{f_builtins}
  \ttindex{f_restricted}}

Special writable attributes: \member{f_trace}, if not \code{None}, is
a function called at the start of each source code line (this is used
by the debugger); \member{f_exc_type}, \member{f_exc_value},
\member{f_exc_traceback} represent the last exception raised in the
parent frame provided another exception was ever raised in the current
frame (in all other cases they are None); \member{f_lineno} is the
current line number of the frame --- writing to this from within a
trace function jumps to the given line (only for the bottom-most
frame).  A debugger can implement a Jump command (aka Set Next
Statement) by writing to f_lineno.
\withsubitem{(frame attribute)}{
  \ttindex{f_trace}
  \ttindex{f_exc_type}
  \ttindex{f_exc_value}
  \ttindex{f_exc_traceback}
  \ttindex{f_lineno}}

\item[Traceback objects] \label{traceback}
Traceback objects represent a stack trace of an exception.  A
traceback object is created when an exception occurs.  When the search
for an exception handler unwinds the execution stack, at each unwound
level a traceback object is inserted in front of the current
traceback.  When an exception handler is entered, the stack trace is
made available to the program.
(See section~\ref{try}, ``The \code{try} statement.'')
It is accessible as \code{sys.exc_traceback}, and also as the third
item of the tuple returned by \code{sys.exc_info()}.  The latter is
the preferred interface, since it works correctly when the program is
using multiple threads.
When the program contains no suitable handler, the stack trace is written
(nicely formatted) to the standard error stream; if the interpreter is
interactive, it is also made available to the user as
\code{sys.last_traceback}.
\obindex{traceback}
\indexii{stack}{trace}
\indexii{exception}{handler}
\indexii{execution}{stack}
\withsubitem{(in module sys)}{
  \ttindex{exc_info}
  \ttindex{exc_traceback}
  \ttindex{last_traceback}}
\ttindex{sys.exc_info}
\ttindex{sys.exc_traceback}
\ttindex{sys.last_traceback}

Special read-only attributes: \member{tb_next} is the next level in the
stack trace (towards the frame where the exception occurred), or
\code{None} if there is no next level; \member{tb_frame} points to the
execution frame of the current level; \member{tb_lineno} gives the line
number where the exception occurred; \member{tb_lasti} indicates the
precise instruction.  The line number and last instruction in the
traceback may differ from the line number of its frame object if the
exception occurred in a \keyword{try} statement with no matching
except clause or with a finally clause.
\withsubitem{(traceback attribute)}{
  \ttindex{tb_next}
  \ttindex{tb_frame}
  \ttindex{tb_lineno}
  \ttindex{tb_lasti}}
\stindex{try}

\item[Slice objects]
Slice objects are used to represent slices when \emph{extended slice
syntax} is used.  This is a slice using two colons, or multiple slices
or ellipses separated by commas, e.g., \code{a[i:j:step]}, \code{a[i:j,
k:l]}, or \code{a[..., i:j]}.  They are also created by the built-in
\function{slice()}\bifuncindex{slice} function.

Special read-only attributes: \member{start} is the lower bound;
\member{stop} is the upper bound; \member{step} is the step value; each is
\code{None} if omitted. These attributes can have any type.
\withsubitem{(slice object attribute)}{
  \ttindex{start}
  \ttindex{stop}
  \ttindex{step}}

Slice objects support one method:

\begin{methoddesc}[slice]{indices}{self, length}
This method takes a single integer argument \var{length} and computes
information about the extended slice that the slice object would
describe if applied to a sequence of \var{length} items.  It returns a
tuple of three integers; respectively these are the \var{start} and
\var{stop} indices and the \var{step} or stride length of the slice.
Missing or out-of-bounds indices are handled in a manner consistent
with regular slices.
\versionadded{2.3}
\end{methoddesc}

\item[Static method objects]
Static method objects provide a way of defeating the transformation
of function objects to method objects described above. A static method
object is a wrapper around any other object, usually a user-defined
method object. When a static method object is retrieved from a class
or a class instance, the object actually returned is the wrapped object,
which is not subject to any further transformation. Static method
objects are not themselves callable, although the objects they
wrap usually are. Static method objects are created by the built-in
\function{staticmethod()} constructor.

\item[Class method objects]
A class method object, like a static method object, is a wrapper
around another object that alters the way in which that object
is retrieved from classes and class instances. The behaviour of
class method objects upon such retrieval is described above,
under ``User-defined methods''. Class method objects are created
by the built-in \function{classmethod()} constructor.

\end{description} % Internal types

\end{description} % Types

%=========================================================================
\section{New-style and classic classes}

Classes and instances come in two flavors: old-style or classic, and new-style.  

Up to Python 2.1, old-style classes were the only flavour available to the
user.  The concept of (old-style) class is unrelated to the concept of type: if
\var{x} is an instance of an old-style class, then \code{x.__class__}
designates the class of \var{x}, but \code{type(x)} is always \code{<type
'instance'>}.  This reflects the fact that all old-style instances,
independently of their class, are implemented with a single built-in type,
called \code{instance}.

New-style classes were introduced in Python 2.2 to unify classes and types.  A
new-style class neither more nor less than a user-defined type.  If \var{x} is
an instance of a new-style class, then \code{type(x)} is the same as
\code{x.__class__}.

The major motivation for introducing new-style classes is to provide a unified
object model with a full meta-model.  It also has a number of immediate
benefits, like the ability to subclass most built-in types, or the introduction
of "descriptors", which enable computed properties.

For compatibility reasons, classes are still old-style by default.  New-style
classes are created by specifying another new-style class (i.e.\ a type) as a
parent class, or the "top-level type" \class{object} if no other parent is
needed.  The behaviour of new-style classes differs from that of old-style
classes in a number of important details in addition to what \function{type}
returns.  Some of these changes are fundamental to the new object model, like
the way special methods are invoked.  Others are "fixes" that could not be
implemented before for compatibility concerns, like the method resolution order
in case of multiple inheritance.

This manual is not up-to-date with respect to new-style classes.  For now,
please see \url{http://www.python.org/doc/newstyle/} for more information.

The plan is to eventually drop old-style classes, leaving only the semantics of
new-style classes.  This change will probably only be feasible in Python 3.0.
\index{class}{new-style}
\index{class}{classic}
\index{class}{old-style}

%=========================================================================
\section{Special method names\label{specialnames}}

A class can implement certain operations that are invoked by special
syntax (such as arithmetic operations or subscripting and slicing) by
defining methods with special names.\indexii{operator}{overloading}
This is Python's approach to \dfn{operator overloading}, allowing
classes to define their own behavior with respect to language
operators.  For instance, if a class defines
a method named \method{__getitem__()}, and \code{x} is an instance of
this class, then \code{x[i]} is equivalent\footnote{This, and other
statements, are only roughly true for instances of new-style
classes.} to
\code{x.__getitem__(i)}.  Except where mentioned, attempts to execute
an operation raise an exception when no appropriate method is defined.
\withsubitem{(mapping object method)}{\ttindex{__getitem__()}}

When implementing a class that emulates any built-in type, it is
important that the emulation only be implemented to the degree that it
makes sense for the object being modelled.  For example, some
sequences may work well with retrieval of individual elements, but
extracting a slice may not make sense.  (One example of this is the
\class{NodeList} interface in the W3C's Document Object Model.)


\subsection{Basic customization\label{customization}}

\begin{methoddesc}[object]{__new__}{cls\optional{, \moreargs}}
Called to create a new instance of class \var{cls}.  \method{__new__()}
is a static method (special-cased so you need not declare it as such)
that takes the class of which an instance was requested as its first
argument.  The remaining arguments are those passed to the object
constructor expression (the call to the class).  The return value of
\method{__new__()} should be the new object instance (usually an
instance of \var{cls}).

Typical implementations create a new instance of the class by invoking
the superclass's \method{__new__()} method using
\samp{super(\var{currentclass}, \var{cls}).__new__(\var{cls}[, ...])}
with appropriate arguments and then modifying the newly-created instance
as necessary before returning it.

If \method{__new__()} returns an instance of \var{cls}, then the new
instance's \method{__init__()} method will be invoked like
\samp{__init__(\var{self}[, ...])}, where \var{self} is the new instance
and the remaining arguments are the same as were passed to
\method{__new__()}.

If \method{__new__()} does not return an instance of \var{cls}, then the
new instance's \method{__init__()} method will not be invoked.

\method{__new__()} is intended mainly to allow subclasses of
immutable types (like int, str, or tuple) to customize instance
creation.
\end{methoddesc}

\begin{methoddesc}[object]{__init__}{self\optional{, \moreargs}}
Called\indexii{class}{constructor} when the instance is created.  The
arguments are those passed to the class constructor expression.  If a
base class has an \method{__init__()} method, the derived class's
\method{__init__()} method, if any, must explicitly call it to ensure proper
initialization of the base class part of the instance; for example:
\samp{BaseClass.__init__(\var{self}, [\var{args}...])}.  As a special
constraint on constructors, no value may be returned; doing so will
cause a \exception{TypeError} to be raised at runtime.
\end{methoddesc}


\begin{methoddesc}[object]{__del__}{self}
Called when the instance is about to be destroyed.  This is also
called a destructor\index{destructor}.  If a base class
has a \method{__del__()} method, the derived class's \method{__del__()}
method, if any,
must explicitly call it to ensure proper deletion of the base class
part of the instance.  Note that it is possible (though not recommended!)
for the \method{__del__()}
method to postpone destruction of the instance by creating a new
reference to it.  It may then be called at a later time when this new
reference is deleted.  It is not guaranteed that
\method{__del__()} methods are called for objects that still exist when
the interpreter exits.
\stindex{del}

\begin{notice}
\samp{del x} doesn't directly call
\code{x.__del__()} --- the former decrements the reference count for
\code{x} by one, and the latter is only called when \code{x}'s reference
count reaches zero.  Some common situations that may prevent the
reference count of an object from going to zero include: circular
references between objects (e.g., a doubly-linked list or a tree data
structure with parent and child pointers); a reference to the object
on the stack frame of a function that caught an exception (the
traceback stored in \code{sys.exc_traceback} keeps the stack frame
alive); or a reference to the object on the stack frame that raised an
unhandled exception in interactive mode (the traceback stored in
\code{sys.last_traceback} keeps the stack frame alive).  The first
situation can only be remedied by explicitly breaking the cycles; the
latter two situations can be resolved by storing \code{None} in
\code{sys.exc_traceback} or \code{sys.last_traceback}.  Circular
references which are garbage are detected when the option cycle
detector is enabled (it's on by default), but can only be cleaned up
if there are no Python-level \method{__del__()} methods involved.
Refer to the documentation for the \ulink{\module{gc}
module}{../lib/module-gc.html} for more information about how
\method{__del__()} methods are handled by the cycle detector,
particularly the description of the \code{garbage} value.
\end{notice}

\begin{notice}[warning]
Due to the precarious circumstances under which
\method{__del__()} methods are invoked, exceptions that occur during their
execution are ignored, and a warning is printed to \code{sys.stderr}
instead.  Also, when \method{__del__()} is invoked in response to a module
being deleted (e.g., when execution of the program is done), other
globals referenced by the \method{__del__()} method may already have been
deleted.  For this reason, \method{__del__()} methods should do the
absolute minimum needed to maintain external invariants.  Starting with
version 1.5, Python guarantees that globals whose name begins with a single
underscore are deleted from their module before other globals are deleted;
if no other references to such globals exist, this may help in assuring that
imported modules are still available at the time when the
\method{__del__()} method is called.
\end{notice}
\end{methoddesc}

\begin{methoddesc}[object]{__repr__}{self}
Called by the \function{repr()}\bifuncindex{repr} built-in function
and by string conversions (reverse quotes) to compute the ``official''
string representation of an object.  If at all possible, this should
look like a valid Python expression that could be used to recreate an
object with the same value (given an appropriate environment).  If
this is not possible, a string of the form \samp{<\var{...some useful
description...}>} should be returned.  The return value must be a
string object.
If a class defines \method{__repr__()} but not \method{__str__()},
then \method{__repr__()} is also used when an ``informal'' string
representation of instances of that class is required.		     

This is typically used for debugging, so it is important that the
representation is information-rich and unambiguous.
\indexii{string}{conversion}
\indexii{reverse}{quotes}
\indexii{backward}{quotes}
\index{back-quotes}
\end{methoddesc}

\begin{methoddesc}[object]{__str__}{self}
Called by the \function{str()}\bifuncindex{str} built-in function and
by the \keyword{print}\stindex{print} statement to compute the
``informal'' string representation of an object.  This differs from
\method{__repr__()} in that it does not have to be a valid Python
expression: a more convenient or concise representation may be used
instead.  The return value must be a string object.
\end{methoddesc}

\begin{methoddesc}[object]{__lt__}{self, other}
\methodline[object]{__le__}{self, other}
\methodline[object]{__eq__}{self, other}
\methodline[object]{__ne__}{self, other}
\methodline[object]{__gt__}{self, other}
\methodline[object]{__ge__}{self, other}
\versionadded{2.1}
These are the so-called ``rich comparison'' methods, and are called
for comparison operators in preference to \method{__cmp__()} below.
The correspondence between operator symbols and method names is as
follows:
\code{\var{x}<\var{y}} calls \code{\var{x}.__lt__(\var{y})},
\code{\var{x}<=\var{y}} calls \code{\var{x}.__le__(\var{y})},
\code{\var{x}==\var{y}} calls \code{\var{x}.__eq__(\var{y})},
\code{\var{x}!=\var{y}} and \code{\var{x}<>\var{y}} call
\code{\var{x}.__ne__(\var{y})},
\code{\var{x}>\var{y}} calls \code{\var{x}.__gt__(\var{y})}, and
\code{\var{x}>=\var{y}} calls \code{\var{x}.__ge__(\var{y})}.

A rich comparison method may return the singleton \code{NotImplemented} if it
does not implement the operation for a given pair of arguments.
By convention, \code{False} and \code{True} are returned for a successful
comparison. However, these methods can return any value, so if the
comparison operator is used in a Boolean context (e.g., in the condition
of an \code{if} statement), Python will call \function{bool()} on the
value to determine if the result is true or false.

There are no implied relationships among the comparison operators.
The truth of \code{\var{x}==\var{y}} does not imply that \code{\var{x}!=\var{y}}
is false.  Accordingly, when defining \method{__eq__()}, one should also
define \method{__ne__()} so that the operators will behave as expected.

There are no reflected (swapped-argument) versions of these methods
(to be used when the left argument does not support the operation but
the right argument does); rather, \method{__lt__()} and
\method{__gt__()} are each other's reflection, \method{__le__()} and
\method{__ge__()} are each other's reflection, and \method{__eq__()}
and \method{__ne__()} are their own reflection.

Arguments to rich comparison methods are never coerced.
\end{methoddesc}

\begin{methoddesc}[object]{__cmp__}{self, other}
Called by comparison operations if rich comparison (see above) is not
defined.  Should return a negative integer if \code{self < other},
zero if \code{self == other}, a positive integer if \code{self >
other}.  If no \method{__cmp__()}, \method{__eq__()} or
\method{__ne__()} operation is defined, class instances are compared
by object identity (``address'').  See also the description of
\method{__hash__()} for some important notes on creating objects which
support custom comparison operations and are usable as dictionary
keys.
(Note: the restriction that exceptions are not propagated by
\method{__cmp__()} has been removed since Python 1.5.)
\bifuncindex{cmp}
\index{comparisons}
\end{methoddesc}

\begin{methoddesc}[object]{__rcmp__}{self, other}
  \versionchanged[No longer supported]{2.1}
\end{methoddesc}

\begin{methoddesc}[object]{__hash__}{self}
Called for the key object for dictionary \obindex{dictionary}
operations, and by the built-in function
\function{hash()}\bifuncindex{hash}.  Should return a 32-bit integer
usable as a hash value
for dictionary operations.  The only required property is that objects
which compare equal have the same hash value; it is advised to somehow
mix together (e.g., using exclusive or) the hash values for the
components of the object that also play a part in comparison of
objects.  If a class does not define a \method{__cmp__()} method it should
not define a \method{__hash__()} operation either; if it defines
\method{__cmp__()} or \method{__eq__()} but not \method{__hash__()},
its instances will not be usable as dictionary keys.  If a class
defines mutable objects and implements a \method{__cmp__()} or
\method{__eq__()} method, it should not implement \method{__hash__()},
since the dictionary implementation requires that a key's hash value
is immutable (if the object's hash value changes, it will be in the
wrong hash bucket).

\versionchanged[\method{__hash__()} may now also return a long
integer object; the 32-bit integer is then derived from the hash
of that object]{2.5}

\withsubitem{(object method)}{\ttindex{__cmp__()}}
\end{methoddesc}

\begin{methoddesc}[object]{__nonzero__}{self}
Called to implement truth value testing, and the built-in operation
\code{bool()}; should return \code{False} or \code{True}, or their
integer equivalents \code{0} or \code{1}.
When this method is not defined, \method{__len__()} is
called, if it is defined (see below).  If a class defines neither
\method{__len__()} nor \method{__nonzero__()}, all its instances are
considered true.
\withsubitem{(mapping object method)}{\ttindex{__len__()}}
\end{methoddesc}

\begin{methoddesc}[object]{__unicode__}{self}
Called to implement \function{unicode()}\bifuncindex{unicode} builtin;
should return a Unicode object. When this method is not defined, string
conversion is attempted, and the result of string conversion is converted
to Unicode using the system default encoding.
\end{methoddesc}


\subsection{Customizing attribute access\label{attribute-access}}

The following methods can be defined to customize the meaning of
attribute access (use of, assignment to, or deletion of \code{x.name})
for class instances.

\begin{methoddesc}[object]{__getattr__}{self, name}
Called when an attribute lookup has not found the attribute in the
usual places (i.e. it is not an instance attribute nor is it found in
the class tree for \code{self}).  \code{name} is the attribute name.
This method should return the (computed) attribute value or raise an
\exception{AttributeError} exception.

Note that if the attribute is found through the normal mechanism,
\method{__getattr__()} is not called.  (This is an intentional
asymmetry between \method{__getattr__()} and \method{__setattr__()}.)
This is done both for efficiency reasons and because otherwise
\method{__setattr__()} would have no way to access other attributes of
the instance.  Note that at least for instance variables, you can fake
total control by not inserting any values in the instance attribute
dictionary (but instead inserting them in another object).  See the
\method{__getattribute__()} method below for a way to actually get
total control in new-style classes.
\withsubitem{(object method)}{\ttindex{__setattr__()}}
\end{methoddesc}

\begin{methoddesc}[object]{__setattr__}{self, name, value}
Called when an attribute assignment is attempted.  This is called
instead of the normal mechanism (i.e.\ store the value in the instance
dictionary).  \var{name} is the attribute name, \var{value} is the
value to be assigned to it.

If \method{__setattr__()} wants to assign to an instance attribute, it 
should not simply execute \samp{self.\var{name} = value} --- this
would cause a recursive call to itself.  Instead, it should insert the
value in the dictionary of instance attributes, e.g.,
\samp{self.__dict__[\var{name}] = value}.  For new-style classes,
rather than accessing the instance dictionary, it should call the base
class method with the same name, for example,
\samp{object.__setattr__(self, name, value)}.
\withsubitem{(instance attribute)}{\ttindex{__dict__}}
\end{methoddesc}

\begin{methoddesc}[object]{__delattr__}{self, name}
Like \method{__setattr__()} but for attribute deletion instead of
assignment.  This should only be implemented if \samp{del
obj.\var{name}} is meaningful for the object.
\end{methoddesc}

\subsubsection{More attribute access for new-style classes \label{new-style-attribute-access}}

The following methods only apply to new-style classes.

\begin{methoddesc}[object]{__getattribute__}{self, name}
Called unconditionally to implement attribute accesses for instances
of the class. If the class also defines \method{__getattr__()}, the latter 
will not be called unless \method{__getattribute__()} either calls it 
explicitly or raises an \exception{AttributeError}.
This method should return the (computed) attribute
value or raise an \exception{AttributeError} exception.
In order to avoid infinite recursion in this method, its
implementation should always call the base class method with the same
name to access any attributes it needs, for example,
\samp{object.__getattribute__(self, name)}.
\end{methoddesc}

\subsubsection{Implementing Descriptors \label{descriptors}}

The following methods only apply when an instance of the class
containing the method (a so-called \emph{descriptor} class) appears in
the class dictionary of another new-style class, known as the
\emph{owner} class. In the examples below, ``the attribute'' refers to
the attribute whose name is the key of the property in the owner
class' \code{__dict__}.  Descriptors can only be implemented as
new-style classes themselves.

\begin{methoddesc}[object]{__get__}{self, instance, owner}
Called to get the attribute of the owner class (class attribute access)
or of an instance of that class (instance attribute access).
\var{owner} is always the owner class, while \var{instance} is the
instance that the attribute was accessed through, or \code{None} when
the attribute is accessed through the \var{owner}.  This method should
return the (computed) attribute value or raise an
\exception{AttributeError} exception.
\end{methoddesc}

\begin{methoddesc}[object]{__set__}{self, instance, value}
Called to set the attribute on an instance \var{instance} of the owner
class to a new value, \var{value}.
\end{methoddesc}

\begin{methoddesc}[object]{__delete__}{self, instance}
Called to delete the attribute on an instance \var{instance} of the
owner class.
\end{methoddesc}


\subsubsection{Invoking Descriptors \label{descriptor-invocation}}

In general, a descriptor is an object attribute with ``binding behavior'',
one whose attribute access has been overridden by methods in the descriptor
protocol:  \method{__get__()}, \method{__set__()}, and \method{__delete__()}.
If any of those methods are defined for an object, it is said to be a
descriptor.

The default behavior for attribute access is to get, set, or delete the
attribute from an object's dictionary. For instance, \code{a.x} has a
lookup chain starting with \code{a.__dict__['x']}, then
\code{type(a).__dict__['x']}, and continuing 
through the base classes of \code{type(a)} excluding metaclasses.

However, if the looked-up value is an object defining one of the descriptor
methods, then Python may override the default behavior and invoke the
descriptor method instead.  Where this occurs in the precedence chain depends
on which descriptor methods were defined and how they were called.  Note that
descriptors are only invoked for new style objects or classes
(ones that subclass \class{object()} or \class{type()}).

The starting point for descriptor invocation is a binding, \code{a.x}.
How the arguments are assembled depends on \code{a}:

\begin{itemize}
                      
  \item[Direct Call] The simplest and least common call is when user code
    directly invokes a descriptor method:    \code{x.__get__(a)}.

  \item[Instance Binding]  If binding to a new-style object instance,
    \code{a.x} is transformed into the call:
    \code{type(a).__dict__['x'].__get__(a, type(a))}.
                     
  \item[Class Binding]  If binding to a new-style class, \code{A.x}
    is transformed into the call: \code{A.__dict__['x'].__get__(None, A)}.

  \item[Super Binding] If \code{a} is an instance of \class{super},
    then the binding \code{super(B, obj).m()} searches
    \code{obj.__class__.__mro__} for the base class \code{A} immediately
    preceding \code{B} and then invokes the descriptor with the call:
    \code{A.__dict__['m'].__get__(obj, A)}.
                     
\end{itemize}

For instance bindings, the precedence of descriptor invocation depends
on the which descriptor methods are defined.  Data descriptors define
both \method{__get__()} and \method{__set__()}.  Non-data descriptors have
just the \method{__get__()} method.  Data descriptors always override
a redefinition in an instance dictionary.  In contrast, non-data
descriptors can be overridden by instances.

Python methods (including \function{staticmethod()} and \function{classmethod()})
are implemented as non-data descriptors.  Accordingly, instances can
redefine and override methods.  This allows individual instances to acquire
behaviors that differ from other instances of the same class.                     

The \function{property()} function is implemented as a data descriptor.
Accordingly, instances cannot override the behavior of a property.


\subsubsection{__slots__\label{slots}}

By default, instances of both old and new-style classes have a dictionary
for attribute storage.  This wastes space for objects having very few instance
variables.  The space consumption can become acute when creating large numbers
of instances.

The default can be overridden by defining \var{__slots__} in a new-style class
definition.  The \var{__slots__} declaration takes a sequence of instance
variables and reserves just enough space in each instance to hold a value
for each variable.  Space is saved because \var{__dict__} is not created for
each instance.
    
\begin{datadesc}{__slots__}
This class variable can be assigned a string, iterable, or sequence of strings
with variable names used by instances.  If defined in a new-style class,
\var{__slots__} reserves space for the declared variables
and prevents the automatic creation of \var{__dict__} and \var{__weakref__}
for each instance.
\versionadded{2.2}                     
\end{datadesc}

\noindent
Notes on using \var{__slots__}

\begin{itemize}

\item Without a \var{__dict__} variable, instances cannot be assigned new
variables not listed in the \var{__slots__} definition.  Attempts to assign
to an unlisted variable name raises \exception{AttributeError}. If dynamic
assignment of new variables is desired, then add \code{'__dict__'} to the
sequence of strings in the \var{__slots__} declaration.                                     
\versionchanged[Previously, adding \code{'__dict__'} to the \var{__slots__}
declaration would not enable the assignment of new attributes not
specifically listed in the sequence of instance variable names]{2.3}                     

\item Without a \var{__weakref__} variable for each instance, classes
defining \var{__slots__} do not support weak references to its instances.
If weak reference support is needed, then add \code{'__weakref__'} to the
sequence of strings in the \var{__slots__} declaration.                    
\versionchanged[Previously, adding \code{'__weakref__'} to the \var{__slots__}
declaration would not enable support for weak references]{2.3}                                            

\item \var{__slots__} are implemented at the class level by creating
descriptors (\ref{descriptors}) for each variable name.  As a result,
class attributes cannot be used to set default values for instance
variables defined by \var{__slots__}; otherwise, the class attribute would
overwrite the descriptor assignment. 

\item If a class defines a slot also defined in a base class, the instance
variable defined by the base class slot is inaccessible (except by retrieving
its descriptor directly from the base class). This renders the meaning of the
program undefined.  In the future, a check may be added to prevent this.

\item The action of a \var{__slots__} declaration is limited to the class
where it is defined.  As a result, subclasses will have a \var{__dict__}
unless they also define  \var{__slots__}.                     

\item \var{__slots__} do not work for classes derived from ``variable-length''
built-in types such as \class{long}, \class{str} and \class{tuple}. 

\item Any non-string iterable may be assigned to \var{__slots__}.
Mappings may also be used; however, in the future, special meaning may
be assigned to the values corresponding to each key.                      

\end{itemize}


\subsection{Customizing class creation\label{metaclasses}}

By default, new-style classes are constructed using \function{type()}.
A class definition is read into a separate namespace and the value
of class name is bound to the result of \code{type(name, bases, dict)}.

When the class definition is read, if \var{__metaclass__} is defined
then the callable assigned to it will be called instead of \function{type()}.
The allows classes or functions to be written which monitor or alter the class
creation process:

\begin{itemize}
\item Modifying the class dictionary prior to the class being created.
\item Returning an instance of another class -- essentially performing
the role of a factory function.
\end{itemize}

\begin{datadesc}{__metaclass__}
This variable can be any callable accepting arguments for \code{name},
\code{bases}, and \code{dict}.  Upon class creation, the callable is
used instead of the built-in \function{type()}.
\versionadded{2.2}                     
\end{datadesc}

The appropriate metaclass is determined by the following precedence rules:

\begin{itemize}

\item If \code{dict['__metaclass__']} exists, it is used.

\item Otherwise, if there is at least one base class, its metaclass is used
(this looks for a \var{__class__} attribute first and if not found, uses its
type).

\item Otherwise, if a global variable named __metaclass__ exists, it is used.

\item Otherwise, the old-style, classic metaclass (types.ClassType) is used.

\end{itemize}      

The potential uses for metaclasses are boundless. Some ideas that have
been explored including logging, interface checking, automatic delegation,
automatic property creation, proxies, frameworks, and automatic resource
locking/synchronization.


\subsection{Emulating callable objects\label{callable-types}}

\begin{methoddesc}[object]{__call__}{self\optional{, args...}}
Called when the instance is ``called'' as a function; if this method
is defined, \code{\var{x}(arg1, arg2, ...)} is a shorthand for
\code{\var{x}.__call__(arg1, arg2, ...)}.
\indexii{call}{instance}
\end{methoddesc}


\subsection{Emulating container types\label{sequence-types}}

The following methods can be defined to implement container
objects.  Containers usually are sequences (such as lists or tuples)
or mappings (like dictionaries), but can represent other containers as
well.  The first set of methods is used either to emulate a
sequence or to emulate a mapping; the difference is that for a
sequence, the allowable keys should be the integers \var{k} for which
\code{0 <= \var{k} < \var{N}} where \var{N} is the length of the
sequence, or slice objects, which define a range of items. (For backwards
compatibility, the method \method{__getslice__()} (see below) can also be
defined to handle simple, but not extended slices.) It is also recommended
that mappings provide the methods \method{keys()}, \method{values()},
\method{items()}, \method{has_key()}, \method{get()}, \method{clear()},
\method{setdefault()}, \method{iterkeys()}, \method{itervalues()},
\method{iteritems()}, \method{pop()}, \method{popitem()},		     
\method{copy()}, and \method{update()} behaving similar to those for
Python's standard dictionary objects.  The \module{UserDict} module
provides a \class{DictMixin} class to help create those methods
from a base set of \method{__getitem__()}, \method{__setitem__()},
\method{__delitem__()}, and \method{keys()}.		     
Mutable sequences should provide
methods \method{append()}, \method{count()}, \method{index()},
\method{extend()},		     
\method{insert()}, \method{pop()}, \method{remove()}, \method{reverse()}
and \method{sort()}, like Python standard list objects.  Finally,
sequence types should implement addition (meaning concatenation) and
multiplication (meaning repetition) by defining the methods
\method{__add__()}, \method{__radd__()}, \method{__iadd__()},
\method{__mul__()}, \method{__rmul__()} and \method{__imul__()} described
below; they should not define \method{__coerce__()} or other numerical
operators.  It is recommended that both mappings and sequences
implement the \method{__contains__()} method to allow efficient use of
the \code{in} operator; for mappings, \code{in} should be equivalent
of \method{has_key()}; for sequences, it should search through the
values.  It is further recommended that both mappings and sequences
implement the \method{__iter__()} method to allow efficient iteration
through the container; for mappings, \method{__iter__()} should be
the same as \method{iterkeys()}; for sequences, it should iterate
through the values.
\withsubitem{(mapping object method)}{
  \ttindex{keys()}
  \ttindex{values()}
  \ttindex{items()}
  \ttindex{iterkeys()}
  \ttindex{itervalues()}
  \ttindex{iteritems()}    
  \ttindex{has_key()}
  \ttindex{get()}
  \ttindex{setdefault()}
  \ttindex{pop()}      
  \ttindex{popitem()}    
  \ttindex{clear()}
  \ttindex{copy()}
  \ttindex{update()}
  \ttindex{__contains__()}}
\withsubitem{(sequence object method)}{
  \ttindex{append()}
  \ttindex{count()}
  \ttindex{extend()}    
  \ttindex{index()}
  \ttindex{insert()}
  \ttindex{pop()}
  \ttindex{remove()}
  \ttindex{reverse()}
  \ttindex{sort()}
  \ttindex{__add__()}
  \ttindex{__radd__()}
  \ttindex{__iadd__()}
  \ttindex{__mul__()}
  \ttindex{__rmul__()}
  \ttindex{__imul__()}
  \ttindex{__contains__()}
  \ttindex{__iter__()}}		     
\withsubitem{(numeric object method)}{\ttindex{__coerce__()}}

\begin{methoddesc}[container object]{__len__}{self}
Called to implement the built-in function
\function{len()}\bifuncindex{len}.  Should return the length of the
object, an integer \code{>=} 0.  Also, an object that doesn't define a
\method{__nonzero__()} method and whose \method{__len__()} method
returns zero is considered to be false in a Boolean context.
\withsubitem{(object method)}{\ttindex{__nonzero__()}}
\end{methoddesc}

\begin{methoddesc}[container object]{__getitem__}{self, key}
Called to implement evaluation of \code{\var{self}[\var{key}]}.
For sequence types, the accepted keys should be integers and slice
objects.\obindex{slice}  Note that
the special interpretation of negative indexes (if the class wishes to
emulate a sequence type) is up to the \method{__getitem__()} method.
If \var{key} is of an inappropriate type, \exception{TypeError} may be
raised; if of a value outside the set of indexes for the sequence
(after any special interpretation of negative values),
\exception{IndexError} should be raised.
For mapping types, if \var{key} is missing (not in the container),
\exception{KeyError} should be raised.                     
\note{\keyword{for} loops expect that an
\exception{IndexError} will be raised for illegal indexes to allow
proper detection of the end of the sequence.}
\end{methoddesc}

\begin{methoddesc}[container object]{__setitem__}{self, key, value}
Called to implement assignment to \code{\var{self}[\var{key}]}.  Same
note as for \method{__getitem__()}.  This should only be implemented
for mappings if the objects support changes to the values for keys, or
if new keys can be added, or for sequences if elements can be
replaced.  The same exceptions should be raised for improper
\var{key} values as for the \method{__getitem__()} method.
\end{methoddesc}

\begin{methoddesc}[container object]{__delitem__}{self, key}
Called to implement deletion of \code{\var{self}[\var{key}]}.  Same
note as for \method{__getitem__()}.  This should only be implemented
for mappings if the objects support removal of keys, or for sequences
if elements can be removed from the sequence.  The same exceptions
should be raised for improper \var{key} values as for the
\method{__getitem__()} method.
\end{methoddesc}

\begin{methoddesc}[container object]{__iter__}{self}
This method is called when an iterator is required for a container.
This method should return a new iterator object that can iterate over
all the objects in the container.  For mappings, it should iterate
over the keys of the container, and should also be made available as
the method \method{iterkeys()}.

Iterator objects also need to implement this method; they are required
to return themselves.  For more information on iterator objects, see
``\ulink{Iterator Types}{../lib/typeiter.html}'' in the
\citetitle[../lib/lib.html]{Python Library Reference}.
\end{methoddesc}

The membership test operators (\keyword{in} and \keyword{not in}) are
normally implemented as an iteration through a sequence.  However,
container objects can supply the following special method with a more
efficient implementation, which also does not require the object be a
sequence.

\begin{methoddesc}[container object]{__contains__}{self, item}
Called to implement membership test operators.  Should return true if
\var{item} is in \var{self}, false otherwise.  For mapping objects,
this should consider the keys of the mapping rather than the values or
the key-item pairs.
\end{methoddesc}


\subsection{Additional methods for emulation of sequence types
  \label{sequence-methods}}

The following optional methods can be defined to further emulate sequence
objects.  Immutable sequences methods should at most only define
\method{__getslice__()}; mutable sequences might define all three
methods.

\begin{methoddesc}[sequence object]{__getslice__}{self, i, j}
\deprecated{2.0}{Support slice objects as parameters to the
\method{__getitem__()} method.}
Called to implement evaluation of \code{\var{self}[\var{i}:\var{j}]}.
The returned object should be of the same type as \var{self}.  Note
that missing \var{i} or \var{j} in the slice expression are replaced
by zero or \code{sys.maxint}, respectively.  If negative indexes are
used in the slice, the length of the sequence is added to that index.
If the instance does not implement the \method{__len__()} method, an
\exception{AttributeError} is raised.
No guarantee is made that indexes adjusted this way are not still
negative.  Indexes which are greater than the length of the sequence
are not modified.
If no \method{__getslice__()} is found, a slice
object is created instead, and passed to \method{__getitem__()} instead.
\end{methoddesc}

\begin{methoddesc}[sequence object]{__setslice__}{self, i, j, sequence}
Called to implement assignment to \code{\var{self}[\var{i}:\var{j}]}.
Same notes for \var{i} and \var{j} as for \method{__getslice__()}.

This method is deprecated. If no \method{__setslice__()} is found,
or for extended slicing of the form
\code{\var{self}[\var{i}:\var{j}:\var{k}]}, a
slice object is created, and passed to \method{__setitem__()},
instead of \method{__setslice__()} being called.
\end{methoddesc}

\begin{methoddesc}[sequence object]{__delslice__}{self, i, j}
Called to implement deletion of \code{\var{self}[\var{i}:\var{j}]}.
Same notes for \var{i} and \var{j} as for \method{__getslice__()}.
This method is deprecated. If no \method{__delslice__()} is found,
or for extended slicing of the form
\code{\var{self}[\var{i}:\var{j}:\var{k}]}, a
slice object is created, and passed to \method{__delitem__()},
instead of \method{__delslice__()} being called.
\end{methoddesc}

Notice that these methods are only invoked when a single slice with a
single colon is used, and the slice method is available.  For slice
operations involving extended slice notation, or in absence of the
slice methods, \method{__getitem__()}, \method{__setitem__()} or
\method{__delitem__()} is called with a slice object as argument.

The following example demonstrate how to make your program or module
compatible with earlier versions of Python (assuming that methods
\method{__getitem__()}, \method{__setitem__()} and \method{__delitem__()}
support slice objects as arguments):

\begin{verbatim}
class MyClass:
    ...
    def __getitem__(self, index):
        ...
    def __setitem__(self, index, value):
        ...
    def __delitem__(self, index):
        ...

    if sys.version_info < (2, 0):
        # They won't be defined if version is at least 2.0 final

        def __getslice__(self, i, j):
            return self[max(0, i):max(0, j):]
        def __setslice__(self, i, j, seq):
            self[max(0, i):max(0, j):] = seq
        def __delslice__(self, i, j):
            del self[max(0, i):max(0, j):]
    ...
\end{verbatim}

Note the calls to \function{max()}; these are necessary because of
the handling of negative indices before the
\method{__*slice__()} methods are called.  When negative indexes are
used, the \method{__*item__()} methods receive them as provided, but
the \method{__*slice__()} methods get a ``cooked'' form of the index
values.  For each negative index value, the length of the sequence is
added to the index before calling the method (which may still result
in a negative index); this is the customary handling of negative
indexes by the built-in sequence types, and the \method{__*item__()}
methods are expected to do this as well.  However, since they should
already be doing that, negative indexes cannot be passed in; they must
be constrained to the bounds of the sequence before being passed to
the \method{__*item__()} methods.
Calling \code{max(0, i)} conveniently returns the proper value.


\subsection{Emulating numeric types\label{numeric-types}}

The following methods can be defined to emulate numeric objects.
Methods corresponding to operations that are not supported by the
particular kind of number implemented (e.g., bitwise operations for
non-integral numbers) should be left undefined.

\begin{methoddesc}[numeric object]{__add__}{self, other}
\methodline[numeric object]{__sub__}{self, other}
\methodline[numeric object]{__mul__}{self, other}
\methodline[numeric object]{__floordiv__}{self, other}
\methodline[numeric object]{__mod__}{self, other}
\methodline[numeric object]{__divmod__}{self, other}
\methodline[numeric object]{__pow__}{self, other\optional{, modulo}}
\methodline[numeric object]{__lshift__}{self, other}
\methodline[numeric object]{__rshift__}{self, other}
\methodline[numeric object]{__and__}{self, other}
\methodline[numeric object]{__xor__}{self, other}
\methodline[numeric object]{__or__}{self, other}
These methods are
called to implement the binary arithmetic operations (\code{+},
\code{-}, \code{*}, \code{//}, \code{\%},
\function{divmod()}\bifuncindex{divmod},
\function{pow()}\bifuncindex{pow}, \code{**}, \code{<<},
\code{>>}, \code{\&}, \code{\^}, \code{|}).  For instance, to
evaluate the expression \var{x}\code{+}\var{y}, where \var{x} is an
instance of a class that has an \method{__add__()} method,
\code{\var{x}.__add__(\var{y})} is called.  The \method{__divmod__()}
method should be the equivalent to using \method{__floordiv__()} and
\method{__mod__()}; it should not be related to \method{__truediv__()}
(described below).  Note that
\method{__pow__()} should be defined to accept an optional third
argument if the ternary version of the built-in
\function{pow()}\bifuncindex{pow} function is to be supported.

If one of those methods does not support the operation with the
supplied arguments, it should return \code{NotImplemented}.
\end{methoddesc}

\begin{methoddesc}[numeric object]{__div__}{self, other}
\methodline[numeric object]{__truediv__}{self, other}
The division operator (\code{/}) is implemented by these methods.  The
\method{__truediv__()} method is used when \code{__future__.division}
is in effect, otherwise \method{__div__()} is used.  If only one of
these two methods is defined, the object will not support division in
the alternate context; \exception{TypeError} will be raised instead.
\end{methoddesc}

\begin{methoddesc}[numeric object]{__radd__}{self, other}
\methodline[numeric object]{__rsub__}{self, other}
\methodline[numeric object]{__rmul__}{self, other}
\methodline[numeric object]{__rdiv__}{self, other}
\methodline[numeric object]{__rtruediv__}{self, other}
\methodline[numeric object]{__rfloordiv__}{self, other}	     
\methodline[numeric object]{__rmod__}{self, other}
\methodline[numeric object]{__rdivmod__}{self, other}
\methodline[numeric object]{__rpow__}{self, other}
\methodline[numeric object]{__rlshift__}{self, other}
\methodline[numeric object]{__rrshift__}{self, other}
\methodline[numeric object]{__rand__}{self, other}
\methodline[numeric object]{__rxor__}{self, other}
\methodline[numeric object]{__ror__}{self, other}
These methods are
called to implement the binary arithmetic operations (\code{+},
\code{-}, \code{*}, \code{/}, \code{\%},
\function{divmod()}\bifuncindex{divmod},
\function{pow()}\bifuncindex{pow}, \code{**}, \code{<<},
\code{>>}, \code{\&}, \code{\^}, \code{|}) with reflected
(swapped) operands.  These functions are only called if the left
operand does not support the corresponding operation and the
operands are of different types.\footnote{
    For operands of the same type, it is assumed that if the
    non-reflected method (such as \method{__add__()}) fails the
    operation is not supported, which is why the reflected method
    is not called.} 
For instance, to evaluate the expression \var{x}\code{-}\var{y},
where \var{y} is an instance of a class that has an
\method{__rsub__()} method, \code{\var{y}.__rsub__(\var{x})}
is called if \code{\var{x}.__sub__(\var{y})} returns
\var{NotImplemented}.

Note that ternary
\function{pow()}\bifuncindex{pow} will not try calling
\method{__rpow__()} (the coercion rules would become too
complicated).

\note{If the right operand's type is a subclass of the left operand's
      type and that subclass provides the reflected method for the
      operation, this method will be called before the left operand's
      non-reflected method.  This behavior allows subclasses to
      override their ancestors' operations.}
\end{methoddesc}

\begin{methoddesc}[numeric object]{__iadd__}{self, other}
\methodline[numeric object]{__isub__}{self, other}
\methodline[numeric object]{__imul__}{self, other}
\methodline[numeric object]{__idiv__}{self, other}
\methodline[numeric object]{__itruediv__}{self, other}
\methodline[numeric object]{__ifloordiv__}{self, other}
\methodline[numeric object]{__imod__}{self, other}		     
\methodline[numeric object]{__ipow__}{self, other\optional{, modulo}}
\methodline[numeric object]{__ilshift__}{self, other}
\methodline[numeric object]{__irshift__}{self, other}
\methodline[numeric object]{__iand__}{self, other}
\methodline[numeric object]{__ixor__}{self, other}
\methodline[numeric object]{__ior__}{self, other}
These methods are called to implement the augmented arithmetic
operations (\code{+=}, \code{-=}, \code{*=}, \code{/=}, \code{//=},
\code{\%=}, \code{**=}, \code{<<=}, \code{>>=}, \code{\&=},
\code{\textasciicircum=}, \code{|=}).  These methods should attempt to do the
operation in-place (modifying \var{self}) and return the result (which
could be, but does not have to be, \var{self}).  If a specific method
is not defined, the augmented operation falls back to the normal
methods.  For instance, to evaluate the expression
\var{x}\code{+=}\var{y}, where \var{x} is an instance of a class that
has an \method{__iadd__()} method, \code{\var{x}.__iadd__(\var{y})} is
called.  If \var{x} is an instance of a class that does not define a
\method{__iadd__()} method, \code{\var{x}.__add__(\var{y})} and
\code{\var{y}.__radd__(\var{x})} are considered, as with the
evaluation of \var{x}\code{+}\var{y}.
\end{methoddesc}

\begin{methoddesc}[numeric object]{__neg__}{self}
\methodline[numeric object]{__pos__}{self}
\methodline[numeric object]{__abs__}{self}
\methodline[numeric object]{__invert__}{self}
Called to implement the unary arithmetic operations (\code{-},
\code{+}, \function{abs()}\bifuncindex{abs} and \code{\~{}}).
\end{methoddesc}

\begin{methoddesc}[numeric object]{__complex__}{self}
\methodline[numeric object]{__int__}{self}
\methodline[numeric object]{__long__}{self}
\methodline[numeric object]{__float__}{self}
Called to implement the built-in functions
\function{complex()}\bifuncindex{complex},
\function{int()}\bifuncindex{int}, \function{long()}\bifuncindex{long},
and \function{float()}\bifuncindex{float}.  Should return a value of
the appropriate type.
\end{methoddesc}

\begin{methoddesc}[numeric object]{__oct__}{self}
\methodline[numeric object]{__hex__}{self}
Called to implement the built-in functions
\function{oct()}\bifuncindex{oct} and
\function{hex()}\bifuncindex{hex}.  Should return a string value.
\end{methoddesc}

\begin{methoddesc}[numeric object]{__index__}{self}
Called to implement \function{operator.index()}.  Also called whenever
Python needs an integer object (such as in slicing).  Must return an
integer (int or long).
\versionadded{2.5}
\end{methoddesc}

\begin{methoddesc}[numeric object]{__coerce__}{self, other}
Called to implement ``mixed-mode'' numeric arithmetic.  Should either
return a 2-tuple containing \var{self} and \var{other} converted to
a common numeric type, or \code{None} if conversion is impossible.  When
the common type would be the type of \code{other}, it is sufficient to
return \code{None}, since the interpreter will also ask the other
object to attempt a coercion (but sometimes, if the implementation of
the other type cannot be changed, it is useful to do the conversion to
the other type here).  A return value of \code{NotImplemented} is
equivalent to returning \code{None}.
\end{methoddesc}

\subsection{Coercion rules\label{coercion-rules}}

This section used to document the rules for coercion.  As the language
has evolved, the coercion rules have become hard to document
precisely; documenting what one version of one particular
implementation does is undesirable.  Instead, here are some informal
guidelines regarding coercion.  In Python 3.0, coercion will not be
supported.

\begin{itemize}

\item

If the left operand of a \% operator is a string or Unicode object, no
coercion takes place and the string formatting operation is invoked
instead.

\item

It is no longer recommended to define a coercion operation.
Mixed-mode operations on types that don't define coercion pass the
original arguments to the operation.

\item

New-style classes (those derived from \class{object}) never invoke the
\method{__coerce__()} method in response to a binary operator; the only
time \method{__coerce__()} is invoked is when the built-in function
\function{coerce()} is called.

\item

For most intents and purposes, an operator that returns
\code{NotImplemented} is treated the same as one that is not
implemented at all.

\item

Below, \method{__op__()} and \method{__rop__()} are used to signify
the generic method names corresponding to an operator;
\method{__iop__()} is used for the corresponding in-place operator.  For
example, for the operator `\code{+}', \method{__add__()} and
\method{__radd__()} are used for the left and right variant of the
binary operator, and \method{__iadd__()} for the in-place variant.

\item

For objects \var{x} and \var{y}, first \code{\var{x}.__op__(\var{y})}
is tried.  If this is not implemented or returns \code{NotImplemented},
\code{\var{y}.__rop__(\var{x})} is tried.  If this is also not
implemented or returns \code{NotImplemented}, a \exception{TypeError}
exception is raised.  But see the following exception:

\item

Exception to the previous item: if the left operand is an instance of
a built-in type or a new-style class, and the right operand is an instance
of a proper subclass of that type or class and overrides the base's
\method{__rop__()} method, the right operand's \method{__rop__()} method
is tried \emph{before} the left operand's \method{__op__()} method.

This is done so that a subclass can completely override binary operators.
Otherwise, the left operand's \method{__op__()} method would always
accept the right operand: when an instance of a given class is expected,
an instance of a subclass of that class is always acceptable.

\item

When either operand type defines a coercion, this coercion is called
before that type's \method{__op__()} or \method{__rop__()} method is
called, but no sooner.  If the coercion returns an object of a
different type for the operand whose coercion is invoked, part of the
process is redone using the new object.

\item

When an in-place operator (like `\code{+=}') is used, if the left
operand implements \method{__iop__()}, it is invoked without any
coercion.  When the operation falls back to \method{__op__()} and/or
\method{__rop__()}, the normal coercion rules apply.

\item

In \var{x}\code{+}\var{y}, if \var{x} is a sequence that implements
sequence concatenation, sequence concatenation is invoked.

\item

In \var{x}\code{*}\var{y}, if one operator is a sequence that
implements sequence repetition, and the other is an integer
(\class{int} or \class{long}), sequence repetition is invoked.

\item

Rich comparisons (implemented by methods \method{__eq__()} and so on)
never use coercion.  Three-way comparison (implemented by
\method{__cmp__()}) does use coercion under the same conditions as
other binary operations use it.

\item

In the current implementation, the built-in numeric types \class{int},
\class{long} and \class{float} do not use coercion; the type
\class{complex} however does use it.  The difference can become
apparent when subclassing these types.  Over time, the type
\class{complex} may be fixed to avoid coercion.  All these types
implement a \method{__coerce__()} method, for use by the built-in
\function{coerce()} function.

\end{itemize}

\subsection{With Statement Context Managers\label{context-managers}}

\versionadded{2.5}

A \dfn{context manager} is an object that defines the runtime
context to be established when executing a \keyword{with}
statement. The context manager handles the entry into,
and the exit from, the desired runtime context for the execution
of the block of code.  Context managers are normally invoked using
the \keyword{with} statement (described in section~\ref{with}), but
can also be used by directly invoking their methods.

\stindex{with}
\index{context manager}

Typical uses of context managers include saving and
restoring various kinds of global state, locking and unlocking
resources, closing opened files, etc.

For more information on context managers, see
``\ulink{Context Types}{../lib/typecontextmanager.html}'' in the
\citetitle[../lib/lib.html]{Python Library Reference}.

\begin{methoddesc}[context manager]{__enter__}{self}
Enter the runtime context related to this object. The \keyword{with}
statement will bind this method's return value to the target(s)
specified in the \keyword{as} clause of the statement, if any.
\end{methoddesc}

\begin{methoddesc}[context manager]{__exit__}
{self, exc_type, exc_value, traceback}
Exit the runtime context related to this object. The parameters
describe the exception that caused the context to be exited. If
the context was exited without an exception, all three arguments
will be \constant{None}.

If an exception is supplied, and the method wishes to suppress the
exception (i.e., prevent it from being propagated), it should return a
true value. Otherwise, the exception will be processed normally upon
exit from this method.

Note that \method{__exit__} methods should not reraise the passed-in
exception; this is the caller's responsibility.
\end{methoddesc}

\begin{seealso}
  \seepep{0343}{The "with" statement}
         {The specification, background, and examples for the
          Python \keyword{with} statement.}
\end{seealso}

		% Data model
\chapter{Execution model \label{execmodel}}
\index{execution model}


\section{Naming and binding \label{naming}}
\indexii{code}{block}
\index{namespace}
\index{scope}

\dfn{Names}\index{name} refer to objects.  Names are introduced by
name binding operations.  Each occurrence of a name in the program
text refers to the \dfn{binding}\indexii{binding}{name} of that name
established in the innermost function block containing the use.

A \dfn{block}\index{block} is a piece of Python program text that is
executed as a unit.  The following are blocks: a module, a function
body, and a class definition.  Each command typed interactively is a
block.  A script file (a file given as standard input to the
interpreter or specified on the interpreter command line the first
argument) is a code block.  A script command (a command specified on
the interpreter command line with the `\strong{-c}' option) is a code
block.  The file read by the built-in function \function{execfile()}
is a code block.  The string argument passed to the built-in function
\function{eval()} and to the \keyword{exec} statement is a code block.
The expression read and evaluated by the built-in function
\function{input()} is a code block.

A code block is executed in an \dfn{execution
frame}\indexii{execution}{frame}.  A frame contains some
administrative information (used for debugging) and determines where
and how execution continues after the code block's execution has
completed.

A \dfn{scope}\index{scope} defines the visibility of a name within a
block.  If a local variable is defined in a block, its scope includes
that block.  If the definition occurs in a function block, the scope
extends to any blocks contained within the defining one, unless a
contained block introduces a different binding for the name.  The
scope of names defined in a class block is limited to the class block;
it does not extend to the code blocks of methods.

When a name is used in a code block, it is resolved using the nearest
enclosing scope.  The set of all such scopes visible to a code block
is called the block's \dfn{environment}\index{environment}.  

If a name is bound in a block, it is a local variable of that block.
If a name is bound at the module level, it is a global variable.  (The
variables of the module code block are local and global.)  If a
variable is used in a code block but not defined there, it is a
\dfn{free variable}\indexii{free}{variable}.

When a name is not found at all, a
\exception{NameError}\withsubitem{(built-in
exception)}{\ttindex{NameError}} exception is raised.  If the name
refers to a local variable that has not been bound, a
\exception{UnboundLocalError}\ttindex{UnboundLocalError} exception is
raised.  \exception{UnboundLocalError} is a subclass of
\exception{NameError}.

The following constructs bind names: formal parameters to functions,
\keyword{import} statements, class and function definitions (these
bind the class or function name in the defining block), and targets
that are identifiers if occurring in an assignment, \keyword{for} loop
header, or in the second position of an \keyword{except} clause
header.  The \keyword{import} statement of the form ``\samp{from
\ldots import *}''\stindex{from} binds all names defined in the
imported module, except those beginning with an underscore.  This form
may only be used at the module level.

A target occurring in a \keyword{del} statement is also considered bound
for this purpose (though the actual semantics are to unbind the
name).  It is illegal to unbind a name that is referenced by an
enclosing scope; the compiler will report a \exception{SyntaxError}.

Each assignment or import statement occurs within a block defined by a
class or function definition or at the module level (the top-level
code block).

If a name binding operation occurs anywhere within a code block, all
uses of the name within the block are treated as references to the
current block.  This can lead to errors when a name is used within a
block before it is bound.
This rule is subtle.  Python lacks declarations and allows
name binding operations to occur anywhere within a code block.  The
local variables of a code block can be determined by scanning the
entire text of the block for name binding operations.

If the global statement occurs within a block, all uses of the name
specified in the statement refer to the binding of that name in the
top-level namespace.  Names are resolved in the top-level namespace by
searching the global namespace, i.e. the namespace of the module
containing the code block, and the builtin namespace, the namespace of
the module \module{__builtin__}.  The global namespace is searched
first.  If the name is not found there, the builtin namespace is
searched.  The global statement must precede all uses of the name.

The built-in namespace associated with the execution of a code block
is actually found by looking up the name \code{__builtins__} in its
global namespace; this should be a dictionary or a module (in the
latter case the module's dictionary is used).  By default, when in the
\module{__main__} module, \code{__builtins__} is the built-in module
\module{__builtin__} (note: no `s'); when in any other module,
\code{__builtins__} is an alias for the dictionary of the
\module{__builtin__} module itself.  \code{__builtins__} can be set
to a user-created dictionary to create a weak form of restricted
execution\indexii{restricted}{execution}.

\begin{notice}
  Users should not touch \code{__builtins__}; it is strictly an
  implementation detail.  Users wanting to override values in the
  built-in namespace should \keyword{import} the \module{__builtin__}
  (no `s') module and modify its attributes appropriately.
\end{notice}

The namespace for a module is automatically created the first time a
module is imported.  The main module for a script is always called
\module{__main__}\refbimodindex{__main__}.

The global statement has the same scope as a name binding operation
in the same block.  If the nearest enclosing scope for a free variable
contains a global statement, the free variable is treated as a global.

A class definition is an executable statement that may use and define
names.  These references follow the normal rules for name resolution.
The namespace of the class definition becomes the attribute dictionary
of the class.  Names defined at the class scope are not visible in
methods. 

\subsection{Interaction with dynamic features \label{dynamic-features}}

There are several cases where Python statements are illegal when
used in conjunction with nested scopes that contain free
variables.

If a variable is referenced in an enclosing scope, it is illegal
to delete the name.  An error will be reported at compile time.

If the wild card form of import --- \samp{import *} --- is used in a
function and the function contains or is a nested block with free
variables, the compiler will raise a \exception{SyntaxError}.

If \keyword{exec} is used in a function and the function contains or
is a nested block with free variables, the compiler will raise a
\exception{SyntaxError} unless the exec explicitly specifies the local
namespace for the \keyword{exec}.  (In other words, \samp{exec obj}
would be illegal, but \samp{exec obj in ns} would be legal.)

The \function{eval()}, \function{execfile()}, and \function{input()}
functions and the \keyword{exec} statement do not have access to the
full environment for resolving names.  Names may be resolved in the
local and global namespaces of the caller.  Free variables are not
resolved in the nearest enclosing namespace, but in the global
namespace.\footnote{This limitation occurs because the code that is
    executed by these operations is not available at the time the
    module is compiled.}
The \keyword{exec} statement and the \function{eval()} and
\function{execfile()} functions have optional arguments to override
the global and local namespace.  If only one namespace is specified,
it is used for both.

\section{Exceptions \label{exceptions}}
\index{exception}

Exceptions are a means of breaking out of the normal flow of control
of a code block in order to handle errors or other exceptional
conditions.  An exception is
\emph{raised}\index{raise an exception} at the point where the error
is detected; it may be \emph{handled}\index{handle an exception} by
the surrounding code block or by any code block that directly or
indirectly invoked the code block where the error occurred.
\index{exception handler}
\index{errors}
\index{error handling}

The Python interpreter raises an exception when it detects a run-time
error (such as division by zero).  A Python program can also
explicitly raise an exception with the \keyword{raise} statement.
Exception handlers are specified with the \keyword{try} ... \keyword{except}
statement.  The \keyword{try} ... \keyword{finally} statement
specifies cleanup code which does not handle the exception, but is
executed whether an exception occurred or not in the preceding code.

Python uses the ``termination''\index{termination model} model of
error handling: an exception handler can find out what happened and
continue execution at an outer level, but it cannot repair the cause
of the error and retry the failing operation (except by re-entering
the offending piece of code from the top).

When an exception is not handled at all, the interpreter terminates
execution of the program, or returns to its interactive main loop.  In
either case, it prints a stack backtrace, except when the exception is 
\exception{SystemExit}\withsubitem{(built-in
exception)}{\ttindex{SystemExit}}.

Exceptions are identified by class instances.  The \keyword{except}
clause is selected depending on the class of the instance: it must
reference the class of the instance or a base class thereof.  The
instance can be received by the handler and can carry additional
information about the exceptional condition.

Exceptions can also be identified by strings, in which case the
\keyword{except} clause is selected by object identity.  An arbitrary
value can be raised along with the identifying string which can be
passed to the handler.

\deprecated{2.5}{String exceptions should not be used in new code.
They will not be supported in a future version of Python.  Old code
should be rewritten to use class exceptions instead.}

\begin{notice}[warning]
Messages to exceptions are not part of the Python API.  Their contents may
change from one version of Python to the next without warning and should not
be relied on by code which will run under multiple versions of the
interpreter.
\end{notice}

See also the description of the \keyword{try} statement in
section~\ref{try} and \keyword{raise} statement in
section~\ref{raise}.
		% Execution model
\chapter{Expressions\label{expressions}}
\index{expression}

This chapter explains the meaning of the elements of expressions in
Python.

\strong{Syntax Notes:} In this and the following chapters, extended
BNF\index{BNF} notation will be used to describe syntax, not lexical
analysis.  When (one alternative of) a syntax rule has the form

\begin{productionlist}[*]
  \production{name}{\token{othername}}
\end{productionlist}

and no semantics are given, the semantics of this form of \code{name}
are the same as for \code{othername}.
\index{syntax}


\section{Arithmetic conversions\label{conversions}}
\indexii{arithmetic}{conversion}

When a description of an arithmetic operator below uses the phrase
``the numeric arguments are converted to a common type,'' the
arguments are coerced using the coercion rules listed at
~\ref{coercion-rules}.  If both arguments are standard numeric types,
the following coercions are applied:

\begin{itemize}
\item	If either argument is a complex number, the other is converted
	to complex;
\item	otherwise, if either argument is a floating point number,
	the other is converted to floating point;
\item	otherwise, if either argument is a long integer,
	the other is converted to long integer;
\item	otherwise, both must be plain integers and no conversion
	is necessary.
\end{itemize}

Some additional rules apply for certain operators (e.g., a string left
argument to the `\%' operator). Extensions can define their own
coercions.


\section{Atoms\label{atoms}}
\index{atom}

Atoms are the most basic elements of expressions.  The simplest atoms
are identifiers or literals.  Forms enclosed in
reverse quotes or in parentheses, brackets or braces are also
categorized syntactically as atoms.  The syntax for atoms is:

\begin{productionlist}
  \production{atom}
             {\token{identifier} | \token{literal} | \token{enclosure}}
  \production{enclosure}
             {\token{parenth_form} | \token{list_display}}
  \productioncont{| \token{generator_expression} | \token{dict_display}}
  \productioncont{| \token{string_conversion} | \token{yield_atom}}
\end{productionlist}


\subsection{Identifiers (Names)\label{atom-identifiers}}
\index{name}
\index{identifier}

An identifier occurring as an atom is a name.  See
section \ref{identifiers} for lexical definition and
section~\ref{naming} for documentation of naming and binding.

When the name is bound to an object, evaluation of the atom yields
that object.  When a name is not bound, an attempt to evaluate it
raises a \exception{NameError} exception.
\exindex{NameError}

\strong{Private name mangling:}
\indexii{name}{mangling}%
\indexii{private}{names}%
When an identifier that textually occurs in a class definition begins
with two or more underscore characters and does not end in two or more
underscores, it is considered a \dfn{private name} of that class.
Private names are transformed to a longer form before code is
generated for them.  The transformation inserts the class name in
front of the name, with leading underscores removed, and a single
underscore inserted in front of the class name.  For example, the
identifier \code{__spam} occurring in a class named \code{Ham} will be
transformed to \code{_Ham__spam}.  This transformation is independent
of the syntactical context in which the identifier is used.  If the
transformed name is extremely long (longer than 255 characters),
implementation defined truncation may happen.  If the class name
consists only of underscores, no transformation is done.


\subsection{Literals\label{atom-literals}}
\index{literal}

Python supports string literals and various numeric literals:

\begin{productionlist}
  \production{literal}
             {\token{stringliteral} | \token{integer} | \token{longinteger}}
  \productioncont{| \token{floatnumber} | \token{imagnumber}}
\end{productionlist}

Evaluation of a literal yields an object of the given type (string,
integer, long integer, floating point number, complex number) with the
given value.  The value may be approximated in the case of floating
point and imaginary (complex) literals.  See section \ref{literals}
for details.

All literals correspond to immutable data types, and hence the
object's identity is less important than its value.  Multiple
evaluations of literals with the same value (either the same
occurrence in the program text or a different occurrence) may obtain
the same object or a different object with the same value.
\indexiii{immutable}{data}{type}
\indexii{immutable}{object}


\subsection{Parenthesized forms\label{parenthesized}}
\index{parenthesized form}

A parenthesized form is an optional expression list enclosed in
parentheses:

\begin{productionlist}
  \production{parenth_form}
             {"(" [\token{expression_list}] ")"}
\end{productionlist}

A parenthesized expression list yields whatever that expression list
yields: if the list contains at least one comma, it yields a tuple;
otherwise, it yields the single expression that makes up the
expression list.

An empty pair of parentheses yields an empty tuple object.  Since
tuples are immutable, the rules for literals apply (i.e., two
occurrences of the empty tuple may or may not yield the same object).
\indexii{empty}{tuple}

Note that tuples are not formed by the parentheses, but rather by use
of the comma operator.  The exception is the empty tuple, for which
parentheses \emph{are} required --- allowing unparenthesized ``nothing''
in expressions would cause ambiguities and allow common typos to
pass uncaught.
\index{comma}
\indexii{tuple}{display}


\subsection{List displays\label{lists}}
\indexii{list}{display}
\indexii{list}{comprehensions}

A list display is a possibly empty series of expressions enclosed in
square brackets:

\begin{productionlist}
  \production{list_display}
             {"[" [\token{expression_list} | \token{list_comprehension}] "]"}
  \production{list_comprehension}
             {\token{expression} \token{list_for}}
  \production{list_for}
             {"for" \token{target_list} "in" \token{old_expression_list}
              [\token{list_iter}]}
  \production{old_expression_list}
             {\token{old_expression}
              [("," \token{old_expression})+ [","]]}
  \production{list_iter}
             {\token{list_for} | \token{list_if}}
  \production{list_if}
             {"if" \token{old_expression} [\token{list_iter}]}
\end{productionlist}

A list display yields a new list object.  Its contents are specified
by providing either a list of expressions or a list comprehension.
\indexii{list}{comprehensions}
When a comma-separated list of expressions is supplied, its elements are
evaluated from left to right and placed into the list object in that
order.  When a list comprehension is supplied, it consists of a
single expression followed by at least one \keyword{for} clause and zero or
more \keyword{for} or \keyword{if} clauses.  In this
case, the elements of the new list are those that would be produced
by considering each of the \keyword{for} or \keyword{if} clauses a block,
nesting from
left to right, and evaluating the expression to produce a list element
each time the innermost block is reached\footnote{In Python 2.3, a
list comprehension "leaks" the control variables of each
\samp{for} it contains into the containing scope.  However, this
behavior is deprecated, and relying on it will not work once this
bug is fixed in a future release}.
\obindex{list}
\indexii{empty}{list}


\subsection{Generator expressions\label{genexpr}}
\indexii{generator}{expression}

A generator expression is a compact generator notation in parentheses:

\begin{productionlist}
  \production{generator_expression}
             {"(" \token{expression} \token{genexpr_for} ")"}
  \production{genexpr_for}
             {"for" \token{target_list} "in" \token{or_test}
              [\token{genexpr_iter}]}
  \production{genexpr_iter}
             {\token{genexpr_for} | \token{genexpr_if}}
  \production{genexpr_if}
             {"if" \token{old_expression} [\token{genexpr_iter}]}
\end{productionlist}

A generator expression yields a new generator object.
\obindex{generator}
It consists of a single expression followed by at least one
\keyword{for} clause and zero or more \keyword{for} or \keyword{if}
clauses.  The iterating values of the new generator are those that
would be produced by considering each of the \keyword{for} or
\keyword{if} clauses a block, nesting from left to right, and
evaluating the expression to yield a value that is reached the
innermost block for each iteration.

Variables used in the generator expression are evaluated lazily
when the \method{next()} method is called for generator object
(in the same fashion as normal generators). However, the leftmost
\keyword{for} clause is immediately evaluated so that error produced
by it can be seen before any other possible error in the code that
handles the generator expression.
Subsequent \keyword{for} clauses cannot be evaluated immediately since
they may depend on the previous \keyword{for} loop.
For example: \samp{(x*y for x in range(10) for y in bar(x))}.

The parentheses can be omitted on calls with only one argument.
See section \ref{calls} for the detail.


\subsection{Dictionary displays\label{dict}}
\indexii{dictionary}{display}

A dictionary display is a possibly empty series of key/datum pairs
enclosed in curly braces:
\index{key}
\index{datum}
\index{key/datum pair}

\begin{productionlist}
  \production{dict_display}
             {"\{" [\token{key_datum_list}] "\}"}
  \production{key_datum_list}
             {\token{key_datum} ("," \token{key_datum})* [","]}
  \production{key_datum}
             {\token{expression} ":" \token{expression}}
\end{productionlist}

A dictionary display yields a new dictionary object.
\obindex{dictionary}

The key/datum pairs are evaluated from left to right to define the
entries of the dictionary: each key object is used as a key into the
dictionary to store the corresponding datum.

Restrictions on the types of the key values are listed earlier in
section \ref{types}.  (To summarize, the key type should be hashable,
which excludes all mutable objects.)  Clashes between duplicate keys
are not detected; the last datum (textually rightmost in the display)
stored for a given key value prevails.
\indexii{immutable}{object}


\subsection{String conversions\label{string-conversions}}
\indexii{string}{conversion}
\indexii{reverse}{quotes}
\indexii{backward}{quotes}
\index{back-quotes}

A string conversion is an expression list enclosed in reverse (a.k.a.
backward) quotes:

\begin{productionlist}
  \production{string_conversion}
             {"`" \token{expression_list} "`"}
\end{productionlist}

A string conversion evaluates the contained expression list and
converts the resulting object into a string according to rules
specific to its type.

If the object is a string, a number, \code{None}, or a tuple, list or
dictionary containing only objects whose type is one of these, the
resulting string is a valid Python expression which can be passed to
the built-in function \function{eval()} to yield an expression with the
same value (or an approximation, if floating point numbers are
involved).

(In particular, converting a string adds quotes around it and converts
``funny'' characters to escape sequences that are safe to print.)

Recursive objects (for example, lists or dictionaries that contain a
reference to themselves, directly or indirectly) use \samp{...} to
indicate a recursive reference, and the result cannot be passed to
\function{eval()} to get an equal value (\exception{SyntaxError} will
be raised instead).
\obindex{recursive}

The built-in function \function{repr()} performs exactly the same
conversion in its argument as enclosing it in parentheses and reverse
quotes does.  The built-in function \function{str()} performs a
similar but more user-friendly conversion.
\bifuncindex{repr}
\bifuncindex{str}


\subsection{Yield expressions\label{yieldexpr}}
\kwindex{yield}
\indexii{yield}{expression}
\indexii{generator}{function}

\begin{productionlist}
  \production{yield_atom}
             {"(" \token{yield_expression} ")"}
  \production{yield_expression}
             {"yield" [\token{expression_list}]}
\end{productionlist}

\versionadded{2.5}

The \keyword{yield} expression is only used when defining a generator
function, and can only be used in the body of a function definition.
Using a \keyword{yield} expression in a function definition is
sufficient to cause that definition to create a generator function
instead of a normal function.

When a generator function is called, it returns an iterator known as a
generator.  That generator then controls the execution of a generator
function.  The execution starts when one of the generator's methods is
called.  At that time, the execution proceeds to the first
\keyword{yield} expression, where it is suspended again, returning the
value of \grammartoken{expression_list} to generator's caller.  By
suspended we mean that all local state is retained, including the
current bindings of local variables, the instruction pointer, and the
internal evaluation stack.  When the execution is resumed by calling
one of the generator's methods, the function can proceed exactly as
if the \keyword{yield} expression was just another external call.
The value of the \keyword{yield} expression after resuming depends on
the method which resumed the execution.

\index{coroutine}

All of this makes generator functions quite similar to coroutines; they
yield multiple times, they have more than one entry point and their
execution can be suspended.  The only difference is that a generator
function cannot control where should the execution continue after it
yields; the control is always transfered to the generator's caller.

\obindex{generator}

The following generator's methods can be used to control the execution
of a generator function:

\exindex{StopIteration}

\begin{methoddesc}[generator]{next}{}
  Starts the execution of a generator function or resumes it at the
  last executed \keyword{yield} expression.  When a generator function
  is resumed with a \method{next()} method, the current \keyword{yield}
  expression always evaluates to \constant{None}.  The execution then
  continues to the next \keyword{yield} expression, where the generator
  is suspended again, and the value of the
  \grammartoken{expression_list} is returned to \method{next()}'s
  caller. If the generator exits without yielding another value, a
  \exception{StopIteration} exception is raised.
\end{methoddesc}

\begin{methoddesc}[generator]{send}{value}
  Resumes the execution and ``sends'' a value into the generator
  function.  The \code{value} argument becomes the result of the
  current \keyword{yield} expression.  The \method{send()} method
  returns the next value yielded by the generator, or raises
  \exception{StopIteration} if the generator exits without yielding
  another value.
  When \method{send()} is called to start the generator, it must be
  called with \constant{None} as the argument, because there is no
  \keyword{yield} expression that could receieve the value.
\end{methoddesc}

\begin{methoddesc}[generator]{throw}
                  {type\optional{, value\optional{, traceback}}}
  Raises an exception of type \code{type} at the point where generator
  was paused, and returns the next value yielded by the generator
  function.  If the generator exits without yielding another value, a
  \exception{StopIteration} exception is raised.  If the generator
  function does not catch the passed-in exception, or raises a
  different exception, then that exception propagates to the caller.
\end{methoddesc}

\exindex{GeneratorExit}

\begin{methoddesc}[generator]{close}{}
  Raises a \exception{GeneratorExit} at the point where the generator
  function was paused.  If the generator function then raises
  \exception{StopIteration} (by exiting normally, or due to already
  being closed) or \exception{GeneratorExit} (by not catching the
  exception), close returns to its caller.  If the generator yields a
  value, a \exception{RuntimeError} is raised.  If the generator raises
  any other exception, it is propagated to the caller.  \method{close}
  does nothing if the generator has already exited due to an exception
  or normal exit.
\end{methoddesc}

Here is a simple example that demonstrates the behavior of generators
and generator functions:

\begin{verbatim}
>>> def echo(value=None):
...     print "Execution starts when 'next()' is called for the first time."
...     try:
...         while True:
...             try:
...                 value = (yield value)
...             except GeneratorExit:
...                 # never catch GeneratorExit
...                 raise
...             except Exception, e:
...                 value = e
...     finally:
...         print "Don't forget to clean up when 'close()' is called."
...
>>> generator = echo(1)
>>> print generator.next()
Execution starts when 'next()' is called for the first time.
1
>>> print generator.next()
None
>>> print generator.send(2)
2
>>> generator.throw(TypeError, "spam")
TypeError('spam',)
>>> generator.close()
Don't forget to clean up when 'close()' is called.
\end{verbatim}

\begin{seealso}
  \seepep{0342}{Coroutines via Enhanced Generators}
         {The proposal to enhance the API and syntax of generators,
          making them usable as simple coroutines.}
\end{seealso}


\section{Primaries\label{primaries}}
\index{primary}

Primaries represent the most tightly bound operations of the language.
Their syntax is:

\begin{productionlist}
  \production{primary}
             {\token{atom} | \token{attributeref}
              | \token{subscription} | \token{slicing} | \token{call}}
\end{productionlist}


\subsection{Attribute references\label{attribute-references}}
\indexii{attribute}{reference}

An attribute reference is a primary followed by a period and a name:

\begin{productionlist}
  \production{attributeref}
             {\token{primary} "." \token{identifier}}
\end{productionlist}

The primary must evaluate to an object of a type that supports
attribute references, e.g., a module, list, or an instance.  This
object is then asked to produce the attribute whose name is the
identifier.  If this attribute is not available, the exception
\exception{AttributeError}\exindex{AttributeError} is raised.
Otherwise, the type and value of the object produced is determined by
the object.  Multiple evaluations of the same attribute reference may
yield different objects.
\obindex{module}
\obindex{list}


\subsection{Subscriptions\label{subscriptions}}
\index{subscription}

A subscription selects an item of a sequence (string, tuple or list)
or mapping (dictionary) object:
\obindex{sequence}
\obindex{mapping}
\obindex{string}
\obindex{tuple}
\obindex{list}
\obindex{dictionary}
\indexii{sequence}{item}

\begin{productionlist}
  \production{subscription}
             {\token{primary} "[" \token{expression_list} "]"}
\end{productionlist}

The primary must evaluate to an object of a sequence or mapping type.

If the primary is a mapping, the expression list must evaluate to an
object whose value is one of the keys of the mapping, and the
subscription selects the value in the mapping that corresponds to that
key.  (The expression list is a tuple except if it has exactly one
item.)

If the primary is a sequence, the expression (list) must evaluate to a
plain integer.  If this value is negative, the length of the sequence
is added to it (so that, e.g., \code{x[-1]} selects the last item of
\code{x}.)  The resulting value must be a nonnegative integer less
than the number of items in the sequence, and the subscription selects
the item whose index is that value (counting from zero).

A string's items are characters.  A character is not a separate data
type but a string of exactly one character.
\index{character}
\indexii{string}{item}


\subsection{Slicings\label{slicings}}
\index{slicing}
\index{slice}

A slicing selects a range of items in a sequence object (e.g., a
string, tuple or list).  Slicings may be used as expressions or as
targets in assignment or \keyword{del} statements.  The syntax for a
slicing:
\obindex{sequence}
\obindex{string}
\obindex{tuple}
\obindex{list}

\begin{productionlist}
  \production{slicing}
             {\token{simple_slicing} | \token{extended_slicing}}
  \production{simple_slicing}
             {\token{primary} "[" \token{short_slice} "]"}
  \production{extended_slicing}
             {\token{primary} "[" \token{slice_list} "]" }
  \production{slice_list}
             {\token{slice_item} ("," \token{slice_item})* [","]}
  \production{slice_item}
             {\token{expression} | \token{proper_slice} | \token{ellipsis}}
  \production{proper_slice}
             {\token{short_slice} | \token{long_slice}}
  \production{short_slice}
             {[\token{lower_bound}] ":" [\token{upper_bound}]}
  \production{long_slice}
             {\token{short_slice} ":" [\token{stride}]}
  \production{lower_bound}
             {\token{expression}}
  \production{upper_bound}
             {\token{expression}}
  \production{stride}
             {\token{expression}}
  \production{ellipsis}
             {"..."}
\end{productionlist}

There is ambiguity in the formal syntax here: anything that looks like
an expression list also looks like a slice list, so any subscription
can be interpreted as a slicing.  Rather than further complicating the
syntax, this is disambiguated by defining that in this case the
interpretation as a subscription takes priority over the
interpretation as a slicing (this is the case if the slice list
contains no proper slice nor ellipses).  Similarly, when the slice
list has exactly one short slice and no trailing comma, the
interpretation as a simple slicing takes priority over that as an
extended slicing.\indexii{extended}{slicing}

The semantics for a simple slicing are as follows.  The primary must
evaluate to a sequence object.  The lower and upper bound expressions,
if present, must evaluate to plain integers; defaults are zero and the
\code{sys.maxint}, respectively.  If either bound is negative, the
sequence's length is added to it.  The slicing now selects all items
with index \var{k} such that
\code{\var{i} <= \var{k} < \var{j}} where \var{i}
and \var{j} are the specified lower and upper bounds.  This may be an
empty sequence.  It is not an error if \var{i} or \var{j} lie outside the
range of valid indexes (such items don't exist so they aren't
selected).

The semantics for an extended slicing are as follows.  The primary
must evaluate to a mapping object, and it is indexed with a key that
is constructed from the slice list, as follows.  If the slice list
contains at least one comma, the key is a tuple containing the
conversion of the slice items; otherwise, the conversion of the lone
slice item is the key.  The conversion of a slice item that is an
expression is that expression.  The conversion of an ellipsis slice
item is the built-in \code{Ellipsis} object.  The conversion of a
proper slice is a slice object (see section \ref{types}) whose
\member{start}, \member{stop} and \member{step} attributes are the
values of the expressions given as lower bound, upper bound and
stride, respectively, substituting \code{None} for missing
expressions.
\withsubitem{(slice object attribute)}{\ttindex{start}
  \ttindex{stop}\ttindex{step}}


\subsection{Calls\label{calls}}
\index{call}

A call calls a callable object (e.g., a function) with a possibly empty
series of arguments:
\obindex{callable}

\begin{productionlist}
  \production{call}
             {\token{primary} "(" [\token{argument_list} [","]}
  \productioncont{            | \token{expression} \token{genexpr_for}] ")"}
  \production{argument_list}
             {\token{positional_arguments} ["," \token{keyword_arguments}]}
  \productioncont{                     ["," "*" \token{expression}]}
  \productioncont{                     ["," "**" \token{expression}]}
  \productioncont{| \token{keyword_arguments} ["," "*" \token{expression}]}
  \productioncont{                    ["," "**" \token{expression}]}
  \productioncont{| "*" \token{expression} ["," "**" \token{expression}]}
  \productioncont{| "**" \token{expression}}
  \production{positional_arguments}
             {\token{expression} ("," \token{expression})*}
  \production{keyword_arguments}
             {\token{keyword_item} ("," \token{keyword_item})*}
  \production{keyword_item}
             {\token{identifier} "=" \token{expression}}
\end{productionlist}

A trailing comma may be present after the positional and keyword
arguments but does not affect the semantics.

The primary must evaluate to a callable object (user-defined
functions, built-in functions, methods of built-in objects, class
objects, methods of class instances, and certain class instances
themselves are callable; extensions may define additional callable
object types).  All argument expressions are evaluated before the call
is attempted.  Please refer to section \ref{function} for the syntax
of formal parameter lists.

If keyword arguments are present, they are first converted to
positional arguments, as follows.  First, a list of unfilled slots is
created for the formal parameters.  If there are N positional
arguments, they are placed in the first N slots.  Next, for each
keyword argument, the identifier is used to determine the
corresponding slot (if the identifier is the same as the first formal
parameter name, the first slot is used, and so on).  If the slot is
already filled, a \exception{TypeError} exception is raised.
Otherwise, the value of the argument is placed in the slot, filling it
(even if the expression is \code{None}, it fills the slot).  When all
arguments have been processed, the slots that are still unfilled are
filled with the corresponding default value from the function
definition.  (Default values are calculated, once, when the function
is defined; thus, a mutable object such as a list or dictionary used
as default value will be shared by all calls that don't specify an
argument value for the corresponding slot; this should usually be
avoided.)  If there are any unfilled slots for which no default value
is specified, a \exception{TypeError} exception is raised.  Otherwise,
the list of filled slots is used as the argument list for the call.

If there are more positional arguments than there are formal parameter
slots, a \exception{TypeError} exception is raised, unless a formal
parameter using the syntax \samp{*identifier} is present; in this
case, that formal parameter receives a tuple containing the excess
positional arguments (or an empty tuple if there were no excess
positional arguments).

If any keyword argument does not correspond to a formal parameter
name, a \exception{TypeError} exception is raised, unless a formal
parameter using the syntax \samp{**identifier} is present; in this
case, that formal parameter receives a dictionary containing the
excess keyword arguments (using the keywords as keys and the argument
values as corresponding values), or a (new) empty dictionary if there
were no excess keyword arguments.

If the syntax \samp{*expression} appears in the function call,
\samp{expression} must evaluate to a sequence.  Elements from this
sequence are treated as if they were additional positional arguments;
if there are postional arguments \var{x1},...,\var{xN} , and
\samp{expression} evaluates to a sequence \var{y1},...,\var{yM}, this
is equivalent to a call with M+N positional arguments
\var{x1},...,\var{xN},\var{y1},...,\var{yM}.

A consequence of this is that although the \samp{*expression} syntax
appears \emph{after} any keyword arguments, it is processed
\emph{before} the keyword arguments (and the
\samp{**expression} argument, if any -- see below).  So:

\begin{verbatim}
>>> def f(a, b):
...  print a, b
...
>>> f(b=1, *(2,))
2 1
>>> f(a=1, *(2,))
Traceback (most recent call last):
  File "<stdin>", line 1, in ?
TypeError: f() got multiple values for keyword argument 'a'
>>> f(1, *(2,))
1 2
\end{verbatim}

It is unusual for both keyword arguments and the
\samp{*expression} syntax to be used in the same call, so in practice
this confusion does not arise.

If the syntax \samp{**expression} appears in the function call,
\samp{expression} must evaluate to a (subclass of) dictionary, the
contents of which are treated as additional keyword arguments.  In the
case of a keyword appearing in both \samp{expression} and as an
explicit keyword argument, a \exception{TypeError} exception is
raised.

Formal parameters using the syntax \samp{*identifier} or
\samp{**identifier} cannot be used as positional argument slots or
as keyword argument names.  Formal parameters using the syntax
\samp{(sublist)} cannot be used as keyword argument names; the
outermost sublist corresponds to a single unnamed argument slot, and
the argument value is assigned to the sublist using the usual tuple
assignment rules after all other parameter processing is done.

A call always returns some value, possibly \code{None}, unless it
raises an exception.  How this value is computed depends on the type
of the callable object.

If it is---

\begin{description}

\item[a user-defined function:] The code block for the function is
executed, passing it the argument list.  The first thing the code
block will do is bind the formal parameters to the arguments; this is
described in section \ref{function}.  When the code block executes a
\keyword{return} statement, this specifies the return value of the
function call.
\indexii{function}{call}
\indexiii{user-defined}{function}{call}
\obindex{user-defined function}
\obindex{function}

\item[a built-in function or method:] The result is up to the
interpreter; see the \citetitle[../lib/built-in-funcs.html]{Python
Library Reference} for the descriptions of built-in functions and
methods.
\indexii{function}{call}
\indexii{built-in function}{call}
\indexii{method}{call}
\indexii{built-in method}{call}
\obindex{built-in method}
\obindex{built-in function}
\obindex{method}
\obindex{function}

\item[a class object:] A new instance of that class is returned.
\obindex{class}
\indexii{class object}{call}

\item[a class instance method:] The corresponding user-defined
function is called, with an argument list that is one longer than the
argument list of the call: the instance becomes the first argument.
\obindex{class instance}
\obindex{instance}
\indexii{class instance}{call}

\item[a class instance:] The class must define a \method{__call__()}
method; the effect is then the same as if that method was called.
\indexii{instance}{call}
\withsubitem{(object method)}{\ttindex{__call__()}}

\end{description}


\section{The power operator\label{power}}

The power operator binds more tightly than unary operators on its
left; it binds less tightly than unary operators on its right.  The
syntax is:

\begin{productionlist}
  \production{power}
             {\token{primary} ["**" \token{u_expr}]}
\end{productionlist}

Thus, in an unparenthesized sequence of power and unary operators, the
operators are evaluated from right to left (this does not constrain
the evaluation order for the operands).

The power operator has the same semantics as the built-in
\function{pow()} function, when called with two arguments: it yields
its left argument raised to the power of its right argument.  The
numeric arguments are first converted to a common type.  The result
type is that of the arguments after coercion.

With mixed operand types, the coercion rules for binary arithmetic
operators apply. For int and long int operands, the result has the
same type as the operands (after coercion) unless the second argument
is negative; in that case, all arguments are converted to float and a
float result is delivered. For example, \code{10**2} returns \code{100},
but \code{10**-2} returns \code{0.01}. (This last feature was added in
Python 2.2. In Python 2.1 and before, if both arguments were of integer
types and the second argument was negative, an exception was raised).

Raising \code{0.0} to a negative power results in a
\exception{ZeroDivisionError}.  Raising a negative number to a
fractional power results in a \exception{ValueError}.


\section{Unary arithmetic operations \label{unary}}
\indexiii{unary}{arithmetic}{operation}
\indexiii{unary}{bit-wise}{operation}

All unary arithmetic (and bit-wise) operations have the same priority:

\begin{productionlist}
  \production{u_expr}
             {\token{power} | "-" \token{u_expr}
              | "+" \token{u_expr} | "{\~}" \token{u_expr}}
\end{productionlist}

The unary \code{-} (minus) operator yields the negation of its
numeric argument.
\index{negation}
\index{minus}

The unary \code{+} (plus) operator yields its numeric argument
unchanged.
\index{plus}

The unary \code{\~} (invert) operator yields the bit-wise inversion
of its plain or long integer argument.  The bit-wise inversion of
\code{x} is defined as \code{-(x+1)}.  It only applies to integral
numbers.
\index{inversion}

In all three cases, if the argument does not have the proper type,
a \exception{TypeError} exception is raised.
\exindex{TypeError}


\section{Binary arithmetic operations\label{binary}}
\indexiii{binary}{arithmetic}{operation}

The binary arithmetic operations have the conventional priority
levels.  Note that some of these operations also apply to certain
non-numeric types.  Apart from the power operator, there are only two
levels, one for multiplicative operators and one for additive
operators:

\begin{productionlist}
  \production{m_expr}
             {\token{u_expr} | \token{m_expr} "*" \token{u_expr}
              | \token{m_expr} "//" \token{u_expr}
              | \token{m_expr} "/" \token{u_expr}}
  \productioncont{| \token{m_expr} "\%" \token{u_expr}}
  \production{a_expr}
             {\token{m_expr} | \token{a_expr} "+" \token{m_expr}
              | \token{a_expr} "-" \token{m_expr}}
\end{productionlist}

The \code{*} (multiplication) operator yields the product of its
arguments.  The arguments must either both be numbers, or one argument
must be an integer (plain or long) and the other must be a sequence.
In the former case, the numbers are converted to a common type and
then multiplied together.  In the latter case, sequence repetition is
performed; a negative repetition factor yields an empty sequence.
\index{multiplication}

The \code{/} (division) and \code{//} (floor division) operators yield
the quotient of their arguments.  The numeric arguments are first
converted to a common type.  Plain or long integer division yields an
integer of the same type; the result is that of mathematical division
with the `floor' function applied to the result.  Division by zero
raises the
\exception{ZeroDivisionError} exception.
\exindex{ZeroDivisionError}
\index{division}

The \code{\%} (modulo) operator yields the remainder from the
division of the first argument by the second.  The numeric arguments
are first converted to a common type.  A zero right argument raises
the \exception{ZeroDivisionError} exception.  The arguments may be floating
point numbers, e.g., \code{3.14\%0.7} equals \code{0.34} (since
\code{3.14} equals \code{4*0.7 + 0.34}.)  The modulo operator always
yields a result with the same sign as its second operand (or zero);
the absolute value of the result is strictly smaller than the absolute
value of the second operand\footnote{
    While \code{abs(x\%y) < abs(y)} is true mathematically, for
    floats it may not be true numerically due to roundoff.  For
    example, and assuming a platform on which a Python float is an
    IEEE 754 double-precision number, in order that \code{-1e-100 \% 1e100}
    have the same sign as \code{1e100}, the computed result is
    \code{-1e-100 + 1e100}, which is numerically exactly equal
    to \code{1e100}.  Function \function{fmod()} in the \module{math}
    module returns a result whose sign matches the sign of the
    first argument instead, and so returns \code{-1e-100} in this case.
    Which approach is more appropriate depends on the application.
}.
\index{modulo}

The integer division and modulo operators are connected by the
following identity: \code{x == (x/y)*y + (x\%y)}.  Integer division and
modulo are also connected with the built-in function \function{divmod()}:
\code{divmod(x, y) == (x/y, x\%y)}.  These identities don't hold for
floating point numbers; there similar identities hold
approximately where \code{x/y} is replaced by \code{floor(x/y)} or
\code{floor(x/y) - 1}\footnote{
    If x is very close to an exact integer multiple of y, it's
    possible for \code{floor(x/y)} to be one larger than
    \code{(x-x\%y)/y} due to rounding.  In such cases, Python returns
    the latter result, in order to preserve that \code{divmod(x,y)[0]
    * y + x \%{} y} be very close to \code{x}.
}.

In addition to performing the modulo operation on numbers, the \code{\%}
operator is also overloaded by string and unicode objects to perform
string formatting (also known as interpolation). The syntax for string
formatting is described in the
\citetitle[../lib/typesseq-strings.html]{Python Library Reference},
section ``Sequence Types''.

\deprecated{2.3}{The floor division operator, the modulo operator,
and the \function{divmod()} function are no longer defined for complex
numbers.  Instead, convert to a floating point number using the
\function{abs()} function if appropriate.}

The \code{+} (addition) operator yields the sum of its arguments.
The arguments must either both be numbers or both sequences of the
same type.  In the former case, the numbers are converted to a common
type and then added together.  In the latter case, the sequences are
concatenated.
\index{addition}

The \code{-} (subtraction) operator yields the difference of its
arguments.  The numeric arguments are first converted to a common
type.
\index{subtraction}


\section{Shifting operations\label{shifting}}
\indexii{shifting}{operation}

The shifting operations have lower priority than the arithmetic
operations:

\begin{productionlist}
  \production{shift_expr}
             {\token{a_expr}
              | \token{shift_expr} ( "<<" | ">>" ) \token{a_expr}}
\end{productionlist}

These operators accept plain or long integers as arguments.  The
arguments are converted to a common type.  They shift the first
argument to the left or right by the number of bits given by the
second argument.

A right shift by \var{n} bits is defined as division by
\code{pow(2,\var{n})}.  A left shift by \var{n} bits is defined as
multiplication with \code{pow(2,\var{n})}; for plain integers there is
no overflow check so in that case the operation drops bits and flips
the sign if the result is not less than \code{pow(2,31)} in absolute
value.  Negative shift counts raise a \exception{ValueError}
exception.
\exindex{ValueError}


\section{Binary bit-wise operations\label{bitwise}}
\indexiii{binary}{bit-wise}{operation}

Each of the three bitwise operations has a different priority level:

\begin{productionlist}
  \production{and_expr}
             {\token{shift_expr} | \token{and_expr} "\&" \token{shift_expr}}
  \production{xor_expr}
             {\token{and_expr} | \token{xor_expr} "\textasciicircum" \token{and_expr}}
  \production{or_expr}
             {\token{xor_expr} | \token{or_expr} "|" \token{xor_expr}}
\end{productionlist}

The \code{\&} operator yields the bitwise AND of its arguments, which
must be plain or long integers.  The arguments are converted to a
common type.
\indexii{bit-wise}{and}

The \code{\^} operator yields the bitwise XOR (exclusive OR) of its
arguments, which must be plain or long integers.  The arguments are
converted to a common type.
\indexii{bit-wise}{xor}
\indexii{exclusive}{or}

The \code{|} operator yields the bitwise (inclusive) OR of its
arguments, which must be plain or long integers.  The arguments are
converted to a common type.
\indexii{bit-wise}{or}
\indexii{inclusive}{or}


\section{Comparisons\label{comparisons}}
\index{comparison}

Unlike C, all comparison operations in Python have the same priority,
which is lower than that of any arithmetic, shifting or bitwise
operation.  Also unlike C, expressions like \code{a < b < c} have the
interpretation that is conventional in mathematics:
\indexii{C}{language}

\begin{productionlist}
  \production{comparison}
             {\token{or_expr} ( \token{comp_operator} \token{or_expr} )*}
  \production{comp_operator}
             {"<" | ">" | "==" | ">=" | "<=" | "<>" | "!="}
  \productioncont{| "is" ["not"] | ["not"] "in"}
\end{productionlist}

Comparisons yield boolean values: \code{True} or \code{False}.

Comparisons can be chained arbitrarily, e.g., \code{x < y <= z} is
equivalent to \code{x < y and y <= z}, except that \code{y} is
evaluated only once (but in both cases \code{z} is not evaluated at all
when \code{x < y} is found to be false).
\indexii{chaining}{comparisons}

Formally, if \var{a}, \var{b}, \var{c}, \ldots, \var{y}, \var{z} are
expressions and \var{opa}, \var{opb}, \ldots, \var{opy} are comparison
operators, then \var{a opa b opb c} \ldots \var{y opy z} is equivalent
to \var{a opa b} \keyword{and} \var{b opb c} \keyword{and} \ldots
\var{y opy z}, except that each expression is evaluated at most once.

Note that \var{a opa b opb c} doesn't imply any kind of comparison
between \var{a} and \var{c}, so that, e.g., \code{x < y > z} is
perfectly legal (though perhaps not pretty).

The forms \code{<>} and \code{!=} are equivalent; for consistency with
C, \code{!=} is preferred; where \code{!=} is mentioned below
\code{<>} is also accepted.  The \code{<>} spelling is considered
obsolescent.

The operators \code{<}, \code{>}, \code{==}, \code{>=}, \code{<=}, and
\code{!=} compare
the values of two objects.  The objects need not have the same type.
If both are numbers, they are converted to a common type.  Otherwise,
objects of different types \emph{always} compare unequal, and are
ordered consistently but arbitrarily.  You can control comparison
behavior of objects of non-builtin types by defining a \code{__cmp__}
method or rich comparison methods like \code{__gt__}, described in
section~\ref{specialnames}.

(This unusual definition of comparison was used to simplify the
definition of operations like sorting and the \keyword{in} and
\keyword{not in} operators.  In the future, the comparison rules for
objects of different types are likely to change.)

Comparison of objects of the same type depends on the type:

\begin{itemize}

\item
Numbers are compared arithmetically.

\item
Strings are compared lexicographically using the numeric equivalents
(the result of the built-in function \function{ord()}) of their
characters.  Unicode and 8-bit strings are fully interoperable in this
behavior.

\item
Tuples and lists are compared lexicographically using comparison of
corresponding elements.  This means that to compare equal, each
element must compare equal and the two sequences must be of the same
type and have the same length.

If not equal, the sequences are ordered the same as their first
differing elements.  For example, \code{cmp([1,2,x], [1,2,y])} returns
the same as \code{cmp(x,y)}.  If the corresponding element does not
exist, the shorter sequence is ordered first (for example,
\code{[1,2] < [1,2,3]}).

\item
Mappings (dictionaries) compare equal if and only if their sorted
(key, value) lists compare equal.\footnote{The implementation computes
   this efficiently, without constructing lists or sorting.}
Outcomes other than equality are resolved consistently, but are not
otherwise defined.\footnote{Earlier versions of Python used
  lexicographic comparison of the sorted (key, value) lists, but this
  was very expensive for the common case of comparing for equality.  An
  even earlier version of Python compared dictionaries by identity only,
  but this caused surprises because people expected to be able to test
  a dictionary for emptiness by comparing it to \code{\{\}}.}

\item
Most other objects of builtin types compare unequal unless they are
the same object;
the choice whether one object is considered smaller or larger than
another one is made arbitrarily but consistently within one
execution of a program.

\end{itemize}

The operators \keyword{in} and \keyword{not in} test for set
membership.  \code{\var{x} in \var{s}} evaluates to true if \var{x}
is a member of the set \var{s}, and false otherwise.  \code{\var{x}
not in \var{s}} returns the negation of \code{\var{x} in \var{s}}.
The set membership test has traditionally been bound to sequences; an
object is a member of a set if the set is a sequence and contains an
element equal to that object.  However, it is possible for an object
to support membership tests without being a sequence.  In particular,
dictionaries support membership testing as a nicer way of spelling
\code{\var{key} in \var{dict}}; other mapping types may follow suit.

For the list and tuple types, \code{\var{x} in \var{y}} is true if and
only if there exists an index \var{i} such that
\code{\var{x} == \var{y}[\var{i}]} is true.

For the Unicode and string types, \code{\var{x} in \var{y}} is true if
and only if \var{x} is a substring of \var{y}.  An equivalent test is
\code{y.find(x) != -1}.  Note, \var{x} and \var{y} need not be the
same type; consequently, \code{u'ab' in 'abc'} will return \code{True}.
Empty strings are always considered to be a substring of any other string,
so \code{"" in "abc"} will return \code{True}.
\versionchanged[Previously, \var{x} was required to be a string of
length \code{1}]{2.3}

For user-defined classes which define the \method{__contains__()} method,
\code{\var{x} in \var{y}} is true if and only if
\code{\var{y}.__contains__(\var{x})} is true.

For user-defined classes which do not define \method{__contains__()} and
do define \method{__getitem__()}, \code{\var{x} in \var{y}} is true if
and only if there is a non-negative integer index \var{i} such that
\code{\var{x} == \var{y}[\var{i}]}, and all lower integer indices
do not raise \exception{IndexError} exception. (If any other exception
is raised, it is as if \keyword{in} raised that exception).

The operator \keyword{not in} is defined to have the inverse true value
of \keyword{in}.
\opindex{in}
\opindex{not in}
\indexii{membership}{test}
\obindex{sequence}

The operators \keyword{is} and \keyword{is not} test for object identity:
\code{\var{x} is \var{y}} is true if and only if \var{x} and \var{y}
are the same object.  \code{\var{x} is not \var{y}} yields the inverse
truth value.
\opindex{is}
\opindex{is not}
\indexii{identity}{test}


\section{Boolean operations\label{Booleans}}
\indexii{Conditional}{expression}
\indexii{Boolean}{operation}

Boolean operations have the lowest priority of all Python operations:

\begin{productionlist}
  \production{expression}
             {\token{conditional_expression} | \token{lambda_form}}
  \production{old_expression}
             {\token{or_test} | \token{old_lambda_form}}
  \production{conditional_expression}
             {\token{or_test} ["if" \token{or_test} "else" \token{expression}]}
  \production{or_test}
             {\token{and_test} | \token{or_test} "or" \token{and_test}}
  \production{and_test}
             {\token{not_test} | \token{and_test} "and" \token{not_test}}
  \production{not_test}
             {\token{comparison} | "not" \token{not_test}}
\end{productionlist}

In the context of Boolean operations, and also when expressions are
used by control flow statements, the following values are interpreted
as false: \code{False}, \code{None}, numeric zero of all types, and empty
strings and containers (including strings, tuples, lists, dictionaries,
sets and frozensets).  All other values are interpreted as true.

The operator \keyword{not} yields \code{True} if its argument is false,
\code{False} otherwise.
\opindex{not}

The expression \code{\var{x} if \var{C} else \var{y}} first evaluates
\var{C} (\emph{not} \var{x}); if \var{C} is true, \var{x} is evaluated and
its value is returned; otherwise, \var{y} is evaluated and its value is
returned.  \versionadded{2.5}

The expression \code{\var{x} and \var{y}} first evaluates \var{x}; if
\var{x} is false, its value is returned; otherwise, \var{y} is
evaluated and the resulting value is returned.
\opindex{and}

The expression \code{\var{x} or \var{y}} first evaluates \var{x}; if
\var{x} is true, its value is returned; otherwise, \var{y} is
evaluated and the resulting value is returned.
\opindex{or}

(Note that neither \keyword{and} nor \keyword{or} restrict the value
and type they return to \code{False} and \code{True}, but rather return the
last evaluated argument.
This is sometimes useful, e.g., if \code{s} is a string that should be
replaced by a default value if it is empty, the expression
\code{s or 'foo'} yields the desired value.  Because \keyword{not} has to
invent a value anyway, it does not bother to return a value of the
same type as its argument, so e.g., \code{not 'foo'} yields \code{False},
not \code{''}.)

\section{Lambdas\label{lambdas}}
\indexii{lambda}{expression}
\indexii{lambda}{form}
\indexii{anonymous}{function}

\begin{productionlist}
  \production{lambda_form}
             {"lambda" [\token{parameter_list}]: \token{expression}}
  \production{old_lambda_form}
             {"lambda" [\token{parameter_list}]: \token{old_expression}}
\end{productionlist}

Lambda forms (lambda expressions) have the same syntactic position as
expressions.  They are a shorthand to create anonymous functions; the
expression \code{lambda \var{arguments}: \var{expression}}
yields a function object.  The unnamed object behaves like a function
object defined with

\begin{verbatim}
def name(arguments):
    return expression
\end{verbatim}

See section \ref{function} for the syntax of parameter lists.  Note
that functions created with lambda forms cannot contain statements.
\label{lambda}

\section{Expression lists\label{exprlists}}
\indexii{expression}{list}

\begin{productionlist}
  \production{expression_list}
             {\token{expression} ( "," \token{expression} )* [","]}
\end{productionlist}

An expression list containing at least one comma yields a
tuple.  The length of the tuple is the number of expressions in the
list.  The expressions are evaluated from left to right.
\obindex{tuple}

The trailing comma is required only to create a single tuple (a.k.a. a
\emph{singleton}); it is optional in all other cases.  A single
expression without a trailing comma doesn't create a
tuple, but rather yields the value of that expression.
(To create an empty tuple, use an empty pair of parentheses:
\code{()}.)
\indexii{trailing}{comma}

\section{Evaluation order\label{evalorder}}
\indexii{evaluation}{order}

Python evaluates expressions from left to right. Notice that while
evaluating an assignment, the right-hand side is evaluated before
the left-hand side.

In the following lines, expressions will be evaluated in the
arithmetic order of their suffixes:

\begin{verbatim}
expr1, expr2, expr3, expr4
(expr1, expr2, expr3, expr4)
{expr1: expr2, expr3: expr4}
expr1 + expr2 * (expr3 - expr4)
func(expr1, expr2, *expr3, **expr4)
expr3, expr4 = expr1, expr2
\end{verbatim}

\section{Summary\label{summary}}

The following table summarizes the operator
precedences\indexii{operator}{precedence} in Python, from lowest
precedence (least binding) to highest precedence (most binding).
Operators in the same box have the same precedence.  Unless the syntax
is explicitly given, operators are binary.  Operators in the same box
group left to right (except for comparisons, including tests, which all
have the same precedence and chain from left to right --- see section
\ref{comparisons} -- and exponentiation, which groups from right to left).

\begin{tableii}{c|l}{textrm}{Operator}{Description}
    \lineii{\keyword{lambda}}			{Lambda expression}
  \hline
    \lineii{\keyword{or}}			{Boolean OR}
  \hline
    \lineii{\keyword{and}}			{Boolean AND}
  \hline
    \lineii{\keyword{not} \var{x}}		{Boolean NOT}
  \hline
    \lineii{\keyword{in}, \keyword{not} \keyword{in}}{Membership tests}
    \lineii{\keyword{is}, \keyword{is not}}{Identity tests}
    \lineii{\code{<}, \code{<=}, \code{>}, \code{>=},
            \code{<>}, \code{!=}, \code{==}}
	   {Comparisons}
  \hline
    \lineii{\code{|}}				{Bitwise OR}
  \hline
    \lineii{\code{\^}}				{Bitwise XOR}
  \hline
    \lineii{\code{\&}}				{Bitwise AND}
  \hline
    \lineii{\code{<<}, \code{>>}}		{Shifts}
  \hline
    \lineii{\code{+}, \code{-}}{Addition and subtraction}
  \hline
    \lineii{\code{*}, \code{/}, \code{\%}}
           {Multiplication, division, remainder}
  \hline
    \lineii{\code{+\var{x}}, \code{-\var{x}}}	{Positive, negative}
    \lineii{\code{\~\var{x}}}			{Bitwise not}
  \hline
    \lineii{\code{**}}				{Exponentiation}
  \hline
    \lineii{\code{\var{x}.\var{attribute}}}	{Attribute reference}
    \lineii{\code{\var{x}[\var{index}]}}	{Subscription}
    \lineii{\code{\var{x}[\var{index}:\var{index}]}}	{Slicing}
    \lineii{\code{\var{f}(\var{arguments}...)}}	{Function call}
  \hline
    \lineii{\code{(\var{expressions}\ldots)}}	{Binding or tuple display}
    \lineii{\code{[\var{expressions}\ldots]}}	{List display}
    \lineii{\code{\{\var{key}:\var{datum}\ldots\}}}{Dictionary display}
    \lineii{\code{`\var{expressions}\ldots`}}	{String conversion}
\end{tableii}
		% Expressions and conditions
\chapter{Simple statements \label{simple}}
\indexii{simple}{statement}

Simple statements are comprised within a single logical line.
Several simple statements may occur on a single line separated
by semicolons.  The syntax for simple statements is:

\begin{productionlist}
  \production{simple_stmt}{\token{expression_stmt}}
  \productioncont{| \token{assert_stmt}}
  \productioncont{| \token{assignment_stmt}}
  \productioncont{| \token{augmented_assignment_stmt}}
  \productioncont{| \token{pass_stmt}}
  \productioncont{| \token{del_stmt}}
  \productioncont{| \token{print_stmt}}
  \productioncont{| \token{return_stmt}}
  \productioncont{| \token{yield_stmt}}
  \productioncont{| \token{raise_stmt}}
  \productioncont{| \token{break_stmt}}
  \productioncont{| \token{continue_stmt}}
  \productioncont{| \token{import_stmt}}
  \productioncont{| \token{global_stmt}}
  \productioncont{| \token{exec_stmt}}
\end{productionlist}


\section{Expression statements \label{exprstmts}}
\indexii{expression}{statement}

Expression statements are used (mostly interactively) to compute and
write a value, or (usually) to call a procedure (a function that
returns no meaningful result; in Python, procedures return the value
\code{None}).  Other uses of expression statements are allowed and
occasionally useful.  The syntax for an expression statement is:

\begin{productionlist}
  \production{expression_stmt}
             {\token{expression_list}}
\end{productionlist}

An expression statement evaluates the expression list (which may be a
single expression).
\indexii{expression}{list}

In interactive mode, if the value is not \code{None}, it is converted
to a string using the built-in \function{repr()}\bifuncindex{repr}
function and the resulting string is written to standard output (see
section~\ref{print}) on a line by itself.  (Expression statements
yielding \code{None} are not written, so that procedure calls do not
cause any output.)
\obindex{None}
\indexii{string}{conversion}
\index{output}
\indexii{standard}{output}
\indexii{writing}{values}
\indexii{procedure}{call}


\section{Assert statements \label{assert}}

Assert statements\stindex{assert} are a convenient way to insert
debugging assertions\indexii{debugging}{assertions} into a program:

\begin{productionlist}
  \production{assert_stmt}
             {"assert" \token{expression} ["," \token{expression}]}
\end{productionlist}

The simple form, \samp{assert expression}, is equivalent to

\begin{verbatim}
if __debug__:
   if not expression: raise AssertionError
\end{verbatim}

The extended form, \samp{assert expression1, expression2}, is
equivalent to

\begin{verbatim}
if __debug__:
   if not expression1: raise AssertionError, expression2
\end{verbatim}

These equivalences assume that \code{__debug__}\ttindex{__debug__} and
\exception{AssertionError}\exindex{AssertionError} refer to the built-in
variables with those names.  In the current implementation, the
built-in variable \code{__debug__} is \code{True} under normal
circumstances, \code{False} when optimization is requested (command line
option -O).  The current code generator emits no code for an assert
statement when optimization is requested at compile time.  Note that it
is unnecessary to include the source code for the expression that failed
in the error message;
it will be displayed as part of the stack trace.

Assignments to \code{__debug__} are illegal.  The value for the
built-in variable is determined when the interpreter starts.


\section{Assignment statements \label{assignment}}

Assignment statements\indexii{assignment}{statement} are used to
(re)bind names to values and to modify attributes or items of mutable
objects:
\indexii{binding}{name}
\indexii{rebinding}{name}
\obindex{mutable}
\indexii{attribute}{assignment}

\begin{productionlist}
  \production{assignment_stmt}
             {(\token{target_list} "=")+
              (\token{expression_list} | \token{yield_expression})}
  \production{target_list}
             {\token{target} ("," \token{target})* [","]}
  \production{target}
             {\token{identifier}}
  \productioncont{| "(" \token{target_list} ")"}
  \productioncont{| "[" \token{target_list} "]"}
  \productioncont{| \token{attributeref}}
  \productioncont{| \token{subscription}}
  \productioncont{| \token{slicing}}
\end{productionlist}

(See section~\ref{primaries} for the syntax definitions for the last
three symbols.)

An assignment statement evaluates the expression list (remember that
this can be a single expression or a comma-separated list, the latter
yielding a tuple) and assigns the single resulting object to each of
the target lists, from left to right.
\indexii{expression}{list}

Assignment is defined recursively depending on the form of the target
(list).  When a target is part of a mutable object (an attribute
reference, subscription or slicing), the mutable object must
ultimately perform the assignment and decide about its validity, and
may raise an exception if the assignment is unacceptable.  The rules
observed by various types and the exceptions raised are given with the
definition of the object types (see section~\ref{types}).
\index{target}
\indexii{target}{list}

Assignment of an object to a target list is recursively defined as
follows.
\indexiii{target}{list}{assignment}

\begin{itemize}
\item
If the target list is a single target: The object is assigned to that
target.

\item
If the target list is a comma-separated list of targets: The object
must be a sequence with the same number of items as there are
targets in the target list, and the items are assigned, from left to
right, to the corresponding targets.  (This rule is relaxed as of
Python 1.5; in earlier versions, the object had to be a tuple.  Since
strings are sequences, an assignment like \samp{a, b = "xy"} is
now legal as long as the string has the right length.)

\end{itemize}

Assignment of an object to a single target is recursively defined as
follows.

\begin{itemize} % nested

\item
If the target is an identifier (name):

\begin{itemize}

\item
If the name does not occur in a \keyword{global} statement in the current
code block: the name is bound to the object in the current local
namespace.
\stindex{global}

\item
Otherwise: the name is bound to the object in the current global
namespace.

\end{itemize} % nested

The name is rebound if it was already bound.  This may cause the
reference count for the object previously bound to the name to reach
zero, causing the object to be deallocated and its
destructor\index{destructor} (if it has one) to be called.

\item
If the target is a target list enclosed in parentheses or in square
brackets: The object must be a sequence with the same number of items
as there are targets in the target list, and its items are assigned,
from left to right, to the corresponding targets.

\item
If the target is an attribute reference: The primary expression in the
reference is evaluated.  It should yield an object with assignable
attributes; if this is not the case, \exception{TypeError} is raised.  That
object is then asked to assign the assigned object to the given
attribute; if it cannot perform the assignment, it raises an exception
(usually but not necessarily \exception{AttributeError}).
\indexii{attribute}{assignment}

\item
If the target is a subscription: The primary expression in the
reference is evaluated.  It should yield either a mutable sequence
object (such as a list) or a mapping object (such as a dictionary). Next,
the subscript expression is evaluated.
\indexii{subscription}{assignment}
\obindex{mutable}

If the primary is a mutable sequence object (such as a list), the subscript
must yield a plain integer.  If it is negative, the sequence's length
is added to it.  The resulting value must be a nonnegative integer
less than the sequence's length, and the sequence is asked to assign
the assigned object to its item with that index.  If the index is out
of range, \exception{IndexError} is raised (assignment to a subscripted
sequence cannot add new items to a list).
\obindex{sequence}
\obindex{list}

If the primary is a mapping object (such as a dictionary), the subscript must
have a type compatible with the mapping's key type, and the mapping is
then asked to create a key/datum pair which maps the subscript to
the assigned object.  This can either replace an existing key/value
pair with the same key value, or insert a new key/value pair (if no
key with the same value existed).
\obindex{mapping}
\obindex{dictionary}

\item
If the target is a slicing: The primary expression in the reference is
evaluated.  It should yield a mutable sequence object (such as a list).  The
assigned object should be a sequence object of the same type.  Next,
the lower and upper bound expressions are evaluated, insofar they are
present; defaults are zero and the sequence's length.  The bounds
should evaluate to (small) integers.  If either bound is negative, the
sequence's length is added to it.  The resulting bounds are clipped to
lie between zero and the sequence's length, inclusive.  Finally, the
sequence object is asked to replace the slice with the items of the
assigned sequence.  The length of the slice may be different from the
length of the assigned sequence, thus changing the length of the
target sequence, if the object allows it.
\indexii{slicing}{assignment}

\end{itemize}
        
(In the current implementation, the syntax for targets is taken
to be the same as for expressions, and invalid syntax is rejected
during the code generation phase, causing less detailed error
messages.)

WARNING: Although the definition of assignment implies that overlaps
between the left-hand side and the right-hand side are `safe' (for example
\samp{a, b = b, a} swaps two variables), overlaps \emph{within} the
collection of assigned-to variables are not safe!  For instance, the
following program prints \samp{[0, 2]}:

\begin{verbatim}
x = [0, 1]
i = 0
i, x[i] = 1, 2
print x
\end{verbatim}


\subsection{Augmented assignment statements \label{augassign}}

Augmented assignment is the combination, in a single statement, of a binary
operation and an assignment statement:
\indexii{augmented}{assignment}
\index{statement!assignment, augmented}

\begin{productionlist}
  \production{augmented_assignment_stmt}
             {\token{target} \token{augop}
              (\token{expression_list} | \token{yield_expression})}
  \production{augop}
             {"+=" | "-=" | "*=" | "/=" | "\%=" | "**="}
  \productioncont{| ">>=" | "<<=" | "\&=" | "\textasciicircum=" | "|="}
\end{productionlist}

(See section~\ref{primaries} for the syntax definitions for the last
three symbols.)

An augmented assignment evaluates the target (which, unlike normal
assignment statements, cannot be an unpacking) and the expression
list, performs the binary operation specific to the type of assignment
on the two operands, and assigns the result to the original
target.  The target is only evaluated once.

An augmented assignment expression like \code{x += 1} can be rewritten as
\code{x = x + 1} to achieve a similar, but not exactly equal effect. In the
augmented version, \code{x} is only evaluated once. Also, when possible, the
actual operation is performed \emph{in-place}, meaning that rather than
creating a new object and assigning that to the target, the old object is
modified instead.

With the exception of assigning to tuples and multiple targets in a single
statement, the assignment done by augmented assignment statements is handled
the same way as normal assignments. Similarly, with the exception of the
possible \emph{in-place} behavior, the binary operation performed by
augmented assignment is the same as the normal binary operations.

For targets which are attribute references, the initial value is
retrieved with a \method{getattr()} and the result is assigned with a
\method{setattr()}.  Notice that the two methods do not necessarily
refer to the same variable.  When \method{getattr()} refers to a class
variable, \method{setattr()} still writes to an instance variable.
For example:

\begin{verbatim}
class A:
    x = 3    # class variable
a = A()
a.x += 1     # writes a.x as 4 leaving A.x as 3
\end{verbatim}


\section{The \keyword{pass} statement \label{pass}}
\stindex{pass}

\begin{productionlist}
  \production{pass_stmt}
             {"pass"}
\end{productionlist}

\keyword{pass} is a null operation --- when it is executed, nothing
happens.  It is useful as a placeholder when a statement is
required syntactically, but no code needs to be executed, for example:
\indexii{null}{operation}

\begin{verbatim}
def f(arg): pass    # a function that does nothing (yet)

class C: pass       # a class with no methods (yet)
\end{verbatim}


\section{The \keyword{del} statement \label{del}}
\stindex{del}

\begin{productionlist}
  \production{del_stmt}
             {"del" \token{target_list}}
\end{productionlist}

Deletion is recursively defined very similar to the way assignment is
defined. Rather that spelling it out in full details, here are some
hints.
\indexii{deletion}{target}
\indexiii{deletion}{target}{list}

Deletion of a target list recursively deletes each target, from left
to right.

Deletion of a name removes the binding of that name 
from the local or global namespace, depending on whether the name
occurs in a \keyword{global} statement in the same code block.  If the
name is unbound, a \exception{NameError} exception will be raised.
\stindex{global}
\indexii{unbinding}{name}

It is illegal to delete a name from the local namespace if it occurs
as a free variable\indexii{free}{variable} in a nested block.

Deletion of attribute references, subscriptions and slicings
is passed to the primary object involved; deletion of a slicing
is in general equivalent to assignment of an empty slice of the
right type (but even this is determined by the sliced object).
\indexii{attribute}{deletion}


\section{The \keyword{print} statement \label{print}}
\stindex{print}

\begin{productionlist}
  \production{print_stmt}
             {"print" ([\token{expression} ("," \token{expression})* [","]}
  \productioncont{| ">>" \token{expression}
                  [("," \token{expression})+ [","])}
\end{productionlist}

\keyword{print} evaluates each expression in turn and writes the
resulting object to standard output (see below).  If an object is not
a string, it is first converted to a string using the rules for string
conversions.  The (resulting or original) string is then written.  A
space is written before each object is (converted and) written, unless
the output system believes it is positioned at the beginning of a
line.  This is the case (1) when no characters have yet been written
to standard output, (2) when the last character written to standard
output is \character{\e n}, or (3) when the last write operation on
standard output was not a \keyword{print} statement.  (In some cases
it may be functional to write an empty string to standard output for
this reason.)  \note{Objects which act like file objects but which are
not the built-in file objects often do not properly emulate this
aspect of the file object's behavior, so it is best not to rely on
this.}
\index{output}
\indexii{writing}{values}

A \character{\e n} character is written at the end, unless the
\keyword{print} statement ends with a comma.  This is the only action
if the statement contains just the keyword \keyword{print}.
\indexii{trailing}{comma}
\indexii{newline}{suppression}

Standard output is defined as the file object named \code{stdout}
in the built-in module \module{sys}.  If no such object exists, or if
it does not have a \method{write()} method, a \exception{RuntimeError}
exception is raised.
\indexii{standard}{output}
\refbimodindex{sys}
\withsubitem{(in module sys)}{\ttindex{stdout}}
\exindex{RuntimeError}

\keyword{print} also has an extended\index{extended print statement}
form, defined by the second portion of the syntax described above.
This form is sometimes referred to as ``\keyword{print} chevron.''
In this form, the first expression after the \code{>>} must
evaluate to a ``file-like'' object, specifically an object that has a
\method{write()} method as described above.  With this extended form,
the subsequent expressions are printed to this file object.  If the
first expression evaluates to \code{None}, then \code{sys.stdout} is
used as the file for output.


\section{The \keyword{return} statement \label{return}}
\stindex{return}

\begin{productionlist}
  \production{return_stmt}
             {"return" [\token{expression_list}]}
\end{productionlist}

\keyword{return} may only occur syntactically nested in a function
definition, not within a nested class definition.
\indexii{function}{definition}
\indexii{class}{definition}

If an expression list is present, it is evaluated, else \code{None}
is substituted.

\keyword{return} leaves the current function call with the expression
list (or \code{None}) as return value.

When \keyword{return} passes control out of a \keyword{try} statement
with a \keyword{finally} clause, that \keyword{finally} clause is executed
before really leaving the function.
\kwindex{finally}

In a generator function, the \keyword{return} statement is not allowed
to include an \grammartoken{expression_list}.  In that context, a bare
\keyword{return} indicates that the generator is done and will cause
\exception{StopIteration} to be raised.


\section{The \keyword{yield} statement \label{yield}}
\stindex{yield}

\begin{productionlist}
  \production{yield_stmt}
             {\token{yield_expression}}
\end{productionlist}

\index{generator!function}
\index{generator!iterator}
\index{function!generator}
\exindex{StopIteration}

The \keyword{yield} statement is only used when defining a generator
function, and is only used in the body of the generator function.
Using a \keyword{yield} statement in a function definition is
sufficient to cause that definition to create a generator function
instead of a normal function.

When a generator function is called, it returns an iterator known as a
generator iterator, or more commonly, a generator.  The body of the
generator function is executed by calling the generator's
\method{next()} method repeatedly until it raises an exception.

When a \keyword{yield} statement is executed, the state of the
generator is frozen and the value of \grammartoken{expression_list} is
returned to \method{next()}'s caller.  By ``frozen'' we mean that all
local state is retained, including the current bindings of local
variables, the instruction pointer, and the internal evaluation stack:
enough information is saved so that the next time \method{next()} is
invoked, the function can proceed exactly as if the \keyword{yield}
statement were just another external call.

As of Python version 2.5, the \keyword{yield} statement is now
allowed in the \keyword{try} clause of a \keyword{try} ...\ 
\keyword{finally} construct.  If the generator is not resumed before
it is finalized (by reaching a zero reference count or by being garbage
collected), the generator-iterator's \method{close()} method will be
called, allowing any pending \keyword{finally} clauses to execute.

\begin{notice}
In Python 2.2, the \keyword{yield} statement is only allowed
when the \code{generators} feature has been enabled.  It will always
be enabled in Python 2.3.  This \code{__future__} import statement can
be used to enable the feature:

\begin{verbatim}
from __future__ import generators
\end{verbatim}
\end{notice}


\begin{seealso}
  \seepep{0255}{Simple Generators}
         {The proposal for adding generators and the \keyword{yield}
          statement to Python.}

  \seepep{0342}{Coroutines via Enhanced Generators}
         {The proposal that, among other generator enhancements,
          proposed allowing \keyword{yield} to appear inside a
          \keyword{try} ... \keyword{finally} block.}
\end{seealso}


\section{The \keyword{raise} statement \label{raise}}
\stindex{raise}

\begin{productionlist}
  \production{raise_stmt}
             {"raise" [\token{expression} ["," \token{expression}
              ["," \token{expression}]]]}
\end{productionlist}

If no expressions are present, \keyword{raise} re-raises the last
exception that was active in the current scope.  If no exception is
active in the current scope, a \exception{TypeError} exception is
raised indicating that this is an error (if running under IDLE, a
\exception{Queue.Empty} exception is raised instead).
\index{exception}
\indexii{raising}{exception}

Otherwise, \keyword{raise} evaluates the expressions to get three
objects, using \code{None} as the value of omitted expressions.  The
first two objects are used to determine the \emph{type} and
\emph{value} of the exception.

If the first object is an instance, the type of the exception is the
class of the instance, the instance itself is the value, and the
second object must be \code{None}.

If the first object is a class, it becomes the type of the exception.
The second object is used to determine the exception value: If it is
an instance of the class, the instance becomes the exception value.
If the second object is a tuple, it is used as the argument list for
the class constructor; if it is \code{None}, an empty argument list is
used, and any other object is treated as a single argument to the
constructor.  The instance so created by calling the constructor is
used as the exception value.

If a third object is present and not \code{None}, it must be a
traceback\obindex{traceback} object (see section~\ref{traceback}), and
it is substituted instead of the current location as the place where
the exception occurred.  If the third object is present and not a
traceback object or \code{None}, a \exception{TypeError} exception is
raised.  The three-expression form of \keyword{raise} is useful to
re-raise an exception transparently in an except clause, but
\keyword{raise} with no expressions should be preferred if the
exception to be re-raised was the most recently active exception in
the current scope.

Additional information on exceptions can be found in
section~\ref{exceptions}, and information about handling exceptions is
in section~\ref{try}.


\section{The \keyword{break} statement \label{break}}
\stindex{break}

\begin{productionlist}
  \production{break_stmt}
             {"break"}
\end{productionlist}

\keyword{break} may only occur syntactically nested in a \keyword{for}
or \keyword{while} loop, but not nested in a function or class definition
within that loop.
\stindex{for}
\stindex{while}
\indexii{loop}{statement}

It terminates the nearest enclosing loop, skipping the optional
\keyword{else} clause if the loop has one.
\kwindex{else}

If a \keyword{for} loop is terminated by \keyword{break}, the loop control
target keeps its current value.
\indexii{loop control}{target}

When \keyword{break} passes control out of a \keyword{try} statement
with a \keyword{finally} clause, that \keyword{finally} clause is executed
before really leaving the loop.
\kwindex{finally}


\section{The \keyword{continue} statement \label{continue}}
\stindex{continue}

\begin{productionlist}
  \production{continue_stmt}
             {"continue"}
\end{productionlist}

\keyword{continue} may only occur syntactically nested in a \keyword{for} or
\keyword{while} loop, but not nested in a function or class definition or
\keyword{finally} statement within that loop.\footnote{It may
occur within an \keyword{except} or \keyword{else} clause.  The
restriction on occurring in the \keyword{try} clause is implementor's
laziness and will eventually be lifted.}
It continues with the next cycle of the nearest enclosing loop.
\stindex{for}
\stindex{while}
\indexii{loop}{statement}
\kwindex{finally}


\section{The \keyword{import} statement \label{import}}
\stindex{import}
\index{module!importing}
\indexii{name}{binding}
\kwindex{from}

\begin{productionlist}
  \production{import_stmt}
             {"import" \token{module} ["as" \token{name}]
                ( "," \token{module} ["as" \token{name}] )*}
  \productioncont{| "from" \token{relative_module} "import" \token{identifier}
                    ["as" \token{name}]}
  \productioncont{  ( "," \token{identifier} ["as" \token{name}] )*}
  \productioncont{| "from" \token{relative_module} "import" "("
                    \token{identifier} ["as" \token{name}]}
  \productioncont{  ( "," \token{identifier} ["as" \token{name}] )* [","] ")"}
  \productioncont{| "from" \token{module} "import" "*"}
  \production{module}
             {(\token{identifier} ".")* \token{identifier}}
  \production{relative_module}
             {"."* \token{module} | "."+}
  \production{name}
             {\token{identifier}}
\end{productionlist}

Import statements are executed in two steps: (1) find a module, and
initialize it if necessary; (2) define a name or names in the local
namespace (of the scope where the \keyword{import} statement occurs).
The first form (without \keyword{from}) repeats these steps for each
identifier in the list.  The form with \keyword{from} performs step
(1) once, and then performs step (2) repeatedly.

In this context, to ``initialize'' a built-in or extension module means to
call an initialization function that the module must provide for the purpose
(in the reference implementation, the function's name is obtained by
prepending string ``init'' to the module's name); to ``initialize'' a
Python-coded module means to execute the module's body.
  
The system maintains a table of modules that have been or are being
initialized,
indexed by module name.  This table is
accessible as \code{sys.modules}.  When a module name is found in
this table, step (1) is finished.  If not, a search for a module
definition is started.  When a module is found, it is loaded.  Details
of the module searching and loading process are implementation and
platform specific.  It generally involves searching for a ``built-in''
module with the given name and then searching a list of locations
given as \code{sys.path}.
\withsubitem{(in module sys)}{\ttindex{modules}}
\ttindex{sys.modules}
\indexii{module}{name}
\indexii{built-in}{module}
\indexii{user-defined}{module}
\refbimodindex{sys}
\indexii{filename}{extension}
\indexiii{module}{search}{path}

If a built-in module is found,\indexii{module}{initialization} its
built-in initialization code is executed and step (1) is finished.  If
no matching file is found,
\exception{ImportError}\exindex{ImportError} is raised.
\index{code block}If a file is found, it is parsed,
yielding an executable code block.  If a syntax error occurs,
\exception{SyntaxError}\exindex{SyntaxError} is raised.  Otherwise, an
empty module of the given name is created and inserted in the module
table, and then the code block is executed in the context of this
module.  Exceptions during this execution terminate step (1).

When step (1) finishes without raising an exception, step (2) can
begin.

The first form of \keyword{import} statement binds the module name in
the local namespace to the module object, and then goes on to import
the next identifier, if any.  If the module name is followed by
\keyword{as}, the name following \keyword{as} is used as the local
name for the module. 

The \keyword{from} form does not bind the module name: it goes through the
list of identifiers, looks each one of them up in the module found in step
(1), and binds the name in the local namespace to the object thus found. 
As with the first form of \keyword{import}, an alternate local name can be
supplied by specifying "\keyword{as} localname".  If a name is not found,
\exception{ImportError} is raised.  If the list of identifiers is replaced
by a star (\character{*}), all public names defined in the module are
bound in the local namespace of the \keyword{import} statement..
\indexii{name}{binding}
\exindex{ImportError}

The \emph{public names} defined by a module are determined by checking
the module's namespace for a variable named \code{__all__}; if
defined, it must be a sequence of strings which are names defined or
imported by that module.  The names given in \code{__all__} are all
considered public and are required to exist.  If \code{__all__} is not
defined, the set of public names includes all names found in the
module's namespace which do not begin with an underscore character
(\character{_}).  \code{__all__} should contain the entire public API.
It is intended to avoid accidentally exporting items that are not part
of the API (such as library modules which were imported and used within
the module).
\withsubitem{(optional module attribute)}{\ttindex{__all__}}

The \keyword{from} form with \samp{*} may only occur in a module
scope.  If the wild card form of import --- \samp{import *} --- is
used in a function and the function contains or is a nested block with
free variables, the compiler will raise a \exception{SyntaxError}.

\kwindex{from}
\stindex{from}

\strong{Hierarchical module names:}\indexiii{hierarchical}{module}{names}
when the module names contains one or more dots, the module search
path is carried out differently.  The sequence of identifiers up to
the last dot is used to find a ``package''\index{packages}; the final
identifier is then searched inside the package.  A package is
generally a subdirectory of a directory on \code{sys.path} that has a
file \file{__init__.py}.\ttindex{__init__.py}
%
[XXX Can't be bothered to spell this out right now; see the URL
\url{http://www.python.org/doc/essays/packages.html} for more details, also
about how the module search works from inside a package.]

The built-in function \function{__import__()} is provided to support
applications that determine which modules need to be loaded
dynamically; refer to \ulink{Built-in
Functions}{../lib/built-in-funcs.html} in the
\citetitle[../lib/lib.html]{Python Library Reference} for additional
information.
\bifuncindex{__import__}

\subsection{Future statements \label{future}}

A \dfn{future statement}\indexii{future}{statement} is a directive to
the compiler that a particular module should be compiled using syntax
or semantics that will be available in a specified future release of
Python.  The future statement is intended to ease migration to future
versions of Python that introduce incompatible changes to the
language.  It allows use of the new features on a per-module basis
before the release in which the feature becomes standard.

\begin{productionlist}[*]
  \production{future_statement}
             {"from" "__future__" "import" feature ["as" name]}
  \productioncont{  ("," feature ["as" name])*}
  \productioncont{| "from" "__future__" "import" "(" feature ["as" name]}
  \productioncont{  ("," feature ["as" name])* [","] ")"}
  \production{feature}{identifier}
  \production{name}{identifier}
\end{productionlist}

A future statement must appear near the top of the module.  The only
lines that can appear before a future statement are:

\begin{itemize}

\item the module docstring (if any),
\item comments,
\item blank lines, and
\item other future statements.

\end{itemize}

The features recognized by Python 2.5 are \samp{absolute_import},
\samp{division}, \samp{generators}, \samp{nested_scopes} and
\samp{with_statement}.  \samp{generators} and \samp{nested_scopes} 
are redundant in Python version 2.3 and above because they are always
enabled. 

A future statement is recognized and treated specially at compile
time: Changes to the semantics of core constructs are often
implemented by generating different code.  It may even be the case
that a new feature introduces new incompatible syntax (such as a new
reserved word), in which case the compiler may need to parse the
module differently.  Such decisions cannot be pushed off until
runtime.

For any given release, the compiler knows which feature names have been
defined, and raises a compile-time error if a future statement contains
a feature not known to it.

The direct runtime semantics are the same as for any import statement:
there is a standard module \module{__future__}, described later, and
it will be imported in the usual way at the time the future statement
is executed.

The interesting runtime semantics depend on the specific feature
enabled by the future statement.

Note that there is nothing special about the statement:

\begin{verbatim}
import __future__ [as name]
\end{verbatim}

That is not a future statement; it's an ordinary import statement with
no special semantics or syntax restrictions.

Code compiled by an \keyword{exec} statement or calls to the builtin functions
\function{compile()} and \function{execfile()} that occur in a module
\module{M} containing a future statement will, by default, use the new 
syntax or semantics associated with the future statement.  This can,
starting with Python 2.2 be controlled by optional arguments to
\function{compile()} --- see the documentation of that function in the
\citetitle[../lib/built-in-funcs.html]{Python Library Reference} for
details.

A future statement typed at an interactive interpreter prompt will
take effect for the rest of the interpreter session.  If an
interpreter is started with the \programopt{-i} option, is passed a
script name to execute, and the script includes a future statement, it
will be in effect in the interactive session started after the script
is executed.

\section{The \keyword{global} statement \label{global}}
\stindex{global}

\begin{productionlist}
  \production{global_stmt}
             {"global" \token{identifier} ("," \token{identifier})*}
\end{productionlist}

The \keyword{global} statement is a declaration which holds for the
entire current code block.  It means that the listed identifiers are to be
interpreted as globals.  It would be impossible to assign to a global
variable without \keyword{global}, although free variables may refer
to globals without being declared global.
\indexiii{global}{name}{binding}

Names listed in a \keyword{global} statement must not be used in the same
code block textually preceding that \keyword{global} statement.

Names listed in a \keyword{global} statement must not be defined as formal
parameters or in a \keyword{for} loop control target, \keyword{class}
definition, function definition, or \keyword{import} statement.

(The current implementation does not enforce the latter two
restrictions, but programs should not abuse this freedom, as future
implementations may enforce them or silently change the meaning of the
program.)

\strong{Programmer's note:}
the \keyword{global} is a directive to the parser.  It
applies only to code parsed at the same time as the \keyword{global}
statement.  In particular, a \keyword{global} statement contained in an
\keyword{exec} statement does not affect the code block \emph{containing}
the \keyword{exec} statement, and code contained in an \keyword{exec}
statement is unaffected by \keyword{global} statements in the code
containing the \keyword{exec} statement.  The same applies to the
\function{eval()}, \function{execfile()} and \function{compile()} functions.
\stindex{exec}
\bifuncindex{eval}
\bifuncindex{execfile}
\bifuncindex{compile}


\section{The \keyword{exec} statement \label{exec}}
\stindex{exec}

\begin{productionlist}
  \production{exec_stmt}
             {"exec" \token{or_expr}
              ["in" \token{expression} ["," \token{expression}]]}
\end{productionlist}

This statement supports dynamic execution of Python code.  The first
expression should evaluate to either a string, an open file object, or
a code object.  If it is a string, the string is parsed as a suite of
Python statements which is then executed (unless a syntax error
occurs).  If it is an open file, the file is parsed until \EOF{} and
executed.  If it is a code object, it is simply executed.  In all
cases, the code that's executed is expected to be valid as file
input (see section~\ref{file-input}, ``File input'').  Be aware that
the \keyword{return} and \keyword{yield} statements may not be used
outside of function definitions even within the context of code passed
to the \keyword{exec} statement.

In all cases, if the optional parts are omitted, the code is executed
in the current scope.  If only the first expression after \keyword{in}
is specified, it should be a dictionary, which will be used for both
the global and the local variables.  If two expressions are given,
they are used for the global and local variables, respectively.
If provided, \var{locals} can be any mapping object.
\versionchanged[formerly \var{locals} was required to be a dictionary]{2.4}

As a side effect, an implementation may insert additional keys into
the dictionaries given besides those corresponding to variable names
set by the executed code.  For example, the current implementation
may add a reference to the dictionary of the built-in module
\module{__builtin__} under the key \code{__builtins__} (!).
\ttindex{__builtins__}
\refbimodindex{__builtin__}

\strong{Programmer's hints:}
dynamic evaluation of expressions is supported by the built-in
function \function{eval()}.  The built-in functions
\function{globals()} and \function{locals()} return the current global
and local dictionary, respectively, which may be useful to pass around
for use by \keyword{exec}.
\bifuncindex{eval}
\bifuncindex{globals}
\bifuncindex{locals}

  



		% Simple statements
\chapter{Compound statements\label{compound}}
\indexii{compound}{statement}

Compound statements contain (groups of) other statements; they affect
or control the execution of those other statements in some way.  In
general, compound statements span multiple lines, although in simple
incarnations a whole compound statement may be contained in one line.

The \keyword{if}, \keyword{while} and \keyword{for} statements implement
traditional control flow constructs.  \keyword{try} specifies exception
handlers and/or cleanup code for a group of statements.  Function and
class definitions are also syntactically compound statements.

Compound statements consist of one or more `clauses.'  A clause
consists of a header and a `suite.'  The clause headers of a
particular compound statement are all at the same indentation level.
Each clause header begins with a uniquely identifying keyword and ends
with a colon.  A suite is a group of statements controlled by a
clause.  A suite can be one or more semicolon-separated simple
statements on the same line as the header, following the header's
colon, or it can be one or more indented statements on subsequent
lines.  Only the latter form of suite can contain nested compound
statements; the following is illegal, mostly because it wouldn't be
clear to which \keyword{if} clause a following \keyword{else} clause would
belong:
\index{clause}
\index{suite}

\begin{verbatim}
if test1: if test2: print x
\end{verbatim}

Also note that the semicolon binds tighter than the colon in this
context, so that in the following example, either all or none of the
\keyword{print} statements are executed:

\begin{verbatim}
if x < y < z: print x; print y; print z
\end{verbatim}

Summarizing:

\begin{productionlist}
  \production{compound_stmt}
             {\token{if_stmt}}
  \productioncont{| \token{while_stmt}}
  \productioncont{| \token{for_stmt}}
  \productioncont{| \token{try_stmt}}
  \productioncont{| \token{with_stmt}}
  \productioncont{| \token{funcdef}}
  \productioncont{| \token{classdef}}
  \production{suite}
             {\token{stmt_list} NEWLINE
              | NEWLINE INDENT \token{statement}+ DEDENT}
  \production{statement}
             {\token{stmt_list} NEWLINE | \token{compound_stmt}}
  \production{stmt_list}
             {\token{simple_stmt} (";" \token{simple_stmt})* [";"]}
\end{productionlist}

Note that statements always end in a
\code{NEWLINE}\index{NEWLINE token} possibly followed by a
\code{DEDENT}.\index{DEDENT token} Also note that optional
continuation clauses always begin with a keyword that cannot start a
statement, thus there are no ambiguities (the `dangling
\keyword{else}' problem is solved in Python by requiring nested
\keyword{if} statements to be indented).
\indexii{dangling}{else}

The formatting of the grammar rules in the following sections places
each clause on a separate line for clarity.


\section{The \keyword{if} statement\label{if}}
\stindex{if}

The \keyword{if} statement is used for conditional execution:

\begin{productionlist}
  \production{if_stmt}
             {"if" \token{expression} ":" \token{suite}}
  \productioncont{( "elif" \token{expression} ":" \token{suite} )*}
  \productioncont{["else" ":" \token{suite}]}
\end{productionlist}

It selects exactly one of the suites by evaluating the expressions one
by one until one is found to be true (see section~\ref{Booleans} for
the definition of true and false); then that suite is executed (and no
other part of the \keyword{if} statement is executed or evaluated).  If
all expressions are false, the suite of the \keyword{else} clause, if
present, is executed.
\kwindex{elif}
\kwindex{else}


\section{The \keyword{while} statement\label{while}}
\stindex{while}
\indexii{loop}{statement}

The \keyword{while} statement is used for repeated execution as long
as an expression is true:

\begin{productionlist}
  \production{while_stmt}
             {"while" \token{expression} ":" \token{suite}}
  \productioncont{["else" ":" \token{suite}]}
\end{productionlist}

This repeatedly tests the expression and, if it is true, executes the
first suite; if the expression is false (which may be the first time it
is tested) the suite of the \keyword{else} clause, if present, is
executed and the loop terminates.
\kwindex{else}

A \keyword{break} statement executed in the first suite terminates the
loop without executing the \keyword{else} clause's suite.  A
\keyword{continue} statement executed in the first suite skips the rest
of the suite and goes back to testing the expression.
\stindex{break}
\stindex{continue}


\section{The \keyword{for} statement\label{for}}
\stindex{for}
\indexii{loop}{statement}

The \keyword{for} statement is used to iterate over the elements of a
sequence (such as a string, tuple or list) or other iterable object:
\obindex{sequence}

\begin{productionlist}
  \production{for_stmt}
             {"for" \token{target_list} "in" \token{expression_list}
              ":" \token{suite}}
  \productioncont{["else" ":" \token{suite}]}
\end{productionlist}

The expression list is evaluated once; it should yield an iterable
object.  An iterator is created for the result of the
{}\code{expression_list}.  The suite is then executed once for each
item provided by the iterator, in the
order of ascending indices.  Each item in turn is assigned to the
target list using the standard rules for assignments, and then the
suite is executed.  When the items are exhausted (which is immediately
when the sequence is empty), the suite in the \keyword{else} clause, if
present, is executed, and the loop terminates.
\kwindex{in}
\kwindex{else}
\indexii{target}{list}

A \keyword{break} statement executed in the first suite terminates the
loop without executing the \keyword{else} clause's suite.  A
\keyword{continue} statement executed in the first suite skips the rest
of the suite and continues with the next item, or with the \keyword{else}
clause if there was no next item.
\stindex{break}
\stindex{continue}

The suite may assign to the variable(s) in the target list; this does
not affect the next item assigned to it.

The target list is not deleted when the loop is finished, but if the
sequence is empty, it will not have been assigned to at all by the
loop.  Hint: the built-in function \function{range()} returns a
sequence of integers suitable to emulate the effect of Pascal's
\code{for i := a to b do};
e.g., \code{range(3)} returns the list \code{[0, 1, 2]}.
\bifuncindex{range}
\indexii{Pascal}{language}

\warning{There is a subtlety when the sequence is being modified
by the loop (this can only occur for mutable sequences, i.e. lists).
An internal counter is used to keep track of which item is used next,
and this is incremented on each iteration.  When this counter has
reached the length of the sequence the loop terminates.  This means that
if the suite deletes the current (or a previous) item from the
sequence, the next item will be skipped (since it gets the index of
the current item which has already been treated).  Likewise, if the
suite inserts an item in the sequence before the current item, the
current item will be treated again the next time through the loop.
This can lead to nasty bugs that can be avoided by making a temporary
copy using a slice of the whole sequence, e.g.,
\index{loop!over mutable sequence}
\index{mutable sequence!loop over}}

\begin{verbatim}
for x in a[:]:
    if x < 0: a.remove(x)
\end{verbatim}


\section{The \keyword{try} statement\label{try}}
\stindex{try}

The \keyword{try} statement specifies exception handlers and/or cleanup
code for a group of statements:

\begin{productionlist}
  \production{try_stmt} {try1_stmt | try2_stmt}
  \production{try1_stmt}
             {"try" ":" \token{suite}}
  \productioncont{("except" [\token{expression}
                             ["," \token{target}]] ":" \token{suite})+}
  \productioncont{["else" ":" \token{suite}]}
  \productioncont{["finally" ":" \token{suite}]}
  \production{try2_stmt}
             {"try" ":" \token{suite}}
  \productioncont{"finally" ":" \token{suite}}
\end{productionlist}

\versionchanged[In previous versions of Python,
\keyword{try}...\keyword{except}...\keyword{finally} did not work.
\keyword{try}...\keyword{except} had to be nested in
\keyword{try}...\keyword{finally}]{2.5}

The \keyword{except} clause(s) specify one or more exception handlers.
When no exception occurs in the
\keyword{try} clause, no exception handler is executed.  When an
exception occurs in the \keyword{try} suite, a search for an exception
handler is started.  This search inspects the except clauses in turn until
one is found that matches the exception.  An expression-less except
clause, if present, must be last; it matches any exception.  For an
except clause with an expression, that expression is evaluated, and the
clause matches the exception if the resulting object is ``compatible''
with the exception.  An object is compatible with an exception if it
is the class or a base class of the exception object, a tuple
containing an item compatible with the exception, or, in the
(deprecated) case of string exceptions, is the raised string itself
(note that the object identities must match, i.e. it must be the same
string object, not just a string with the same value).
\kwindex{except}

If no except clause matches the exception, the search for an exception
handler continues in the surrounding code and on the invocation stack.
\footnote{The exception is propogated to the invocation stack only if
there is no \keyword{finally} clause that negates the exception.}

If the evaluation of an expression in the header of an except clause
raises an exception, the original search for a handler is canceled
and a search starts for the new exception in the surrounding code and
on the call stack (it is treated as if the entire \keyword{try} statement
raised the exception).

When a matching except clause is found, the exception is assigned to
the target specified in that except clause, if present, and the except
clause's suite is executed.  All except clauses must have an
executable block.  When the end of this block is reached, execution
continues normally after the entire try statement.  (This means that
if two nested handlers exist for the same exception, and the exception
occurs in the try clause of the inner handler, the outer handler will
not handle the exception.)

Before an except clause's suite is executed, details about the
exception are assigned to three variables in the
\module{sys}\refbimodindex{sys} module: \code{sys.exc_type} receives
the object identifying the exception; \code{sys.exc_value} receives
the exception's parameter; \code{sys.exc_traceback} receives a
traceback object\obindex{traceback} (see section~\ref{traceback})
identifying the point in the program where the exception occurred.
These details are also available through the \function{sys.exc_info()}
function, which returns a tuple \code{(\var{exc_type}, \var{exc_value},
\var{exc_traceback})}.  Use of the corresponding variables is
deprecated in favor of this function, since their use is unsafe in a
threaded program.  As of Python 1.5, the variables are restored to
their previous values (before the call) when returning from a function
that handled an exception.
\withsubitem{(in module sys)}{\ttindex{exc_type}
  \ttindex{exc_value}\ttindex{exc_traceback}}

The optional \keyword{else} clause is executed if and when control
flows off the end of the \keyword{try} clause.\footnote{
  Currently, control ``flows off the end'' except in the case of an
  exception or the execution of a \keyword{return},
  \keyword{continue}, or \keyword{break} statement.
} Exceptions in the \keyword{else} clause are not handled by the
preceding \keyword{except} clauses.
\kwindex{else}
\stindex{return}
\stindex{break}
\stindex{continue}

If \keyword{finally} is present, it specifies a `cleanup' handler.  The
\keyword{try} clause is executed, including any \keyword{except} and
\keyword{else} clauses.  If an exception occurs in any of the clauses
and is not handled, the exception is temporarily saved. The
\keyword{finally} clause is executed.  If there is a saved exception,
it is re-raised at the end of the \keyword{finally} clause.
If the \keyword{finally} clause raises another exception or
executes a \keyword{return} or \keyword{break} statement, the saved
exception is lost.  The exception information is not available to the
program during execution of the \keyword{finally} clause.
\kwindex{finally}

When a \keyword{return}, \keyword{break} or \keyword{continue} statement is
executed in the \keyword{try} suite of a \keyword{try}...\keyword{finally}
statement, the \keyword{finally} clause is also executed `on the way out.' A
\keyword{continue} statement is illegal in the \keyword{finally} clause.
(The reason is a problem with the current implementation --- this
restriction may be lifted in the future).
\stindex{return}
\stindex{break}
\stindex{continue}

Additional information on exceptions can be found in
section~\ref{exceptions}, and information on using the \keyword{raise}
statement to generate exceptions may be found in section~\ref{raise}.


\section{The \keyword{with} statement\label{with}}
\stindex{with}

\versionadded{2.5}

The \keyword{with} statement is used to wrap the execution of a block
with methods defined by a context manager (see
section~\ref{context-managers}). This allows common
\keyword{try}...\keyword{except}...\keyword{finally} usage patterns to
be encapsulated for convenient reuse.

\begin{productionlist}
  \production{with_stmt}
  {"with" \token{expression} ["as" \token{target}] ":" \token{suite}}
\end{productionlist}

The execution of the \keyword{with} statement proceeds as follows:

\begin{enumerate}

\item The context expression is evaluated to obtain a context manager.

\item The context manager's \method{__enter__()} method is invoked.

\item If a target was included in the \keyword{with}
statement, the return value from \method{__enter__()} is assigned to it.

\note{The \keyword{with} statement guarantees that if the
\method{__enter__()} method returns without an error, then
\method{__exit__()} will always be called. Thus, if an error occurs
during the assignment to the target list, it will be treated the same as
an error occurring within the suite would be. See step 5 below.}

\item The suite is executed.

\item The context manager's \method{__exit__()} method is invoked. If
an exception caused the suite to be exited, its type, value, and
traceback are passed as arguments to \method{__exit__()}. Otherwise,
three \constant{None} arguments are supplied.

If the suite was exited due to an exception, and the return
value from the \method{__exit__()} method was false, the exception is
reraised. If the return value was true, the exception is suppressed, and
execution continues with the statement following the \keyword{with}
statement.

If the suite was exited for any reason other than an exception, the
return value from \method{__exit__()} is ignored, and execution proceeds
at the normal location for the kind of exit that was taken.

\end{enumerate}

\begin{notice}
In Python 2.5, the \keyword{with} statement is only allowed
when the \code{with_statement} feature has been enabled.  It will always
be enabled in Python 2.6.  This \code{__future__} import statement can
be used to enable the feature:

\begin{verbatim}
from __future__ import with_statement
\end{verbatim}
\end{notice}

\begin{seealso}
  \seepep{0343}{The "with" statement}
         {The specification, background, and examples for the
          Python \keyword{with} statement.}
\end{seealso}

\section{Function definitions\label{function}}
\indexii{function}{definition}
\stindex{def}

A function definition defines a user-defined function object (see
section~\ref{types}):
\obindex{user-defined function}
\obindex{function}

\begin{productionlist}
  \production{funcdef}
             {[\token{decorators}] "def" \token{funcname} "(" [\token{parameter_list}] ")"
              ":" \token{suite}}
  \production{decorators}
             {\token{decorator}+}
  \production{decorator}
             {"@" \token{dotted_name} ["(" [\token{argument_list} [","]] ")"] NEWLINE}
  \production{dotted_name}
             {\token{identifier} ("." \token{identifier})*}
  \production{parameter_list}
                 {(\token{defparameter} ",")*}
  \productioncont{(~~"*" \token{identifier} [, "**" \token{identifier}]}
  \productioncont{ | "**" \token{identifier}}
  \productioncont{ | \token{defparameter} [","] )}
  \production{defparameter}
             {\token{parameter} ["=" \token{expression}]}
  \production{sublist}
             {\token{parameter} ("," \token{parameter})* [","]}
  \production{parameter}
             {\token{identifier} | "(" \token{sublist} ")"}
  \production{funcname}
             {\token{identifier}}
\end{productionlist}

A function definition is an executable statement.  Its execution binds
the function name in the current local namespace to a function object
(a wrapper around the executable code for the function).  This
function object contains a reference to the current global namespace
as the global namespace to be used when the function is called.
\indexii{function}{name}
\indexii{name}{binding}

The function definition does not execute the function body; this gets
executed only when the function is called.

A function definition may be wrapped by one or more decorator expressions.
Decorator expressions are evaluated when the function is defined, in the scope
that contains the function definition.  The result must be a callable,
which is invoked with the function object as the only argument.
The returned value is bound to the function name instead of the function
object.  Multiple decorators are applied in nested fashion.
For example, the following code:

\begin{verbatim}
@f1(arg)
@f2
def func(): pass
\end{verbatim}

is equivalent to:

\begin{verbatim}
def func(): pass
func = f1(arg)(f2(func))
\end{verbatim}

When one or more top-level parameters have the form \var{parameter}
\code{=} \var{expression}, the function is said to have ``default
parameter values.''  For a parameter with a
default value, the corresponding argument may be omitted from a call,
in which case the parameter's default value is substituted.  If a
parameter has a default value, all following parameters must also have
a default value --- this is a syntactic restriction that is not
expressed by the grammar.
\indexiii{default}{parameter}{value}

\strong{Default parameter values are evaluated when the function
definition is executed.}  This means that the expression is evaluated
once, when the function is defined, and that that same
``pre-computed'' value is used for each call.  This is especially
important to understand when a default parameter is a mutable object,
such as a list or a dictionary: if the function modifies the object
(e.g. by appending an item to a list), the default value is in effect
modified.  This is generally not what was intended.  A way around this 
is to use \code{None} as the default, and explicitly test for it in
the body of the function, e.g.:

\begin{verbatim}
def whats_on_the_telly(penguin=None):
    if penguin is None:
        penguin = []
    penguin.append("property of the zoo")
    return penguin
\end{verbatim}

Function call semantics are described in more detail in
section~\ref{calls}.
A function call always assigns values to all parameters mentioned in
the parameter list, either from position arguments, from keyword
arguments, or from default values.  If the form ``\code{*identifier}''
is present, it is initialized to a tuple receiving any excess
positional parameters, defaulting to the empty tuple.  If the form
``\code{**identifier}'' is present, it is initialized to a new
dictionary receiving any excess keyword arguments, defaulting to a
new empty dictionary.

It is also possible to create anonymous functions (functions not bound
to a name), for immediate use in expressions.  This uses lambda forms,
described in section~\ref{lambda}.  Note that the lambda form is
merely a shorthand for a simplified function definition; a function
defined in a ``\keyword{def}'' statement can be passed around or
assigned to another name just like a function defined by a lambda
form.  The ``\keyword{def}'' form is actually more powerful since it
allows the execution of multiple statements.
\indexii{lambda}{form}

\strong{Programmer's note:} Functions are first-class objects.  A
``\code{def}'' form executed inside a function definition defines a
local function that can be returned or passed around.  Free variables
used in the nested function can access the local variables of the
function containing the def.  See section~\ref{naming} for details.


\section{Class definitions\label{class}}
\indexii{class}{definition}
\stindex{class}

A class definition defines a class object (see section~\ref{types}):
\obindex{class}

\begin{productionlist}
  \production{classdef}
             {"class" \token{classname} [\token{inheritance}] ":"
              \token{suite}}
  \production{inheritance}
             {"(" [\token{expression_list}] ")"}
  \production{classname}
             {\token{identifier}}
\end{productionlist}

A class definition is an executable statement.  It first evaluates the
inheritance list, if present.  Each item in the inheritance list
should evaluate to a class object or class type which allows
subclassing.  The class's suite is then executed
in a new execution frame (see section~\ref{naming}), using a newly
created local namespace and the original global namespace.
(Usually, the suite contains only function definitions.)  When the
class's suite finishes execution, its execution frame is discarded but
its local namespace is saved.  A class object is then created using
the inheritance list for the base classes and the saved local
namespace for the attribute dictionary.  The class name is bound to this
class object in the original local namespace.
\index{inheritance}
\indexii{class}{name}
\indexii{name}{binding}
\indexii{execution}{frame}

\strong{Programmer's note:} Variables defined in the class definition
are class variables; they are shared by all instances.  To define
instance variables, they must be given a value in the
\method{__init__()} method or in another method.  Both class and
instance variables are accessible through the notation
``\code{self.name}'', and an instance variable hides a class variable
with the same name when accessed in this way.  Class variables with
immutable values can be used as defaults for instance variables.
For new-style classes, descriptors can be used to create instance
variables with different implementation details.
		% Compound statements
\chapter{Top-level components\label{top-level}}

The Python interpreter can get its input from a number of sources:
from a script passed to it as standard input or as program argument,
typed in interactively, from a module source file, etc.  This chapter
gives the syntax used in these cases.
\index{interpreter}


\section{Complete Python programs\label{programs}}
\index{program}

While a language specification need not prescribe how the language
interpreter is invoked, it is useful to have a notion of a complete
Python program.  A complete Python program is executed in a minimally
initialized environment: all built-in and standard modules are
available, but none have been initialized, except for \module{sys}
(various system services), \module{__builtin__} (built-in functions,
exceptions and \code{None}) and \module{__main__}.  The latter is used
to provide the local and global namespace for execution of the
complete program.
\refbimodindex{sys}
\refbimodindex{__main__}
\refbimodindex{__builtin__}

The syntax for a complete Python program is that for file input,
described in the next section.

The interpreter may also be invoked in interactive mode; in this case,
it does not read and execute a complete program but reads and executes
one statement (possibly compound) at a time.  The initial environment
is identical to that of a complete program; each statement is executed
in the namespace of \module{__main__}.
\index{interactive mode}
\refbimodindex{__main__}

Under \UNIX, a complete program can be passed to the interpreter in
three forms: with the \programopt{-c} \var{string} command line option, as a
file passed as the first command line argument, or as standard input.
If the file or standard input is a tty device, the interpreter enters
interactive mode; otherwise, it executes the file as a complete
program.
\index{UNIX}
\index{command line}
\index{standard input}


\section{File input\label{file-input}}

All input read from non-interactive files has the same form:

\begin{productionlist}
  \production{file_input}
             {(NEWLINE | \token{statement})*}
\end{productionlist}

This syntax is used in the following situations:

\begin{itemize}

\item when parsing a complete Python program (from a file or from a string);

\item when parsing a module;

\item when parsing a string passed to the \keyword{exec} statement;

\end{itemize}


\section{Interactive input\label{interactive}}

Input in interactive mode is parsed using the following grammar:

\begin{productionlist}
  \production{interactive_input}
             {[\token{stmt_list}] NEWLINE | \token{compound_stmt} NEWLINE}
\end{productionlist}

Note that a (top-level) compound statement must be followed by a blank
line in interactive mode; this is needed to help the parser detect the
end of the input.


\section{Expression input\label{expression-input}}
\index{input}

There are two forms of expression input.  Both ignore leading
whitespace.
The string argument to \function{eval()} must have the following form:
\bifuncindex{eval}

\begin{productionlist}
  \production{eval_input}
             {\token{expression_list} NEWLINE*}
\end{productionlist}

The input line read by \function{input()} must have the following form:
\bifuncindex{input}

\begin{productionlist}
  \production{input_input}
             {\token{expression_list} NEWLINE}
\end{productionlist}

Note: to read `raw' input line without interpretation, you can use the
built-in function \function{raw_input()} or the \method{readline()} method
of file objects.
\obindex{file}
\index{input!raw}
\index{raw input}
\bifuncindex{raw_input}
\withsubitem{(file method)}{\ttindex{readline()}}
		% Top-level components

\appendix

\chapter{History and License}
\section{History of the software}

Python was created in the early 1990s by Guido van Rossum at Stichting
Mathematisch Centrum (CWI, see \url{http://www.cwi.nl/}) in the Netherlands
as a successor of a language called ABC.  Guido remains Python's
principal author, although it includes many contributions from others.

In 1995, Guido continued his work on Python at the Corporation for
National Research Initiatives (CNRI, see \url{http://www.cnri.reston.va.us/})
in Reston, Virginia where he released several versions of the
software.

In May 2000, Guido and the Python core development team moved to
BeOpen.com to form the BeOpen PythonLabs team.  In October of the same
year, the PythonLabs team moved to Digital Creations (now Zope
Corporation; see \url{http://www.zope.com/}).  In 2001, the Python
Software Foundation (PSF, see \url{http://www.python.org/psf/}) was
formed, a non-profit organization created specifically to own
Python-related Intellectual Property.  Zope Corporation is a
sponsoring member of the PSF.

All Python releases are Open Source (see
\url{http://www.opensource.org/} for the Open Source Definition).
Historically, most, but not all, Python releases have also been
GPL-compatible; the table below summarizes the various releases.

\begin{tablev}{c|c|c|c|c}{textrm}%
  {Release}{Derived from}{Year}{Owner}{GPL compatible?}
  \linev{0.9.0 thru 1.2}{n/a}{1991-1995}{CWI}{yes}
  \linev{1.3 thru 1.5.2}{1.2}{1995-1999}{CNRI}{yes}
  \linev{1.6}{1.5.2}{2000}{CNRI}{no}
  \linev{2.0}{1.6}{2000}{BeOpen.com}{no}
  \linev{1.6.1}{1.6}{2001}{CNRI}{no}
  \linev{2.1}{2.0+1.6.1}{2001}{PSF}{no}
  \linev{2.0.1}{2.0+1.6.1}{2001}{PSF}{yes}
  \linev{2.1.1}{2.1+2.0.1}{2001}{PSF}{yes}
  \linev{2.2}{2.1.1}{2001}{PSF}{yes}
  \linev{2.1.2}{2.1.1}{2002}{PSF}{yes}
  \linev{2.1.3}{2.1.2}{2002}{PSF}{yes}
  \linev{2.2.1}{2.2}{2002}{PSF}{yes}
  \linev{2.2.2}{2.2.1}{2002}{PSF}{yes}
  \linev{2.2.3}{2.2.2}{2002-2003}{PSF}{yes}
  \linev{2.3}{2.2.2}{2002-2003}{PSF}{yes}
  \linev{2.3.1}{2.3}{2002-2003}{PSF}{yes}
  \linev{2.3.2}{2.3.1}{2003}{PSF}{yes}
  \linev{2.3.3}{2.3.2}{2003}{PSF}{yes}
  \linev{2.3.4}{2.3.3}{2004}{PSF}{yes}
  \linev{2.3.5}{2.3.4}{2005}{PSF}{yes}
  \linev{2.4}{2.3}{2004}{PSF}{yes}
  \linev{2.4.1}{2.4}{2005}{PSF}{yes}
  \linev{2.4.2}{2.4.1}{2005}{PSF}{yes}
  \linev{2.4.3}{2.4.2}{2006}{PSF}{yes}
  \linev{2.4.4}{2.4.3}{2006}{PSF}{yes}
  \linev{2.5}{2.4}{2006}{PSF}{yes}
  \linev{2.5.1}{2.5}{2007}{PSF}{yes}
  \linev{2.5.2}{2.5.1}{2008}{PSF}{yes}
\end{tablev}

\note{GPL-compatible doesn't mean that we're distributing
Python under the GPL.  All Python licenses, unlike the GPL, let you
distribute a modified version without making your changes open source.
The GPL-compatible licenses make it possible to combine Python with
other software that is released under the GPL; the others don't.}

Thanks to the many outside volunteers who have worked under Guido's
direction to make these releases possible.


\section{Terms and conditions for accessing or otherwise using Python}

\centerline{\strong{PSF LICENSE AGREEMENT FOR PYTHON \version}}

\begin{enumerate}
\item
This LICENSE AGREEMENT is between the Python Software Foundation
(``PSF''), and the Individual or Organization (``Licensee'') accessing
and otherwise using Python \version{} software in source or binary
form and its associated documentation.

\item
Subject to the terms and conditions of this License Agreement, PSF
hereby grants Licensee a nonexclusive, royalty-free, world-wide
license to reproduce, analyze, test, perform and/or display publicly,
prepare derivative works, distribute, and otherwise use Python
\version{} alone or in any derivative version, provided, however, that
PSF's License Agreement and PSF's notice of copyright, i.e.,
``Copyright \copyright{} 2001-2008 Python Software Foundation; All
Rights Reserved'' are retained in Python \version{} alone or in any
derivative version prepared by Licensee.

\item
In the event Licensee prepares a derivative work that is based on
or incorporates Python \version{} or any part thereof, and wants to
make the derivative work available to others as provided herein, then
Licensee hereby agrees to include in any such work a brief summary of
the changes made to Python \version.

\item
PSF is making Python \version{} available to Licensee on an ``AS IS''
basis.  PSF MAKES NO REPRESENTATIONS OR WARRANTIES, EXPRESS OR
IMPLIED.  BY WAY OF EXAMPLE, BUT NOT LIMITATION, PSF MAKES NO AND
DISCLAIMS ANY REPRESENTATION OR WARRANTY OF MERCHANTABILITY OR FITNESS
FOR ANY PARTICULAR PURPOSE OR THAT THE USE OF PYTHON \version{} WILL
NOT INFRINGE ANY THIRD PARTY RIGHTS.

\item
PSF SHALL NOT BE LIABLE TO LICENSEE OR ANY OTHER USERS OF PYTHON
\version{} FOR ANY INCIDENTAL, SPECIAL, OR CONSEQUENTIAL DAMAGES OR
LOSS AS A RESULT OF MODIFYING, DISTRIBUTING, OR OTHERWISE USING PYTHON
\version, OR ANY DERIVATIVE THEREOF, EVEN IF ADVISED OF THE
POSSIBILITY THEREOF.

\item
This License Agreement will automatically terminate upon a material
breach of its terms and conditions.

\item
Nothing in this License Agreement shall be deemed to create any
relationship of agency, partnership, or joint venture between PSF and
Licensee.  This License Agreement does not grant permission to use PSF
trademarks or trade name in a trademark sense to endorse or promote
products or services of Licensee, or any third party.

\item
By copying, installing or otherwise using Python \version, Licensee
agrees to be bound by the terms and conditions of this License
Agreement.
\end{enumerate}


\centerline{\strong{BEOPEN.COM LICENSE AGREEMENT FOR PYTHON 2.0}}

\centerline{\strong{BEOPEN PYTHON OPEN SOURCE LICENSE AGREEMENT VERSION 1}}

\begin{enumerate}
\item
This LICENSE AGREEMENT is between BeOpen.com (``BeOpen''), having an
office at 160 Saratoga Avenue, Santa Clara, CA 95051, and the
Individual or Organization (``Licensee'') accessing and otherwise
using this software in source or binary form and its associated
documentation (``the Software'').

\item
Subject to the terms and conditions of this BeOpen Python License
Agreement, BeOpen hereby grants Licensee a non-exclusive,
royalty-free, world-wide license to reproduce, analyze, test, perform
and/or display publicly, prepare derivative works, distribute, and
otherwise use the Software alone or in any derivative version,
provided, however, that the BeOpen Python License is retained in the
Software, alone or in any derivative version prepared by Licensee.

\item
BeOpen is making the Software available to Licensee on an ``AS IS''
basis.  BEOPEN MAKES NO REPRESENTATIONS OR WARRANTIES, EXPRESS OR
IMPLIED.  BY WAY OF EXAMPLE, BUT NOT LIMITATION, BEOPEN MAKES NO AND
DISCLAIMS ANY REPRESENTATION OR WARRANTY OF MERCHANTABILITY OR FITNESS
FOR ANY PARTICULAR PURPOSE OR THAT THE USE OF THE SOFTWARE WILL NOT
INFRINGE ANY THIRD PARTY RIGHTS.

\item
BEOPEN SHALL NOT BE LIABLE TO LICENSEE OR ANY OTHER USERS OF THE
SOFTWARE FOR ANY INCIDENTAL, SPECIAL, OR CONSEQUENTIAL DAMAGES OR LOSS
AS A RESULT OF USING, MODIFYING OR DISTRIBUTING THE SOFTWARE, OR ANY
DERIVATIVE THEREOF, EVEN IF ADVISED OF THE POSSIBILITY THEREOF.

\item
This License Agreement will automatically terminate upon a material
breach of its terms and conditions.

\item
This License Agreement shall be governed by and interpreted in all
respects by the law of the State of California, excluding conflict of
law provisions.  Nothing in this License Agreement shall be deemed to
create any relationship of agency, partnership, or joint venture
between BeOpen and Licensee.  This License Agreement does not grant
permission to use BeOpen trademarks or trade names in a trademark
sense to endorse or promote products or services of Licensee, or any
third party.  As an exception, the ``BeOpen Python'' logos available
at http://www.pythonlabs.com/logos.html may be used according to the
permissions granted on that web page.

\item
By copying, installing or otherwise using the software, Licensee
agrees to be bound by the terms and conditions of this License
Agreement.
\end{enumerate}


\centerline{\strong{CNRI LICENSE AGREEMENT FOR PYTHON 1.6.1}}

\begin{enumerate}
\item
This LICENSE AGREEMENT is between the Corporation for National
Research Initiatives, having an office at 1895 Preston White Drive,
Reston, VA 20191 (``CNRI''), and the Individual or Organization
(``Licensee'') accessing and otherwise using Python 1.6.1 software in
source or binary form and its associated documentation.

\item
Subject to the terms and conditions of this License Agreement, CNRI
hereby grants Licensee a nonexclusive, royalty-free, world-wide
license to reproduce, analyze, test, perform and/or display publicly,
prepare derivative works, distribute, and otherwise use Python 1.6.1
alone or in any derivative version, provided, however, that CNRI's
License Agreement and CNRI's notice of copyright, i.e., ``Copyright
\copyright{} 1995-2001 Corporation for National Research Initiatives;
All Rights Reserved'' are retained in Python 1.6.1 alone or in any
derivative version prepared by Licensee.  Alternately, in lieu of
CNRI's License Agreement, Licensee may substitute the following text
(omitting the quotes): ``Python 1.6.1 is made available subject to the
terms and conditions in CNRI's License Agreement.  This Agreement
together with Python 1.6.1 may be located on the Internet using the
following unique, persistent identifier (known as a handle):
1895.22/1013.  This Agreement may also be obtained from a proxy server
on the Internet using the following URL:
\url{http://hdl.handle.net/1895.22/1013}.''

\item
In the event Licensee prepares a derivative work that is based on
or incorporates Python 1.6.1 or any part thereof, and wants to make
the derivative work available to others as provided herein, then
Licensee hereby agrees to include in any such work a brief summary of
the changes made to Python 1.6.1.

\item
CNRI is making Python 1.6.1 available to Licensee on an ``AS IS''
basis.  CNRI MAKES NO REPRESENTATIONS OR WARRANTIES, EXPRESS OR
IMPLIED.  BY WAY OF EXAMPLE, BUT NOT LIMITATION, CNRI MAKES NO AND
DISCLAIMS ANY REPRESENTATION OR WARRANTY OF MERCHANTABILITY OR FITNESS
FOR ANY PARTICULAR PURPOSE OR THAT THE USE OF PYTHON 1.6.1 WILL NOT
INFRINGE ANY THIRD PARTY RIGHTS.

\item
CNRI SHALL NOT BE LIABLE TO LICENSEE OR ANY OTHER USERS OF PYTHON
1.6.1 FOR ANY INCIDENTAL, SPECIAL, OR CONSEQUENTIAL DAMAGES OR LOSS AS
A RESULT OF MODIFYING, DISTRIBUTING, OR OTHERWISE USING PYTHON 1.6.1,
OR ANY DERIVATIVE THEREOF, EVEN IF ADVISED OF THE POSSIBILITY THEREOF.

\item
This License Agreement will automatically terminate upon a material
breach of its terms and conditions.

\item
This License Agreement shall be governed by the federal
intellectual property law of the United States, including without
limitation the federal copyright law, and, to the extent such
U.S. federal law does not apply, by the law of the Commonwealth of
Virginia, excluding Virginia's conflict of law provisions.
Notwithstanding the foregoing, with regard to derivative works based
on Python 1.6.1 that incorporate non-separable material that was
previously distributed under the GNU General Public License (GPL), the
law of the Commonwealth of Virginia shall govern this License
Agreement only as to issues arising under or with respect to
Paragraphs 4, 5, and 7 of this License Agreement.  Nothing in this
License Agreement shall be deemed to create any relationship of
agency, partnership, or joint venture between CNRI and Licensee.  This
License Agreement does not grant permission to use CNRI trademarks or
trade name in a trademark sense to endorse or promote products or
services of Licensee, or any third party.

\item
By clicking on the ``ACCEPT'' button where indicated, or by copying,
installing or otherwise using Python 1.6.1, Licensee agrees to be
bound by the terms and conditions of this License Agreement.
\end{enumerate}

\centerline{ACCEPT}



\centerline{\strong{CWI LICENSE AGREEMENT FOR PYTHON 0.9.0 THROUGH 1.2}}

Copyright \copyright{} 1991 - 1995, Stichting Mathematisch Centrum
Amsterdam, The Netherlands.  All rights reserved.

Permission to use, copy, modify, and distribute this software and its
documentation for any purpose and without fee is hereby granted,
provided that the above copyright notice appear in all copies and that
both that copyright notice and this permission notice appear in
supporting documentation, and that the name of Stichting Mathematisch
Centrum or CWI not be used in advertising or publicity pertaining to
distribution of the software without specific, written prior
permission.

STICHTING MATHEMATISCH CENTRUM DISCLAIMS ALL WARRANTIES WITH REGARD TO
THIS SOFTWARE, INCLUDING ALL IMPLIED WARRANTIES OF MERCHANTABILITY AND
FITNESS, IN NO EVENT SHALL STICHTING MATHEMATISCH CENTRUM BE LIABLE
FOR ANY SPECIAL, INDIRECT OR CONSEQUENTIAL DAMAGES OR ANY DAMAGES
WHATSOEVER RESULTING FROM LOSS OF USE, DATA OR PROFITS, WHETHER IN AN
ACTION OF CONTRACT, NEGLIGENCE OR OTHER TORTIOUS ACTION, ARISING OUT
OF OR IN CONNECTION WITH THE USE OR PERFORMANCE OF THIS SOFTWARE.


\section{Licenses and Acknowledgements for Incorporated Software}

This section is an incomplete, but growing list of licenses and
acknowledgements for third-party software incorporated in the
Python distribution.


\subsection{Mersenne Twister}

The \module{_random} module includes code based on a download from
\url{http://www.math.keio.ac.jp/~matumoto/MT2002/emt19937ar.html}.
The following are the verbatim comments from the original code:

\begin{verbatim}
A C-program for MT19937, with initialization improved 2002/1/26.
Coded by Takuji Nishimura and Makoto Matsumoto.

Before using, initialize the state by using init_genrand(seed)
or init_by_array(init_key, key_length).

Copyright (C) 1997 - 2002, Makoto Matsumoto and Takuji Nishimura,
All rights reserved.

Redistribution and use in source and binary forms, with or without
modification, are permitted provided that the following conditions
are met:

 1. Redistributions of source code must retain the above copyright
    notice, this list of conditions and the following disclaimer.

 2. Redistributions in binary form must reproduce the above copyright
    notice, this list of conditions and the following disclaimer in the
    documentation and/or other materials provided with the distribution.

 3. The names of its contributors may not be used to endorse or promote
    products derived from this software without specific prior written
    permission.

THIS SOFTWARE IS PROVIDED BY THE COPYRIGHT HOLDERS AND CONTRIBUTORS
"AS IS" AND ANY EXPRESS OR IMPLIED WARRANTIES, INCLUDING, BUT NOT
LIMITED TO, THE IMPLIED WARRANTIES OF MERCHANTABILITY AND FITNESS FOR
A PARTICULAR PURPOSE ARE DISCLAIMED.  IN NO EVENT SHALL THE COPYRIGHT OWNER OR
CONTRIBUTORS BE LIABLE FOR ANY DIRECT, INDIRECT, INCIDENTAL, SPECIAL,
EXEMPLARY, OR CONSEQUENTIAL DAMAGES (INCLUDING, BUT NOT LIMITED TO,
PROCUREMENT OF SUBSTITUTE GOODS OR SERVICES; LOSS OF USE, DATA, OR
PROFITS; OR BUSINESS INTERRUPTION) HOWEVER CAUSED AND ON ANY THEORY OF
LIABILITY, WHETHER IN CONTRACT, STRICT LIABILITY, OR TORT (INCLUDING
NEGLIGENCE OR OTHERWISE) ARISING IN ANY WAY OUT OF THE USE OF THIS
SOFTWARE, EVEN IF ADVISED OF THE POSSIBILITY OF SUCH DAMAGE.


Any feedback is very welcome.
http://www.math.keio.ac.jp/matumoto/emt.html
email: matumoto@math.keio.ac.jp
\end{verbatim}



\subsection{Sockets}

The \module{socket} module uses the functions, \function{getaddrinfo},
and \function{getnameinfo}, which are coded in separate source files
from the WIDE Project, \url{http://www.wide.ad.jp/about/index.html}.

\begin{verbatim}      
Copyright (C) 1995, 1996, 1997, and 1998 WIDE Project.
All rights reserved.
 
Redistribution and use in source and binary forms, with or without
modification, are permitted provided that the following conditions
are met:
1. Redistributions of source code must retain the above copyright
   notice, this list of conditions and the following disclaimer.
2. Redistributions in binary form must reproduce the above copyright
   notice, this list of conditions and the following disclaimer in the
   documentation and/or other materials provided with the distribution.
3. Neither the name of the project nor the names of its contributors
   may be used to endorse or promote products derived from this software
   without specific prior written permission.

THIS SOFTWARE IS PROVIDED BY THE PROJECT AND CONTRIBUTORS ``AS IS'' AND
GAI_ANY EXPRESS OR IMPLIED WARRANTIES, INCLUDING, BUT NOT LIMITED TO, THE
IMPLIED WARRANTIES OF MERCHANTABILITY AND FITNESS FOR A PARTICULAR PURPOSE
ARE DISCLAIMED.  IN NO EVENT SHALL THE PROJECT OR CONTRIBUTORS BE LIABLE
FOR GAI_ANY DIRECT, INDIRECT, INCIDENTAL, SPECIAL, EXEMPLARY, OR CONSEQUENTIAL
DAMAGES (INCLUDING, BUT NOT LIMITED TO, PROCUREMENT OF SUBSTITUTE GOODS
OR SERVICES; LOSS OF USE, DATA, OR PROFITS; OR BUSINESS INTERRUPTION)
HOWEVER CAUSED AND ON GAI_ANY THEORY OF LIABILITY, WHETHER IN CONTRACT, STRICT
LIABILITY, OR TORT (INCLUDING NEGLIGENCE OR OTHERWISE) ARISING IN GAI_ANY WAY
OUT OF THE USE OF THIS SOFTWARE, EVEN IF ADVISED OF THE POSSIBILITY OF
SUCH DAMAGE.
\end{verbatim}



\subsection{Floating point exception control}

The source for the \module{fpectl} module includes the following notice:

\begin{verbatim}
     ---------------------------------------------------------------------  
    /                       Copyright (c) 1996.                           \ 
   |          The Regents of the University of California.                 |
   |                        All rights reserved.                           |
   |                                                                       |
   |   Permission to use, copy, modify, and distribute this software for   |
   |   any purpose without fee is hereby granted, provided that this en-   |
   |   tire notice is included in all copies of any software which is or   |
   |   includes  a  copy  or  modification  of  this software and in all   |
   |   copies of the supporting documentation for such software.           |
   |                                                                       |
   |   This  work was produced at the University of California, Lawrence   |
   |   Livermore National Laboratory under  contract  no.  W-7405-ENG-48   |
   |   between  the  U.S.  Department  of  Energy and The Regents of the   |
   |   University of California for the operation of UC LLNL.              |
   |                                                                       |
   |                              DISCLAIMER                               |
   |                                                                       |
   |   This  software was prepared as an account of work sponsored by an   |
   |   agency of the United States Government. Neither the United States   |
   |   Government  nor the University of California nor any of their em-   |
   |   ployees, makes any warranty, express or implied, or  assumes  any   |
   |   liability  or  responsibility  for the accuracy, completeness, or   |
   |   usefulness of any information,  apparatus,  product,  or  process   |
   |   disclosed,   or  represents  that  its  use  would  not  infringe   |
   |   privately-owned rights. Reference herein to any specific  commer-   |
   |   cial  products,  process,  or  service  by trade name, trademark,   |
   |   manufacturer, or otherwise, does not  necessarily  constitute  or   |
   |   imply  its endorsement, recommendation, or favoring by the United   |
   |   States Government or the University of California. The views  and   |
   |   opinions  of authors expressed herein do not necessarily state or   |
   |   reflect those of the United States Government or  the  University   |
   |   of  California,  and shall not be used for advertising or product   |
    \  endorsement purposes.                                              / 
     ---------------------------------------------------------------------
\end{verbatim}



\subsection{MD5 message digest algorithm}

The source code for the \module{md5} module contains the following notice:

\begin{verbatim}
  Copyright (C) 1999, 2002 Aladdin Enterprises.  All rights reserved.

  This software is provided 'as-is', without any express or implied
  warranty.  In no event will the authors be held liable for any damages
  arising from the use of this software.

  Permission is granted to anyone to use this software for any purpose,
  including commercial applications, and to alter it and redistribute it
  freely, subject to the following restrictions:

  1. The origin of this software must not be misrepresented; you must not
     claim that you wrote the original software. If you use this software
     in a product, an acknowledgment in the product documentation would be
     appreciated but is not required.
  2. Altered source versions must be plainly marked as such, and must not be
     misrepresented as being the original software.
  3. This notice may not be removed or altered from any source distribution.

  L. Peter Deutsch
  ghost@aladdin.com

  Independent implementation of MD5 (RFC 1321).

  This code implements the MD5 Algorithm defined in RFC 1321, whose
  text is available at
	http://www.ietf.org/rfc/rfc1321.txt
  The code is derived from the text of the RFC, including the test suite
  (section A.5) but excluding the rest of Appendix A.  It does not include
  any code or documentation that is identified in the RFC as being
  copyrighted.

  The original and principal author of md5.h is L. Peter Deutsch
  <ghost@aladdin.com>.  Other authors are noted in the change history
  that follows (in reverse chronological order):

  2002-04-13 lpd Removed support for non-ANSI compilers; removed
	references to Ghostscript; clarified derivation from RFC 1321;
	now handles byte order either statically or dynamically.
  1999-11-04 lpd Edited comments slightly for automatic TOC extraction.
  1999-10-18 lpd Fixed typo in header comment (ansi2knr rather than md5);
	added conditionalization for C++ compilation from Martin
	Purschke <purschke@bnl.gov>.
  1999-05-03 lpd Original version.
\end{verbatim}



\subsection{Asynchronous socket services}

The \module{asynchat} and \module{asyncore} modules contain the
following notice:

\begin{verbatim}      
 Copyright 1996 by Sam Rushing

                         All Rights Reserved

 Permission to use, copy, modify, and distribute this software and
 its documentation for any purpose and without fee is hereby
 granted, provided that the above copyright notice appear in all
 copies and that both that copyright notice and this permission
 notice appear in supporting documentation, and that the name of Sam
 Rushing not be used in advertising or publicity pertaining to
 distribution of the software without specific, written prior
 permission.

 SAM RUSHING DISCLAIMS ALL WARRANTIES WITH REGARD TO THIS SOFTWARE,
 INCLUDING ALL IMPLIED WARRANTIES OF MERCHANTABILITY AND FITNESS, IN
 NO EVENT SHALL SAM RUSHING BE LIABLE FOR ANY SPECIAL, INDIRECT OR
 CONSEQUENTIAL DAMAGES OR ANY DAMAGES WHATSOEVER RESULTING FROM LOSS
 OF USE, DATA OR PROFITS, WHETHER IN AN ACTION OF CONTRACT,
 NEGLIGENCE OR OTHER TORTIOUS ACTION, ARISING OUT OF OR IN
 CONNECTION WITH THE USE OR PERFORMANCE OF THIS SOFTWARE.
\end{verbatim}


\subsection{Cookie management}

The \module{Cookie} module contains the following notice:

\begin{verbatim}
 Copyright 2000 by Timothy O'Malley <timo@alum.mit.edu>

                All Rights Reserved

 Permission to use, copy, modify, and distribute this software
 and its documentation for any purpose and without fee is hereby
 granted, provided that the above copyright notice appear in all
 copies and that both that copyright notice and this permission
 notice appear in supporting documentation, and that the name of
 Timothy O'Malley  not be used in advertising or publicity
 pertaining to distribution of the software without specific, written
 prior permission.

 Timothy O'Malley DISCLAIMS ALL WARRANTIES WITH REGARD TO THIS
 SOFTWARE, INCLUDING ALL IMPLIED WARRANTIES OF MERCHANTABILITY
 AND FITNESS, IN NO EVENT SHALL Timothy O'Malley BE LIABLE FOR
 ANY SPECIAL, INDIRECT OR CONSEQUENTIAL DAMAGES OR ANY DAMAGES
 WHATSOEVER RESULTING FROM LOSS OF USE, DATA OR PROFITS,
 WHETHER IN AN ACTION OF CONTRACT, NEGLIGENCE OR OTHER TORTIOUS
 ACTION, ARISING OUT OF OR IN CONNECTION WITH THE USE OR
 PERFORMANCE OF THIS SOFTWARE.
\end{verbatim}      



\subsection{Profiling}

The \module{profile} and \module{pstats} modules contain
the following notice:

\begin{verbatim}
 Copyright 1994, by InfoSeek Corporation, all rights reserved.
 Written by James Roskind

 Permission to use, copy, modify, and distribute this Python software
 and its associated documentation for any purpose (subject to the
 restriction in the following sentence) without fee is hereby granted,
 provided that the above copyright notice appears in all copies, and
 that both that copyright notice and this permission notice appear in
 supporting documentation, and that the name of InfoSeek not be used in
 advertising or publicity pertaining to distribution of the software
 without specific, written prior permission.  This permission is
 explicitly restricted to the copying and modification of the software
 to remain in Python, compiled Python, or other languages (such as C)
 wherein the modified or derived code is exclusively imported into a
 Python module.

 INFOSEEK CORPORATION DISCLAIMS ALL WARRANTIES WITH REGARD TO THIS
 SOFTWARE, INCLUDING ALL IMPLIED WARRANTIES OF MERCHANTABILITY AND
 FITNESS. IN NO EVENT SHALL INFOSEEK CORPORATION BE LIABLE FOR ANY
 SPECIAL, INDIRECT OR CONSEQUENTIAL DAMAGES OR ANY DAMAGES WHATSOEVER
 RESULTING FROM LOSS OF USE, DATA OR PROFITS, WHETHER IN AN ACTION OF
 CONTRACT, NEGLIGENCE OR OTHER TORTIOUS ACTION, ARISING OUT OF OR IN
 CONNECTION WITH THE USE OR PERFORMANCE OF THIS SOFTWARE.
\end{verbatim}



\subsection{Execution tracing}

The \module{trace} module contains the following notice:

\begin{verbatim}
 portions copyright 2001, Autonomous Zones Industries, Inc., all rights...
 err...  reserved and offered to the public under the terms of the
 Python 2.2 license.
 Author: Zooko O'Whielacronx
 http://zooko.com/
 mailto:zooko@zooko.com

 Copyright 2000, Mojam Media, Inc., all rights reserved.
 Author: Skip Montanaro

 Copyright 1999, Bioreason, Inc., all rights reserved.
 Author: Andrew Dalke

 Copyright 1995-1997, Automatrix, Inc., all rights reserved.
 Author: Skip Montanaro

 Copyright 1991-1995, Stichting Mathematisch Centrum, all rights reserved.


 Permission to use, copy, modify, and distribute this Python software and
 its associated documentation for any purpose without fee is hereby
 granted, provided that the above copyright notice appears in all copies,
 and that both that copyright notice and this permission notice appear in
 supporting documentation, and that the name of neither Automatrix,
 Bioreason or Mojam Media be used in advertising or publicity pertaining to
 distribution of the software without specific, written prior permission.
\end{verbatim} 



\subsection{UUencode and UUdecode functions}

The \module{uu} module contains the following notice:

\begin{verbatim}
 Copyright 1994 by Lance Ellinghouse
 Cathedral City, California Republic, United States of America.
                        All Rights Reserved
 Permission to use, copy, modify, and distribute this software and its
 documentation for any purpose and without fee is hereby granted,
 provided that the above copyright notice appear in all copies and that
 both that copyright notice and this permission notice appear in
 supporting documentation, and that the name of Lance Ellinghouse
 not be used in advertising or publicity pertaining to distribution
 of the software without specific, written prior permission.
 LANCE ELLINGHOUSE DISCLAIMS ALL WARRANTIES WITH REGARD TO
 THIS SOFTWARE, INCLUDING ALL IMPLIED WARRANTIES OF MERCHANTABILITY AND
 FITNESS, IN NO EVENT SHALL LANCE ELLINGHOUSE CENTRUM BE LIABLE
 FOR ANY SPECIAL, INDIRECT OR CONSEQUENTIAL DAMAGES OR ANY DAMAGES
 WHATSOEVER RESULTING FROM LOSS OF USE, DATA OR PROFITS, WHETHER IN AN
 ACTION OF CONTRACT, NEGLIGENCE OR OTHER TORTIOUS ACTION, ARISING OUT
 OF OR IN CONNECTION WITH THE USE OR PERFORMANCE OF THIS SOFTWARE.

 Modified by Jack Jansen, CWI, July 1995:
 - Use binascii module to do the actual line-by-line conversion
   between ascii and binary. This results in a 1000-fold speedup. The C
   version is still 5 times faster, though.
 - Arguments more compliant with python standard
\end{verbatim}



\subsection{XML Remote Procedure Calls}

The \module{xmlrpclib} module contains the following notice:

\begin{verbatim}
     The XML-RPC client interface is

 Copyright (c) 1999-2002 by Secret Labs AB
 Copyright (c) 1999-2002 by Fredrik Lundh

 By obtaining, using, and/or copying this software and/or its
 associated documentation, you agree that you have read, understood,
 and will comply with the following terms and conditions:

 Permission to use, copy, modify, and distribute this software and
 its associated documentation for any purpose and without fee is
 hereby granted, provided that the above copyright notice appears in
 all copies, and that both that copyright notice and this permission
 notice appear in supporting documentation, and that the name of
 Secret Labs AB or the author not be used in advertising or publicity
 pertaining to distribution of the software without specific, written
 prior permission.

 SECRET LABS AB AND THE AUTHOR DISCLAIMS ALL WARRANTIES WITH REGARD
 TO THIS SOFTWARE, INCLUDING ALL IMPLIED WARRANTIES OF MERCHANT-
 ABILITY AND FITNESS.  IN NO EVENT SHALL SECRET LABS AB OR THE AUTHOR
 BE LIABLE FOR ANY SPECIAL, INDIRECT OR CONSEQUENTIAL DAMAGES OR ANY
 DAMAGES WHATSOEVER RESULTING FROM LOSS OF USE, DATA OR PROFITS,
 WHETHER IN AN ACTION OF CONTRACT, NEGLIGENCE OR OTHER TORTIOUS
 ACTION, ARISING OUT OF OR IN CONNECTION WITH THE USE OR PERFORMANCE
 OF THIS SOFTWARE.
\end{verbatim}


\documentclass{manual}

\title{Python Reference Manual}

\input{boilerplate}

\makeindex

\begin{document}

\maketitle

\ifhtml
\chapter*{Front Matter\label{front}}
\fi

\input{copyright}

\begin{abstract}

\noindent
Python is an interpreted, object-oriented, high-level programming
language with dynamic semantics.  Its high-level built in data
structures, combined with dynamic typing and dynamic binding, make it
very attractive for rapid application development, as well as for use
as a scripting or glue language to connect existing components
together.  Python's simple, easy to learn syntax emphasizes
readability and therefore reduces the cost of program
maintenance.  Python supports modules and packages, which encourages
program modularity and code reuse.  The Python interpreter and the
extensive standard library are available in source or binary form
without charge for all major platforms, and can be freely distributed.

This reference manual describes the syntax and ``core semantics'' of
the language.  It is terse, but attempts to be exact and complete.
The semantics of non-essential built-in object types and of the
built-in functions and modules are described in the
\citetitle[../lib/lib.html]{Python Library Reference}.  For an
informal introduction to the language, see the
\citetitle[../tut/tut.html]{Python Tutorial}.  For C or
\Cpp{} programmers, two additional manuals exist:
\citetitle[../ext/ext.html]{Extending and Embedding the Python
Interpreter} describes the high-level picture of how to write a Python
extension module, and the \citetitle[../api/api.html]{Python/C API
Reference Manual} describes the interfaces available to
C/\Cpp{} programmers in detail.

\end{abstract}

\tableofcontents

\input{ref1}		% Introduction
\input{ref2}		% Lexical analysis
\input{ref3}		% Data model
\input{ref4}		% Execution model
\input{ref5}		% Expressions and conditions
\input{ref6}		% Simple statements
\input{ref7}		% Compound statements
\input{ref8}		% Top-level components

\appendix

\chapter{History and License}
\input{license}

\input{ref.ind}

\end{document}


\end{document}


\end{document}


\end{document}
